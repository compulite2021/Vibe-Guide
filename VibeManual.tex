% Options for packages loaded elsewhere
\PassOptionsToPackage{unicode}{hyperref}
\PassOptionsToPackage{hyphens}{url}
%
\documentclass[
]{article}
\title{Vibe Manual}
\author{Compulite Systems (2000) Ltd}
\date{}

\usepackage{amsmath,amssymb}
\usepackage{lmodern}
\usepackage{iftex}
\ifPDFTeX
  \usepackage[T1]{fontenc}
  \usepackage[utf8]{inputenc}
  \usepackage{textcomp} % provide euro and other symbols
\else % if luatex or xetex
  \usepackage{unicode-math}
  \defaultfontfeatures{Scale=MatchLowercase}
  \defaultfontfeatures[\rmfamily]{Ligatures=TeX,Scale=1}
\fi
% Use upquote if available, for straight quotes in verbatim environments
\IfFileExists{upquote.sty}{\usepackage{upquote}}{}
\IfFileExists{microtype.sty}{% use microtype if available
  \usepackage[]{microtype}
  \UseMicrotypeSet[protrusion]{basicmath} % disable protrusion for tt fonts
}{}
\makeatletter
\@ifundefined{KOMAClassName}{% if non-KOMA class
  \IfFileExists{parskip.sty}{%
    \usepackage{parskip}
  }{% else
    \setlength{\parindent}{0pt}
    \setlength{\parskip}{6pt plus 2pt minus 1pt}}
}{% if KOMA class
  \KOMAoptions{parskip=half}}
\makeatother
\usepackage{xcolor}
\IfFileExists{xurl.sty}{\usepackage{xurl}}{} % add URL line breaks if available
\IfFileExists{bookmark.sty}{\usepackage{bookmark}}{\usepackage{hyperref}}
\hypersetup{
  pdftitle={Vibe Manual},
  pdfauthor={Compulite Systems (2000) Ltd},
  hidelinks,
  pdfcreator={LaTeX via pandoc}}
\urlstyle{same} % disable monospaced font for URLs
\usepackage[margin=1in]{geometry}
\usepackage{longtable,booktabs,array}
\usepackage{calc} % for calculating minipage widths
% Correct order of tables after \paragraph or \subparagraph
\usepackage{etoolbox}
\makeatletter
\patchcmd\longtable{\par}{\if@noskipsec\mbox{}\fi\par}{}{}
\makeatother
% Allow footnotes in longtable head/foot
\IfFileExists{footnotehyper.sty}{\usepackage{footnotehyper}}{\usepackage{footnote}}
\makesavenoteenv{longtable}
\usepackage{graphicx}
\makeatletter
\def\maxwidth{\ifdim\Gin@nat@width>\linewidth\linewidth\else\Gin@nat@width\fi}
\def\maxheight{\ifdim\Gin@nat@height>\textheight\textheight\else\Gin@nat@height\fi}
\makeatother
% Scale images if necessary, so that they will not overflow the page
% margins by default, and it is still possible to overwrite the defaults
% using explicit options in \includegraphics[width, height, ...]{}
\setkeys{Gin}{width=\maxwidth,height=\maxheight,keepaspectratio}
% Set default figure placement to htbp
\makeatletter
\def\fps@figure{htbp}
\makeatother
\setlength{\emergencystretch}{3em} % prevent overfull lines
\providecommand{\tightlist}{%
  \setlength{\itemsep}{0pt}\setlength{\parskip}{0pt}}
\setcounter{secnumdepth}{5}
\ifLuaTeX
  \usepackage{selnolig}  % disable illegal ligatures
\fi

\begin{document}
\maketitle

{
\setcounter{tocdepth}{2}
\tableofcontents
}
\hypertarget{introduction}{%
\section{Introduction}\label{introduction}}

This chapter discusses getting started with your Compulite Vibe console.

\textbf{The following is covered in this chapter:}

\begin{itemize}
\tightlist
\item
  \href{https://vibemanual.compulite.com/index.html\#unboxing}{1.1. Unboxing}
\item
  \href{https://vibemanual.compulite.com/index.html\#getting-started}{1.2. Getting Started}
\item
  \href{https://vibemanual.compulite.com/index.html\#guide-conventions}{1.3. Guide Conventions}
\item
  \href{https://vibemanual.compulite.com/index.html\#important-terms}{1.4. Important Terms}
\item
  \href{https://vibemanual.compulite.com/index.html\#releases}{1.5 Releases}
\end{itemize}

\hypertarget{unboxing}{%
\subsection{Unboxing}\label{unboxing}}

TBA.

\hypertarget{getting-started}{%
\subsection{Getting Started}\label{getting-started}}

\textbf{Power requirements:}

\begin{itemize}
\tightlist
\item
  Vibe uses a power supply that supports 90-264V at 47-63 Hz.
\item
  Approximately 3.2 amps at 230V.
\item
  Input uses a Neutrik powerCON connector ended with the appropriate male connector for the destination country.
\end{itemize}

\textbf{Desk Light:}

\begin{itemize}
\tightlist
\item
  Insert the goose neck desk light into the 3 pin XLR on the rear of the console.
\item
  {Desk light intensity is controlled in the \{System Settings\} \{Hardware\} tab.}
\end{itemize}

\textbf{Monitor intensity:}

\begin{itemize}
\tightlist
\item
  {Monitor intensity is controlled in the \{System Settings\} \{Hardware\} tab.}
\end{itemize}

\textbf{Power up the console:}

\begin{itemize}
\tightlist
\item
  Insert the supplied powerCON male connector into the female connector located on the rear of the console. Align the guide slots and turn clockwise untill the connector locks.
\item
  Turn on the power supply switch above the powerCON connector.
\item
  Press the Power button located at the upper right corner of the monitor panel above the USB connectors.
\item
  The operating system will boot and the Vibe console software will automatically launch.
\end{itemize}

\textbf{Power down the console:}

\begin{itemize}
\tightlist
\item
  Tap the \{Vibe\} menu key beside the command line on the main display - The Vibe Menu will open.
\item
  Tap \{Exit\} - The \{Exit Vibe\} Pop-up will appear.
\item
  Tap \{Yes\} or \{No\}.
\item
  The Vibe application will shutdown and a screen will appear with the following options:

  \begin{itemize}
  \tightlist
  \item
    {\{Shut Down the Console\} - System will power off.}
  \item
    {\{Restart Vibe\} - Vibe console application will restart.}
  \item
    {\{Open Offline Tool\} - the Vibe utilities app will open. Hardware and software updates are preformed in the Offline Tools app.}
  \end{itemize}
\item
  Choose Shut Down the Console - The console will safely power off.
\end{itemize}

\hypertarget{guide-conventions}{%
\subsection{Guide Conventions}\label{guide-conventions}}

\begin{itemize}
\item
  \textbf{{[}XXXX{]}} - Physical key
\item
  \textbf{\{XXXX\}} - Soft key
\item
  \textbf{{[} {]}} \textbf{\{ \}} - Example
\item
  \textbf{Press} - Action applied to a physical key
\item
  \textbf{Tap} - Action applied to softkey or virtual key
\item
  \textbf{Toggle} - Action that actives and deactivates with alternating taps or presses
\item
  \textbf{Key} - Any physical or virtual key that is NOT used for controller activation
\item
  \textbf{Button} - Any virtual or physical key that is used to activate a controller
\item
  \textbf{Keypad} - Embedded numeric keypad
\item
  \textbf{Keyboard} - Console's pull-out ASCII keyboard or any external keyboard
\item
  \textbf{SK -} Object softkeys (Libraries, Groups, Snap, Macros, etc.)
\item
  👉 \textbf{-} Hint or suggestion
\item
  {\textbf{RED}} \textbf{-} Important information or Caution
\item
  \textbf{{[}Here{]}} - Short form for Press an appropriate destination softkey or controller button to complete the command
\end{itemize}

\hypertarget{important-terms}{%
\subsection{Important Terms}\label{important-terms}}

\begin{itemize}
\item
  \textbf{DMX 512 Output/Input Port} - One of the 8 physical 5 pin XLR connectors on the back of the console. Although the gender is female, any connector may be configured as an input but must use a turnaround adapter.
\item
  \textbf{DMX 512 Universe} - Virtual DMX outputs 0 ⟶ 256 that may be routed to the 8 physical outputs and/or any of the three supported DMX over Ethernet protocols.

  ◾ Compulite VC protocol

  ◾ w Art-Net 3 w

  ◾ sACN
\item
  \textbf{SET} = Fixture numbering syste

  \begin{itemize}
  \tightlist
  \item
    0 = Fixtures, (Default, includes all sets)
  \item
    1 = Channels (Conventional fixtures)
  \item
    2 = Spots (Automated heads and mirrors)
  \item
    3 = Matrix (LED pixel-based fixtures)
  \item
    4 = Servers (Video servers)
  \item
    5 ⟶ 1000 = User defined sets
  \end{itemize}
\item
  \textbf{Parameter} - An individual feature of a fixture such as dimmer, pan, tilt, color wheel, gobo.
\item
  \textbf{Controllers} - Any physical or virtual control that executes or plays back system objects such as cues, scenes, macros, and snapshots.
\item
  \textbf{Submaster} - A special type of Scene where stored fixtures inhibit their dimmer values proportionate to their assigned slider controller value.
\item
  \textbf{Master Controller} - Playback section of the console with large {[}GO{]}, {[}BACK{]}, {[}HOLD{]} keys.
\item
  \textbf{Slider} - Slide potentiometer often referred to as a \textbf{Fader}.
\item
  \textbf{Qkey} - A single button Controller.
\item
  \textbf{Assert} - Takes back control of parameters that have been ``robbed'' by other controllers.
\item
  \textbf{HTP (Highest-Takes-Precedence)} - When values from two or more controllers are summed, The highest value will output to the stage. Used mainly for dimmer values in theater applications.
\item
  \textbf{LTP (Latest-Takes-Precedence)} - When similar parameters are on two or more controllers, the last asserted parameter will output it's values to the stage no matter what the value.
\item
  \textbf{{[}SELECT{]}} - Multi-use key, mainly used to assign any controller as the Master Controller. Commands not specifying a Qlist destination will be applied directly to the {[}SELECT{]} Master Controller.
\item
  \textbf{Qlist} - A Folder containing cues. Each Qlist has its own unique numbering system from 0.001 to 999.999.
\item
  \textbf{Cue} - A look (record) contained within a Qlist. Cues contain stored parameter values \textbf{and} properties such as text label and time values.
\item
  \textbf{Scene} - A look that exists independent of a Qlist.
\item
  \textbf{Library} - Container of frequently used fixture parameter values (Bank Library) or system objects (Object Library) . Bank Libraries may be created as referenced or unreferenced. The default for most libraries is referenced.
\item
  \textbf{Options} - Used to modify object properties ``on the fly''.
\item
  \textbf{Settings} - Used to modify object properties and behaviors \textbf{after} an object has been created.
\item
  \textbf{Cue Zero} - The starting point at the top of a Qlist. Release may or may not return to the top of a Qlist depending on Qlist Properties. Cue Zero will always return to the top of the Qlist. The command Cue 0 GOTO is valid.
\end{itemize}

\begin{quote}
👉 For the sake of this guide, \textbf{Channel, Spot, Matrix, and Server Sets} will be referred to as \textbf{Fixture} because the Fixture Set can reference all the other sets
\end{quote}

\begin{quote}
👉 Some keys have two functions when used with the {[}VIBE{]} key (shift key). In this case, the key label on the bottom is the default, and top label is the shifted function.
\end{quote}

\hypertarget{releases}{%
\subsection{Releases}\label{releases}}

\hypertarget{r.3.0}{%
\subsubsection{R.3.0}\label{r.3.0}}

\begin{itemize}
\tightlist
\item
  Release Date: \textbf{06/09/2020}
\item
  DB version: \textbf{R\_3.0}
\item
  Desktop version: \textbf{R\_3.0}
\item
  LPU version: \textbf{R\_3.0}
\item
  HAL version: \textbf{R\_3.0}
\item
  Device Pack version: \textbf{DP\_3.0}
\end{itemize}

\hypertarget{version-highlights}{%
\paragraph{Version Highlights}\label{version-highlights}}

Version R\_3.0 brings new and updated features as listed below:

\textbf{Settings}

\begin{itemize}
\item
  \textbf{Controller Default Actions}- It is now possible to set a default controller action for each object type on each controller type.
\item
  \textbf{Exit Button Location} - The location for the button to exit the system or close a window was moved
\item
  \textbf{Clear Device IP History}- Added this option to clear the cache under the system settings.
\item
  \textbf{Show/Hide View Headers}- Added an option to show or hide view headers in order to make room on screen.
\item
  \textbf{Softkeys Size Improvements}- It is now possible to make the softkey view very compact
\item
  \textbf{Unicast Option for Art-Net And VC}- It is now possible to set specific IPs to send the output to
\item
  \textbf{Defaults Changes}- There are some new defaults for: Update popup -- new values to master pb Bottom Button on submasters -- flash Pan \& Tilt Effect default size
\end{itemize}

\textbf{Patch}

\begin{itemize}
\tightlist
\item
  \textbf{Edit Sets}- There is an option to edit user-sets on the Sets view
\end{itemize}

\textbf{Workflow}

\begin{itemize}
\item
  \textbf{Media Bank}- There is a new bank called Media on the Small Screen
\item
  \textbf{Toggle Between Banks}- The combination of SHIFT+\# on the keypad will switch a bank on the small screen
\item
  \textbf{Cell Time Fan}- It is now possible to fan cell times via the keypad
\item
  \textbf{Parameter Profile}- It is now possible to set a profile for parameters to change their fade behavior
\end{itemize}

\textbf{Timeline}

\begin{itemize}
\item
  \textbf{Grid View Filters}- It is now possible to filter timelines on the grid view
\item
  \textbf{Time Format}- It is now possible to decide the time format on the Timeline View
\item
  \textbf{Moving Events}- It is now possible to move events on the Timeline with the wheels
\item
  \textbf{Events Selection}- It is not possible to select events by creating a rectangle of selection
\item
  \textbf{State Machine Support}- It is now possible to trigger Timeline commands from the keypad + toolbar commands
\end{itemize}

\textbf{Popups}

\begin{itemize}
\item
  \textbf{New ``Loading'' Popup}- There is a new popup for continuous loading
\item
  \textbf{Remove Function}- On the effects popup, it is now possible to remove an added function
\item
  \textbf{CITP Popup}- There is a new popup for CITP configuration and commands
\end{itemize}

\textbf{General}

\begin{itemize}
\tightlist
\item
  \textbf{Auto Patch from Capture}- Using CITP protocol, it is now possible to retrieve patch information from Capture
\end{itemize}

\textbf{Fixture Builder}

\begin{itemize}
\item
  \textbf{Option to edit Shutter Picker commands}- It now possible to create/edit commands for the Shutter picker
\item
  \textbf{Option to edit Frost Picker commands}- It now possible to create/edit commands for the Frost picker
\item
  \textbf{Option to edit Iris Picker commands}- It now possible to create/edit commands for the Iris picker
\item
  \textbf{Option to edit Zoom Picker commands}- It now possible to create/edit commands for the Zoom picker
\item
  \textbf{Option to edit Blade Picker commands}- It now possible to create/edit commands for the Blade picker
\item
  \textbf{Option to edit Focus Picker commands}- It now possible to create/edit commands for the Focus picker
\item
  \textbf{New Parameter Types}- Added new parameter types to Vibe
\end{itemize}

\textbf{Offline Tool}

\begin{itemize}
\item
  \textbf{Panel Test}- New Panel test mode to check the hardware
\item
  \textbf{Midi / SMPTE Test}- New Midi / SMPTE test mode
\item
  \textbf{Update Firmware Options}- There are new possibilities to update the console firmware
\end{itemize}

\hypertarget{bug-fixes}{%
\paragraph{Bug Fixes}\label{bug-fixes}}

\begin{itemize}
\item
  Fixed a crash when removing a point from the Function editor
\item
  Fixed a crash when searching Exam View and pressing SHIFT+RESET
\item
  Fixed a potential crash on startup
\item
  Fixed a potential crash on Undo popup
\item
  Fixed potential crash on HAL
\item
  Fixed a crash when pressing Actions on the Timeline view
\item
  Fixed a crash when pressing apply on the Link Cue popup
\item
  Fixed a crash on LPU when deleting a cue that is next on a PB and it's values are active as Look Ahead
\item
  Fixed a crash on LPU when releasing a playback during loop could cause it on some cases
\item
  Fixed a crash on filters popup when deleting the last parameter from the list
\item
  Fixed a crash when deleting a device that had no fixtures in the show
\item
  Loading a big show could cause the system to get stuck
\item
  Rem Dim could cause errors on DB
\item
  Grab Active Master PB could cause errors on DB
\item
  Live View - When parameter jumping is on its should works all the cases
\item
  Performance fixes in specific sequences as well as general system speed
\item
  Fixture selection was slow and caused many updates to small screen, especially when selecting with `+'
\item
  When fixture has values, entering to Blind editor where the values there are at home, will make a graphic flick
\item
  ``Exclude from Snap'' and ``Exclude from Override'' did not work correctly \textbf{Snaps stored with these options need to be rebuilt!}
\item
  It was not possible to select/deselect a cell on the first column in Cell Time popup
\item
  Exam of a new device did not show the info on Exam View
\item
  Exam view did not refresh after deleting a Qlist
\item
  Wheel picker sometimes was not accurate and selected the wrong step
\item
  Address test delta value was showing wrong value
\item
  Changes to Address properties did not output unless applying the popup
\item
  Fixtures with more than 1 color layer could make the HIS picker stuck
\item
  Data on the status view was not always refreshed
\item
  Filters popup -- changes were not saved if user switched between filters
\item
  Activate a snap after deleting Qlists loaded them in a ghost mode
\item
  Deleting Libraries, Macros, Scenes did not remove them from the snaps they were stored in
\item
  Deleting Snaps did not clear the data correctly
\item
  Master PB did not change according to the data stored in the Snap
\item
  Fixture selection numbers did not cut to the actual maximum number on echoline
\item
  When Lib stored, echoline looks strange
\item
  Sometimes \protect\hyperlink{transpose}{Transpose} doesn't ordering the fixtures correctly
\item
  Park on parked parameters did not work correctly with Grand Master
\item
  A/B Slider behavior was wrong for manual fade.
\item
  Faders kept sending data even though the value did not change
\item
  Cue time units on popup did not change
\item
  SHIFT+GRAB took a lot of time
\item
  Delete a cue took a lot of time
\item
  Updating cues caused an error on DB and sometimes caused wrong values to be stored
\item
  Release value from Cue did not work correctly
\item
  Updating a copied cue/scene did not update it
\item
  Update cue values from the Tracksheet view did not always work
\item
  Load track data to cue did not load from a block cue
\item
  When updating a cue that after it there is a cue with virtual params stored, the params are changed
\item
  When Lib is stored as block, update the previous cue will make track to the block
\item
  After releasing a parameter from a cue, values look like they `Block' even though they are tracked
\item
  After remove a reference from a Lib, all cues that are using any Libs will lose their reference
\item
  When updating a Hard value using a referenced Lib it should select the source cue in the update popup
\item
  Updating a Virtual Color Cue to Physical Color via update pop-up will track to the next cue if stored with Virtual Color
\item
  Cannot release a param so it will be tracked again from a Lib
\item
  Release Virtual color to physical color, vice versa, will release to Home
\item
  Release a Virtual Lib to Virtual hard value or Physical Lib to Physical hard value, keeps the notation of the Lib in the Live view
\item
  When storing a Lib, the Library notation and reference in Live is being applied only on the selected fixtures
\item
  Virtual and Physical params are acting weird when ``maintain last value'' is active
\item
  Can't delete Highlight or Lowlight cues
\item
  Link Cue popup showed error when opened from the toolbar
\item
  If a cue had time-in, time-out and follow in CUT, it was skipping the cue
\item
  Color values are jumping when Qlist is tracking off and follow time is CUT
\item
  Release at bottom took the Qlist release time instead of working in CUT
\item
  Clear editor on master Go, clears the editor in any go
\item
  Color icon is not changing when in Blind mode
\item
  If param is selected, and then you change to `Parameter Steps' view, it does not show the steps of the param
\item
  Small screen effects showed wrong parameters
\item
  When Randomize fixtures in the effect, and more than 1 pattern is selected, it should shuffle them in the same way
\item
  Randomize an existing effect that has 2 params on the same row via small effect engine after selecting the params is wrong
\item
  Randomize fixtures should randomize only the order of the fixtures
\item
  When Pan effect was running, pressing the Tilt made a jump to 0
\item
  Rate window is flickering with the cues fades
\item
  Context view got back to idle after selecting a library
\item
  Context did not display Filter SK
\item
  The name of context view when its tabbed is wrong
\item
  SMPTE FPS value was wrong
\item
  Group Select/Deselect/Release did not work on Qlists and Scenes
\item
  `Group Release' button stuck the params it releases
\item
  {[}Shift{]}+Master {[}Go{]}{[}Back{]} should make a Go\Back on master in Cut time
\item
  RDM settings are not stored in show file
\item
  Edit control - caret remains visible after focusing out
\item
  Fixture Builder:

  \begin{itemize}
  \item
    Errors on edit factory devices with overlapping steps
  \item
    Editing a show device could cause it to be cloned infinite times
  \item
    Parameter Step tab did not update the navigation buttons
  \item
    Edit boxes in Parameter Steps table did not work
  \item
    Memory leaks fixes
  \end{itemize}
\item
  Vector show conversion bugs
\end{itemize}

\hypertarget{general-specification}{%
\section{General Specification}\label{general-specification}}

This chapter covers basic specifications.

\textbf{The following is covered in this chapter:}

\begin{itemize}
\tightlist
\item
  \href{https://vibemanual.compulite.com/general-specification.html\#controls}{2.1. Controls}
\item
  \href{https://vibemanual.compulite.com/general-specification.html\#io}{2.2. I/O}
\item
  \href{https://vibemanual.compulite.com/general-specification.html\#capacity}{2.3. Capacity}
\end{itemize}

\hypertarget{controls}{%
\subsection{Controls}\label{controls}}

\begin{itemize}
\item
  4 large backlit push encoder wheels.
\item
  4 small interactive backlit push encoder wheels.
\item
  Backlit trackball.
\item
  Dedicated dimmer encoder wheel.
\item
  15 \textbf{Motorized Faders}.
\item
  20 \textbf{Qkey} controller keys.
\item
  20 Multi-use \textbf{Auxiliary Qkeys} .
\item
  5 general purpose non-motorized \textbf{Global Faders}.
\end{itemize}

\hypertarget{io}{%
\subsection{I/O}\label{io}}

\begin{itemize}
\item
  8 Physical DMX 512 input/output ports (RDM supported).
\item
  Standard 64 DMX over Ethernet universes. 96, 128, 256, DMX universes optional via license.
\item
  Support for Compulite VC, Art-Net 3, and sACN protocols. RDM is currently supported locally and over Compulite VC networks. (Art-Net and sACN in the near future)
\item
  MIDI In/Out.
\item
  SMPTE In.
\item
  1 Ethernet data network with 2 etherCON ports.
\item
  2 Ethernet networks for accessories and additional devices such as NAS storage.
\item
  4 USB 3 ports on the back panel.
\item
  2 USB 2 ports on the front panel.
\item
  3.5mm audio line in/out.
\item
  3.5mm Mic in.
\item
  3.5mm Speaker out.
\item
  2 DisplayPort video outputs.
\end{itemize}

\hypertarget{capacity}{%
\subsection{Capacity}\label{capacity}}

\begin{itemize}
\item
  100 physical/virtual motorized fader controllers.
\item
  100 physical/virtual Qkey single button controllers.
\item
  100 physical/virtual Auxiliary Qkey single button controllers.
\item
  5 global multi-purpose non-motorized Slider controllers.
\item
  30 individual pages for motorized Sliders, Qkeys, and Auxiliary Qkeys.
\item
  9000 total physical/virtual controllers.
\item
  1000 Qlists.
\item
  Virtually unlimited cues.
\item
  1000 Scenes.
\item
  1000 fixture Groups.
\item
  1000 Libraries per bank type.
\item
  1000 Effects.
\item
  1000 Snapshots.
\item
  1000 Macros.
\end{itemize}

\hypertarget{vibe-hardware}{%
\section{Vibe Hardware}\label{vibe-hardware}}

This chapter provides an overview of Vibe hardware.

\textbf{The following is covered in this chapter:}

\begin{itemize}
\tightlist
\item
  \href{https://vibemanual.compulite.com/vibe-hardware.html\#console}{3.1. Console}
\item
  \href{https://vibemanual.compulite.com/vibe-hardware.html\#editor-controls}{3.2. Editor Controls}
\item
  \href{https://vibemanual.compulite.com/vibe-hardware.html\#motorized-sliders-and-qkeys}{3.3. Motorized Sliders and Qkeys}
\item
  \href{https://vibemanual.compulite.com/vibe-hardware.html\#master-controller-and-control-keys}{3.4. Master Controller and Control Keys}
\item
  \href{https://vibemanual.compulite.com/vibe-hardware.html\#aux-qkeys-grand-master-and-global-controllers}{3.5. Aux Qkeys, Grand Master, and Global Controllers}
\end{itemize}

\hypertarget{console}{%
\subsection{Console}\label{console}}

Front Panel View

Rear Panel View

\hypertarget{editor-controls}{%
\subsection{Editor Controls}\label{editor-controls}}

Editor Controls

Editor Controls are used to program and edit parameter values for all fixtures as well as create and edit objects such as cues and libraries.

• \textbf{{[}Vibe{]}} Key (Shift Key)

• Numeric Key Pad § 4 Main encoder push wheels

• Trackball

• \textbf{{[}Next{]}/{[}Prev{]}} Keys

• XYZ lock keys

• \textbf{{[}Res{]}} Wheel resolution key

• Dedicated dimmer wheel

• Command select keys

• Object select keys

• {Navigation keys}

\hypertarget{motorized-sliders-and-qkeys}{%
\subsection{Motorized Sliders and Qkeys}\label{motorized-sliders-and-qkeys}}

Motorized Slider and Qkey Controllers

15 individually paging 60mm Motorized slider potentiometers and 15 individually paging Qkey single button controllers, used to execute and control objects such as:

• Qlist Cues

• Scenes

• Effects Rate and Size

• Group selects

• Group Masters (Submasters)

• Libraries

• {Macros}

• {Snap}

\hypertarget{master-controller-and-control-keys}{%
\subsection{Master Controller and Control Keys}\label{master-controller-and-control-keys}}

Master Controller and Control Keys

• Non-motorized 100mm Sliders, used for Theatrical playback of cues

• Large \textbf{{[}GO/BACK/HOLD{]}} Keys for executing cues on the Master Controller

• Master Controller \textbf{{[}SELECT{]}} key

• Controller \textbf{{[}RELEASE{]}} Key

• \textbf{{[}LOAD{]}} key to pre-load a controller for execution

• Controller \textbf{{[}FREE{]}} Key (unload)

• \textbf{12 {[}Controller Keys{]}} to provide additional or missing functions to controllers.

• \textbf{{[}RATE{]}} for overall or individual rate override of controllers

• \textbf{{[}TEACH{]}} for BBM tap time

\hypertarget{aux-qkeys-grand-master-and-global-controllers}{%
\subsection{Aux Qkeys, Grand Master, and Global Controllers}\label{aux-qkeys-grand-master-and-global-controllers}}

Aux Qkeys, Grand Master, Blackout, and General Sliders

• 20 individually paging Auxiliary \textbf{Qkey Controllers} for single button execution of Cues, Scenes, Group Selects, {Snaps and Macros}.

• 5 Non-motorized single button Sliders for execution and control of Scenes, Cues, Group Masters (Submasters), {Rate Masters, and Flash Masters}.

• Grand Master Slider.

• \textbf{{[}BO{]}} Blackout key with blackout option.

• \textbf{\protect\hyperlink{patch}{PATCH} {[}PROGRAM{]}} and \textbf{{[}PLAY-B{]}} Workspace Template keys available for accessing display pages.

• {\textbf{{[}View{]}}, stores or executes display snapshots.}

\hypertarget{graphical-user-interface-gui}{%
\section{Graphical User Interface (GUI)}\label{graphical-user-interface-gui}}

This chapter provides an overview of Vibe's Graphical User Interface (GUI).

\textbf{The following is covered in this chapter:}

\begin{itemize}
\tightlist
\item
  \href{https://vibemanual.compulite.com/graphical-user-interface-gui.html\#introduction-1}{4.1. Introduction}
\item
  \href{https://vibemanual.compulite.com/graphical-user-interface-gui.html\#vibe-menu}{4.2. \{VIBE\} Menu}
\item
  \href{https://vibemanual.compulite.com/graphical-user-interface-gui.html\#configuring-workspaces}{4.3 Configuring WorkSpaces}
\item
  \href{https://vibemanual.compulite.com/graphical-user-interface-gui.html\#layouts}{4.4. Layouts}
\item
  \href{https://vibemanual.compulite.com/graphical-user-interface-gui.html\#smart-screen}{4.5. Smart Screen}
\item
  \href{https://vibemanual.compulite.com/graphical-user-interface-gui.html\#toolbars}{4.6. Toolbars}
\item
  \href{https://vibemanual.compulite.com/graphical-user-interface-gui.html\#live-display}{4.7. Live Display}
\item
  \href{https://vibemanual.compulite.com/graphical-user-interface-gui.html\#live-parameter-display}{4.8. Live Parameter Display}
\end{itemize}

\hypertarget{introduction-1}{%
\subsection{Introduction}\label{introduction-1}}

Vibe uses a Multi-touch based interface with simple familiar gestures for navigation.

\begin{itemize}
\item
  Supported gestures:

  ◾ Two fingers for swipe and scrolling.

  ◾ One finger select.

  ◾ Multi-selection via ️️⤴ ⤵ shaped gestures.

  ◾ {Multi-selection via window.}
\item
  Vibe directly supports up to 4 monitors, two embedded and two external. (Additional monitors may be added using DisplayPort 1.2 or higher MST splitter for a total of 4 external monitors at 1920 x 1080 resolution). The large 21'' embedded screen is general-purpose like the external monitors.
\item
  The smaller 11.6'' \textbf{Smart Screen} is dedicated to displaying context sensitive interactive parameter information such as color pickers and gobo pickers.
\item
  Each monitor, (excluding the \textbf{Smart Screen}) can contain one Window Frame.
\item
  Window frames can contain one of three \textbf{Workspace Templates}.
\item
  Each Template has a dedicated display in the lower area. This area will have different fixed options and information depending on the selected Workspace Template.
\item
  The remainder of the template may be freely customized by the user.
\item
  Each Template can contain up to 4 \textbf{Pages} that can be toggling using the \textbf{WorkSpace} \textbf{Template} \textbf{Keys} or by tapping the yellow \textbf{\{VIBE\}} Menu key at the bottom left of each Workspace Template and tapping the \textbf{\{PAGES\}} key to the right of the \textbf{\{VIBE\}} key to open the page's display.
\item
  Each Monitor (excluding the \textbf{Smart Screen}) can contain its own independent set of \textbf{WorkSpace} \textbf{Templates}.
\item
  Workspace Templates are organized into three familiar categories, each with a dedicated display key. These keys are found above the top left corner of the \textbf{Smart Screen}:

  ◾ \textbf{\protect\hyperlink{patch}{PATCH}}

  ◾ \textbf{{[}PROG{]}} - (Program)

  ◾ \textbf{{[}PLAY-B{]}} - (Playback Controllers)
\item
  The Workspace Template Keys are specific to the monitor that has focus (last touched or selected via a mouse).
\item
  \textbf{The Patch Template} by default has a fixed command line and patch tool bar at the bottom, and a pre-built patch display on page 1. Except for the bottom toolbar, the patch display may be freely customized via the \textbf{\{VIBE\}} Menu.
\item
  \textbf{The Program Template} contains a command line and fixed display of the 15 motorized Slider controllers. Everything above this fixed display is user customizable via the \textbf{\{VIBE\}} Menu. The controller display may be toggled to display the Qkey controller display using the \textbf{{[}SWAP DISPLAY{]}} key at the top left corner of the controller panel.
\item
  \textbf{The Playback Template} contains a command line and a fixed display of both the 15 motorized fader controllers and the 15 Qkey controllers at the same time. Everything above this fixed display is user customizable via the \textbf{\{VIBE\}} Menu. The controller displays may be swapped to display the Qkey controllers on the bottom row via the \textbf{{[}SWAP DISPLAY{]}} key at the top left corner of the controller panel.
\end{itemize}

\hypertarget{vibe-menu}{%
\subsection{VIBE Menu}\label{vibe-menu}}

The \textbf{\{VIBE\}} Menu is a hierarchical menu containing logically grouped sub-menus. It is used to access common system functions such as \textbf{Save} and \textbf{System Settings} as well as configure \textbf{Workspace Template} displays.

\includegraphics{https://files.gitbook.com/v0/b/gitbook-x-prod.appspot.com/o/spaces\%2F3kS90tLsADGm1ocbe7q9\%2Fuploads\%2FhnmbOYCxpKpFeS3PTChB\%2F4.2.webp?alt=media\&token=127302c8-d5cc-43aa-92f1-7eee592f68a8}

\hypertarget{exit-button}{%
\subsubsection{Exit Button}\label{exit-button}}

The Exit and Close buttons are located in the Windows sub-menu.

\textbf{To exit Vibe or close a window}

\begin{enumerate}
\def\labelenumi{\arabic{enumi}.}
\item
  Open Vibe menu from the Vibe button
\item
  Tap Windows
\item
  Tap Exit/Close
\end{enumerate}

\includegraphics{https://files.gitbook.com/v0/b/gitbook-x-prod.appspot.com/o/spaces\%2F3kS90tLsADGm1ocbe7q9\%2Fuploads\%2FFV8W9WgLTCerYjssYKJD\%2Fimage.png?alt=media\&token=a171e965-52eb-47d5-b975-7bd50df1ac00}

\hypertarget{configuring-workspaces}{%
\subsection{Configuring WorkSpaces}\label{configuring-workspaces}}

\hypertarget{basics}{%
\subsubsection{Basics}\label{basics}}

\begin{itemize}
\item
  When configuring workspaces, any object at the end of the hierarchical sub-menu can be tapped to open its object on a blank portion of the current workspace or dragged and dropped to the workspace surface. Once on the surface it may be moved or sized freely, space permitting.
\item
  If an object is dragged on top of another both will turn green and become one \textbf{tabbed} object.
\item
  To edit the placement of objects, the \textbf{\{Lock\}} icon beside the \textbf{\{VIBE\}} menu key must be toggled to unlock the workspace display. After changes are made, it must be toggled again to lock the display and allow normal usage.
\end{itemize}

\includegraphics{https://files.gitbook.com/v0/b/gitbook-x-prod.appspot.com/o/spaces\%2F3kS90tLsADGm1ocbe7q9\%2Fuploads\%2FRgauawhpMT9fs8UyREov\%2F4.2.1.webp?alt=media\&token=9ba0114d-9b28-4942-b8cc-835d775dae8b}

\hypertarget{showhide-view-headers}{%
\subsubsection{Show/Hide View Headers}\label{showhide-view-headers}}

This option is valid when the layout is unlocked and the view is selected. Showing or hiding a view's header is saved in the layout once layout is saved. Loading old layouts will load view with their headers as they were stored before.

\textbf{To Show/Hide View Header:}

\begin{enumerate}
\def\labelenumi{\arabic{enumi}.}
\item
  Go to edit mode by tapping the Vibe button and unlocking the layout
\item
  Select the view you want to toggle its header on or off
\item
  Tap the Toggle Header button
\end{enumerate}

\includegraphics{https://files.gitbook.com/v0/b/gitbook-x-prod.appspot.com/o/spaces\%2F3kS90tLsADGm1ocbe7q9\%2Fuploads\%2F9B5NTIysFGLAcAACtoIr\%2Fimage.png?alt=media\&token=f0d63ddd-259d-43c7-bd02-709fb7315cc1}

\hypertarget{layouts}{%
\subsection{Layouts}\label{layouts}}

\textbf{Layouts:}

\begin{itemize}
\tightlist
\item
  After all workspaces and associated pages are configured, the display configurations of the workspaces can be saved to a \textbf{Layout} by tapping the \textbf{\{VIBE\}} Menu and then the \textbf{\{LAYOUT\}} key.
\end{itemize}

A Sub-menu will open with options to \textbf{\{SAVE LAYOUT\}}, \textbf{\{LOAD LAYOUT\}}.

\begin{itemize}
\item
  After configuring external monitors, a layout should be saved to preserve the external monitor layout on reboot.
\item
  Layouts remember the configuration of external monitors and a warning message will appear if the system is booted without the expected external monitor.
\end{itemize}

\hypertarget{smart-screen}{%
\subsection{Smart Screen}\label{smart-screen}}

The 11.6'' multi-touch monitor embedded in the upper section of the Vibe console is referred to as the \textbf{Smart Screen}. The \textbf{Smart Screen} is dedicated to displaying context sensitive interactive bank, wheel, and parameter information.

\textbf{For more detail see:}

\href{https://compulite2021.github.io/Vibe-Guide/programming-basics.html\#smart-screen-1}{9.3.4. Smart Screen}

\hypertarget{toolbars}{%
\subsection{Toolbars}\label{toolbars}}

\textbf{The Editor Toolbar contains is a set of interactive Softkeys that are used to access less common commands:}

\includegraphics{https://files.gitbook.com/v0/b/gitbook-x-prod.appspot.com/o/spaces\%2F3kS90tLsADGm1ocbe7q9\%2Fuploads\%2FEtMwno5lafumX8bqox4j\%2F4.5.webp?alt=media\&token=c082bd36-0eb9-4402-8a03-bede1ce46ec1}

\textbf{The Patch Toolbar contains two pages of options:}

\begin{itemize}
\tightlist
\item
  Idle state toolbar
\end{itemize}

\includegraphics{https://files.gitbook.com/v0/b/gitbook-x-prod.appspot.com/o/spaces\%2F3kS90tLsADGm1ocbe7q9\%2Fuploads\%2Fk1fZqMaKocauppohAgHw\%2F4.5.1.webp?alt=media\&token=c8cafb32-4d46-4284-ab22-4dfc5699cf4b}

\begin{itemize}
\tightlist
\item
  Fixture selection state toolbar
\end{itemize}

\includegraphics{https://files.gitbook.com/v0/b/gitbook-x-prod.appspot.com/o/spaces\%2F3kS90tLsADGm1ocbe7q9\%2Fuploads\%2Fn6vIb0nGIrhdgceYRjec\%2F4.5.2.webp?alt=media\&token=0ef6d1a9-45f8-4015-b8c0-c1e3a115df30}

\hypertarget{live-display}{%
\subsection{Live Display}\label{live-display}}

Vibe has two ways to display parameter data:

\begin{enumerate}
\def\labelenumi{\arabic{enumi}.}
\tightlist
\item
  \textbf{Live Display} - Displays cue and editor data as a spreadsheet.
\item
  \textbf{Live Parameter Display} - shows values in a table format instead of spreadsheet format.
\end{enumerate}

\textbf{Live Display}

\includegraphics{https://files.gitbook.com/v0/b/gitbook-x-prod.appspot.com/o/spaces\%2F3kS90tLsADGm1ocbe7q9\%2Fuploads\%2F6qvBRbxpA4zVqzi1eqFx\%2F4.6.webp?alt=media\&token=9f4db473-e88e-439e-9c84-3ea59a199c95}

\textbf{The Live Display has a number of options that may be set under the Options field:}

\begin{itemize}
\item
  \href{image.png}{} - Opens Live View Settings. (See \href{/s/3kS90tLsADGm1ocbe7q9/~/changes/WdkBzgj8gGqCInAC3DkK/4.-graphical-user-interface-gui/4.-graphical-user-interface-gui/4.8.1.-live-view-settings}{4.6.1.Live View Settings}.)
\item
  \textbf{Fit} - Opens column width options.

  ◾ Fit to Window

  ◾ Fit to Value

  ◾ Fix by Text

  ◾ Default
\item
  \textbf{Active on Stage} - Only parameter columns that are selected in the editor or have values derived from the active stage look are shown. All others are hidden.
\end{itemize}

\includegraphics{https://files.gitbook.com/v0/b/gitbook-x-prod.appspot.com/o/spaces\%2F3kS90tLsADGm1ocbe7q9\%2Fuploads\%2FACLxFHojrpM1W1uVZfRf\%2F4.6.1.webp?alt=media\&token=cf1c4fbf-74a4-47f5-9bab-085fb9cb925e}

\begin{itemize}
\item
  \textbf{Sets} - Select which of the fixture sets to display.

  ◾ Fixture s Channel

  ◾ Spot s Matrix

  ◾ Media Server

  ◾ Other custom Sets if created
\item
  \textbf{Format} - Controls what format the data is displayed in.

  ◾ \% - Values in percentage

  ◾ Decimal - DMX values 0 ⟶ 255

  ◾ Text - Parameter step text

  ◾ Text + Percentage
\end{itemize}

\hypertarget{live-view-settings}{%
\subsubsection{Live View Settings}\label{live-view-settings}}

Live View Settings pop-up - Display Tab.

\includegraphics{https://files.gitbook.com/v0/b/gitbook-x-prod.appspot.com/o/spaces\%2F3kS90tLsADGm1ocbe7q9\%2Fuploads\%2F5uGp3hwefoypGXdWveRx\%2F4.7.webp?alt=media\&token=ab20ee96-1d11-43a8-bdf2-b4183886bcb0}

\textbf{Live View Scheme} - Parameter columns may be freely added, removed, moved in using the Live View Scheme.

\textbf{Move columns:}

\begin{enumerate}
\def\labelenumi{\arabic{enumi}.}
\item
  Tap a parameter column heading or multiple parameter column headings - the green indicator light will turn on.
\item
  Use the \textbf{\{\textbar\textless\}} or \textbf{\{\textgreater\textbar\}} keys to move the parameter column forward or backwards.
\end{enumerate}

\textbf{Move columns to the first or last column position:}

\begin{enumerate}
\def\labelenumi{\arabic{enumi}.}
\item
  Tap a parameter column heading or multiple parameter column headings - the green indicator light will turn on.
\item
  Use the \textbf{\{\textbar\textless\textless\}} or \textbf{\{\textgreater\textgreater\textbar\}} keys to move the parameter column first position or last position.
\end{enumerate}

\textbf{Remove} \textbf{columns:}

\begin{enumerate}
\def\labelenumi{\arabic{enumi}.}
\item
  Tap a parameter column heading or multiple parameter column headings - the green indicator light will turn on.
\item
  Tap the \{Remove\} key.
\end{enumerate}

\textbf{Dock} - Locks the column scroll at the end of the parameter column that is docked. A red line will show on the live display to show the dock position. (Group By Banks and All icons in 1 cell are not implemented yet)

\textbf{Add Columns} - Columns are added using the \textbf{Banks} section of the pop-up.

\begin{enumerate}
\def\labelenumi{\arabic{enumi}.}
\item
  Select a bank - All parameters for the bank will appear in the box adjacent to the banks.
\item
  Select the parameter or parameters to add to the live display. 👉 Swipe in the the parameter box to browse hidden parameters.
\item
  Tap a parameter that is not already selected in green and it will be added to the end of the column list.
\end{enumerate}

\begin{quote}
👉 {Parameter keys are toggles so if the indicator light is on, touching the parameter will remove it. Tapping again will put it back at the end of the columns.
}.
\end{quote}

\begin{enumerate}
\def\labelenumi{\arabic{enumi}.}
\setcounter{enumi}{3}
\tightlist
\item
  Use the \textbf{\{\textbar\textless\}} or \textbf{\{\textgreater\textbar\} \{\textbar\textless\textless\}} or \textbf{\{\textgreater\textgreater\textbar\}} to position the column.
\end{enumerate}

\textbf{Interactive Icons:}

\begin{itemize}
\tightlist
\item
  Adds or removes the icon columns.
\end{itemize}

\begin{quote}
👉 {Interactive Icon keys are toggles so if the indicator light is on, touching an icon key will remove it. Tapping again will put it back at the end of the columns.}.
\end{quote}

\textbf{General:}

\begin{itemize}
\tightlist
\item
  Global Fixture Number - Fixture's hidden unique system number.
\item
  Fixture Layer Number - Adds or removes Fixture Number column.
\item
  Fixture Name - Adds or removes Fixture Name column.
\end{itemize}

\begin{quote}
👉 {Fixture \# and Fixture Name are toggles so if the indicator light is on, touching an icon key will remove it. Tapping again will put it back at the end of the columns.}
\end{quote}

\textbf{Live View Settings pop-up - Behavior Tab}

\includegraphics{https://files.gitbook.com/v0/b/gitbook-x-prod.appspot.com/o/spaces\%2F3kS90tLsADGm1ocbe7q9\%2Fuploads\%2F0Cf1rvhFttH5PpWmRGsr\%2F4.7.1.webp?alt=media\&token=add0fffb-777d-4be4-b946-9baf95669f4e}

\textbf{Fixture Jumping}:

\begin{itemize}
\tightlist
\item
  Off
\item
  Jump if needed - live display pages the display to show the selected fixtures if they are not shown on the current live page.
\item
  Lowest Selection to Top - Jumps the lowest fixture number of the selection to the top of the live display.
\end{itemize}

\textbf{Parameter Jumping}:

\begin{itemize}
\tightlist
\item
  Jumps the parameter's column and related bank parameters to the end of the dock position. (First scrollable column).
\end{itemize}

{\textbf{Interactive Fixture Selection:}}

\begin{itemize}
\tightlist
\item
  {Turns on the ability to make fixture selections from the live display grids.}
\end{itemize}

{\textbf{Interactive} \textbf{Cell Editing:}}

\begin{itemize}
\tightlist
\item
  {Turns on the ability to directly edit cells in the live display grid.}
\end{itemize}

\textbf{Interactive Header selection} - Currently implemented but may not be turned off.

\begin{itemize}
\tightlist
\item
  {Enables parameters to be directly selected from the live display headers.}
\end{itemize}

\textbf{Submaster Values:}

\begin{itemize}
\tightlist
\item
  Show Stored Values - Dimmer parameters under the control of Group Submasters will continue to show their stored values and the \href{image.png}{} icon will also be shown to indicate when the parameter is inhibited.
\end{itemize}

\textbf{Show Output Values} - When the Submaster is pulled down, the actual dimmer output values will be shown in the live display dimmer column.

\hypertarget{live-parameter-display}{%
\subsection{Live Parameter Display}\label{live-parameter-display}}

The Live Parameter Display shows values in a table format instead of spreadsheet format.

\textbf{There are three Live Parameter Modes:}

\begin{itemize}
\item
  Live Parameters
\item
  Live Banks
\item
  Live Fixtures
\end{itemize}

Live Parameter Mode is mainly used to display single parameter fixtures similar to theatrical consoles but is far more versatile.

\includegraphics{https://files.gitbook.com/v0/b/gitbook-x-prod.appspot.com/o/spaces\%2F3kS90tLsADGm1ocbe7q9\%2Fuploads\%2FG5hlBdfBjmea6tjbutqi\%2Fimage.png?alt=media\&token=2a667d38-c262-4bbf-b802-dedade61d1a2}

\hypertarget{parameter-mode}{%
\subsubsection{Parameter Mode}\label{parameter-mode}}

Unlike most theatrical consoles, parameters other than dimmer may be shown in Live Parameter Mode.

\includegraphics{https://files.gitbook.com/v0/b/gitbook-x-prod.appspot.com/o/spaces\%2F3kS90tLsADGm1ocbe7q9\%2Fuploads\%2FNzhdWIduBdr1JlJxRnpw\%2F4.8.webp?alt=media\&token=68d8f738-7d03-4ae9-ba14-8670879bf203}

Parameters visible in Live Parameter Mode are filtered using the Parameter View Settings Pop-up.

\textbf{Filter Parameters:}

\begin{enumerate}
\def\labelenumi{\arabic{enumi}.}
\item
  Tap the \href{image.png}{} icon - The Parameter View Settings pop-up will open.
\item
  Tap parameters on either side to move them from one box to the other.
\item
  Close the pop-up using {[}ENTER{]} or \href{image.png}{}
\end{enumerate}

\includegraphics{https://files.gitbook.com/v0/b/gitbook-x-prod.appspot.com/o/spaces\%2F3kS90tLsADGm1ocbe7q9\%2Fuploads\%2FtzAZJ7gsczYgIUPMQYit\%2F4.8.1.webp?alt=media\&token=9923932f-be18-402f-92aa-5307f27cfd3a}

\hypertarget{bank-mode}{%
\subsubsection{Bank Mode}\label{bank-mode}}

Fixtures may also be graphically displayed by banks.

\textbf{Currently, three banks can be graphically viewed}: {Image will be implemented soon.}

\begin{itemize}
\item
  \textbf{Intensity} - In the form of an illuminated disk.
\item
  \textbf{Position} - In the form of a cursor that moves in a black rectangle representing the stage.
\item
  \textbf{Color} - In the form of a circular color swatch - Color wheels are not currently supported.
\end{itemize}

\includegraphics{https://files.gitbook.com/v0/b/gitbook-x-prod.appspot.com/o/spaces\%2F3kS90tLsADGm1ocbe7q9\%2Fuploads\%2FiiqnBUvvCzFFqXrjLXkE\%2F4.9.webp?alt=media\&token=c9eab077-dd10-4792-bf41-696023b32b2a}

\hypertarget{fixture-mode}{%
\subsubsection{Fixture Mode}\label{fixture-mode}}

Graphically Displays Intensity, Position, and Color parameter icons as a single ``tombstone'' fixture

\includegraphics{https://files.gitbook.com/v0/b/gitbook-x-prod.appspot.com/o/spaces\%2F3kS90tLsADGm1ocbe7q9\%2Fuploads\%2FbVNqz9O6Ncel6mSwo6Nj\%2F4.10.webp?alt=media\&token=1313790f-2563-4b84-a48d-4da3f71042d3}

\hypertarget{file-management}{%
\section{File Management}\label{file-management}}

This chapter deals with saving and loading Vibe show files as well as the importing of Compulite Vector show files.

\textbf{The following is covered in this chapter:}

\begin{itemize}
\tightlist
\item
  \href{https://vibemanual.compulite.com/file-management.html\#show-menu}{5.1 Show Menu}
\item
  \href{https://vibemanual.compulite.com/file-management.html\#load-vector-show}{5.2 Load Vector Show}
\end{itemize}

\hypertarget{show-menu}{%
\subsection{Show Menu}\label{show-menu}}

The Show menu is the first item of the \{VIBE\} Menu. It contains 5 sub menus:

\includegraphics{https://files.gitbook.com/v0/b/gitbook-x-prod.appspot.com/o/spaces\%2F3kS90tLsADGm1ocbe7q9\%2Fuploads\%2Fd1qUl43R5aNuzM0Vd4zj\%2F5.1.png?alt=media\&token=2a502a88-1fe4-4648-8ed2-0606cc272d37}

\begin{enumerate}
\def\labelenumi{\arabic{enumi}.}
\item
  \textbf{\{New Show\}} - Opens the \textbf{New Show} pop-up with options for \{Cancel\} \{Don't Save\} \{Save\}.
\item
  \textbf{\{Save Show\}} - Directly updates the current show file. A Save Progress pop-up will open and automatically close on completion of the save.
\item
  \textbf{\{Save Show As\ldots\}} - Opens the Save Show/Load Show Browser. New show files may be created or existing files may be overwritten.
\item
  \textbf{\{Load Show\}} - Opens Save Show/Load Show Browser. Shows may be viewed as folders or lists.
\end{enumerate}

\includegraphics{https://files.gitbook.com/v0/b/gitbook-x-prod.appspot.com/o/spaces\%2F3kS90tLsADGm1ocbe7q9\%2Fuploads\%2FRPNEzph0cLWgAv4qbh5L\%2F5.1.1.webp?alt=media\&token=e14277ce-fdef-434a-b09f-b14005ca425b}

Save Show/Load Show Browser

\begin{enumerate}
\def\labelenumi{\arabic{enumi}.}
\setcounter{enumi}{4}
\tightlist
\item
  \textbf{\{Load Vector Show\}} - (\href{https://vibemanual.compulite.com/file-management.html\#load-vector-show}{5.2 Load Vector Show}).
\end{enumerate}

\hypertarget{load-vector-show}{%
\subsection{Load Vector Show}\label{load-vector-show}}

Vector Shows may be loaded with the following limitations:

\textbf{Devices:}

\begin{verbatim}
◾ If a matching device exists in Vibe that device will be used. Modes will be matched. 

◾ If a similar device exists, it will be used. Modes will try to be matched. 

◾ If no match or close match is found, a generic device will be created.
\end{verbatim}

\textbf{Patch:}

\begin{verbatim}
◾ Vector device parameter DMX will be mapped to Vibe parameter DMX.
\end{verbatim}

\textbf{Qlists}:

\begin{verbatim}
◾ Qlists will be matched.
\end{verbatim}

\textbf{Cues and Cue Values:}

\begin{verbatim}
◾ Matching Cues will be created with Vector cue parameter values matched to Vibe cue parameter values.
\end{verbatim}

\textbf{Temp Cues:}

\begin{verbatim}
◾ Temp Cues are converted to Vibe Scenes.
\end{verbatim}

\textbf{Vector Submasters:}

\begin{verbatim}
◾ Submasters are converted to Vibe Scene Submasters.
\end{verbatim}

\textbf{Groups:}

\begin{verbatim}
◾ Groups will be matched.
\end{verbatim}

\textbf{Snaps:}

\begin{verbatim}
◾ Snaps will be mapped with Vector physical playbacks being mapped to Vibe virtual controllers when the number of Vector Playbacks exceeds the amount of Vibe Controllers.
\end{verbatim}

\textbf{Macros:}

\begin{verbatim}
◾ Currently not supported.
\end{verbatim}

\textbf{Effects:}

\begin{verbatim}
◾ Currently not supported but under development.
\end{verbatim}

\textbf{Vector Playbacks} ⟶ \textbf{Vibe Controllers:}

\begin{verbatim}
◾ Paging is matched and all button functions common to Vector and Vibe are matched.
\end{verbatim}

\hypertarget{patch}{%
\section{Patch}\label{patch}}

This chapter deals with creating and patching fixtures.

\textbf{The following is covered in this chapter:}

\begin{itemize}
\tightlist
\item
  \href{https://vibemanual.compulite.com/patch.html\#fixture-sets}{6.1. Fixture Sets}
\item
  \href{https://vibemanual.compulite.com/patch.html\#import-devices}{6.2. Import Devices}
\item
  \href{https://vibemanual.compulite.com/patch.html\#advanced-search}{6.3. Advanced Search}
\item
  \href{https://vibemanual.compulite.com/patch.html\#create-and-patch-imported-fixtures}{6.4. Create and Patch imported fixtures}
\item
  \href{https://vibemanual.compulite.com/patch.html\#configuring-physical-dmx-10}{6.5. Configuring Physical DMX 1/0}
\item
  \href{https://vibemanual.compulite.com/patch.html\#patching-inputs}{6.6. Patching Inputs}
\item
  \href{https://vibemanual.compulite.com/patch.html\#parking-addresses-and-fixtures}{6.7. Parking Addresses and Fixtures}
\end{itemize}

\hypertarget{fixture-sets}{%
\subsection{Fixture Sets}\label{fixture-sets}}

Like other lighting consoles, Vibe contains an extensive \textbf{Device Library} with DMX parameter mapping for fixtures, media servers, and common miscellaneous devices like smoke machines. Before a device can be programmed it must be imported and patched to DMX addresses. Vibe has an advanced numbering system that allows devices to be grouped into five familiar categories called \textbf{SETS}. Custom \textbf{SETS} may also be created. Numbering in all sets may be freely changed if there are no overlapping numbers. Each Set has a unique ID.

\textbf{Set ID numbers:}

\begin{itemize}
\tightlist
\item
  0 = \textbf{{[}Fixture{]}} is a special global numbering set that must also reference one of the other sets. All devices must be assigned a Fixture number and one of the other set numbers.
\item
  1 = \textbf{{[}Channel{]}} is a numbering set that would normally be used for conventional single parameter devices.
\item
  2 = \textbf{{[}SPOT{]}} is a numbering set that would mainly be used for multi-parameter devices like moving lights.
\item
  3 = \textbf{{[}MATRIX{]}} is a numbering set that would mainly be used for RGB or other non-moving LED fixtures. (also useful in pixel mapping devices)
\item
  4 = \textbf{{[}SERVER{]}} is a numbering system used for media servers.
\end{itemize}

Each \textbf{SET} may have Fixtures numbered from 1 ⟶ 999999 To patch a device, you must first move focus to the main monitor and press the \protect\hyperlink{patch}{PATCH} Workspace Template key. By default, the first page will contain all the displays needed for patching.

\hypertarget{edit-sets}{%
\subsubsection{Edit Sets}\label{edit-sets}}

There is an option to edit a user-created set. This is done via the Sets browser view.

\textbf{To Edit a Set}

\begin{enumerate}
\def\labelenumi{\arabic{enumi}.}
\tightlist
\item
  Add a Sets browser view :
\end{enumerate}

\begin{enumerate}
\def\labelenumi{\alph{enumi}.}
\tightlist
\item
  Open \emph{Vibe} menu
\item
  Tap Patch
\item
  Tap Sets
\end{enumerate}

\begin{enumerate}
\def\labelenumi{\arabic{enumi}.}
\setcounter{enumi}{1}
\item
  On the Sets browser view select a user-created set
\item
  Tap Edit
\end{enumerate}

\includegraphics{https://files.gitbook.com/v0/b/gitbook-x-prod.appspot.com/o/spaces\%2F3kS90tLsADGm1ocbe7q9\%2Fuploads\%2FDHXsAk0ZCFREw06w6uOp\%2Fimage.png?alt=media\&token=622ad1f2-ecbc-4e2b-8f04-96d6857896a1}

\hypertarget{import-devices}{%
\subsection{Import Devices}\label{import-devices}}

\textbf{Once the Patch Workspace is open:}

\begin{enumerate}
\def\labelenumi{\arabic{enumi}.}
\item
  Make sure the \{Imported Devices\} tab just above the white FIXTURE selection display area at the bottom left of the workspace is selected.
\item
  Tap \{Import\}, the Manufactures picker list will appear.
\item
  Flick or drag along the list to browse to the desired manufacturer then tap the manufacturer's name.
\item
  A new list and search window will open. If the exact device name is known, type it in the search window, if not flick or drag to browse the list until the desired device is found.

  \includegraphics{https://files.gitbook.com/v0/b/gitbook-x-prod.appspot.com/o/spaces\%2F3kS90tLsADGm1ocbe7q9\%2Fuploads\%2F6OPIN7M2INVvd6Qb68PC\%2F6.2.webp?alt=media\&token=a9d3bfed-3a49-4174-bdb6-3c32843653ee}
\item
  The \textless{} key may be pressed at any time to return to the Imported Devices list.

  \includegraphics{https://files.gitbook.com/v0/b/gitbook-x-prod.appspot.com/o/spaces\%2F3kS90tLsADGm1ocbe7q9\%2Fuploads\%2FMJNhnkyYAYn3wnGiHWkk\%2F6.2.1.webp?alt=media\&token=cb596e2e-e61d-424f-ba50-aa0eee3a0c55}

  \includegraphics{https://files.gitbook.com/v0/b/gitbook-x-prod.appspot.com/o/spaces\%2F3kS90tLsADGm1ocbe7q9\%2Fuploads\%2FrcWDm2caVqizbpYrIRBb\%2F6.2.2.webp?alt=media\&token=62e204dc-17b9-4fd2-8566-18780860a2b8}
\end{enumerate}

\hypertarget{advanced-search}{%
\subsection{Advanced Search}\label{advanced-search}}

Fixtures may be searched directly by name using the Advanced Device Search. Searches may also be filtered by parameter order and DMX Channel Count to limit the search. The filters are very helpful in finding close matches for fixtures that are not currently in the device library.

\textbf{To access Advanced Search:}

\begin{enumerate}
\def\labelenumi{\arabic{enumi}.}
\item
  Tap \{Import\} from the main page - The tab will now read \{Advanced Search\}.
\item
  Tap \{Advanced Search\} - The Advanced Search Pop-up will appear.
\item
  Type the fixture's name in the Fixture Name box or Set the filters to limit the selection.
\item
  Select the fixture from under the Matching Devices header.
\end{enumerate}

\includegraphics{https://files.gitbook.com/v0/b/gitbook-x-prod.appspot.com/o/spaces\%2F3kS90tLsADGm1ocbe7q9\%2Fuploads\%2F4CX90yhir4RUCFfG8kNP\%2F6.3.png?alt=media\&token=38c25f2a-507d-4f3b-99cd-6208f5cc4863}

\begin{enumerate}
\def\labelenumi{\arabic{enumi}.}
\setcounter{enumi}{4}
\tightlist
\item
  The DMX Modes pop-up will open.
\item
  Review the DMX implementation and select the desired mode.
\item
  Tap \href{image.png}{} or press {[}ENTER{]} to close the pop-up and add the fixture to the Imported Devices list.
\end{enumerate}

\includegraphics{https://files.gitbook.com/v0/b/gitbook-x-prod.appspot.com/o/spaces\%2F3kS90tLsADGm1ocbe7q9\%2Fuploads\%2FW9vK61A11wEoGR23P58y\%2F6.3.1.webp?alt=media\&token=ac62e3aa-1045-4cc4-87d6-5715e511417d}

\hypertarget{create-and-patch-imported-fixtures}{%
\subsection{Create and Patch imported fixtures}\label{create-and-patch-imported-fixtures}}

\begin{enumerate}
\def\labelenumi{\arabic{enumi}.}
\item
  Import the desired device. (See \href{https://vibemanual.compulite.com/patch.html\#import-devices}{6.2. Import Devices}).
\item
  The device will now be in the list below the Imported Devices header.
\item
  Tap the device you wish to patch and the Create and Patch Fixtures popup will appear.
\end{enumerate}

\includegraphics{https://files.gitbook.com/v0/b/gitbook-x-prod.appspot.com/o/spaces\%2F3kS90tLsADGm1ocbe7q9\%2Fuploads\%2FcFXeBwqGbF1LU2qRzhsk\%2F6.4.webp?alt=media\&token=cdca54cf-4b59-440f-ad0d-d1b0f09960b0}

\begin{enumerate}
\def\labelenumi{\arabic{enumi}.}
\item
  In the \textbf{Create and Patch} fixtures pop-up, type the quantity in the red highlighted field via the keypad or flick/drag to browse the list until the desired quantity number is found then tap the number.
\item
  If the \textbf{Quantity} number is not highlighted in {Red}, tap the field to make it active.
\item
  Press the ⟶ key on the keypad to tab to the next field, or tap in the \{SET\} field. The appropriate SET for the selected fixture should automatically be selected, but as some fixtures don't always cleanly fit into standard categories, check that the correct Set is selected. If not, flick/drag to browse to the correct SET then tap it. You may also type the Set number (0-5).
\item
  Press the ⟶ key on the keypad to tab to the \{Set Number\} field.
\item
  Set the Channel, Spot, Matrix, or Server number for the fixture in the red highlighted field using the keypad or scroll to browse the list until the desired start number is found, then tap the number.
  👉 By default, the first available number for the selected SET is pre-assigned.
\item
  Press the ⟶ key on the keypad to tab to the next field, or tap any box in the Fixture Number field to select it.
\item
  By default, the same number as the SET Number will appear as the Fixture Number. Many users like to group Fixtures by 10ʼs or 100ʼs so it is common to type 101 for the start of one type of fixture, 201 for a second type and so on. The SET number and Fixture number do not have to be the same.
\item
  Next to the Fixture Number field is a data entry box that displays the fixture's default short name. This may be edited at any time while in this pop-up. The name will be updated when the fixtures are created upon closing the pop-up.
\item
  In the second section of the pop-up, fixtures that were selected for creation can be patched via the DMX patch wizard. Press the ⟶ key on the keypad to tab to the next field, or tap a number in the UNIVERSE field. If the desired universe is not visible, flick/drag to browse to the correct universe number then tap it.
\item
  Press the ⟶ key on the keypad to tab to the next field, or tap any number in the \textbf{ADDRESS} field.
\end{enumerate}

\begin{quote}
👉 By default, the interval between fixtures of the same type is fixed.
Using the \textbf{INTERVAL} field, a custom interval may be set.
\end{quote}

\begin{enumerate}
\def\labelenumi{\arabic{enumi}.}
\setcounter{enumi}{10}
\tightlist
\item
  To complete \textbf{CREATE and PATCH} tap the \textbf{\{APPLY \& CONTINUE\}} key in the bottom \textbf{right corner of the pop-up}, Press \textbf{{[}ENTER{]}}, or tap \href{image.png}{} to close the pop-up and return to the patch workspace.
\end{enumerate}

\textbf{Theatrical Multi-Patching can also be done manually using keypad syntax:}

\begin{verbatim}
◾ Toggle off the **{MAIN PATCH}** key at the top of the DMX patch wizard before closing the pop-up. The fixtures will be created but will remain unpatched.
\end{verbatim}

\textbf{To Patch fixtures using traditional keypad syntax:}

\begin{verbatim}
◾ [FIXTURE] [#] [ADDRESS] [#].[#] (👉 By default, the first available DMX address of the selected Universe will be shown in the **ADDRESS** field. If the default start address is not appropriate, type or scroll the list to the desired start address and tap it.Universe.Address), [STORE]. **Or** 

◾ [ADDRESS] [#].[#] +/ - ⟶ [#].[#] [FIXTURE] [#] [STORE].
\end{verbatim}

\begin{quote}
👉 Addresses may also be typed in absolute format: 1 ⟶ 32,768 (64U niverses standard). The system will automatically assign the correct universe.
\end{quote}

\begin{quote}
👉 When starting from a \{New Show\} only \{Physical DMX outputs\} are enabled. Network universes must be enabled in \textbf{PATCH} \{DMX Settings\}
\end{quote}

\hypertarget{drag-and-drop-patch}{%
\subsubsection{Drag and Drop Patch}\label{drag-and-drop-patch}}

\textbf{Drag and Drop Patch} - If \{Main Patch\} is toggled off in the Create and Patch pop-up, fixtures can be created but not patched. Created fixtures may then be dragged and dropped onto the Universe View.

\textbf{Drag and Drop from the Fixtures tab:}

\begin{enumerate}
\def\labelenumi{\arabic{enumi}.}
\item
  Create fixtures - (See: \href{https://vibemanual.compulite.com/patch.html\#create-and-patch-imported-fixtures}{6.4. Create and patch imported fixtures}).
\item
  Select the Fixture \textbf{Set} you wish to patch from in the lower right of the main patch workspace.
\item
  Select the \textbf{Fixtures} tab on the left side of the main patch workspace directly above the white FIXTURE selection box. A list of created fixtures will appear.
\item
  Make sure the \textbf{Universes View} tab at the bottom of the main patch workspace is selected.
\item
  Select a destination Universe from the \textbf{Universes} list in the top right of the main patch workspace. - Scroll the list to browse for hidden universes.
\item
  \textbf{Press and Hold} a fixture, it will turn into a floating icon of the fixture.
\item
  Drag the fixture icon from the Fixtures list onto the Universe View grid. All associated DMX addresses for the fixture will turn green. Drop the fixture icon on the desired DMX start address for the fixture. - The fixture is now patched.
\end{enumerate}

\begin{quote}
👉 Multiple fixtures may be selected by swiping ⤴ or ⤵ in the Fixtures list.
\end{quote}

\includegraphics{https://files.gitbook.com/v0/b/gitbook-x-prod.appspot.com/o/spaces\%2F3kS90tLsADGm1ocbe7q9\%2Fuploads\%2FFTYnnKHRjr3DV3bFTnMN\%2F6.4.1.webp?alt=media\&token=7a7991b5-3817-4e6d-8fd0-cbd9c031c839}

\textbf{Make multiple selections of fixtures using the keypad:}

\begin{enumerate}
\def\labelenumi{\arabic{enumi}.}
\item
  Type a range of fixtures using the keypad - the fixture selection will be displayed numerically in the white fixture selection box.
\item
  \textbf{Press and Hold} anywhere in the white fixture selection box - a fixture icon will appear.
\item
  Drag the icon to the Universe View grid and drop it at the destination start address - the range of fixtures is now patched.
\end{enumerate}

\textbf{Multi-Patch using Drag and Drop:}

\begin{enumerate}
\def\labelenumi{\arabic{enumi}.}
\tightlist
\item
  Drag and drop the same fixture to multiple DMX offsets.
\end{enumerate}

\begin{quote}
👉 All types of fixtures may be multi-patched but caution should be used when multi-patching multi-parameter fixtures.
\end{quote}

\textbf{Unpatch fixtures using drag and drop:}

\begin{enumerate}
\def\labelenumi{\arabic{enumi}.}
\item
  \textbf{Press and Hold} any patched DMX offset - The DMX addresses for the fixtures patched to it will turn green and a fixture icon will be shown.
\item
  Drag the selected block of fixtures towards the bottom of the screen - a waste basket icon will appear at the bottom of the workspace.
\item
  Drag the block until the fixture icon is over the waste basket and release the fixture - the fixture will now be unpatched.
\end{enumerate}

\hypertarget{modify-patch}{%
\subsubsection{Modify Patch}\label{modify-patch}}

Once patched fixtures can be modified using keypad syntax or using the two Patch Toolbars in the Patch Workspace.

\textbf{Toolbar with no fixture selection:}

\includegraphics{https://files.gitbook.com/v0/b/gitbook-x-prod.appspot.com/o/spaces\%2F3kS90tLsADGm1ocbe7q9\%2Fuploads\%2FYBzeJ8a39N937xjx5S3n\%2F6.4.2.png?alt=media\&token=d1b58421-5fff-40dd-8c9a-f14f6944e6c8}

\begin{verbatim}
◾ {CLEAR PATCH} - Opens up Clear Patch pop-up. 

◾ {RENUMBER} - Opens up RENUMBER: FIXTURE pop-up. 

◾ {ADD FIXTURE} - Opens the Create And Patch pop-up. 

◾ {DELETE FIXTURE} - Opens up Delete Fixtures pop-up.
\end{verbatim}

\textbf{Toolbar with fixture selection:}

\includegraphics{https://files.gitbook.com/v0/b/gitbook-x-prod.appspot.com/o/spaces\%2F3kS90tLsADGm1ocbe7q9\%2Fuploads\%2FFrU1z04uJEKhtVcZTvFA\%2F6.4.2.1.webp?alt=media\&token=4f322af9-9116-4d1e-b01b-cc15538d482a}

\begin{verbatim}
◾ [FIXTURE] [#] {PATCH} - Opens up Patch Fixture pop-up. 

◾ [FIXTURE] [#] {RENUMBER} - Opens up RENUMBER: FIXTURE pop-up with fixture already selected. 

◾ [FIXTURE] [#] {FIXTURE CALIBRATION} - Opens up Fixture Calibration pop-up with the fixture selected. 
  
\end{verbatim}

See:
\href{https://vibemanual.compulite.com/xyz-fixture-calibration.html}{20. XYZ Fixture Calibration}

\begin{verbatim}
◾ [FIXTURE] [#] {CLEAR PATCH} - Opens the Clear Patch confirmation popup 

◾ [FIXTURE] [#] {DELETE FIXTURE} - Opens the Delete Fixture confirmation pop-up
\end{verbatim}

\textbf{Modify the patch using keypad syntax:}

\begin{verbatim}
◾ [FIXTURE] [#] {CLEAR PATCH} - From the Editor toolbar. 

◾ [FIXTURE] [#] [DELETE] - Only works in the Patch Workspace.
\end{verbatim}

\hypertarget{configuring-physical-dmx-10}{%
\subsection{Configuring Physical DMX 1/0}\label{configuring-physical-dmx-10}}

By default, DMX is mapped to the physical ports on a 1:1 basis.
DMX is always transmitted but can be remapped to anyone of the supported 256 DMX universes.

\includegraphics{https://files.gitbook.com/v0/b/gitbook-x-prod.appspot.com/o/spaces\%2F3kS90tLsADGm1ocbe7q9\%2Fuploads\%2F373A2JvOjPY60S5OzfN5\%2F6.5.png?alt=media\&token=2c5c721d-669e-4b91-b171-6abbcc354f10}

\textbf{Remap Physical outputs:}

\begin{enumerate}
\def\labelenumi{\arabic{enumi}.}
\item
  With no fixtures selected, tap \{DMX SETTINGS\} on the patch toolbar - The DMX Properties pop-up will open.
\item
  In the Protocol column, Tap \{Physical output\}.
\item
  Under the Source Universe heading, tap the desired source universe number or once red type the universe number you wish to be the source.
\item
  Under the Physical output Destination Universe heading, tap one of the 8 displayed physical outputs - the source Universe will now be patched to the output.
\end{enumerate}

\textbf{Change output to input} - Currently, Vibe supports 2 inputs but multiple connectors may be assigned to the same input.

\begin{enumerate}
\def\labelenumi{\arabic{enumi}.}
\item
  Tap a patched output - Two additional options will appear below the Output Destination Universe box.

  ◾ \{Make Input 1\} - Makes the selected outputs into Input 1. The Physical connector display will turn white to indicate it is now an input.

  ◾ \{Make Input 2\} - Makes the selected outputs into Input 2. The Physical connector display will turn white to indicate it is now an input.
\end{enumerate}

\textbf{Change input back to output:}

\begin{enumerate}
\def\labelenumi{\arabic{enumi}.}
\item
  Tap a patched input - an additional Make Output option will appear below the Output Destination Universe box.
\item
  Tap \{Make Output\} - The Connector will now output DMX.
\end{enumerate}

\textbf{Change DMX refresh rate:}

◾ Vibe will aways transmit DMX at the fastest refresh rate it can (38 - 44Hz). Sometimes older equipment cannot handle the high refresh rate, so it may be reduced using the DMX Max Rate wheel. It is not recommended to go below 28Hz as some jitter may be observed.

\hypertarget{patching-inputs}{%
\subsection{Patching Inputs}\label{patching-inputs}}

\textbf{The following objects may be mapped to DMX inputs}:

\begin{itemize}
\item
  DMX input mapped directly to the console's DMX outputs
\item
  DMX input mapped to Macros
\item
  DMX input mapped to Controllers
\item
  DMX input mapped to fixtures
\end{itemize}

\textbf{Map a DMX output to a DMX input:}

\begin{enumerate}
\def\labelenumi{\arabic{enumi}.}
\item
  Press {[}ADDRESS{]} {[}\#{]}
\item
  Tap \{INPUT PATCH TO\} on the toolbar - Command line will say INPUT PATCH TO
\item
  Press {[}ADDRESS{]} {[}\#{]}
\item
  Press {[}STORE{]}
\end{enumerate}

\begin{quote}
\textbf{E.g}. {[}ADDRERSS{]} {[}1{]} \{INPUT PATCH TO\} {[}ADDRESS{]} {[}1{]} {[}STORE{]} - When DMX 1 is received from an external source like a console, DMX 1 will directly output to the Vibes DMX output. No values will be shown in the live display as no actual parameters are involved.
\end{quote}

\textbf{Map a range of DMX outputs to a range of DMX inputs:}

\begin{enumerate}
\def\labelenumi{\arabic{enumi}.}
\item
  Press {[}ADDRESS{]} {[}\#{]} ⟶ {[}\#{]}
\item
  Tap \{INPUT PATCH TO\} on the toolbar - Command line will say INPUT PATCH TO.
\item
  Press {[}ADDRESS{]} {[}\#{]} ⟶ {[}\#{]}
\item
  Press {[}STORE{]}
\end{enumerate}

\textbf{Map a Macro to a DMX input} - Macro will trigger at 50\%

\begin{enumerate}
\def\labelenumi{\arabic{enumi}.}
\item
  Press {[}ADDRESS{]} {[}\#{]}.{[}\#{]}
\item
  Tap \{INPUT PATCH TO\} on the toolbar - Command line will say INPUT PATCH TO.
\item
  Press {[}MACRO{]} {[}\#{]}
\item
  Press {[}STORE{]}
\end{enumerate}

\textbf{Map a controller to a DMX input:}

\begin{enumerate}
\def\labelenumi{\arabic{enumi}.}
\item
  Press {[}ADDRESS{]} {[}\#{]}
\item
  Tap \{INPUT PATCH TO\} on the toolbar - Command line will say INPUT PATCH TO.
\item
  Press {[}PLAY-B{]} (Controller) - The Controller Selection pop-up will appear.
\item
  Choose the Controller Type, Page Number, and Controller number from the pop-up.
\item
  Press ENTER, or tap \href{image.png}{} to close the pop-up and complete the patch.
\end{enumerate}

\includegraphics{https://files.gitbook.com/v0/b/gitbook-x-prod.appspot.com/o/spaces\%2F3kS90tLsADGm1ocbe7q9\%2Fuploads\%2F3DT3je7pCP3pknk8LC7V\%2F6.6.webp?alt=media\&token=3bb762be-4947-4802-a1c0-e89d98163565}

\textbf{Map a Fixture to a DMX input:}

\begin{enumerate}
\def\labelenumi{\arabic{enumi}.}
\item
  Press {[}ADDRESS{]} {[}\#{]}
\item
  Tap \{INPUT PATCH TO\} on the toolbar - command line will say INPUT PATCH TO
\item
  Press {[}Fixture{]} {[}\#{]}
\item
  Press {[}STORE{]}
\end{enumerate}

\begin{quote}
DMX inputs will automatically be mapped to all the fixtures parameters starting at the specified address. {[}ADDRESS{]} {[}1{]} {[}FIXTURE{]} {[}1{]} (A fixture with 42 parameters) {[}STORE{]} will automatically match the 42 parameters to input addresses 1 → 42. The next fixture would have to be mapped starting at address 43.
\end{quote}

\textbf{Delete a DMX Input patch assignment:}

\begin{enumerate}
\def\labelenumi{\arabic{enumi}.}
\item
  Press {[}ADDRESS{]} {[}\#{]}
\item
  Tap \{INPUT PATCH TO\} on the toolbar - Command line will say INPUT PATCH TO
\item
  Press {[}DELETE{]}
\end{enumerate}

\hypertarget{parking-addresses-and-fixtures}{%
\subsection{Parking Addresses and Fixtures}\label{parking-addresses-and-fixtures}}

\textbf{Park an Address:}

\begin{enumerate}
\def\labelenumi{\arabic{enumi}.}
\item
  {[}ADDRESS{]} {[}\#{]} \{PARK\} - The Park pop-up will appear.
\item
  Set a DMX level from 0 ⟶ 255 using tee virtual slider.
\item
  Tap \href{image.png}{} or press {[}ENTER{]} to close the pop-up and park the address.
\end{enumerate}

\textbf{Unpark and address:}

\begin{enumerate}
\def\labelenumi{\arabic{enumi}.}
\item
  {[}ADDRESS{]} {[}\#{]} \{PARK\} - The Park pop-up will appear.
\item
  Tap \{Clear Park\}.
\item
  Tap \href{image.png}{} or press {[}ENTER{]} to close the pop-up and unpark the address.
\end{enumerate}

\textbf{Park a fixture:}

\begin{enumerate}
\def\labelenumi{\arabic{enumi}.}
\item
  {[}FIXTURE{]} {[}\#{]} - Set parameter values in the editor.
\item
  Tap \{PARK\} on the editor toolbar.
\item
  Press {[}STORE{]} - Fixture will now be parked at the editor values.
\end{enumerate}

\textbf{Unpark a fixture:}

\begin{enumerate}
\def\labelenumi{\arabic{enumi}.}
\tightlist
\item
  {[}FIXTURE{]} {[}\#{]} \{PARK\} {[}DELETE{]}
\end{enumerate}

\textbf{View parked fixtures and address:}

(See: \href{https://vibemanual.compulite.com/exam-show-data.html\#patch-exam}{18.2. Patch Exam})

\begin{quote}
{\textgreater{} The Grandmaster does not affect parked addresses and fixtures. The {[}B.O.{]} key \textbf{will} blackout parked addresses and fixtures.}
\end{quote}

\hypertarget{system-default-settings}{%
\section{System Default Settings}\label{system-default-settings}}

This chapter deals with customizing system settings to suit programming styles.

\textbf{The following is covered in this chapter:}

\begin{itemize}
\tightlist
\item
  \href{https://vibemanual.compulite.com/system-default-settings.html\#editing}{7.1. Editing}
\item
  \href{https://vibemanual.compulite.com/system-default-settings.html\#timing}{7.2. Timing}
\item
  \href{https://vibemanual.compulite.com/system-default-settings.html\#network}{7.3. Network}
\item
  \href{https://vibemanual.compulite.com/system-default-settings.html\#hardware}{7.4. Hardware}
\item
  \href{https://vibemanual.compulite.com/system-default-settings.html\#defaults}{7.5. Defaults}
\item
  \href{https://vibemanual.compulite.com/system-default-settings.html\#system}{7.6. System}
\end{itemize}

\hypertarget{editing}{%
\subsection{Editing}\label{editing}}

System Settings defaults are accessed by selecting \{System Settings\} from the \{Vibe\} menu or by tapping the \{System Status\} area at the far right side of the controller displays.

\includegraphics{https://files.gitbook.com/v0/b/gitbook-x-prod.appspot.com/o/spaces\%2F3kS90tLsADGm1ocbe7q9\%2Fuploads\%2FGArAdAtN90yRhuBySf1s\%2F7.1.png?alt=media\&token=1a580760-eefe-4d7b-b7a5-987191750123}

\textbf{Editing Properties} - Editing Section

\includegraphics{https://files.gitbook.com/v0/b/gitbook-x-prod.appspot.com/o/spaces\%2F3kS90tLsADGm1ocbe7q9\%2Fuploads\%2FNhQX0oar9rEo6NUdNBul\%2F7.1.1.webp?alt=media\&token=882d021b-ab31-4b7f-b899-693abe059daf}

\textbf{Default Stage Value:}

\begin{itemize}
\item
  Home Value - When released, fixtures parameters will return to their home values as specified in the device's definition or user \textbf{Home Scene}.
\item
  Maintain Last Value - When released, fixture parameters will remain at their current position but dimmer will go out.
\end{itemize}

\textbf{Auto Parameter Grouping:}

\begin{itemize}
\tightlist
\item
  When on, parameters are grouped as follows:
\end{itemize}

\includegraphics{https://files.gitbook.com/v0/b/gitbook-x-prod.appspot.com/o/spaces\%2F3kS90tLsADGm1ocbe7q9\%2Fuploads\%2FLO8HajrUseN0GShf6J83\%2Fimage.png?alt=media\&token=0c5c416b-2190-4490-b0ae-42c0075a6c40}

\textbf{Clear Editor After Store:}

\begin{itemize}
\tightlist
\item
  Default editor behaviour when in Compulite Mode - After storing a cue, values remain in the editor until {[}RESET{]} or {[}VIBE{]}+{[}RESET{]} is pressed.
\end{itemize}

\textbf{Editor Flash:}

\begin{itemize}
\item
  Sets default values for toolbar \{FLASH\} feature.
\item
  \{Base\} = Start point (low value) \{Time\} = Flash duration \{Size\} = Amplitude (high values)
\item
  Defaults - \{Base\} 0.2 (20\%) \{Time\} 2 seconds \{Size\} 0.6 (80\%)
\end{itemize}

\textbf{Apply Compulite Mode:} - Sets a number of defaults to emulate traditional nontracking Compulite Mode.

\begin{itemize}
\item
  Default Stage Values = \{Maintain Last Value\}
\item
  \{Clear Editor After Store\} = Off
\item
  {\{Advance To Cue After Store\} = Off}
\item
  Qlist Behavior = \{Non-Tracking\}
\end{itemize}

\begin{center}\rule{0.5\linewidth}{0.5pt}\end{center}

\textbf{Editing Properties:} - Cueing Section

Tracking - Non-Tracking:

\begin{itemize}
\tightlist
\item
  Tracking On:
\end{itemize}

◾ Values track forward. The stage look is made up of a combination of hard and tracked values.

\begin{itemize}
\tightlist
\item
  Non-Tracking:
\end{itemize}

◾ Only hard values are recorded and zero dimmer values is treated as a hard zero.

\begin{quote}
{In Non-Tracking mode, parameter values that are not given values will still output to the stage where they were last left. This can accidentally lead to mistakes where when played back, the cue does not look the same as it did when it was recorded. It is suggested that the \textbf{Store Options} \{All Parameters if Active\} or \{All Parameters if Selected\} are used whenever possible to avoid errors.}
\end{quote}

\textbf{Clear Editor on Master GO:}

\begin{itemize}
\tightlist
\item
  The editor will be reset when \textbf{GO} on the \textbf{Master Controller} is pressed.
\end{itemize}

\textbf{Store Scene by Default:}

\begin{itemize}
\item
  By default {[}STORE{]} {[}HERE{]} (press any controller button) will create and assign the first available Scene to the controller instead of a Qlist and Cue.
\item
  Qlists may then be stored by using the syntax {[}STORE{]} {[}CUE{]} {[}HERE{]} (press any controller button).
\end{itemize}

\textbf{Next Cue Increment:}

\begin{itemize}
\tightlist
\item
  When {[}Store{]} {[}Here{]}, {[}STORE{]} {[}STORE{]}, or {[}STORE{]} {[}ENTER{]} are used to append a new cue to the end of a Qlist, the specified spacing will be used.
\end{itemize}

\begin{quote}
E.g. If the increment is 1 and the last cue is 10, the new cue would be 11. If the increment is 10, then the new cue would be 20.
\end{quote}

\textbf{Editing Properties:} - ON Value

\begin{itemize}
\item
  Virtual wheel sets the default editor value for the dimmer parameter when the {[}ON{]} key is pressed.
\item
  {\{First ON Recalls Last\} - Pressing {[}ON{]} the first time will copy the last dimmer value set for a fixture into the next selected fixture. Pressing again will set the dimmer value to the ON default.}
\end{itemize}

\textbf{Restore and Save Defaults:}

\includegraphics{https://files.gitbook.com/v0/b/gitbook-x-prod.appspot.com/o/spaces\%2F3kS90tLsADGm1ocbe7q9\%2Fuploads\%2FxFu9gzPgWtT8ffCJCekJ\%2F7.1.2.webp?alt=media\&token=72e6c0c4-49ed-42a1-9d51-e9e97d4823e3}

\begin{itemize}
\item
  \{Restore Popup Positions\} - {Currently redundant.}
\item
  \{Restore Defaults\} - Restores settings changes to the last time \{Apply to Defaults\} was set.
\item
  \{Restore Factory\} - Restores console to ``read only'' defaults.
\item
  \{Apply to Defaults\} -Updates system setting changes to \textbf{show specific} defaults without exiting the pop-up.
\end{itemize}

\hypertarget{timing}{%
\subsection{Timing}\label{timing}}

\textbf{Timing Properties} - Cue Section

\includegraphics{https://files.gitbook.com/v0/b/gitbook-x-prod.appspot.com/o/spaces\%2F3kS90tLsADGm1ocbe7q9\%2Fuploads\%2FWzxK5PnA0giOYH3ndZkh\%2F7.2.webp?alt=media\&token=7617faa8-e505-4a96-9bd8-7bac6b4af477}

\textbf{Default Back Time:}

\begin{itemize}
\tightlist
\item
  Timing is used when the {[}Back{]} or {[}Hold Back{]} Buttons are pressed.
\end{itemize}

\textbf{Default Assert:}

\begin{itemize}
\item
  Timing is used when a controller's parameters are reasserted with the {[}ASSERT{]} {[}HERE{]} command.
\item
  Use Cue Time - As the Assert command is basically a GOTO command to the same cue that is active, the cue time for the cue being reasserted may be used instead of the Default Assert time.
\end{itemize}

\textbf{Default GOTO Time:}

\begin{itemize}
\tightlist
\item
  Default timing when the GOTO command is used. GOTO time can be overridden using command {[}GOTO{]} {[}CUE{]} {[}\#{]} {[}TIME{]} {[}Value{]}Press \href{image.png}{} or {[}ENTER{]}.
\end{itemize}

\textbf{Default Release Time:}

\begin{itemize}
\tightlist
\item
  Default Fade out time for parameters being released from the editor using the {[}RESET{]}, parameter {[}RELEASE{]} keys, or Controllers being released from the stage (turned off) using the controller {[}RELEASE{]} button.
\end{itemize}

\textbf{Look Ahead} Section:

\begin{itemize}
\item
  Set all controllers on will enable the Look Ahead function for all controllers.
\item
  Set all controllers off will disable the Look Ahead function for all controllers.
\end{itemize}

\begin{quote}
Once all is set on, individual controllers may disable Look Ahead at any time. Once all off, individual controllers may be enabled Look Ahead at any time.
\end{quote}

\textbf{Timing Properties - Snap} Section:

\begin{itemize}
\tightlist
\item
  Default fade time for all controllers being initiated by the Snap.
\end{itemize}

\textbf{Timing Properties - System Time} Section:

\begin{itemize}
\tightlist
\item
  System Time (Editor Time) is applied to values being introduced into the editor and other miscellaneous features that do not otherwise have their own timing settings.
\end{itemize}

\hypertarget{network}{%
\subsection{Network}\label{network}}

\textbf{Network Properties} - Local

\includegraphics{https://files.gitbook.com/v0/b/gitbook-x-prod.appspot.com/o/spaces\%2F3kS90tLsADGm1ocbe7q9\%2Fuploads\%2FmRl35Hvba9wSFzljF7Ap\%2F7.3.png?alt=media\&token=458921a0-d441-4211-a5dc-b0c6819d1772}

\textbf{Local:}

\begin{itemize}
\tightlist
\item
  Option to set the local information of the console such as IP and name.
\end{itemize}

\textbf{Console State:}

\begin{itemize}
\tightlist
\item
  There are 2 possible individual states:

  \begin{itemize}
  \tightlist
  \item
    Stand Alone (Default) -- Console is not in a session but networking for outputs, remotes and external connections is active.
  \item
    Local (Offline) -- Same as Stand Alone but disables the DMX over ethernet.
  \end{itemize}
\end{itemize}

\textbf{User ID:}

\begin{itemize}
\tightlist
\item
  {Not yet implemented}.
\end{itemize}

\textbf{Start Session:}

\begin{itemize}
\tightlist
\item
  The console becomes a Master console, meaning it's hosting a session. Later on this in the Master-Slave and Multiuser chapters.
\end{itemize}

\textbf{Online IP Devises:}

\begin{itemize}
\tightlist
\item
  Displays all Valid IP based devices active on the network.
\end{itemize}

\textbf{Delete Selected / Clear Device IP History}

\begin{itemize}
\tightlist
\item
  Can either select a device to remove from the list or clear the whole list.
\end{itemize}

\textbf{To clear Device IP History}

\begin{enumerate}
\def\labelenumi{\arabic{enumi}.}
\item
  Open System Settings.
\item
  Go to Network tab.
\item
  Tap \emph{Clear IP Devices.}
\end{enumerate}

\textbf{Connect As \ldots{} / Disconnect:}

\begin{itemize}
\tightlist
\item
  Related to starting or ending a session with a master console. Later on this in the Master-Slave and Multiuser chapters.
\end{itemize}

\hypertarget{hardware}{%
\subsection{Hardware}\label{hardware}}

\textbf{Hardware Properties} - Panel Backlight

\includegraphics{https://files.gitbook.com/v0/b/gitbook-x-prod.appspot.com/o/spaces\%2F3kS90tLsADGm1ocbe7q9\%2Fuploads\%2FNTAaCC0zSMDawzSL3jLm\%2F7.4.webp?alt=media\&token=ed771d92-0cc0-48b4-9669-6d83cbe49db8}

\textbf{Inactive keys:}

\begin{itemize}
\item
  Adjusts the LED backlight level of keys that are not valid in the current operational state.
\item
  Adjusts the LED backlight level of controllers that are unassigned.
\end{itemize}

\textbf{Active Keys:}

\begin{itemize}
\item
  Adjusts the LED backlight level of keys that may be used in the current operational state.
\item
  Adjusts the LED backlight level of controllers that are assigned objects.
\end{itemize}

\textbf{Operation System:}

\begin{itemize}
\tightlist
\item
  Opens relevant Windows setting screen.
\end{itemize}

\hypertarget{defaults}{%
\subsection{Defaults}\label{defaults}}

\textbf{Defaults} - Initial settings for most system objects.

\includegraphics{https://files.gitbook.com/v0/b/gitbook-x-prod.appspot.com/o/spaces\%2F3kS90tLsADGm1ocbe7q9\%2Fuploads\%2FU3hN3WzAt97uaM12rHlu\%2F7.5.webp?alt=media\&token=6c4fc2b1-e16e-4cac-aa83-524c93f2a99b}

\textbf{See:}

\begin{itemize}
\tightlist
\item
  \href{https://vibemanual.compulite.com/system-default-settings.html\#controller-settings}{7.5.1. Controller Settings}
\item
  \href{https://vibemanual.compulite.com/system-default-settings.html\#controller-actions}{7.5.2. Controller Actions}
\item
  \href{https://vibemanual.compulite.com/system-default-settings.html\#qlist-properties}{7.5.3. Qlist Properties}
\item
  \href{https://vibemanual.compulite.com/system-default-settings.html\#chaser-properties}{7.5.4. Chaser Properties}
\item
  \href{https://vibemanual.compulite.com/system-default-settings.html\#cue-settings}{7.5.5. Cue Settings}
\item
  \href{https://vibemanual.compulite.com/system-default-settings.html\#cue-store-options}{7.5.6. Cue Store Options}
\item
  \href{https://vibemanual.compulite.com/system-default-settings.html\#scene-settings}{7.5.7. Scene Settings}
\item
  \href{https://vibemanual.compulite.com/system-default-settings.html\#scene-store-options}{7.5.8. Scene Store Options}
\item
  \href{https://vibemanual.compulite.com/system-default-settings.html\#bank-library-options}{7.5.9. Bank Library Options}
\item
  \href{https://vibemanual.compulite.com/system-default-settings.html\#snap-settings}{7.5.10. Snap Settings}
\item
  \href{https://vibemanual.compulite.com/system-default-settings.html\#effects}{7.5.11. Effects}
\end{itemize}

\hypertarget{controller-settings}{%
\subsubsection{Controller Settings}\label{controller-settings}}

\textbf{Defaults} - Controller Settings

\includegraphics{https://files.gitbook.com/v0/b/gitbook-x-prod.appspot.com/o/spaces\%2F3kS90tLsADGm1ocbe7q9\%2Fuploads\%2FIRko0Ph6QW39QUTMFEaI\%2F7.5.1.webp?alt=media\&token=c11c81e4-d394-4237-bd78-b63dacce9566}

\textbf{Flash behaviour:}

\begin{itemize}
\item
  \textbf{Release On Release} - While a Flash button is depressed, the controller will assert its values, when released the controller will be released from affecting the stage.
\item
  \textbf{Pure LTP} - Pressing a Flash key reasserts overridden LTP values from another controller but does not release them when the flash key is released.
\item
  \textbf{Go On Flash} - If a slider is down it will flash the intensity to full \textbf{and} advance the Qlist in cue time with every press of the Flash key.
\item
  \textbf{Latch} - Flash key toggles ON/OFF.
\end{itemize}

\textbf{Slider Behaviour:}

\begin{itemize}
\item
  \textbf{Intensity Only} - Slider only controls the intensity.
\item
  \textbf{Go + Fade Parameters} - Moving the slider off zero will initiate a GO command. All parameters \textbf{except} intensity will fade to their recorded values using cue time. Intensity will remain under the control of the slider.
\item
  \textbf{Go + Jump Parameters} - Parameters Jump to their values as soon as the slider moves off zero. Intensity will remain under the control of the slider.
\item
  \textbf{Release At Bottom} - The controller will be released from affecting the stage when the slider reaches zero. Default System Release Time or Qlist Release Time will be used.
\end{itemize}

\textbf{Defaults} - Controller Settings: Properties

\begin{itemize}
\item
  \textbf{Exclude from Snap} - The controller will not be recorded in Snapshots.
\item
  \textbf{Exclude from (Snap) Override} - Snap commands will be ignored even if the controller was initially recorded in a snap.
\item
  \textbf{Exclude from Free All} - The controller will not be unloaded with a global {[}VIBE{]}+{[}FREE{]} command. It may only be unloaded with {[}FREE{]} {[}HERE{]}.
\item
  \textbf{Exclude from Release All} - The controller will not be released with a global {[}VIBE{]}+{[}RELEASE{]} command. It may only be unloaded with {[}RELEASE{]} {[}HERE{]}.
\item
  \textbf{Prevent from Paging} - Locks a Qlist or Scene to the current page. Paging ⤴ or ⤵ will not affect the Qlist or Scene on the controller.
\item
  \textbf{Prevent from Grab} - The \textbf{Grab} command will not capture the controller's active parameters.
\end{itemize}

\textbf{Defaults} - Controller Settings: Priority

\begin{itemize}
\tightlist
\item
  \textbf{Virtual Priority Wheel} - Sets the controller's LTP group number. Controllers assigned higher numbers cannot be overridden by lower numbers. Controllers assigned the same number will be LTP among themselves. Values may be entered using the virtual wheel or directly by taping the middle of the centre area of the wheel and typing a value from the keypad.
\end{itemize}

\textbf{Defaults} - Controller Settings: Restore defaults

\begin{itemize}
\tightlist
\item
  \textbf{Restores settings} - Resets Controller settings to their last stored configuration.
\end{itemize}

\hypertarget{controller-actions}{%
\subsubsection{Controller Actions}\label{controller-actions}}

\textbf{Defaults} - Controller Default Actions

\includegraphics{https://files.gitbook.com/v0/b/gitbook-x-prod.appspot.com/o/spaces\%2F3kS90tLsADGm1ocbe7q9\%2Fuploads\%2F2QY58B93PfbYXzS3G7uS\%2F7.5.2.png?alt=media\&token=711da5cc-7dbb-4617-94a0-db040ba000fa}

\textbf{Qlists and Scenes:}

\begin{itemize}
\tightlist
\item
  The Default button assignments when a Qlist or Scene is assigned to a controller ( \textbf{See:} \href{https://vibemanual.compulite.com/playback-controllers.html\#configuring-controllers}{14.5. Configuring Controllers})
\end{itemize}

\textbf{Bottom Controller Button for Submaster}

The default behavior for submaster's assignment on the bottom button is \emph{Flash}

\includegraphics{https://files.gitbook.com/v0/b/gitbook-x-prod.appspot.com/o/spaces\%2F3kS90tLsADGm1ocbe7q9\%2Fuploads\%2F46a9txxILwGLKtp8K9sN\%2Fimage.png?alt=media\&token=22b87ea9-af78-4476-9ee4-8eb83de0f6b3}

\textbf{Submasters (Group Submasters)} - When a Scene is configured as a Submaster it has a unique set of functions and button options\textbf{.}

\includegraphics{https://files.gitbook.com/v0/b/gitbook-x-prod.appspot.com/o/spaces\%2F3kS90tLsADGm1ocbe7q9\%2Fuploads\%2Fw0e6FlJP0dNlpzuujuot\%2F7.5.2.1.webp?alt=media\&token=43cdc4db-7246-48fd-82cb-3fe8db74aa4c}

\hypertarget{qlist-properties}{%
\subsubsection{Qlist Properties}\label{qlist-properties}}

7.5.3. Qlist Properties

\textbf{Defaults} - Qlist Properties

\includegraphics{https://files.gitbook.com/v0/b/gitbook-x-prod.appspot.com/o/spaces\%2F3kS90tLsADGm1ocbe7q9\%2Fuploads\%2FHDjMWGWWZT6WTImJ7Lso\%2F7.5.3.webp?alt=media\&token=0a31ada9-f9ae-4abb-ac7b-9a8931c25a6c}

\textbf{Qlist Mode:}

\begin{itemize}
\item
  \textbf{Normal} - Cues do not auto loop and all normal time functions are calculated.
\item
  \textbf{Chaser} - Cues auto-loop from one to another. Only In-Time, and Delay-In Time are calculated. Parameter time is still valid.
\end{itemize}

\textbf{Qlist Direction:}

\begin{itemize}
\item
  \textbf{Backward} - Chasers and Qlists transition highest to lowest.
\item
  \textbf{Forward} - Chases and Qlists transition lowest to highest.
\end{itemize}

\textbf{Tracking (Default):}

\begin{itemize}
\item
  ON - Normal tracking logic will be used.
\item
  OFF {(NOT QONLY)} - Tracking is not calculated and dimmer at zero is treated as a hard zero (Compulite Mode).
\end{itemize}

\textbf{Defaults} - Qlist Properties: Preferences

\begin{itemize}
\item
  \textbf{Back to top on release} - After a controller is released, it will reset to the first cue in the Qlist.
\item
  \textbf{Sequence} - If toggled on, pressing GO will advance the Qlist.
\item
  \textbf{Mark cue at top of the list.}
\item
  \textbf{Loop back to first} - If toggled on, a Qlist will loop from the end of the list to the beginning of the list when GO is pressed.
\item
  \textbf{Trigger} - Toggles on/off the execution of Macros and Snaps that are attached to cues.
\item
  \textbf{Look Ahead} - Enable/Disable the Look Ahead function.
\end{itemize}

\begin{center}\rule{0.5\linewidth}{0.5pt}\end{center}

\textbf{Defaults} - Qlist Properties: Loop GO Behavior

\includegraphics{https://files.gitbook.com/v0/b/gitbook-x-prod.appspot.com/o/spaces\%2FzTP4ECnWc9wORFtULApe\%2Fuploads\%2FjaXMWmwEteT1HZey8pXY\%2F1.png?alt=media\&token=fc69dc17-3117-4da5-8650-dbc640747b56}

\begin{center}\rule{0.5\linewidth}{0.5pt}\end{center}

\textbf{Defaults} - Release time

\begin{itemize}
\item
  \textbf{Virtual Wheel:}

  \begin{itemize}
  \item
    Wheel at Default - System Default Release time is referenced.
  \item
    Using virtual wheel - Rotate the wheel to set a Qlist specific time that a controller references to fade to off.
  \item
    Using keypad - Tap the inside of the virtual wheel until it turns red. Set a release value from the keypad.
  \end{itemize}
\item
  \textbf{Button under wheel:} Toggles between Milliseconds, Seconds, Minutes, Hours.
\end{itemize}

\textbf{Defaults} - Qlist Properties:

\textbf{Restore default} - Resets Qlist properties to their last stored configuration.

\hypertarget{chaser-properties}{%
\subsubsection{Chaser Properties}\label{chaser-properties}}

\textbf{Defaults} - Chase Properties

\includegraphics{https://files.gitbook.com/v0/b/gitbook-x-prod.appspot.com/o/spaces\%2F3kS90tLsADGm1ocbe7q9\%2Fuploads\%2Fx3HOw9fVBOsuOCVxbipY\%2F7.5.4.png?alt=media\&token=e223a6ec-52ab-4896-b1d9-35dcb6425bac}

\textbf{Chase pattern:}

\begin{itemize}
\item
  \textbf{Normal} - Cues loop from one to another in order.
\item
  \textbf{Build} - {Cues are added on to each other until the end of the Qlist and then the build is repeated. (Normal tracking behaviour).}
\item
  \textbf{Bounce} - Cues transition from lowest cue number to highest, then reverse and transition from highest to lowest.
\item
  \textbf{Bounce Build} - {Cues are added on to each other lowest to highest are added to each other highest to lowest.}
\item
  \textbf{Random} - Cue Order is randomized.
\end{itemize}

\textbf{Time:}

\begin{itemize}
\item
  \textbf{Chase Time:} { Independence in-time for dimmer when a chase is started.}
\item
  \textbf{Units Button} - {Sets the units for time settings.}
\item
  \textbf{Loop count:} {Repeats the chase loop the specified count and stops on the last cue.}
\end{itemize}

\textbf{Properties:}

\begin{itemize}
\item
  \textbf{Int Limit:} Proportionately limits the intensity output of the cues assigned to a controller.
\item
  \textbf{X-fade:} Proportionately adjusts the crossfade curve of a chase from full crossfade to jump.
\end{itemize}

\begin{center}\rule{0.5\linewidth}{0.5pt}\end{center}

\textbf{Defaults} - Properties:

Restore default:

\begin{itemize}
\tightlist
\item
  \textbf{Restores default} - Resets Chase properties to their last stored configuration.
\end{itemize}

\hypertarget{cue-settings}{%
\subsubsection{Cue Settings}\label{cue-settings}}

\textbf{Default} - Cue Settings

\includegraphics{https://files.gitbook.com/v0/b/gitbook-x-prod.appspot.com/o/spaces\%2F3kS90tLsADGm1ocbe7q9\%2Fuploads\%2FT8Zv89ZCvjgntdDbmiIZ\%2F7.5.5.png?alt=media\&token=6d1f88b5-7abe-4c31-822d-8ad687c0dfd5}

\textbf{See:}

\begin{itemize}
\item
  \textbf{Time:}

  \begin{itemize}
  \item
    \href{https://vibemanual.compulite.com/programming-cues-and-scenes.html\#assigning-cue-time}{10.11.1. Assigning Cue Time}
  \item
    \href{https://vibemanual.compulite.com/programming-cues-and-scenes.html\#wait-and-follow-times}{10.11.2. Wait and Follow Times}
  \end{itemize}
\item
  \textbf{Properties:}

  \begin{itemize}
  \tightlist
  \item
    \href{https://vibemanual.compulite.com/programming-cues-and-scenes.html\#tracking-options}{10.12.3. Tracking Options}
  \end{itemize}
\end{itemize}

\textbf{Update Popup}

The default for the \emph{New Values} is \emph{Master PB Only}

\includegraphics{https://files.gitbook.com/v0/b/gitbook-x-prod.appspot.com/o/spaces\%2F3kS90tLsADGm1ocbe7q9\%2Fuploads\%2FJVFqmv4vtVYVMeKmMG0X\%2Fimage.png?alt=media\&token=bd3a09bd-086c-4a87-b389-b644b3076ee5}

\hypertarget{cue-store-options}{%
\subsubsection{Cue Store Options}\label{cue-store-options}}

\textbf{Default} - Cue Store Options

\includegraphics{https://files.gitbook.com/v0/b/gitbook-x-prod.appspot.com/o/spaces\%2F3kS90tLsADGm1ocbe7q9\%2Fuploads\%2FjkzU21qPYa1QeWdSjgIe\%2F7.5.6.webp?alt=media\&token=f9d16c19-eb41-479b-9be3-e3a4d2dfd36c}

\textbf{See}: \href{https://vibemanual.compulite.com/programming-cues-and-scenes.html\#cue-store-options-1}{10.7. Cue Store Options}

\hypertarget{scene-settings}{%
\subsubsection{Scene Settings}\label{scene-settings}}

\textbf{Defaults} - Scene Settings

\begin{itemize}
\item
  \textbf{Time In} - Sets the default in-time for Scenes. The default value may be changed using:

  \begin{itemize}
  \item
    Virtual wheel
  \item
    Tapping in the centre of the wheel until it turns red then entering a value using the keypad
  \end{itemize}
\item
  \textbf{Units Button} - Sets the time units (H/M/S/Mil)
\end{itemize}

\hypertarget{scene-store-options}{%
\subsubsection{Scene Store Options}\label{scene-store-options}}

\textbf{Default} - Cue Store Options

\includegraphics{https://files.gitbook.com/v0/b/gitbook-x-prod.appspot.com/o/spaces\%2F3kS90tLsADGm1ocbe7q9\%2Fuploads\%2FDgLaAhLr29G2p63xJTdO\%2F7.5.8.webp?alt=media\&token=739d0f7d-3b86-45c5-a74c-5d035cab9acd}

\textbf{Similar to Cue Store Options. See:} \href{https://vibemanual.compulite.com/programming-cues-and-scenes.html\#cue-store-options-1}{10.7. Cue Store Options}

\hypertarget{bank-library-options}{%
\subsubsection{Bank Library Options}\label{bank-library-options}}

\textbf{Default} - Lib Options

\includegraphics{https://files.gitbook.com/v0/b/gitbook-x-prod.appspot.com/o/spaces\%2F3kS90tLsADGm1ocbe7q9\%2Fuploads\%2FDVM7PFXkDxkbwqmaCdYE\%2F7.5.9.webp?alt=media\&token=f4dd2186-436e-4398-8927-0f51c3c2a1c3}

\textbf{Options are the same for the following:}

\begin{itemize}
\tightlist
\item
  Intensity
\item
  Position
\item
  Color
\item
  Beam
\item
  Image
\item
  Shape
\end{itemize}

\textbf{See:}\\
\href{https://vibemanual.compulite.com/libraries.html\#store-and-modify-bank-libraries}{12.2. Store and Modify Bank Libraries}

\textbf{Options for Effect include an additional option:}

\begin{itemize}
\tightlist
\item
  Include Base - If toggled on, the start points of parameters will be stored with the Effect making them ``Absolute Effects''.
\end{itemize}

\hypertarget{snap-settings}{%
\subsubsection{Snap Settings}\label{snap-settings}}

\textbf{Default} - Snap Settings

\includegraphics{https://files.gitbook.com/v0/b/gitbook-x-prod.appspot.com/o/spaces\%2F3kS90tLsADGm1ocbe7q9\%2Fuploads\%2FcUc9GJIbRrx2Up4cS8iO\%2F7.5.10.webp?alt=media\&token=7fcc6e63-3807-4460-a4c0-f8601d3c0834}

\textbf{Load Pages:}

\begin{itemize}
\tightlist
\item
  Snap will change the current page to the page the Snap was stored in.
\end{itemize}

\textbf{Load Master Controller:}

\begin{itemize}
\tightlist
\item
  Snap will change the {[}Select{]} to to the controller that was Master Controller at the time of the Snap.
\end{itemize}

\textbf{Snap Time:}

\begin{itemize}
\tightlist
\item
  Default fade time for all controllers being initiated by the Snap.
\end{itemize}

\textbf{Units Button:}

\begin{itemize}
\tightlist
\item
  Sets the time units. (H/M/S/Mil)
\end{itemize}

See: \href{https://vibemanual.compulite.com/snaps-snapshots.html}{17. Snaps}

\hypertarget{effects}{%
\subsubsection{Effects}\label{effects}}

When using pre-built effects on Pan \& Tilt parameters, the default size of the effect is 20\%.

\includegraphics{https://files.gitbook.com/v0/b/gitbook-x-prod.appspot.com/o/spaces\%2F3kS90tLsADGm1ocbe7q9\%2Fuploads\%2FSvpUbdsjf1qb4sbjxjRD\%2F7.5.11.png?alt=media\&token=4263c8e1-10ed-415f-a5e4-b32fdae53458}

\hypertarget{system}{%
\subsection{System}\label{system}}

\textbf{Default} - System

\includegraphics{https://files.gitbook.com/v0/b/gitbook-x-prod.appspot.com/o/spaces\%2F3kS90tLsADGm1ocbe7q9\%2Fuploads\%2FzvCIV8NuJjCwGBqHV4cv\%2F7.6.png?alt=media\&token=2cc87dc5-332f-47eb-8ed2-cd59b916dd85}

\textbf{Vibe is made up of four modules:}

\begin{enumerate}
\def\labelenumi{\arabic{enumi}.}
\item
  \textbf{GUI} - Desktop Graphical Interface application.
\item
  \textbf{LPU} (Light Processing Unit) - Calculates core ``real time'' calculations.
\item
  \textbf{SQL} DataBase - Data management. (There is no option to Auto Recover the database as it is always up to date)
\item
  \textbf{HAL} - Hardware Access Layer.
\end{enumerate}

\textbf{Auto Recover:}

\begin{itemize}
\tightlist
\item
  In the case of an unexpected halt, the toggled on modules will attempt a restart.
\end{itemize}

\hypertarget{programming-concepts}{%
\section{Programming Concepts}\label{programming-concepts}}

This chapter deals with Vibe programming philosophy, command line behavior, and syntax conventions.

\textbf{The following is covered in this chapter:}

\begin{itemize}
\tightlist
\item
  \href{https://vibemanual.compulite.com/programming-concepts.html\#operational-philosophy}{8.1. Operational Philosophy}
\item
  \href{https://vibemanual.compulite.com/programming-concepts.html\#command-syntax}{8.2. Command Syntax}
\item
  \href{https://vibemanual.compulite.com/programming-concepts.html\#command-line}{8.3. Command Line}
\end{itemize}

\hypertarget{operational-philosophy}{%
\subsection{Operational Philosophy}\label{operational-philosophy}}

The Vibe is an advanced lighting control system designed to simplify programming in all styles of Entertainment lighting.

\textbf{Abstraction:}

\begin{itemize}
\item
  To simplify programming, Vibe uses an abstract layer that allows fixtures of different hardware types to be presented to the user in a universal way.
\item
  The dedicated graphical ``Smart Screen'' presents global pickers for most parameters including:

  \begin{itemize}
  \item
    HSV color picker (Hue, Saturation, Intensity)
  \item
    P/T XYZ Trackpad
  \item
    Blade picker
  \item
    Beam picker for Iris, Zoom, Focus and Frost effects
  \item
    General slider picker for GOBO, Color, Prism, and Animation wheels
  \item
    Intensity picker for common intensity values and strobe effects
  \end{itemize}
\end{itemize}

\textbf{Real World Units:}

\begin{itemize}
\item
  Where possible picker values are displayed in ``Real World'' units

  \begin{itemize}
  \item
    {XYZ P/T in degrees}
  \item
    Strobe in HZ
  \item
    {Color in HSV 0-360 degrees}
  \item
    {Zoom in degrees}
  \end{itemize}
\end{itemize}

\textbf{Virtual Parameters:}

\begin{itemize}
\item
  To make it easier to work with fixtures using different color systems and XYZ tracking systems, Vibe creates virtual parameters that may not exist in the physical fixture. Examples would be:

  \begin{itemize}
  \item
    HSV - As a common denominator for CMY, RGB, and Multi color LED fixtures such as RGBA, ETC Seven color, RGBW.
  \item
    Virtual Intensity - For controlling LED fixtures that do not have a dimmer parameter.
  \item
    XYZ - for calibration of fixtures that only have a physical Pan/Tilt
  \end{itemize}
\end{itemize}

\textbf{Tracking:}

\begin{itemize}
\tightlist
\item
  Vibe by default is a \textbf{Tracking} console and follows the conventions of modern theatrical consoles. Vibe also has a special track feature called \textbf{SKIP}. If cues are modified using \textbf{SKIP}, instead of \textbf{Qonly}, the relationship with the original hard value is maintained, and changes to the original value ``\textbf{Skip}'' over the modification and preserve the track.
\end{itemize}

\textbf{Qonly Mode:} {Not implemented globally}

\begin{itemize}
\item
  {Modifications, insertions and deletions will be done Qonly by default}
\item
  {Move fades will still be valid}
\end{itemize}

\textbf{Non - Tracking Compulite Mode - Defaults are changed to:}

\begin{itemize}
\item
  Tracking turned off.
\item
  The Editor is not cleared after a Store.
\item
  After Reset, Editor values remain at their last position instead of returning to home values. (Compulite Gray tracking)
\item
  Controllers do not advance after a Store.
\end{itemize}

\textbf{Color spaces:}

\begin{itemize}
\item
  The Vibe can work in three color spaces:

  \begin{itemize}
  \item
    HSV - Color for all types of fixtures can be set using the Color picker or the HSV parameter wheels. HSV virtual parameters are stored in Cues and Libraries just like physical parameters. Automatically CMY or RGB values (depending on the fixture type) will be displayed but not directly referenced in cues using HSV.
  \item
    RGB - If color values are set using the physical RGB parameters, those values are stored in the cue and HSV values will be displayed but not directly referenced .
  \item
    CMY - If color values are set using the CMY parameters, those values are stored in the cue and HSV values will be displayed but not directly referenced.
  \end{itemize}
\end{itemize}

\textbf{Color crossfading:}

\begin{itemize}
\tightlist
\item
  Currently, color crossfading is done using the physical color parameters. In future release it will be possible to crossfade HSV using predefined vectors
\end{itemize}

\textbf{Maintain Selection:}

\begin{itemize}
\item
  To avoid constantly reselecting fixtures after a STORE command, Vibe maintains its selection on reset of the editor.
\item
  To clear the editor and selection, {[}VIBE{]} (Shift) + {[}RESET{]} must be pressed.
\end{itemize}

\hypertarget{command-syntax}{%
\subsection{Command Syntax}\label{command-syntax}}

Vibe uses a hybrid command syntax.

\textbf{When dealing with keypad command line driven operations, the logic is:}

\begin{itemize}
\tightlist
\item
  Source Objects ⟶ Filters ⟶ Options ⟶ Command Completes. (Action Completes Mode)
\end{itemize}

e.g.~\emph{{[}Group 1{]} {[}Full{]} {[}Library 2 (Position) 1{]} {[}Enter{]} {[}Cue 1{]} {[}Store Options{]} \{Red + White\} {[}Store{]}}

\textbf{For softkeys and controllers where touching the object completes the sequence, the logic is:}

\begin{itemize}
\tightlist
\item
  Source Object, Filter, Command, Options, Destination Object Completes (Here Mode) Completes
\end{itemize}

\emph{\{GROUP\} {[}1{]}{[}Full{]}, {[}LIBRARY{]} \{Position\} {[}3{]} {[}Store{]} {[}Store Options{]} \{Red Only\} {[}Press any {[}Button{]} on Slider Controller 1{]}.}

\includegraphics{https://files.gitbook.com/v0/b/gitbook-x-prod.appspot.com/o/spaces\%2F3kS90tLsADGm1ocbe7q9\%2Fuploads\%2FEmYCVzO9Xwt9HSC7m72H\%2F8.2.webp?alt=media\&token=1a34b4b3-592b-4fee-8b27-c3ae19208b19}

\hypertarget{command-line}{%
\subsection{Command Line}\label{command-line}}

The \textbf{Command Line} is the area where commands appear as they are entered. When in Live it is bounded in grey on the big monitor and blue on the small monitor. When in BLIND, both command lines are bounded in yellow. Commands may be backspaced using the Clear Entry {[}CE{]} key, The command line can be clear back to the beginning using the {[}CLEAR{]} key ({[}VIBE{]}+{[}CE{]}) Command are also augmented with the interactive toolbar beside the command line on the big screen.

\includegraphics{https://files.gitbook.com/v0/b/gitbook-x-prod.appspot.com/o/spaces\%2F3kS90tLsADGm1ocbe7q9\%2Fuploads\%2F8xl1NHp28r162Z2ObQ8n\%2F8.3.webp?alt=media\&token=3ffdca98-e9d7-4d71-9c69-bcf68cbc7288}

\hypertarget{programming-basics}{%
\section{Programming Basics}\label{programming-basics}}

This chapter deals with the fundamentals of building stage looks in the editor.

\textbf{The following is covered in this chapter:}

\begin{itemize}
\tightlist
\item
  \href{https://vibemanual.compulite.com/programming-basics.html\#editor}{9.1. Editor}
\item
  \href{https://vibemanual.compulite.com/programming-basics.html\#editor-tools}{9.2. Editor Tools}
\item
  \href{https://vibemanual.compulite.com/programming-basics.html\#working-with-fixtures}{9.3. Working with Fixtures}
\item
  \href{https://vibemanual.compulite.com/programming-basics.html\#groups}{9.4. Groups}
\end{itemize}

\hypertarget{editor}{%
\subsection{Editor}\label{editor}}

Editor Basics:

\begin{itemize}
\tightlist
\item
  \href{https://vibemanual.compulite.com/programming-basics.html\#what-is-the-editor}{9.1.1. What is the Editor}
\item
  \href{https://vibemanual.compulite.com/programming-basics.html\#blind-editor}{9.1.2. Blind Editor}
\end{itemize}

\hypertarget{what-is-the-editor}{%
\subsubsection{What is the Editor}\label{what-is-the-editor}}

All manual parameter modification occurs in the \textbf{EDITOR}. (Programmer in some other consoles). The \textbf{Live Displays} show all values from \textbf{both} Controller outputs (Playbacks or Executors on some consoles) and fixtures in the Editor. Color coding is used to distinguish between values coming from the controllers and values in the \textbf{EDITOR}.

\includegraphics{https://files.gitbook.com/v0/b/gitbook-x-prod.appspot.com/o/spaces\%2F3kS90tLsADGm1ocbe7q9\%2Fuploads\%2FJ376dF1k5YYrLIEw9Guq\%2F9.1.1.webp?alt=media\&token=9bed6d66-ad72-4973-add6-9eb94d134da5}

\textbf{Home a fixture in the Editor:}

\begin{itemize}
\tightlist
\item
  {[}FIXTURE{]} {[}\#{]} {[}HOME{]}
\end{itemize}

\textbf{Home a fixture parameter in the Editor:}

\begin{itemize}
\tightlist
\item
  {[}FIXTURE{]} {[}\#{]} \{BANK\} \{PARAMETER\} {[}HOME{]}
\end{itemize}

\textbf{Release a fixture in the Editor:}

\begin{itemize}
\tightlist
\item
  {[}FIXTURE{]} {[}\#{]} RELEASE{]}
\end{itemize}

\textbf{Release a fixture parameter in the Editor:}

\begin{itemize}
\tightlist
\item
  {[}FIXTURE{]} {[}\#{]} \{BANK\} \{PARAMETER\} {[}RELEASE{]}
\end{itemize}

\textbf{Reset (Clear) the Editor:}

\begin{itemize}
\tightlist
\item
  Press {[}RESET{]} key
\end{itemize}

\textbf{Reset (Clear) the Editor and clear the selection:}

\begin{itemize}
\tightlist
\item
  Press {[}VIBE{]} + {[}RESET{]}
\end{itemize}

See also: \href{https://vibemanual.compulite.com/programming-basics.html\#release-and-home-sequences}{9.3.10. Release and Home Sequences}

\hypertarget{blind-editor}{%
\subsubsection{Blind Editor}\label{blind-editor}}

The Blind Editor is used to create or modify objects without outputting values to the stage

Vibe has two Blind Editor behaviours:

\begin{enumerate}
\def\labelenumi{\arabic{enumi}.}
\item
  Blind acts as a discrete editor separate from the live editor
\item
  Blind emulates the behaviour of single editor consoles where the main editors output is just disabled
\end{enumerate}

\textbf{To build a new cue in the Blind Editor:}

\begin{enumerate}
\def\labelenumi{\arabic{enumi}.}
\item
  Press {[}Blind{]} key flashes \textbf{{yellow}} and the Command Line turns \textbf{yellow}.
\item
  Program values using the Live Display for reference - no values will be outputted to the stage
\item
  Store objects as usual - See also:
\end{enumerate}

\begin{itemize}
\item
  9.4. Creating and Applying Groups
\item
  10.3. Storing Cues and Scenes Directly to Controllers
\item
  10.5. Storing Cues to the Master Controller
\item
  10.6. Storing Qlist Cues to any Controller
\item
  12.2. Store and Modify Bank Libraries
\end{itemize}

\textbf{To edit a cue in the Blind Editor:}

\begin{enumerate}
\def\labelenumi{\arabic{enumi}.}
\item
  Press {[}Blind{]} - key flashes \textbf{{yellow}} and the Command Line turns \textbf{yellow}.
\item
  Press {[}QLIST{]} {[}\#{]} {[}CUE{]} {[}\#{]} {[}EDIT{]} - {[}UPDATE{]} flashes {red}.
\item
  Make modifications to the cue
\item
  Press {[}UPDATE{]} to complete the sequence
\end{enumerate}

\textbf{To emulate single editor blind:}

\begin{enumerate}
\def\labelenumi{\arabic{enumi}.}
\item
  Press {[}VIBE{]}+{[}BLIND{]} - Copies the Live Editor contents to the Blind Editor
\item
  Toggle {[}VIBE {]}+{[}BLIND{]} - Copies the contents of the Blind Editor to the Live Editor (The original contents stay in the Blind Editor
\end{enumerate}

\hypertarget{editor-tools}{%
\subsection{Editor Tools}\label{editor-tools}}

\textbf{Editor Tools:}

\begin{itemize}
\tightlist
\item
  \href{https://vibemanual.compulite.com/programming-basics.html\#selections-tools}{9.2.1. Selections Tools}
\item
  \href{https://vibemanual.compulite.com/programming-basics.html\#pattern}{9.2.2. Transpose}
\item
  \href{https://vibemanual.compulite.com/programming-basics.html\#fan-mode}{9.2.3. Pattern}
\item
  \href{https://vibemanual.compulite.com/programming-basics.html\#fan-mode}{9.2.4. Fan Mode}
\item
  \href{https://vibemanual.compulite.com/programming-basics.html\#highlight-mode}{9.2.5. Highlight Mode}
\item
  \href{https://vibemanual.compulite.com/programming-basics.html\#lowlight-mode}{9.2.6. Lowlight Mode}
\item
  \href{https://vibemanual.compulite.com/programming-basics.html\#remainder-dim-rem-dim}{9.2.7. Remainder Dim \{Rem Dim\}}
\item
  \href{https://vibemanual.compulite.com/programming-basics.html\#home-scene}{9.2.8 Home Scene}
\end{itemize}

\hypertarget{selections-tools}{%
\subsubsection{Selections Tools}\label{selections-tools}}

Currently, Vibe has basic Selection Tools. Advanced tools are under development.

\includegraphics{https://files.gitbook.com/v0/b/gitbook-x-prod.appspot.com/o/spaces\%2F3kS90tLsADGm1ocbe7q9\%2Fuploads\%2F9gX5fSd003KSW9SXkmvY\%2F9.2.1.webp?alt=media\&token=2cf874a1-a246-4be6-b36b-b8b1b3ec6fb9}

\begin{itemize}
\item
  \{ODDS\} - Absolute odd number in the range
\item
  \{Evens\} - Absolute even numbers in the range
\item
  \{2/1\} - Grouping of 2 starting at the first fixture of the range (Relative odd numbers)
\item
  \{2/2\} - Grouping of 2 starting at the second fixture of the range (relative even numbers)
\item
  \{3/1\} - Grouping of 3 starting at the first fixture of the range
\item
  \{4/1\} - Grouping of 4 starting at the first fixture of the range
\item
  \textbf{\{/\}} - Used as a separator. \textbf{Groupings of \#/\# Start fixture of the grouping}
\item
  \textbf{E.g. -} \emph{{[}Fixture{]} {[}1{]} {[}⟶ {]} {[}14{]} \{Section Tools\} {[}3{]} \{/\} {[}1{]} - would select fixtures 1,4,7,10,13 (as soon as the \{/\} is pressed \textbf{EVERY 3/1} will be displayed on the command line).}
\end{itemize}

\emph{{[}Fixture{]} {[}1{]} {[}⟶ {]} {[}14{]} \{Section Tools\} {[}3{]} \{/\} {[}2{]} - would select fixtures 2,5,8,11,14 (as soon as the \{/\} is pressed \textbf{EVERY 3/2} will be displayed on the command line).}

\hypertarget{transpose}{%
\subsubsection{Transpose}\label{transpose}}

Transpose is accessed from the Editor Toolbar. It is used to reorder fixture selection based on a specified offset. This is particularly useful in creating interesting effects and fanning.

\textbf{E.g. 1:}

\emph{- Fixture 1 ⟶ 12 \{Transpose\} 2 {[}ENTER{]} = 1+3+5+7+9+11+2+4+6+8+10+12}

\textbf{E.g. 2:}

\emph{- Fixture 1 ⟶ 12 \{Transpose\} 3 {[}ENTER{]} = 1+4+7+10+2+5+8+11+3+6+9+12}

\textbf{E.g. 3:}

\emph{- Fixture 1⟶ 12 \{Transpose\} 4 {[}ENTER{]} = 1+5+9+2+6+10+3+7+11+4+8+12}

\hypertarget{pattern}{%
\subsubsection{Pattern}\label{pattern}}

Is accessed from the Editor Toolbar. It can be used prior to entering the Advanced Effects editor to create more complex block patterns that are available using the basic \{Block Of\} and \{Sub-Block Of\} key in the effects editor.

\textbf{To set a pattern:}

\begin{enumerate}
\def\labelenumi{\arabic{enumi}.}
\item
  Make a fixture selection
\item
  Tap \{Pattern\} on the Editor Toolbar - The Pattern pop-up will open. 👉 {This is not an effect even though it is a dynamic display.}
\item
  Tap the indicator at the top of the displayed fixtures. - it should turn white.
\item
  Select a number under the Block Size header.
\item
  Tap the \{Spread\} icon. The fixture columns will expand to show the new fixture order.
\item
  If desired make further order changes under the Offset Header and tap \{offset\} icon.
\item
  Make further modifications or reset the selection with the \{Reset\} \{Invert\} \{transpose\} and \{Invert Columns\} keys.
\item
  Once finished, exit the pop-up by pressing \href{image.png}{} or {[}ENTER{]}.
\item
  Open the Advanced Effects editor and program the desired effect.
\end{enumerate}

\begin{itemize}
\tightlist
\item
  See: \href{https://vibemanual.compulite.com/effects-1.html\#advanced-effects-editor}{13.3. Advanced Effects Editor}
\end{itemize}

\begin{quote}
Simple Block and Sub Block groupings may be created within both Effects Editors. The pattern is for advanced blocks and groupings
\end{quote}

\hypertarget{fan-mode}{%
\subsubsection{Fan Mode}\label{fan-mode}}

\textbf{Vibe has two methods to fan fixture parameters:}

\includegraphics{https://files.gitbook.com/v0/b/gitbook-x-prod.appspot.com/o/spaces\%2F3kS90tLsADGm1ocbe7q9\%2Fuploads\%2FRCZsSlnmcSFafQEwMVRY\%2F9.2.4.webp?alt=media\&token=2157359e-bd73-4d57-b692-0826758d2620}

\begin{enumerate}
\def\labelenumi{\arabic{enumi}.}
\item
  Using the {[}Fan{]} key and interface.
\item
  Using keypad syntax.
\end{enumerate}

See: \href{https://vibemanual.compulite.com/programming-basics.html\#fixture-selection-detailed}{9.3.7. Fixture Selection (Detailed)}

\textbf{Using {[}Fan{]} Mode:}

\begin{enumerate}
\def\labelenumi{\arabic{enumi}.}
\item
  Select the range of fixtures to be fanned.
\item
  Press {[}FAN{]} to enter Fan Mode.
\item
  Select a Function from the sliding picker.
\item
  Turn the encoder wheel of the parameter requiring fanning.
\item
  Toggle {[}FAN{]} to turn off Fan Mode.
\end{enumerate}

\begin{quote}
Selecting a different parameter will also turn Fan mode off.
\end{quote}

\textbf{Common Fan Functions are :}

\begin{itemize}
\item
  \{Fade In\} / = lowest to highest
\item
  \{Fade Out\} ~= highest to lowest
\item
  \{Diagonal\} -/- = Standard PAN Fan
\item
  \{Saw Tooth\} /~= Mirror lowest on ends to highest in middle
\item
  \{Mirror\} / = Highest on ends to lowest in middle
\end{itemize}

\begin{quote}
Function types may be freely changes while in Fan Mode to ``audition'' various results.
\end{quote}

\begin{quote}
Any Function created in the Function Editor of the Advanced Effect Editor may be applied to Fan
\end{quote}

\textbf{Block of and Sub Blocks:}

\begin{enumerate}
\def\labelenumi{\arabic{enumi}.}
\item
  Select the range of fixtures to be fanned.
\item
  Press {[}FAN{]} to enter Fan Mode.
\item
  Select a Function from the sliding picker.
\item
  Select \textbf{Blocks of \{\#\}}
\item
  Select \textbf{Sub Blocks \{\#\}}
\item
  Turn the encoder wheel of the parameter requiring fanning.
\item
  Toggle {[}FAN{]} to turn off Fan Mode.
\end{enumerate}

\textbf{Examples of Blocks and Sub Blocks:}

\begin{itemize}
\tightlist
\item
  \{Fade In\} - Blocks of 1 and no Sub Block
\end{itemize}

\includegraphics{https://files.gitbook.com/v0/b/gitbook-x-prod.appspot.com/o/spaces\%2F3kS90tLsADGm1ocbe7q9\%2Fuploads\%2FQzZ38bEmppvXl48ulv8q\%2F9.2.4.1.webp?alt=media\&token=af1d73eb-2f53-462d-88ad-0227cdd79c2f}

\begin{itemize}
\tightlist
\item
  \{Fade In\} - Blocks of 4 no Sub Blocks
\end{itemize}

\includegraphics{https://files.gitbook.com/v0/b/gitbook-x-prod.appspot.com/o/spaces\%2F3kS90tLsADGm1ocbe7q9\%2Fuploads\%2Fygiz82lG0e6QyeKOCsyM\%2F9.2.4.2.webp?alt=media\&token=85ed0077-6d5f-44d5-bdc4-a8b7397d32f3}

\begin{itemize}
\tightlist
\item
  \{Fade In\} - Blocks of 6, Sub Blocks of 2
\end{itemize}

\includegraphics{https://files.gitbook.com/v0/b/gitbook-x-prod.appspot.com/o/spaces\%2F3kS90tLsADGm1ocbe7q9\%2Fuploads\%2F5DhSrryscbhAwmpb6kLe\%2F9.2.4.3.webp?alt=media\&token=041f6ae8-95c7-4515-8a7e-7d6eebfb0da1}

\begin{itemize}
\tightlist
\item
  \{Saw Tooth\} - (Low-High Mirror) Blocks of 12, Sub Blocks 2
\end{itemize}

\includegraphics{https://files.gitbook.com/v0/b/gitbook-x-prod.appspot.com/o/spaces\%2F3kS90tLsADGm1ocbe7q9\%2Fuploads\%2FyNnPhSss8ibiMnNEoCUE\%2F9.2.4.4.webp?alt=media\&token=cac11c41-6744-4ae2-b878-451fe9830e66}

\begin{itemize}
\tightlist
\item
  Moving Light Fixture with \{Fade In\} Dimmer and Color Fan, and \{Diagonal\} Pan Fan
\end{itemize}

\includegraphics{https://files.gitbook.com/v0/b/gitbook-x-prod.appspot.com/o/spaces\%2F3kS90tLsADGm1ocbe7q9\%2Fuploads\%2FW2ZfOpxlsJUa56lG2mgQ\%2F9.2.4.5.webp?alt=media\&token=284e82aa-eec2-4e18-a210-4c1d01ff8752}

Instead of fanning Blocks based on the selection, fixture \textbf{Groups} may also be used as Blocks.

\textbf{To Fan Blocks By Groups:}

\begin{itemize}
\item
  Using the Group SKs or using the keypad, select the Groups to be fanned
\item
  Tap \{Blocks By Group\}
\item
  Use any of the above methods to fan parameters by Group Parameters may also be fanned using keypad syntax.
\item
  \href{https://vibemanual.compulite.com/programming-basics.html\#fixture-selection-detailed}{Fixture Selection (Detailed)}
\end{itemize}

\hypertarget{highlight-mode}{%
\subsubsection{Highlight Mode}\label{highlight-mode}}

Highlight and Lowlight modes are used to help identify the positions of fixtures so that they may be edited more easily.

\textbf{Highlight Mode affects parameters and banks in the following way:}

\begin{itemize}
\item
  Dimmer - Full
\item
  Other Intensity parameters - home values and masked
\item
  Position - unchanged and editable
\item
  Color - home values and masked
\item
  Beam - home values and masked
\item
  Image - home values and masked
\item
  Shape - home values and masked
\end{itemize}

\textbf{To Highlight fixtures or Groups:}

\begin{enumerate}
\def\labelenumi{\arabic{enumi}.}
\item
  Select fixtures using keypad or using Groups.
\item
  Press {[}H.LIGHT{]} - The key will flash in {red} to indicate the mode is active.

  \textbf{All} fixtures of the selection will be set as specified above.
\item
  Press {[}NEXT{]} - The first fixture will remain ``highlighted'' and all other fixtures will return to their editor or stage values.
\item
  Adjust Pan/Tilt as required and press {[}NEXT{]} to move to the next fixture.
\item
  Continue through the fixture selection using {[}NEXT{]} {[}PREV{]} until all fixtures are focused.
\item
  Press {[}H.LIGHT{]} again to turn off Highlight Mode. The red light will go out and fixtures will return to their current editor or stage values.
\end{enumerate}

\begin{quote}
Highlight/Lowlight may be entered before a selection is made. Press {[}ENTER{]} to Highlight the selection
\end{quote}

\textbf{To swap new fixtures into Highlight:}

\begin{itemize}
\tightlist
\item
  Select new fixtures or Groups and press {[}ENTER{]} while still in Highlight Mode - the Highlight selection will be swapped.
\end{itemize}

\begin{quote}
Until {[}NEXT{]} {[}PREV{]} is pressed Groups will be appended to make up a larger Highlight selection.
\end{quote}

Highlight may be customized when the default home values are too generic. Users may want to tighten the zoom or use a specific iris as a fixtures default.

\textbf{To create a custom Highlight Scene:}

\begin{enumerate}
\def\labelenumi{\arabic{enumi}.}
\item
  Set the values in the Editor similar to creating a Cue or a Scene
\item
  Press {[}SCENE{]} {[}H.LIGHT{]} {[}STORE{]}
\end{enumerate}

\begin{quote}
The Highlight Scene only affects the active fixture.
\end{quote}

\textbf{To Delete the Highlight Scene:}

\begin{itemize}
\tightlist
\item
  Press {[}SCENE{]} {[}H.LIGHT{]} {[}DELETE{]}
\end{itemize}

\hypertarget{lowlight-mode}{%
\subsubsection{Lowlight Mode}\label{lowlight-mode}}

Lowlight mode is used when many fixtures are lit up and therefore Highlight is not helpful. \textbf{Lowlight Mode affects parameters and banks in the following way:}

\begin{itemize}
\item
  Dimmer - 30\% all fixtures \textbf{except} active fixture which will be at 100\%
\item
  Other Intensity parameters - home values and masked.
\item
  Position - unchanged and editable
\item
  Color - home values and masked
\item
  Beam - home values and masked
\item
  Image - home values and masked
\item
  Shape - home values and masked Unlike Highlight, Lowlight maintains a value of 30\% for dimmers while active.
\end{itemize}

\textbf{To ``Lowlight'' fixtures or Groups:}

\begin{enumerate}
\def\labelenumi{\arabic{enumi}.}
\item
  Select fixtures using the keypad or using Groups.
\item
  {Press {[}L.LIGHT{]} - All dimmer parameters will go to 30\%. {[}L.LIGHT{]} will flash in red to indicate the mode is active. All other fixtures of the selection will be set as specified above.}
\item
  Press {[}NEXT{]} - The first fixture will go to a dimmer value of 100\% and all other fixtures dimmer values will remain at 30\%. Other parameters will return to their editor or stage values.
\item
  Adjust Pan/Tilt as required and press {[}NEXT{]} to move to the next fixture.
\item
  Continue through the fixture selection using {[}NEXT{]} {[}PREV{]} until all fixtures are focused.
\item
  Press {[}L.LIGHT{]} again to turn off Highlight Mode. The red light will go out and fixtures will return to their current editor or stage values.
\end{enumerate}

\begin{quote}
Highlight/Lowlight may be entered before a selection is made.
Press {[}ENTER{]} to Highlight the selection.
\end{quote}

\textbf{To swap new fixtures into Lowlight:}

\begin{itemize}
\item
  Select new fixtures or Groups and press {[}ENTER{]} while still in Highlight

  Mode - the Highlight selection will be swapped.
  \textgreater Until {[}NEXT{]} {[}PREV{]} is pressed Groups will be appended to make up a larger Highlight selection.
\end{itemize}

Lowlight may be customized when the default home values are are too generic. Frequently users set the Lowlight dimmer to Zero instead of the default 30\% to create a solo effect.

\textbf{To create a custom Lowlight Scene:}

\begin{enumerate}
\def\labelenumi{\arabic{enumi}.}
\tightlist
\item
  Set the values in the Editor similar to creating a Cue or a Scene.
\item
  Press {[}SCENE{]} {[}L.LIGHT{]} {[}STORE{]}
\end{enumerate}

\textbf{To Delete the Highlight Scene:}

\begin{itemize}
\tightlist
\item
  Press {[}SCENE{]} {[}L.LIGHT{]} {[}DELETE{]}
\end{itemize}

\hypertarget{remainder-dim-rem-dim}{%
\subsubsection{Remainder Dim \{Rem Dim\}}\label{remainder-dim-rem-dim}}

\textbf{REM DIM} is an Editor Tool commonly used in theatr environments but can be useful in many other situations.

\begin{itemize}
\item
  \textbf{\{REM DIM\}} Toggles all dimmer values to 0 except those that are selected. If no value is given in the editor, the selected fixtures will maintain their current values and not create a hard move.
\item
  While in \{REM DIM\}, {[}Next{]}/{[}Prev{]} will swap the Rem Dim fixture forwards or backwards.
\item
  If a selection of multiple fixtures is made followed by \{REM DIM\} all unselected fixtures will go to 0. The selected fixtures will not be changed or entered into the editor.
\item
  If a selection of multiple fixtures is made, {[}Next{]}/{[}Prev{]} will force all but the first fixture in the selection to 0. The next press of {[}Next{]}/{[}Prev{]} will force the first fixture to 0 and release the next fixture. Fixtures in the selection will continue to swap from 0 to release with each press of {[}Next{]}/{[}Prev{]}.
\item
  Pressing REM DIM again toggles the feature off and restores the stage look.
\item
  Cues may be stored with \{REM DIM\} on but once stored, \{REM DIM\} is turned off.
\end{itemize}

\hypertarget{home-scene}{%
\subsubsection{Home Scene}\label{home-scene}}

By default released fixtures return to their home values as defined by the device's fixture profile. The device profile is usually based on the manufacturer's suggested defaults, but sometimes the defaults are not appropriate. Vibe allows users to customize Home Values by building a Home Scene. When a fixture is programmed in a Home Scene it will take preference over its home values in the device profile.

\textbf{Create a Home Scene:}

\begin{enumerate}
\def\labelenumi{\arabic{enumi}.}
\item
  Program fixtures in the editor to values that will work best as home values.
\item
  Press {[}SCENE{]} {[}HOME{]} {[}STORE{]} - When released fixtures will now use the Home Scene as a reference.
\end{enumerate}

\textbf{Delete a Home Scene:}

\begin{enumerate}
\def\labelenumi{\arabic{enumi}.}
\tightlist
\item
  Press {[}SCENE{]} {[}HOME{]} {[}DELETE{]} - Home values will return to the device profile's home values.
\end{enumerate}

\hypertarget{working-with-fixtures}{%
\subsection{Working with Fixtures}\label{working-with-fixtures}}

Working with Fixtures:

\begin{itemize}
\item
  \href{https://vibemanual.compulite.com/programming-basics.html\#fixture-selection-basics}{9.3.1. Fixture Selection basics}
\item
  \href{https://vibemanual.compulite.com/programming-basics.html\#what-are-parameters}{9.3.2. What are Parameters}
\item
  \href{https://vibemanual.compulite.com/programming-basics.html\#virtual-parameters}{9.3.3. Virtual Parameters}
\item
  \href{https://vibemanual.compulite.com/programming-basics.html\#smart-screen-1}{9.3.4. Smart Screen}
\item
  \href{https://vibemanual.compulite.com/programming-basics.html\#adjusting-parameters-using-the-wheel-picker}{9.3.5. Adjusting Parameters using the Wheel Picker}
\item
  \href{https://vibemanual.compulite.com/programming-basics.html\#grab}{9.3.6. Grab}
\item
  \href{https://vibemanual.compulite.com/programming-basics.html\#fixture-selection-detailed}{9.3.7. Fixture Selection (Detailed)}
\item
  \href{https://vibemanual.compulite.com/programming-basics.html\#setting-parameters-using-main-encoder-wheels}{9.3.8. Setting Parameters using Main Encoder Wheels}
\item
  \href{https://vibemanual.compulite.com/programming-basics.html\#setting-parameters-using-interactive-encoder-wheels}{9.3.9. Setting Parameters using Interactive Encoder Wheels}
\item
  \href{https://vibemanual.compulite.com/programming-basics.html\#release-and-home-sequences}{9.3.10. Release and Home Sequences}
\end{itemize}

\hypertarget{fixture-selection-basics}{%
\subsubsection{Fixture Selection basics}\label{fixture-selection-basics}}

Fixtures are selected using the \textbf{{[}FIXTURE{]}} key on the keypad. They also may be stored and recalled using the \textbf{{[}GROUP{]}} key. Fixtures may also be interactively selected via the Live Display or Live Parameter Display.

\begin{itemize}
\item
  \textbf{{[}+{]} {[}-{]}} ⟶ keys are valid with Fixture selections and Groups.
\item
  \textbf{{[}FIXTURE{]}} ◾ (period) - Reselects the last selection.
\item
  \textbf{{[}FIXTURE{]}} ⟶ - Reselects all fixtures in the Editor.
\item
  \textbf{{[}CE{]}} - Backspaces the command line one character at a time but currently will not go back one command at a time.
\item
  \textbf{{[}VIBE{]}+{[}CE/CLEAR{]}} - Clears the command line to the beginning and ideal tee system.
\item
  \textbf{{[}RELEASE{]}}:
\end{itemize}

◾ First press releases the active parameters (last adjusted) for selected fixtures.

◾ Second press releases all remaining parameters for selected fixtures from the Editor.

\begin{itemize}
\item
  \textbf{{[}DE SEL{]}} - Deselects all fixtures but maintains values in the Editor. This function is usually used to protect the operator from accidentally adjusting parameters in the Editor when they are temporarily captured. It may also be used to ``idle'' the command line to allow {[}TEXT{]} \{SK\} function while fixtures are still in the Editor.
\item
  \textbf{{[}RESET{]}} - Releases all parameters from the Editor.
\item
  \textbf{{[}VIBE{]}+{[}RESET{]}} - Releases all parameters from the Editor and releases all selected fixtures.
\item
  \textbf{{[}NEXT{]}/ {[}PREV{]}} - Increments or decrements fixtures in a selection. When used with a Group, will be confined to the fixtures in the Group.
\item
  \textbf{Trackball {[}X LOCK{]} / {[}Y LOCK{]}} - Locks the X or Y axis of the trackball.
\item
  \textbf{{[}RES{]}} - Sets global encoder resolution. {Blue} = Low 8 bit, {Green} = Medium 10 bit (Default) {red} = High 16 bit.
\end{itemize}

\begin{quote}
Holding the {[}VIBE{]} key down while turning an encoder temporarily sets the encoder to high resolution 16 bit.
\end{quote}

\begin{quote}
Vibe uses a precision algorithm for encoders. Based on the starting resolution, the slower an encoder is turned, the more resolution it has.
\end{quote}

\hypertarget{what-are-parameters}{%
\subsubsection{What are Parameters}\label{what-are-parameters}}

Parameters are any features of a device that may be manipulated and stored in an object. Examples of parameters would be:

\begin{itemize}
\item
  Dimmer
\item
  Pan/Tilt
\item
  Cyan/Yellow/Magenta
\item
  Zoom
\item
  Gobo 1
\item
  Blade Rotate
\end{itemize}

In Vibe parameters are organized under six standard \textbf{Banks} and one special Bank:

\begin{enumerate}
\def\labelenumi{\arabic{enumi}.}
\item
  Intensity
\item
  Position
\item
  Color
\item
  Beam
\item
  Image
\item
  Shape
\item
  Control - A special bank as its parameters are not stored in cues.
  Instead it sends ``real time'' messages to fixtures control functions.
\item
  Media
\end{enumerate}

The switch between banks can be done either from the touch screen or through the combination SHIFT + \# where the \# represents the bank number.

Parameter values may be set using either Vibe's intuitive \textbf{Smart Screen} picker system or conventional parameter wheels and keypad method.

\hypertarget{virtual-parameters}{%
\subsubsection{Virtual Parameters}\label{virtual-parameters}}

See:

\begin{itemize}
\tightlist
\item
  \href{https://vibemanual.compulite.com/programming-concepts.html\#operational-philosophy}{8.1. Virtual Parameters}
\end{itemize}

\hypertarget{smart-screen-1}{%
\subsubsection{Smart Screen}\label{smart-screen-1}}

The 11.6'' multi-touch monitor directly above the 4 encoder wheels is referred to as the \textbf{Smart Screen.}

The \textbf{Smart Screen} is dedicated to displaying:

\begin{itemize}
\item
  Context sensitive interactive bank, and parameter information.
\item
  Encoder wheel displays.
\item
  Playback information for the controllers that are directly above it.
\item
  Special pop-ups for Libraries, Store options and Grab functions.
\item
  A simplified Effects editor.
\item
  Displays for Fan feature.
\end{itemize}

\textbf{The Smart Screen is divided into six sections:}

Smart Screen - Intensity Bank

\includegraphics{https://files.gitbook.com/v0/b/gitbook-x-prod.appspot.com/o/spaces\%2F3kS90tLsADGm1ocbe7q9\%2Fuploads\%2F188nxyEpMFr3MU2W1LEI\%2F9.3.4.webp?alt=media\&token=194df2d9-fd6e-4a6b-b12b-5a65490fcb0a}

The Context sensitive area changes depending on the bank and fixture type. Where possible, similar parameters from different fixture types will be presented to the user using Vibe's universal fixture model. This means that is not necessary to memorize the manufacturer's specification of each fixture, as all fixtures are programmed in a similar consistent way.

Smart Screen - Position Bank

\includegraphics{https://files.gitbook.com/v0/b/gitbook-x-prod.appspot.com/o/spaces\%2F3kS90tLsADGm1ocbe7q9\%2Fuploads\%2F3fe22ksDN8BgJyU2QVcK\%2F9.3.4.1.webp?alt=media\&token=46d06b06-759b-4bda-9b64-6cc16ae1abb5}

Smart Screen - Color Bank

\includegraphics{https://files.gitbook.com/v0/b/gitbook-x-prod.appspot.com/o/spaces\%2F3kS90tLsADGm1ocbe7q9\%2Fuploads\%2FNN3kRu2VbdAgDsxCJQtE\%2F9.3.4\%20smart\%20screen\%20-\%20color\%20bank.png?alt=media\&token=6d819c90-c204-4edd-9731-4432337e32ec}

Smart Screen - Beam Bank

\includegraphics{https://files.gitbook.com/v0/b/gitbook-x-prod.appspot.com/o/spaces\%2F3kS90tLsADGm1ocbe7q9\%2Fuploads\%2FVVhIXwirfKvdmFyImJQi\%2F9.3.4.1.png?alt=media\&token=c544d391-1994-41e0-bab8-5758ef3a4cce}

Smart Screen - Image Bank

\includegraphics{https://files.gitbook.com/v0/b/gitbook-x-prod.appspot.com/o/spaces\%2F3kS90tLsADGm1ocbe7q9\%2Fuploads\%2FuyC5g599gLzYJcAYWgU3\%2F9.3.4.2.png?alt=media\&token=40cee69d-bc7f-433a-b992-b123ee53b476}

Smart Screen - Shape Bank

\includegraphics{https://files.gitbook.com/v0/b/gitbook-x-prod.appspot.com/o/spaces\%2F3kS90tLsADGm1ocbe7q9\%2Fuploads\%2FpBKNZb06EnOt8VsQf87s\%2F9.3.4.3.png?alt=media\&token=a78d7417-2ce5-4efa-8970-31d64ed3179b}

Smart Screen - Control

\includegraphics{https://files.gitbook.com/v0/b/gitbook-x-prod.appspot.com/o/spaces\%2F3kS90tLsADGm1ocbe7q9\%2Fuploads\%2FV9SuNwD08jogFABBHvBw\%2F9.3.4.6.webp?alt=media\&token=f11d9d9f-6df7-4f09-bcba-9006ec853af6}

\textbf{Smart Screen Preferences:}

\begin{itemize}
\item
  \textbf{Settings:}

  \begin{itemize}
  \item
    Home - Returns menus to Settings level.
  \item
    Format - Sets the view format for wheel displays.

    \begin{itemize}
    \item
      Percent
    \item
      Decimal
    \item
      Text
    \item
      Text + Percent
    \end{itemize}
  \item
    Resolution - Sets the Global wheel resolution. This may also be set by toggling the {[}Res{]} key over the trackball

    \begin{itemize}
    \item
      8 bit - Blue on {[}Res{]} Key
    \item
      10 bit - Green on {[}Res{]} key
    \item
      12 bit - Red on {[}Res{]} key
    \item
      16 bit - Yellow on {[}Res{]} key
    \item
      24 bit - Magenta on {[}Res{]} key
    \end{itemize}
  \item
    View - Sets behaviour of smart screen views.

    \begin{itemize}
    \item
      Auto - Shows programming display when there are fixtures in the editor, and shows the controller displays for Aux Qkeys and Global Sliders when the editor is cleared.
    \item
      Programming - Maintains the programming display even with the editor released.
    \item
      Playbacks - Maintains the playback controller display even with fixtures in the editor. Wheels and wheel displays remain accessible.
    \end{itemize}
  \item
    Wheel Picker Display Modes:

    \begin{itemize}
    \item
      Smart - All wheels for the selected fixture are shown in the interactive wheel picker area.
    \item
      Parameter Steps - Pressing a parameter wheel or tapping the wheel display above it will display a picker list of all feature steps for the selected fixture.
    \item
      Extra Wheels - Displays wheel information for the additional context sensitive encoder wheels
    \end{itemize}
  \end{itemize}
\end{itemize}

Smart Screen Effects Editor - See: \href{https://vibemanual.compulite.com/effects-1.html\#smart-screen-effects-editor}{13.2. Smart Screen Effects Editor}

\hypertarget{adjusting-parameters-using-the-wheel-picker}{%
\subsubsection{Adjusting Parameters using the Wheel Picker}\label{adjusting-parameters-using-the-wheel-picker}}

\includegraphics{https://files.gitbook.com/v0/b/gitbook-x-prod.appspot.com/o/spaces\%2F3kS90tLsADGm1ocbe7q9\%2Fuploads\%2FYRrblpqrjOFnaCqFguQy\%2F9.3.5.webp?alt=media\&token=35ce4615-8aa3-4192-a5c6-7db4e70d41c8}

\textbf{Smart Screen Picker Programming}

\textbf{To adjust fixture wheel parameter values:} (Interactive Wheel Picker method)

\begin{enumerate}
\def\labelenumi{\arabic{enumi}.}
\item
  Select the fixtures to program
\item
  Swipe or Drag the \textbf{Wheel Picker} on the right-hand side of the \textbf{Smart Screen} and browse to the wheel and slot you want to program. Common wheels are:
\end{enumerate}

\begin{itemize}
\item
  Color Wheel
\item
  Static Gobo Wheel
\item
  Rotating Gobo Wheel
\item
  Animation Wheel
\item
  Prism Wheel
\end{itemize}

\begin{enumerate}
\def\labelenumi{\arabic{enumi}.}
\setcounter{enumi}{2}
\item
  Tap the wheel slot you wish to program and the context sensitive area or the \textbf{Smart Screen} will open with the appropriate picker.
\item
  Adjust the virtual controls as needed.
\end{enumerate}

\begin{quote}
Most virtual controls are also mapped to the 4 physical wheels
\end{quote}

\textbf{To adjust parameter values:} (standard programming method)

\begin{enumerate}
\def\labelenumi{\arabic{enumi}.}
\item
  Select the fixtures to program.
\item
  Tap the \textbf{\{BANK\}} key of the parameter you wish to adjust.
\item
  Banks may contain more than 4 wheels of parameters. Tapping the desired \textbf{\{BANK\}} key again, pages to the next set of 4 parameter wheels.
\item
  The Assigned parameter wheel function is displayed on the \textbf{Wheel Display} at the bottom of the Smart Screen directly above the wheels.
\item
  Rotate the large push wheel encoder of the parameter you wish to adjust.
\item
  Values may be added directly from the keypad by:
\end{enumerate}

\begin{itemize}
\item
  Pushing a parameter wheel downward to select the parameter.
\item
  \textbf{AT} will appear on the command line
\item
  Type a value from Zero ⟶ Full.
\end{itemize}

\begin{quote}
{@ is only used for the dimmer parameter}.
\end{quote}

\begin{quote}
{A Parameter value must be entered in 2 digits or one digit and the \textbf{{[}Enter{]}} key for whole numbers. E.g. 05, 50, 5 \textbf{{[}ENTER{]}} =50}.
\end{quote}

\hypertarget{grab}{%
\subsubsection{Grab}\label{grab}}

Grab is a powerful tool for selecting fixtures that meet a number of conditions. By default \protect\hyperlink{grab}{GRAB} selects all fixtures with intensity or virtual intensity values above Zero \textbf{Pressing \protect\hyperlink{grab}{GRAB} without a selection opens up a pop-up on the Smart Screen. It contains the following options:}

\begin{itemize}
\item
  \textbf{\{Filters\}:}

  \begin{itemize}
  \item
    \{Stage\} - Select anything that is live on the stage and meets the conditions below.
  \item
    \{Editor\} -Select anything that is in the Editor and meets the conditions below.
  \item
    \{Master Controller\} (Playback) - Selects only fixtures on the Master Controller that meet the conditions.
  \end{itemize}
\item
  \textbf{\{Conditions\}:}

  \begin{itemize}
  \item
    \{All Fixtures\} - Selects all fixtures that have any changes via the editor.
  \item
    \{Active\} - fixture has intensity value over zero.
  \item
    \{Invisible\} - recorded ``hard'' parameter values for fixtures that do NOT have an intensity value over zero.
  \item
    \{Inactive\} (fixtures that have NO recorded values).
  \end{itemize}
\item
  \textbf{\{Parameter Selection\}}:

  \begin{itemize}
  \item
    \{All for Selected\} Grabs all parameters for selected fixtures (shortcut is {[}VIBE{]}+\protect\hyperlink{grab}{GRAB})
  \item
    \{Active (ENTER)\} Select all ``Active'' fixtures and enters them into the Editor. (Shortcut is \protect\hyperlink{grab}{GRAB} + {[}ENTER{]})
  \item
    {\{Search \& Replace\}}.
  \end{itemize}
\item
  {[}SET{]} \protect\hyperlink{grab}{GRAB} selects all active fixtures filtered by set (Fixture/Channel/Spot/Matrix/Server/Custom)
\end{itemize}

\textbf{To Selectively Grab a Controller:}

\begin{enumerate}
\def\labelenumi{\arabic{enumi}.}
\item
  \textbf{Press \protect\hyperlink{grab}{GRAB}} - Grab Options pop-up will open
\item
  \textbf{Press any Slider/Qkey/Aux Qkey controller button} - All fixtures that are in the Controller's active cue and have dimmer levels above Zero (or meet the Grab conditions selected in the pop-up), will be selected.
\end{enumerate}

\begin{quote}
Pressing ENTER will put the selection in the EDITOR.
\end{quote}

\textbf{Grab Fixtures in a cue based on a \{Library\} SK selection:}

\begin{enumerate}
\def\labelenumi{\arabic{enumi}.}
\item
  Press \protect\hyperlink{grab}{GRAB} tap a \{Library\} SK - All fixtures with matching libraries will be selected
\item
  Tap another \{Library\} SK - The Library will be swapped to the new library
\end{enumerate}

\begin{quote}
This is especially useful for quickly swapping colors in cues containing libraries
\end{quote}

\textbf{Select all parameters for a fixture selection:}

\begin{enumerate}
\def\labelenumi{\arabic{enumi}.}
\item
  Select the fixtures
\item
  Press {[}VIBE{]}+\protect\hyperlink{grab}{GRAB}
\end{enumerate}

\textbf{Select all Parameters with hard and tracked values outputting to the stage:}

\begin{enumerate}
\def\labelenumi{\arabic{enumi}.}
\tightlist
\item
  Press \protect\hyperlink{grab}{GRAB} {[}ENTER{]}
\end{enumerate}

\hypertarget{fixture-selection-detailed}{%
\subsubsection{Fixture Selection (Detailed)}\label{fixture-selection-detailed}}

\begin{quote}
FIn this guide, FIXTURE will be the default SET but any SET (Channel, Spot, Matrix, Server, User) should also be applicable.
\end{quote}

\begin{quote}
Default SET is changed by double pressing the appropriate SET key
\end{quote}

\begin{quote}
Dimmer is implied, but parameter would normally be specified after {[}+ ⟶ -{]} and before value.
\end{quote}

\begin{quote}
\textbf{{[}CE{]}} will backspace the command line one character at a time.
{If more than one state is involved, the state will backspace right up until the idle state}.
\end{quote}

\begin{quote}
\textbf{{[}Shift CE{]}} will clear the command line to the beginning and put the system in an idle state.
\end{quote}

\begin{itemize}
\item
  {[}Fixture{]} \# {[}+ ⟶ -{]} {[}{[}Full{]}/{[}Zero{]} (@100 and @00 are valid if \% or text display format).
\item
  Fixture{]} \# {[}+ ⟶ -{]} {[}@{]} enter value. Two digits are required to complete value entry.
\item
  {[}Fixture{]} \# {[}+ ⟶ -{]} {[}@{]} {[}@{]} will set values to Full.
\item
  {[}Fixture{]} \# {[}+ ⟶ -{]} {[}@{]} single value, {[}ENTER{]}. E.g. {[}Fixture{]} {[}1⟶ 6{]} {[}@{]} {[}5{]} {[}ENTER{]} will set a value of 50\%. {[}@{]} {[}7{]} {[}ENTER{]} would set a value of 70\%.
\item
  Pressing {[}ENTER{]} at the end of value input is optional and does not affect the sequence. In most cases, it will simply terminate the command line.
\item
  {[}Fixture{]} \# will jump the \textbf{live display} screen to the specified fixture if the fixture is not visible on the screen. (Fixture Jumping must be enabled in the \textbf{Live Display, \{Tools\}} (Wrench icon)Pop-up under the \textbf{Behavior} Tab
\item
  {[}Fixture{]} \# {[}+ ⟶ -{]} {[}@{]} {[}+{]} {[}value{]} or {[}@{]} {[}-{]} {[}value{]} will make a \textbf{relative} change to the stage value. E.g. \{If the stage value is 50, @ + 10 will set the value to 60. @ - 30 would set the value to 20\}.
  The rules for single digit plus Enter will also apply.
\item
  {[}Fixture{]} \# {[}⟶ {]} \#, Parameter, {[}@{]} \{value{]} {[}⟶ {]} {[}value{]} will \textbf{fan} the range. (if no parameter is specified, dimmer/intensity will be fanned) 0, and 100\% are also valid as well as {[}Zero{]} and {[}Full{]}.
\item
  {[}Fixture{]} \# {[}⟶ {]} \# Parameter, {[}@{]} {[}value{]} {[}⟶ {]}{[}⟶ {]} {[}value{]} will \textbf{mirror} fan the range. \{1➡⟶ 100 would produce 0, 10, 20, 30, 40, 50, 40, 30, 20, 10, 0\}
\item
  Fixture \textbf{Dot} = Restore last. {Will have a toolbar equivalent.}
\item
  Fixture ⟶ = Reeelect Editor. {Will have a toolbar equivalent. }
\item
  Fixtures and Parameters may be deselected using {[}RELEASE/DE SEL{]} Key. This leaves values in the editor but not under the control of the wheels.
\item
  {[}GROUP{]} \# selects all fixtures that are recorded into the group. *The typed fixture order is preserved in a recorded group.
\item
  With groups, parameters may be fanned via syntax as above, but their recorded order will be considered.
\item
  Control without having to press Shift, Reset, which clears the values in the Editor.
\item
  \textbf{\{LIBRARY\}, \{LIBRARY\}}, will select and enter values from a library into the Editor. This is especially useful when using position libraries as groups with positions.
\end{itemize}

\hypertarget{setting-parameters-using-main-encoder-wheels}{%
\subsubsection{Setting Parameters using Main Encoder Wheels}\label{setting-parameters-using-main-encoder-wheels}}

Vibe features the \textbf{Smart Window} to simplify programming. Programming may also be done traditionally using \{Bank\} keys and the 4 large push wheels:

\textbf{Using parameter wheels:}

\begin{enumerate}
\def\labelenumi{\arabic{enumi}.}
\item
  Fixture \# {[}+ ⟶ -{]} Tap \{Parameter Bank\} on the Smart Screen.
\item
  If the required parameter is not shown on the encoder wheel display, tap the \{parameter Bank\} to scroll through all available parameter pages.
\item
  Turn the encoder wheel of the desired parameter to adjust values.
\end{enumerate}

\begin{quote}
{Parameters other than dimmer do not use the @ key.}
\end{quote}

\textbf{Direct entry of absolute values:}

\begin{enumerate}
\def\labelenumi{\arabic{enumi}.}
\item
  Fixture \# {[}+ ⟶ -{]} Tap \{Parameter Bank\} on Smart Screen.
\item
  Tap the \{Wheel Display\} directly above the physical parameter wheel you wish to enter a value for. The wheel display turns red.
\item
  Enter a value using the keypad.
\end{enumerate}

\textbf{Direct entry of values using push wheels:}

\begin{enumerate}
\def\labelenumi{\arabic{enumi}.}
\item
  Fixture \# {[}+ ⟶ -{]}, Tap \{Parameter Bank\} on Smart Screen.
\item
  Press physical wheel - Wheel display over the physical wheel turns red.
\item
  Enter value via keypad.
\end{enumerate}

\textbf{Select multiple parameters at the same time:}

\begin{enumerate}
\def\labelenumi{\arabic{enumi}.}
\item
  Fixture \# {[}+ ⟶ -{]} Tap \{Parameter Bank\} on the Smart Screen.
\item
  Toggle multiple parameters ON/OFF by tapping \{parameter Keys\}, pressing physical wheel, or toggling desired live display headers. Select all relevant fixture parameters in a parameter bank:
\end{enumerate}

\textbf{Select all relevant fixture parameters in a parameter bank:}

\begin{itemize}
\tightlist
\item
  Select Fixture \# {[}+ ⟶ -{]}, Hold {[}Vibe Key{]} and tap a \{Parameter Bank Key\}.
\end{itemize}

\begin{quote}
Pressing and holding a \{Parameter Bank\} will pop-up three options: \{HOME\}, \{ALL PARAMETERS\}, \{RELEASE ALL\} \{HOME\} - Set the selected parmeters to ``Home values''. \{All\} - Selects all relevant parameters in the parameter bank. \{RELEASE ALL\} - Releases the selected parameter from the Editor.
\end{quote}

\hypertarget{setting-parameters-using-interactive-encoder-wheels}{%
\subsubsection{Setting Parameters using Interactive Encoder Wheels}\label{setting-parameters-using-interactive-encoder-wheels}}

Vibe has 4 additional interactive push encoder wheels along the right hand side of the smart screen. The parameters assigned to the wheels are dependent on the bank and selected fixture type:

\textbf{E.g:}

\begin{itemize}
\item
  Image assigns Focus and Zoom to the interactive wheels
\item
  Color Mix assigns Color Wheel 1+2 to the interactive wheels
\item
  Beam assigns Prism and Prism Rotate to the interactive wheels
\item
  Shape assigns Focus and Zoom to the interactive wheels In future releases it will be possible for users to customize the defaults.
\end{itemize}

In future releases it will be possible for users to customize the defaults.

\includegraphics{https://files.gitbook.com/v0/b/gitbook-x-prod.appspot.com/o/spaces\%2F3kS90tLsADGm1ocbe7q9\%2Fuploads\%2FDGXNEIrHnemjvPSftsn9\%2F9.3.9.webp?alt=media\&token=26fc8f97-07b6-488f-a54e-2e0234b4f8a8}

\hypertarget{release-and-home-sequences}{%
\subsubsection{Release and Home Sequences}\label{release-and-home-sequences}}

\textbf{Release} - Removes fixtures or fixture parameter values from the editor

\textbf{Release all active fixtures:}

\begin{itemize}
\tightlist
\item
  Press \{RELEASE{]} key
\end{itemize}

\textbf{Release fixtures:}

\begin{itemize}
\tightlist
\item
  Fixture \# {[}+ ⟶ -{]} {[}RELEASE{]}
\end{itemize}

\textbf{Release a parameter:}

\begin{itemize}
\tightlist
\item
  Fixture \# {[}+ ⟶ -{]} tap \{Bank\} select \{Parameter\}, {[}RELEASE{]}
\end{itemize}

\textbf{Release a Parameter Bank:}

\begin{itemize}
\tightlist
\item
  Fixture \# {[}+ ⟶ -{]} {[}VIBE{]} \{Bank\} {[}RELEASE{]}
\end{itemize}

If a fixture is selected and then is made active (Red), the first press of release, releases the parameter and then the second press releases the whole fixture.

All devices are assigned home values for parameters based on the manufacturer's DMX chart. By default, these home values are used to tell the system where fixture parameters should be when they are not in use. Vibe has two options for home values. They can be set in the System Settings pop-up in the \{Editing Properties\} tab under the Default Stage Values header. System settings is accessed via the Vibe Menu or by tapping on the \{✱\} icon in the lower right corner of the main monitor controller display.

\begin{enumerate}
\def\labelenumi{\arabic{enumi}.}
\item
  \{Use Home values\} -Released parameters will be shown in grey and at their default home values.
\item
  \{Maintain Last Value\} - Released parameters will be at the last value they were at before being released.
\end{enumerate}

Selected fixtures and/or parameters may be sent to their defaults using the {[}HOME{]} key.
\textbf{Home} - Sets parameter values to home values as defined by the manufacturer or to custom values defined by the user \textbf{Home Scene}.

\textbf{Home works similar to Release:}

\begin{itemize}
\item
  First press ``homes'' the current parameter.
\item
  Second press ``homes'' all parameters for the selected fixtures.
\end{itemize}

\textbf{It is also possible to set a custom ``Home Scene'' that overrides the default Home:}

\begin{enumerate}
\def\labelenumi{\arabic{enumi}.}
\item
  Set the values in the Editor similar to creating a Cue or a Scene.
\item
  Press {[}SCENE{]} {[}HOME{]} {[}STORE{]}
\end{enumerate}

\textbf{To Delete a Home Scene:}

\begin{itemize}
\tightlist
\item
  Press {[}SCENE{]} {[}HOME{]} {[}DELETE{]}
\end{itemize}

\hypertarget{groups}{%
\subsection{Groups}\label{groups}}

\begin{quote}
{The order that fixtures are typed is stored with a group}.
\end{quote}

\textbf{Storing Groups to SKs:}

\begin{itemize}
\tightlist
\item
  {[}FIXTURE{]} {[}\#{]} {[}+ ⟶ -{]} {[}STORE{]} tap desired \{Group Softkey\} - it does not matter if levels are set, they will be ignored.
\end{itemize}

\textbf{Labelling SK Groups:}

\begin{itemize}
\item
  Directly after storing a group, while the focus is still on the group SK, start typing from the keyboard and a Text Entry pop-up will open.
\item
  At any time, pressing the {[}TEXT{]} key followed by tapping the SK will open the Text Entry pop-up.
\end{itemize}

\textbf{Storing Groups via keypad syntax:}

\begin{itemize}
\item
  {[}FIXTURE{]} {[}\#{]} {[}+ ⟶ -{]} {[}\#{]} {[}GROUP{]} {[}\#{]} {[}STORE{]}.
\item
  Pressing the {[}TEXT{]} key after any Group store will open the Text Entry popup.
\end{itemize}

\textbf{Modify Groups using Group Settings pop-up:}

\begin{itemize}
\item
  Using keypad - press {[}GROUP{]} {[}\#{]} {[}Settings{]} - the Group Settings pop-up will open. Range text labelling may be done via this method.
\item
  Using SKs - Tap the source \{GROUP\} SK and then press {[}Settings{]} key
\end{itemize}

\includegraphics{https://files.gitbook.com/v0/b/gitbook-x-prod.appspot.com/o/spaces\%2F3kS90tLsADGm1ocbe7q9\%2Fuploads\%2FTyeIUEfkmEZue9ub8uBx\%2F9.4.png?alt=media\&token=7d1ef45d-a1fe-4c4a-8f78-82154a49df5b}

Group Settings Pop-up

\begin{quote}
A new Group may be created from existing Groups by selecting multiple Groups using Group SKs or via keypad. Press {[}STORE{]} then tap destination \{Group\} SK or Press {[}Group{]} {[}\#{]} {[}STORE{]} to create the new Group. The new Group will not be linked to the original Groups.
\end{quote}

\textbf{Groups may be used to release fixtures:}

\begin{itemize}
\item
  With keypad - {[}GROUP{]} {[}\#{]} {[}RELEASE{]} (Currently only works if ENTER is pressed after group selection)
\item
  With SKs, \{Group \#\}, {[}Release{]}
\end{itemize}

\textbf{Release a fixture from a Group Method 1:}

\begin{itemize}
\item
  With softkeys: Fixture \# {[}+ ⟶ -{]}, {[}STORE{]}, tap \{Group SK\}, Group Efietel pop-up appears, select \{Release\}
\item
  With keypad: Fixture \# {[}+ ⟶ -{]}, press {[}GROUP{]}, \#, {[}STORE{]}, Group Efietel pop-up, appears, select \{Release\}
\end{itemize}

\textbf{Release a fixture from a Group Method 2:}

\begin{itemize}
\item
  {[}GROUP {[}\#{]} {[}EDIT{]} - {[}UPDATE{]} key flashes red ({[}EDIT{]} tap \{GROUP SK\} is also valid)
\item
  {[}FIXTURE{]} {[}\#{]} {[}RELEASE{]}
\item
  Press {[}UPDATE{]} to complete the sequence
\end{itemize}

\textbf{To Delete a Group:}

\begin{itemize}
\item
  {[}GROUP{]} \# {[}DELETE{]}
\item
  Tap \{GROUP\} {[}\#{]} {[}DELETE{]}
\end{itemize}

\hypertarget{programming-cues-and-scenes}{%
\section{Programming Cues, and Scenes}\label{programming-cues-and-scenes}}

This chapter deals with storing and editing Editor values.

\textbf{The following is covered in this chapter:}

\begin{itemize}
\tightlist
\item
  \href{https://vibemanual.compulite.com/programming-cues-and-scenes.html\#what-are-qlists-cues-scenes-and-submasters}{10.1. What Are Qlists, Cues, Scenes, and Submasters?}
\item
  \href{https://vibemanual.compulite.com/programming-cues-and-scenes.html\#tracking-principals}{10.2. Tracking Principals}
\item
  \href{https://vibemanual.compulite.com/programming-cues-and-scenes.html\#storing-cues-and-scenes-directly-to-controllers}{10.3. Storing Cues and Scenes Directly to Controllers}
\item
  \href{https://vibemanual.compulite.com/programming-cues-and-scenes.html\#master-controller}{10.4. Master Controller}
\item
  \href{https://vibemanual.compulite.com/programming-cues-and-scenes.html\#storing-cues-to-the-master-controller}{10.5. Storing Cues to the Master Controller}
\item
  \href{https://vibemanual.compulite.com/programming-cues-and-scenes.html\#storing-qlist-cues-to-any-controller}{10.6. Storing Qlist Cues to any Controller}
\item
  \href{https://vibemanual.compulite.com/programming-cues-and-scenes.html\#cue-store-options-1}{10.7. Cue Store Options}
\item
  \href{https://vibemanual.compulite.com/programming-cues-and-scenes.html\#assigning-text-labels}{10.8. Assigning Text Labels}
\item
  \href{https://vibemanual.compulite.com/programming-cues-and-scenes.html\#modifying-cues-and-scenes}{10.9. Modifying Cues and Scenes}
\item
  \href{https://vibemanual.compulite.com/programming-cues-and-scenes.html\#creating-new-cues-from-existing-cue-data}{10.10. Creating New Cues from Existing Cue Data}
\item
  \href{https://vibemanual.compulite.com/programming-cues-and-scenes.html\#cue-time-properties}{10.11. Cue Time Properties}
\item
  \href{https://vibemanual.compulite.com/programming-cues-and-scenes.html\#additional-cue-properties}{10.12. Additional Cue Properties}
\item
  \href{https://vibemanual.compulite.com/programming-cues-and-scenes.html\#parameter-time}{10.13. Parameter Time}
\item
  \href{https://vibemanual.compulite.com/programming-cues-and-scenes.html\#parameter-profile}{10.14. Parameter Profile}
\item
  \href{https://vibemanual.compulite.com/programming-cues-and-scenes.html\#group-submasters}{10.15. Group Submasters}
\end{itemize}

\hypertarget{what-are-qlists-cues-scenes-and-submasters}{%
\subsection{What Are Qlists, Cues, Scenes, and Submasters?}\label{what-are-qlists-cues-scenes-and-submasters}}

\textbf{Qlists:}

\begin{itemize}
\item
  Qlists are containers that hold cues. They might metaphorically be thought of as file folders.
\item
  Each Qlist has its own unique Cue numbering system.
\item
  Qlists currently may contain up to 8,000 cues.
\item
  Qlists containing Cues must be assigned to Controllers before they can be output to the stage.
\end{itemize}

\textbf{Cues:}

\begin{itemize}
\item
  Cues are stored containers of editor values. They might metaphorically be thought of as files or records.
\item
  Cue may only exist within Qlists.
\item
  Cues may be numbered from 0.001 ⟶ 999.999
\item
  Cues cannot be stored without a Qlist destination.
\item
  \textbf{Cues may be assigned:}

  \begin{itemize}
  \item
    In Time
  \item
    Out Time
  \item
    Delay In Time
  \item
    Delay Out Time
  \item
    Text Label
  \end{itemize}
\end{itemize}

\textbf{Scenes:}

\begin{itemize}
\item
  Scenes are essentially a QList and a Cue combined.
\item
  Scenes may only contain a single ``Look'' of editor values.
\item
  \textbf{Scenes assigned to Slider Controllers are given the default settings of:}

  \begin{itemize}
  \item
    \{GO + Jump Parameters\}
  \item
    \{Release At Bottom\}
  \end{itemize}
\item
  Scenes may only be assigned In-TIme.
\item
  Scenes may be given Text Labels
\end{itemize}

Submasters:
- A special type of Scene where stored fixtures \textbf{inhibit} their dimmer values proportionate to their assigned slider controller's value. All other parameter values are ignored.

\begin{itemize}
\tightlist
\item
  Groups may directly be assigned to controllers to create Group Submasters. \textbf{Group Submasters} reference the original Groups and update automatically if the Groups are modified.
\end{itemize}

\hypertarget{tracking-principals}{%
\subsection{Tracking Principals}\label{tracking-principals}}

Vibe uses the \textbf{Tracking/Move Fade} philosophy common in most current lighting consoles. In tracking consoles, ``Hard Values'' (parameter values with stored information) track forward from cue to cue until new hard values are encountered.

\textbf{If we think of cues and values as a spreadsheet, basic tracking systems have two kinds of parameter value cells:}

\begin{itemize}
\item
  ``Hard Value'' Cells
\item
  ``Tracking Cells'' get their values by ``seeing above'' to the original hard values. They are basically transparent and only hard values are stored for a Cue. This is more efficient.
\end{itemize}

\textbf{Storing Cue information on a cell basis has other advantages:}

\begin{itemize}
\item
  Each parameter cell can have its own individual time which overrides overall cue time. This allows, for example, Pan/Tilt to have a separate transition time or Color Wheels to snap to their values instead of using overall cue time.
\item
  As Vibe's default behaviour for parameters it \textbf{LTP} (The last action takes precedent), a cell assigned its own time will continue to run its time until either the stored time runs out or it encounters a new hard value.
\end{itemize}

\begin{quote}
Some traditional theatre consoles call ``Hard Values'' ``Move Instructions'' and cells that are in transition, ``Move Fades''.
\end{quote}

\begin{itemize}
\tightlist
\item
  Another advantage of tracking is that if you have values that are tracking through a number of cues, all you have to do is change the original hard values and that information will change down to the next hard value. You do not have to copy the change to each cue.
\end{itemize}

\textbf{An issue with tracking is that making a modification to values in one of the cues in the tracked range will change the values in all the cues following it. To solve this problem, Vibe offers two options:}

\begin{enumerate}
\def\labelenumi{\arabic{enumi}.}
\tightlist
\item
  \textbf{Qonly}
\end{enumerate}

\begin{itemize}
\item
  \textbf{\{Qonly\}} - Used in any case where \textbf{cue modification, cue insertion, or cue deletion} must affect \textbf{``This Cue Only''}.
\item
  The look of the cue following any of the above operations will maintain the appearance it had before the operation. This is done by automatically copying the hard values that made up the original look (stage state) and pasting them into the cue following the modification. The downside is that there is no longer a tracking link from the original hard values even though the values match the original values.
\end{itemize}

\begin{enumerate}
\def\labelenumi{\arabic{enumi}.}
\setcounter{enumi}{1}
\tightlist
\item
  \textbf{Skip Track}
\end{enumerate}

\begin{itemize}
\item
  \textbf{\{Skip\}} - Stores modified values as a special kind of cell that acts like a normal \textbf{Tracking Cell} until the \textbf{Skip Cell} is reached. The \textbf{Skip Cell} then behaves as if it has hard values and outputs its value to the stage. When the Qlist is advanced past a cue with the \textbf{Skip Cells}, they again behave as tracking cells, thus restoring the link to the original hard values.
\item
  Storing a Cue modification using \textbf{Skip}, has the same apparent result as storing \textbf{Qonly}. The main difference is that with \textbf{Qonly}, the link to the original hard values is lost, and with \textbf{Skip}, the link is preserved.
\item
  To make it easier for users to identify cues with \textbf{Skip} Cells, the parameter value with the \textbf{Skip} will be displayed in {Magenta}.
\end{itemize}

\textbf{Tracking examples:}

\includegraphics{https://files.gitbook.com/v0/b/gitbook-x-prod.appspot.com/o/spaces\%2F3kS90tLsADGm1ocbe7q9\%2Fuploads\%2FUNCsyi8JpsI7oeqZi6ZK\%2F10.2.0.webp?alt=media\&token=d2092292-5f80-49d7-bcf0-59e42d4535e6}

\includegraphics{https://files.gitbook.com/v0/b/gitbook-x-prod.appspot.com/o/spaces\%2F3kS90tLsADGm1ocbe7q9\%2Fuploads\%2FZvSTEeRxeentEcii0LHv\%2F10.2.webp?alt=media\&token=42722a5a-048d-4b79-a9fd-5e6813fdd5b6}

\includegraphics{https://files.gitbook.com/v0/b/gitbook-x-prod.appspot.com/o/spaces\%2F3kS90tLsADGm1ocbe7q9\%2Fuploads\%2FMizBgP9oiqcr83efFqfN\%2F10.2.1.webp?alt=media\&token=2f61ddfa-a902-4a3b-9078-a5edcee9e13c}

\includegraphics{https://files.gitbook.com/v0/b/gitbook-x-prod.appspot.com/o/spaces\%2F3kS90tLsADGm1ocbe7q9\%2Fuploads\%2F0kaT0pjfbcv2rVuFoaPu\%2F10.2.2.webp?alt=media\&token=24b18404-77dc-4b25-93c4-ad2b4c3d0a59}

\hypertarget{storing-cues-and-scenes-directly-to-controllers}{%
\subsection{Storing Cues and Scenes Directly to Controllers}\label{storing-cues-and-scenes-directly-to-controllers}}

\textbf{Cues and Scenes may be stored:}

\begin{itemize}
\item
  Directly to the Master Controller using command line syntax.
\item
  Directly to a specified Qlist and Cue using syntax.
\item
  Directly to a controller.
\end{itemize}

\textbf{To store the first cue directly to a controller:}

\begin{itemize}
\item
  Set parameter levels in the Editor then press {[}STORE{]} {[}HERE{]} (to \textbf{any} of the buttons of a controller). The first available Qlist will be assigned to the controller and Cue 1 will be created.
\item
  Alternately a Cue \# may be specified in advance: {[}STORE{]} {[}CUE{]} {[}3{]} {[}HERE{]} (press \textbf{any} of the buttons of the destination controller).
\item
  A Qlist \# may also be specified in advance: {[}STORE{]} {[}QLIST{]} {[}3{]} {[}CUE{]} {[}1{]} {[}HERE{]} (press \textbf{any} of the buttons of the destination controller).
\end{itemize}

\textbf{To append additional cues to a controller:}

\begin{itemize}
\tightlist
\item
  Set parameter values in the Editor then press the {[}STORE{]} {[}HERE{]} (press \textbf{any} of the buttons of a controller that already has cues stored on it). A cue will be appended to the first available whole number of the Qlist
\end{itemize}

\textbf{To Store a Scene to a controller:}

\begin{itemize}
\item
  Set parameter levels in the Editor then press {[}STORE{]} {[}SCENE{]} {[}HERE{]} (press \textbf{any} of the buttons of the destination controller). The first available Scene \# will be assigned to the controller. No Qlist will be assigned.
\item
  Alternately a Scene \# may be specified in advance. {[}STORE{]} {[}SCENE{]} {[}3{]} {[}HERE{]} (press \textbf{any} of the buttons of the destination controller).
\end{itemize}

\textbf{To convert a Scene to a Qlist:}

\begin{enumerate}
\def\labelenumi{\arabic{enumi}.}
\item
  Press {[}STORE{]} {[}HERE{]} to an existing Scene.
\item
  The Store to Controller pop-up will appear. Tap \{Create a new Qlist\} the Scene will be assigned the first available Qlist with the Scene's look being converted to Cue 1.
  \includegraphics{https://files.gitbook.com/v0/b/gitbook-x-prod.appspot.com/o/spaces\%2F3kS90tLsADGm1ocbe7q9\%2Fuploads\%2FqjIYoECGpKkw2eosYT1W\%2F10.3.webp?alt=media\&token=92e2fb56-797e-4596-869b-0e1c042b9e01}
\end{enumerate}

\hypertarget{master-controller}{%
\subsection{Master Controller}\label{master-controller}}

In Vibe, at least one controller must be assigned as the {[}SELECT{]} controller. This controller will now be linked to the Master Controller.

\textbf{The Master Controller has 4 large buttons associated with it:}

\begin{enumerate}
\def\labelenumi{\arabic{enumi}.}
\item
  {[}GO{]} - Initiates a forward timed fade between cues in the selected Qlist.
\item
  {[}BACK{]} - Backs up one cue at a time using system timing properties' \textbf{Back Time}.
\item
  {[}HOLD{]} - Pauses the progress of a fade
\item
  {[}GOTO{]} - Advances to the specified cue in system timing properties' \textbf{GOTO Time}.
\end{enumerate}

\includegraphics{https://files.gitbook.com/v0/b/gitbook-x-prod.appspot.com/o/spaces\%2F3kS90tLsADGm1ocbe7q9\%2Fuploads\%2FBHxc5Q4Nsp0bSvXoqXgY\%2F10.4.webp?alt=media\&token=58f1929e-1d7d-4e60-b83f-94cc9678a3cc}

\textbf{Associated with but not limited to the controller assigned as the {[}SELECT{]} Controller are the following 3 buttons:}

\begin{enumerate}
\def\labelenumi{\arabic{enumi}.}
\item
  {[}Load{]} - Preloads a controller with a specified cue \# but does not advance to the cue. Pressing {[}GO{]} will advance to the preloaded cue using cue time.
\item
  {[}Release{]} - Releases the destination controller from affecting the stage (Turns it off).
\item
  {[}Free{]} - Unloads the controller.
\end{enumerate}

\begin{quote}
{[}Load{]} {[}Release{]} and {[}Free{]} do not work with the big Master Controller buttons, only with any button on the controller that is associated with it as the {[}SELECT{]} controller.
\end{quote}

By default cues that do not have a Qlist or Controller specified in their destination will be stored to the controller selected as the Master Controller.

\hypertarget{storing-cues-to-the-master-controller}{%
\subsection{Storing Cues to the Master Controller}\label{storing-cues-to-the-master-controller}}

\begin{quote}
A Qlist must be assigned to the {[}SELECT{]} controller before cues can be stored using theatre style keypad syntax
\end{quote}

\begin{quote}
A Qlist does not have to contain cues to be assigned to a controller
\end{quote}

\textbf{To assign a Qlist to a controller and make it the Master Controller:}

\begin{enumerate}
\def\labelenumi{\arabic{enumi}.}
\item
  Press {[}QLIST{]} {[}\#{]} {[}HERE{]} (press \textbf{any} of the buttons of the destination controller) - The Qlist will be assigned to the controller.
\item
  Press {[}SELECT{]} {[}HERE{]} (press any of the buttons of the destination controller) - The destination controller will now be the Master Controller and the {[}Second Button{]} of the ``Selected'' controller will turn dark blue to indicate that it is now assigned to the Master Controller. The top header of the Controller Display on the main monitor above the controller will also turn {dark blue}.
\end{enumerate}

\textbf{To Store a cue to the Master Controller:}

\begin{enumerate}
\def\labelenumi{\arabic{enumi}.}
\item
  Set Parameter levels in the Editor the press {[}CUE{]} {[}\#{]} {[}STORE{]}. The cue will be added to the Master Controller's assigned Qlist.
\item
  Valid Cue numbers per Qlist = 000.001 → 999.999
\item
  Addition cues may be stored via keypad syntax or the Direct Store method.
\item
  Pressing {[}STORE{]} {[}STORE{]} or {[}STORE{]} {[}ENTER{]} will store editor values to the next whole number cue at the end of a Qlist. (Next spacing can be customized in system settings).
\end{enumerate}

\begin{quote}
Vibe uses the ``Clear Editor After Store'' behaviour common in most theatre consoles.
{The Controller will by default advance to the stored cue and clear the Editor}. This behaviour may be disabled in System Settings.
\end{quote}

\hypertarget{storing-qlist-cues-to-any-controller}{%
\subsection{Storing Qlist Cues to any Controller}\label{storing-qlist-cues-to-any-controller}}

\begin{quote}
{The destination Qlist must be assigned to a controller in advance or after storing the first Cue, the destination Qlist will bump the current cue on the Master Controller out and replace it.}.
\end{quote}

\textbf{To store a Cue to a Qlist that is already assigned to a Controller:} (Usually not the Master Controller)

\begin{enumerate}
\def\labelenumi{\arabic{enumi}.}
\item
  Press {[}Qlist{]} {[}\#{]} {[}HERE{]} to any button of the destination controller. The Qlist will be assigned to the controller
\item
  Set parameter levels in the Editor
\item
  Press {[}Qlist{]} {[}\#{]} {[}CUE{]} {[}\#{]} {[}STORE{]}
\item
  Valid Cue numbers per Qlist = 000.001 → 999.999
\item
  Addition cues may be stored via keypad syntax specifying the Qlist \# in advance or \href{https://vibemanual.compulite.com/programming-cues-and-scenes.html\#storing-cues-and-scenes-directly-to-controllers}{10.3. Storing Cues and Scenes Directly to Controllers}\hspace{0pt}
\end{enumerate}

\hypertarget{cue-store-options-1}{%
\subsection{Cue Store Options}\label{cue-store-options-1}}

In the process of storing a cue, a number of filters and settings are available.

\textbf{To open the Store Options pop-up, two methods are available:}

\begin{itemize}
\tightlist
\item
  \textbf{Store cues using keypad syntax:}
\end{itemize}

\begin{enumerate}
\def\labelenumi{\arabic{enumi}.}
\item
  Set values in the Editor.
\item
  {[}CUE{]} {[}\#{]} {[}OPTIONS{]} Cue Store Options pop-up opens. Make selections and press {[}STORE{]}.
\end{enumerate}

\begin{itemize}
\tightlist
\item
  \textbf{Store directly to a Qlist on a Controller:}
\end{itemize}

\begin{enumerate}
\def\labelenumi{\arabic{enumi}.}
\item
  Set Values in the Editor.
\item
  Press {[}STORE{]} {[}OPTIONS{]} \textbf{Cue Options} pop-up opens. Make selections and press {[}HERE{]} to any button on the destination controller.
\end{enumerate}

\includegraphics{https://files.gitbook.com/v0/b/gitbook-x-prod.appspot.com/o/spaces\%2F3kS90tLsADGm1ocbe7q9\%2Fuploads\%2FknletXzqZk4W9tZ8qBWq\%2F10.7\%20cue\%20store\%20options.png?alt=media\&token=d07a452b-3958-4a12-a42a-abad08bdaa5c}

Cue Store Options Pop-up

\textbf{Options are:}

\begin{itemize}
\tightlist
\item
  Editor as Source:

  \begin{itemize}
  \tightlist
  \item
    \{Red + Grey\} Editor Values in 100\% 90\% will be stored.
  \item
    \{Store Stage\} All fixtures outputting ANY DMX from the console will be stored. This includes tracked values, home values and dimmer values at zero.
  \item
    \{Red Only\} Only Editor values in Red will be stored. This can be a useful way to filter parameters.
  \item
    \{All Param For Selected\} All parameters from the fixture selection will be turned red and stored in the cue.
  \item
    \{All Param If Active\} All parameters of all fixtures in the editor will be stored if the dimmer parameter is greater than zero (WYSIWYG).
  \end{itemize}
\item
  Additional:

  \begin{itemize}
  \tightlist
  \item
    {\{DMX In\} Takes a snapshot of incoming DMX and attaches it to the cue. }
  \item
    \{Force Black\} When it is not desirable to see values move from one cue state to another, Force Black may be used to save programming time. In a cue tagged as Force Black, the dimmer parameter is first ``forced'' to Black Out, parameters they move in black, then the dimmer parameter is faded up again. The total speed of a Force Black operation is relative to the overall cue time.
  \end{itemize}
\item
  Filtering:

  \begin{itemize}
  \tightlist
  \item
    \{Show Filters\} Opens user prebuilt filters in the pop-up.
  \end{itemize}
\item
  Source behaviour when stored:

  \begin{itemize}
  \tightlist
  \item
    \{Track\} Normal system behaviour
  \item
    \{Skip Cue\} - {Tags a cue as a skip cue in the case of inserting a new cue. In the case of updating a cue by storing to an existing cue, only the values being updated will be made into Skip cells. }.
  \item
    {\{Qonly\} Inserted cues will not affect the look of cues following them}
  \end{itemize}
\item
  Look Ahead: (sometimes referred to as Mark Cue) \href{https://vibemanual.compulite.com/dark-parameter-positioning.html\#force-black}{See: 15.3. Force Black}

  \begin{itemize}
  \tightlist
  \item
    Unlike having to Look Ahead automatically preposition all dark parameters for all cues in a Qlist, a cue tagged in \textbf{Cue Store Options} as \{Look Ahead\} will preposition dark parameters for just the tagged cue.
  \end{itemize}
\item
  Force Black - See: \href{https://vibemanual.compulite.com/dark-parameter-positioning.html\#look-ahead-move-in-dark}{15.1. Look Ahead (Move in Dark)}

  \begin{itemize}
  \tightlist
  \item
    Tagged a cue as \{Force Black\} in Cue Store Options, is the same as setting the cue as \{Force Black\} in \{Cue settings\}.
  \end{itemize}
\end{itemize}

\hypertarget{assigning-text-labels}{%
\subsection{Assigning Text Labels}\label{assigning-text-labels}}

\textbf{Assign a Text Label to a Qlist:}

\begin{itemize}
\item
  Method 1 - {[}QLIST{]} {[}\#{]} {[}TEXT{]} opens the Text Entry pop-up.
\item
  Method 2 - {[}QLIST {[}\#{]} {[}SETTINGS{]} opens the Qlist Properties tab or Qlist popup.
\item
  Method 3 - Press the Qlist header above a loaded Controller, The Qlist Properties pop-up opens - Enter Qlist Text in the white box.
\end{itemize}

\includegraphics{https://files.gitbook.com/v0/b/gitbook-x-prod.appspot.com/o/spaces\%2F3kS90tLsADGm1ocbe7q9\%2Fuploads\%2FXJQArmGsu0DelGLmqjg2\%2F10.8\%20Assigning\%20text\%20labels.png?alt=media\&token=b8729961-fa0a-49b3-9e67-d5547c6f59e6}

\textbf{To assign a Text Label to a Cue:}

\begin{itemize}
\item
  Method 1 - {[}QLIST{]} {[}\#{]} {[}CUE{]} {[}\#{]}{[}TEXT{]} opens Text Entry pop-up.
\item
  Method 2 - If assigned to the Master Controller, {[}CUE{]} {[}\#{]} {[}TEXT{]} Opens Text Entry pop-up.
\item
  Method 3 - {[}CUE{]} {[}\#{]} {[}SETTINGS{]} opens \{Cue Properties\} tab of Cue Settings pop-up. Enter Qlist text in the Text Box.
\item
  Method 4 - {[}TEXT{]} {[}HERE{]} press any button on the destination controller.
\item
  Method 5 -

  \begin{enumerate}
  \def\labelenumi{\arabic{enumi}.}
  \tightlist
  \item
    Press \{Cue label\} area of the display above the Controller - The Cue Properties pop-up opens.
  \item
    Select the desired cue.
  \item
    Enter cue text in the text box.
  \item
    Optionally notes may also be entered for the selected cue.
  \end{enumerate}
\end{itemize}

\hypertarget{modifying-cues-and-scenes}{%
\subsection{Modifying Cues and Scenes}\label{modifying-cues-and-scenes}}

\textbf{Modify a Cue or Scene using {[}UPDATE{]}:}

\begin{enumerate}
\def\labelenumi{\arabic{enumi}.}
\item
  Make a change to parameter values already outputting to the stage from an \textbf{active} cue or scene.
\item
  Press the {[}UPDATE{]} and the \textbf{Update pop-up} will open. Two columns will appear. The left column shows \textbf{Cues} available for updating and the right column shows \textbf{Libraries} available for updating. If affected values are not referencing a library, the Cue column will indicate the \textbf{source} cue that the modifications will be updated to.
\item
  If the Editor parameter values are \textbf{new} to all active cues, destination cue. Will be dependent on the \textbf{New Values} setting:
\end{enumerate}

\begin{itemize}
\item
  If \{All Selected Objects\} is selected, a list of all possible active cues will appear in the Cue column.
\item
  If \{Master {PB} Only\} (controller) is selected, new values will be only be updated to the Qlist assigned as Master Controller.
\item
  If \{Don't Store\} is selected, no new parameter values will be updated.
\end{itemize}

\begin{enumerate}
\def\labelenumi{\arabic{enumi}.}
\setcounter{enumi}{3}
\item
  Selected cues may be freely deselected in the Update pop-up.
\item
  Set \textbf{Tracking Options} as required.
\item
  Set \textbf{Source Filters} as required.
\item
  Press \href{image.png}{} or press {[}ENTER{]} to close the pop-up and complete the Update.
\end{enumerate}

\includegraphics{https://files.gitbook.com/v0/b/gitbook-x-prod.appspot.com/o/spaces\%2F3kS90tLsADGm1ocbe7q9\%2Fuploads\%2FWRLE1BnIrudwRn2IJP4Q\%2F10.9.webp?alt=media\&token=03b6843d-4dae-45a1-b5e5-d505b514868b}

\textbf{UPDATE Source Filters:}

\begin{itemize}
\item
  \{Master Controller\} - Updates will only be made to the active cue on the controller assigned to the Master Controller.
\item
  \{Last Activated Controller\} - Updates will be made to the last controller that asserter priority.
\item
  \{Grab All\} - All Active Controllers will be shown in a list and the user must select the destination of the updates.
\end{itemize}

\textbf{UPDATE Tracking Options:}

\begin{itemize}
\item
  \{Default (Track)\} - Changes will track forward to the next cue if it does not have a hard value.
\item
  \{Cue Only\} - Changes will be made to ``This Cue Only'' preserving the look of the next cue (Hard values will be added to next cue).
\item
  \{Skip {Cue}\} - Changes will be made to the current cue but the changes will be ``skipped over'' preserving the link to the original hard values.
\item
  \{Back Track\} - Updates the changes to the \textbf{original} hard value that made the stage look.
\end{itemize}

\textbf{Update directly to a Controller:}

\begin{enumerate}
\def\labelenumi{\arabic{enumi}.}
\item
  Make a change to parameter value outputting to the stage.
\item
  Press {[}UPDATE{]} - The Update pop-up will open.
\item
  Make option changes if needed.
\item
  Press {[}HERE{]} to any button on a destination Controller - the Editor values will be updated into the active cue on the destination Controller.
\end{enumerate}

\textbf{Update values to a cue or range of cues using keypad syntax:}

\begin{enumerate}
\def\labelenumi{\arabic{enumi}.}
\item
  Make a change to parameter value outputting to the stage.
\item
  Type {[}Qlist{]} {[}\#{]} (if destination cue/cues are not on the Master Controller) {[}CUE{]} {[}+ ⟶ -{]} {[}CUE{]} {[}\#{]} {[}UPDATE{]} - Modifications will now be updated to the destination cue selection.
\end{enumerate}

\begin{quote}
Cues may alternately be modified using the universal syntax below:
\end{quote}

\textbf{Modify cues using EDIT:}

\begin{enumerate}
\def\labelenumi{\arabic{enumi}.}
\item
  {[}QLIST{]} {[}\#{]} {[}CUE{]} {[}\#{]} {[}EDIT{]} - Cue values will be placed in the Editor and the {[}UPDATE{]} key will flash red.
\item
  Modify the parameters in the Editor as required.
\item
  Press {[}UPDATE{]} to complete the edit and restore control to the source controller.
\end{enumerate}

\begin{quote}
This method may be used in {[}BLIND{]} and unlike live, fixtures may be released from the source cue. Effects may also be edited
\end{quote}

\textbf{Modifying cues using \{Overwrite\}, \{Update\}, or \{Release\} keys -} (Traditional method common to most consoles)

\begin{enumerate}
\def\labelenumi{\arabic{enumi}.}
\item
  Make a change to parameter value outputting to the stage.
\item
  Press {[}QLIST{]} (in not not on the Master Controller) {[}CUE{]} {[}\#{]} {[}STORE{]}.
\item
  If the cue exits, the Cue Exists pop-up will appear with options for \{Cancel, Update, Overwrite, Release\}.
\end{enumerate}

\begin{itemize}
\item
  Tap \{Overwrite\} to replace all fixtures with those in the Editor.
\item
  Tap \{Update\} to merge the Editor into the selected cue.
\item
  Tap \{Release\} to remove the selected fixtures from the cue.
\end{itemize}

\includegraphics{https://files.gitbook.com/v0/b/gitbook-x-prod.appspot.com/o/spaces\%2F3kS90tLsADGm1ocbe7q9\%2Fuploads\%2FLorXIBo6svKGJIHyVcxC\%2F10.9.1.webp?alt=media\&token=b9c11b6e-cbf7-401f-b5df-66dee8df863b}

Cue Exists pop-up

\includegraphics{https://files.gitbook.com/v0/b/gitbook-x-prod.appspot.com/o/spaces\%2F3kS90tLsADGm1ocbe7q9\%2Fuploads\%2FfuG9DGIRoxtNWzrca99h\%2F10.9.2.webp?alt=media\&token=3d8c1790-cce0-4db6-826c-0eee6ad81c06}

Cue Settings pop-up

\textbf{Cue Settings for Text Label, Cue Time, Cue Linking and Looping, Cue Re-numbering and Cue Tracking options may be changed at any time after they are created using the Cue Settings pop-up:}

\begin{itemize}
\item
  Press {[}CUE{]} {[}\#{]} {[}SETTINGS{]} - The Cue Settings pop-up will appear.
\item
  Alternately tapping the \{Cue \# \} area of the source Controllers' display, will open the Cue Settings pop-Up.
\end{itemize}

\includegraphics{https://files.gitbook.com/v0/b/gitbook-x-prod.appspot.com/o/spaces\%2F3kS90tLsADGm1ocbe7q9\%2Fuploads\%2FMZek3Y7UHMkWYzfH3xor\%2F10.9.3.webp?alt=media\&token=902bcd31-04a0-48f1-ab2e-c7f69a29151f}

\textbf{Controller Display} (Sliders, A/B, Master Controller (MPB), System Settings)

\textbf{To Delete a Cue:}

\begin{itemize}
\tightlist
\item
  {[}Qlist{]} {[}\#{]} (Optional if {[}SELECT{]} controller) {[}CUE {[}\#{]} {[}DELETE{]}
\end{itemize}

\hypertarget{creating-new-cues-from-existing-cue-data}{%
\subsection{Creating New Cues from Existing Cue Data}\label{creating-new-cues-from-existing-cue-data}}

\textbf{To Copy a Cue to a new Cue:}

\begin{enumerate}
\def\labelenumi{\arabic{enumi}.}
\item
  {{[}CUE{]} {[}\#{]} {[}COPY{]}}
\item
  {The \textbf{Copy Cue} Pop-up will open}
\item
  {Three Options are available: }

  \begin{itemize}
  \item
    {\{Actual Cue\} - Copies cue properties as well as stored parameter values}
  \item
    {\{State\} - Copies all of the hard values and tracked values that make up the source cue's ``End State'' (Stage look) }
  \item
    {\{Values Only\} - Copies only the hard values that are stored in the source cue}
  \end{itemize}
\item
  {{[}CUE{]} {[}\#{]} {[}PASTE{]} - The Cue Only Pop-up will open}
\item
  {Three Options are available:}:

  \begin{itemize}
  \item
    {\{Paste\} \{Cue Only\} }
  \item
    {\{Paste\} \{Skip\} }
  \item
    {\{OK\}}
  \item
    {\{Cancel\}}
  \end{itemize}
\item
  Choose an option and the operation will be completed
\end{enumerate}

\textbf{To create a new cue using values from an existing cue, use one of the following two menu commands:}

\begin{itemize}
\item
  {[}CUE{]} {[}\#{]} \{LOAD\} - Loads the cue's \textbf{actual} hard values into the Editor. Tracked values are ignored. \textbf{Or}
\item
  {[}CUE{]} {[}\#{]} \{LOAD STATE\} - Loads the cue's hard values \textbf{and} current \textbf{tracked} values (Stage State) into the Editor.
\item
  Continue modifications and store as a new cue.
\end{itemize}

\hypertarget{cue-time-properties}{%
\subsection{Cue Time Properties}\label{cue-time-properties}}

\textbf{Cue Time Properties:}

\begin{itemize}
\item
  \href{https://vibemanual.compulite.com/programming-cues-and-scenes.html\#assigning-cue-time}{10.11.1. Assigning Cue Time}
\item
  \href{https://vibemanual.compulite.com/programming-cues-and-scenes.html\#wait-and-follow-times}{10.11.2. Wait and Follow Times}
\end{itemize}

\hypertarget{assigning-cue-time}{%
\subsubsection{Assigning Cue Time}\label{assigning-cue-time}}

\textbf{Cues may be assigned the following time properties:}

\begin{itemize}
\item
  Time
\item
  Time In
\item
  Time Out
\item
  Delay In
\item
  Delay Out
\item
  Follow Time
\item
  Wait Time
\end{itemize}

\includegraphics{https://files.gitbook.com/v0/b/gitbook-x-prod.appspot.com/o/spaces\%2F3kS90tLsADGm1ocbe7q9\%2Fuploads\%2Fqk1bKQWhDqyhg1qAysYP\%2F10.11.1.webp?alt=media\&token=7ef7e27d-8ad5-4efe-ae36-6e6966a8045a}

\textbf{Time Pop-up}

Assign a simple in/out time to the current cue:

\begin{enumerate}
\def\labelenumi{\arabic{enumi}.}
\item
  Press {[}CUE{]} {[}\#{]}
\item
  Press {[}TIME{]}, the Time pop-up will appear
\item
  Enter a value with the keypad. The default unit of time will be seconds.
\item
  Press {[}ENTER{]} or \href{image.png}{} to exit the pop-up and assign the cue time.
\end{enumerate}

\textbf{Assign a separate \{Time In\} \{Time Out\}, and/or \{Delay In\} \{Delay Out\} times to the current cue:}

\begin{enumerate}
\def\labelenumi{\arabic{enumi}.}
\item
  Press {[}CUE{]} {[}\#{]}
\item
  Press {[}TIME{]}, the Time pop-up will appear.
\item
  Press {[}TIME{]} again and the display will advance to \{Time In\}.
\item
  Press {[}TIME{]} again and the display will advance to \{Time Out\}.
\item
  Press {[}TIME{]} again and the display will advance to \{Delay\}.
\item
  Press {[}TIME{]} again and the display will advance to \{Delay In\}.
\item
  Press {[}TIME{]} again and the display will advance to \{Delay Out\}.
\end{enumerate}

Alternately times can be randomly entering by tapping the desired time field and making a keypad entry or turning the virtual wheel.

\begin{quote}
When typing any time value, the time indicator of the virtual wheel will turn green for approximately 4 seconds. While it is {\textbf{green}} time values may be entered in the format of: \textbf{Hour.Minute.Second.Millisecond}
\end{quote}

\begin{quote}
If the \{Enable\} key is pressed, the user may manually toggle between time units using the \{Units\} key to the left of it.
\end{quote}

\begin{quote}
{Out times only apply to dimmer parameters.}
\end{quote}

\hypertarget{wait-and-follow-times}{%
\subsubsection{Wait and Follow Times}\label{wait-and-follow-times}}

\textbf{There are two ways to make a cue automatically follow another:}

\begin{enumerate}
\def\labelenumi{\arabic{enumi}.}
\item
  Wait Time - Starts counting down from the {[}GO{]} press before the next cue is executed.
\item
  Follow Time - Advances to the next cue upon completion of the cue time and all of its individual parameter times.
\end{enumerate}

\textbf{Set a Wait Time:}

\begin{enumerate}
\def\labelenumi{\arabic{enumi}.}
\item
  Select the cue you want the Wait to be executed from.
\item
  Press {[}TIME{]} - the Time Pop-up will appear.
\item
  Tap \{Wait\} and either type in the Wait Time or use the virtual time wheel to set the Wait Time.
\end{enumerate}

\begin{quote}
To clear a Wait Time, turn the physical Wait Wheel until its display and the \{Wait\} display say N/A , Press {[}ENTER{]}
\end{quote}

\textbf{Set a Follow Time:}

\begin{enumerate}
\def\labelenumi{\arabic{enumi}.}
\item
  Select the cue you want the Follow from.
\item
  Press {[}TIME{]} - the Time pop-up will appear.
\item
  Tap \{Follow\} and either type in the Follow Time or use the virtual time wheel to set the Follow Time.
\end{enumerate}

\begin{quote}
To clear a Follow, turn the physical Follow Wheel until its display and the \{Follow\} display say N/A Press {[}ENTER\}
\end{quote}

\hypertarget{additional-cue-properties}{%
\subsection{Additional Cue Properties}\label{additional-cue-properties}}

Additional Cue Properties:

\begin{itemize}
\item
  \hspace{0pt}\href{https://vibemanual.compulite.com/programming-cues-and-scenes.html\#link}{10.12.1. Link}\hspace{0pt}
\item
  \hspace{0pt}\href{https://vibemanual.compulite.com/programming-cues-and-scenes.html\#loop}{10.12.2. Loop}\hspace{0pt}
\item
  \hspace{0pt}\href{https://vibemanual.compulite.com/programming-cues-and-scenes.html\#tracking-options}{10.12.3. Tracking Options}
\end{itemize}

\hypertarget{link}{%
\subsubsection{Link}\label{link}}

Cues may be linked out of sequence using any of the following three methods:

\textbf{Using keypad and Toolbar:}

\begin{enumerate}
\def\labelenumi{\arabic{enumi}.}
\item
  Source {[}CUE{]} {[}\#{]} \{LINK\} on the Editor Toolbar - The Link pop-up will appear.
\item
  Type link destination {[}\#{]} (Do not press {[}CUE{]}).
\item
  Press Enter or \href{image.png}{} to complete the operation.
\end{enumerate}

\textbf{Using Cue Setting pop-Up:}

\begin{enumerate}
\def\labelenumi{\arabic{enumi}.}
\item
  Source {[}CUE{]} {[}\#{]} {[}SETTINGS{]} - Cue Settings pop-up will open.
\item
  Tap \{Link\} - The Link pop-up will appear.
\item
  Type, scroll or drag to the desired destination cue.
\item
  Press Enter or \href{image.png}{} to complete the operation.
\item
  Press Enter or \href{image.png}{} again to close the Settings pop-up.
\end{enumerate}

\includegraphics{https://files.gitbook.com/v0/b/gitbook-x-prod.appspot.com/o/spaces\%2F3kS90tLsADGm1ocbe7q9\%2Fuploads\%2FA13nJWROvUuTOtHh3ExP\%2F10.12.1.webp?alt=media\&token=37da5fa1-b7b9-437f-80e6-e42da39e0bb6}

\textbf{Direct using the cue sheet:}

\begin{enumerate}
\def\labelenumi{\arabic{enumi}.}
\item
  Tap in the Link field of the source cue in the \textbf{Live Master Controller Cue Sheet}
\item
  The Link pop-up will appear.
\item
  Type, scroll or drag to the desired destination cue.
\item
  Press Enter or again to close the Settings pop-up.
\end{enumerate}

\textbf{To remove a Link:}

\begin{enumerate}
\def\labelenumi{\arabic{enumi}.}
\item
  Select the source cue and open the Link pop-up by any of the above methods.
\item
  If the proper source cue has been selected, the \{Unlink\} key will appear.
\item
  Tap \{Unlink\} key to remove the Link.
\end{enumerate}

\textbf{Alternate method:}

\begin{enumerate}
\def\labelenumi{\arabic{enumi}.}
\item
  Select the source {[}CUE{]} {[}\#{]} and tap \{Link\} on the Editor Toolbar, the Link pop-up will appear.
\item
  Press {[}RELEASE{]} to remove the link.
\end{enumerate}

\begin{quote}
If the source cue is forgotten, check the \textbf{Link \#} field in the Live Master Controller Cue Sheet.
\end{quote}

\hypertarget{loop}{%
\subsubsection{Loop}\label{loop}}

In many consoles, the only way to create a loop is to create a link from the end of a range of cues back to the beginning. Follow time must then be assigned to all of the cues. In Vibe this is also possible but a shorter more efficient method is also provided. \textbf{LOOP} is a function similar to \textbf{LINK}. It is assigned in the Loop pop-up or via the Cue Settings pop-up.

Loop Pop-up

\textbf{To assign a Loop:}

\textbf{Using keypad and Editor Toolbar:}

\begin{enumerate}
\def\labelenumi{\arabic{enumi}.}
\item
  Start {[}CUE{]} {[}\#{]} tap \{LOOP\} on the Editor Toolbar - The Loop pop-up will appear.
\item
  Type end {[}\#{]} or use \{To Cue No\} field list to select the end cue.
\item
  Press Enter or \href{image.png}{} to complete the operation.
\end{enumerate}

\textbf{Using Cue Setting Pop-Up:}

\begin{enumerate}
\def\labelenumi{\arabic{enumi}.}
\item
  Start {[}CUE{]} {[}\#{]} {[}SETTINGS{]} - Cue Settings pop-up will open.
\item
  Tap \{Loop\} - Loop pop-up will appear.
\item
  Type end Cue {[}3{]} or scroll the value picker to the desired end cue {[}\#{]}.
\item
  Press Enter or \href{image.png}{} to complete the operation.
\item
  Press Enter or \href{image.png}{} again to close the Settings pop-up.
\end{enumerate}

\textbf{Direct via the cue sheet:}

\begin{enumerate}
\def\labelenumi{\arabic{enumi}.}
\item
  Tap in the Loop \# field of the start cue in the Live Master Controller Cue Sheet - The Loop pop-up will appear.
\item
  Type, scroll or drag to the desired end cue.
\item
  Press Enter or to complete the operation.
\end{enumerate}

\textbf{To assign a Loop Count -} A Loop Count may be assigned using any of the above methods to open the Loop pop-up.

\begin{enumerate}
\def\labelenumi{\arabic{enumi}.}
\item
  Make sure \{Auto Loop\} is selected.
\item
  Under the \{\# Of Time To Loop\} field, Type loop count number, or scroll the value picker to the desired loop count number.
\item
  Press Enter or to complete the operation.
\end{enumerate}

\textbf{To enable \{Auto Follow\} -} Auto Follow loops advance to the cue following the loop, after completion of the loop count.

\begin{itemize}
\tightlist
\item
  Toggle \{Auto Follow\} - On/Off as required.
\end{itemize}

\textbf{To set a Loop as Manual Loop} (Does not advance automatically)

\begin{itemize}
\tightlist
\item
  Toggle \{Auto Loop\} - Off
\end{itemize}

\textbf{To remove a Loop:} (Qlist must be released first)

\begin{enumerate}
\def\labelenumi{\arabic{enumi}.}
\item
  Select the start cue and open the Loop pop-up by any of the above methods.
\item
  If the proper start cue has been selected, the \{Unloop\} key will appear.
\item
  Tap the \{Unloop\} key to remove the Link.
\item
  Press Enter or to complete the operation.
\end{enumerate}

\textbf{Alternate method:} (Qlist must be released first)

\begin{itemize}
\item
  Select the start \{CUE{]} {[}\#{]} and tap \{LOOP\} on the Editor Toolbar - The Loop pop-up will appear.
\item
  Press {[}RELEASE{]} to remove the loop.
\end{itemize}

\hypertarget{tracking-options}{%
\subsubsection{Tracking Options}\label{tracking-options}}

\textbf{Individual cues may be tagged with one of four Tracking Options:}

\begin{enumerate}
\def\labelenumi{\arabic{enumi}.}
\item
  Cue Only - Removes tracked values from asserting to the stage and sets them at home values (Not to be confused with Cue Only when used with \textbf{\{UPDATE\}} which works a cell basis).
\item
  Skip Cue - All cells in the cue will behave like Skip Cells. Values will still ``skip around'' a Skip Cue, even if the values in the Skip Cue are modified.
\item
  Block Cue - All cue parameter values whether tracked or hard will be made to act like hard values. Any new values added to cues above the Blocked cue will not track past a cue tagged a Block.
\item
  Track - The systems default behaviour.
\end{enumerate}

\textbf{There are two additional properties a cue can be tagged with:}

\begin{enumerate}
\def\labelenumi{\arabic{enumi}.}
\item
  Force Black - When it is not desirable to see values move from one cue state to another, Force Black may be used to save programming time. In a cue tagged as Force Black, the dimmer parameter is first ``forced'' to Black Out, parameters then move in black, then the dimmer parameter is faded up again. The total speed of a Force Black operation is relative to the overall cue time.
\item
  {Look Ahead cue - When Look Ahead is disabled for a Qlist, Individual cues with dark values may be tagged to perform Look Ahead pre-positioning.}
\end{enumerate}

\hypertarget{parameter-time}{%
\subsection{Parameter Time}\label{parameter-time}}

\textbf{Parameter Time:}

\begin{itemize}
\item
  \href{https://vibemanual.compulite.com/programming-cues-and-scenes.html\#assign-parameter-time-during-cue-creation}{10.13.1. Assign Parameter Time During Cue Creation}
\item
  \href{https://vibemanual.compulite.com/programming-cues-and-scenes.html\#adding-or-modifying-parameter-time-after-cue-creation}{10.13.2. Adding or Modifying Parameter Time after Cue Creation}
\item
  \href{https://vibemanual.compulite.com/programming-cues-and-scenes.html\#fanning-parameter-time}{10.13.3. Fanning Parameter Time}
\end{itemize}

\hypertarget{assign-parameter-time-during-cue-creation}{%
\subsubsection{Assign Parameter Time During Cue Creation}\label{assign-parameter-time-during-cue-creation}}

\begin{enumerate}
\def\labelenumi{\arabic{enumi}.}
\item
  Set fixture parameter values in the Editor.
\item
  Select the fixtures requiring Parameter Time.
\item
  Press {[}TIME{]} - The Parameter Time pop-up will appear.
\item
  oggle on/off the parameters requiring Parameter Time

  \begin{itemize}
  \item
    Parameter view may be filtered by banks.
  \item
    Fixtures may be selected and detected freely in the pop-up.
  \item
    Toggling the parameter header will select reselect all fixtures.
  \end{itemize}
\item
  Set the parameter time for the selected fixtures directly using the keypad, or by using the virtual or physical time wheels.
\item
  Press {[}Enter{]} or tap \href{image.png}{} to close the pop-up.
\item
  Selected fixtures will now show a clock icon in the Live Display.
\item
  Store the Cue using any of the methods described under:
\end{enumerate}

\begin{itemize}
\tightlist
\item
  \href{https://vibemanual.compulite.com/programming-cues-and-scenes.html\#storing-cues-and-scenes-directly-to-controllers}{10.3. Storing Cues and Scenes Directly to Controllers}
\item
  \href{https://vibemanual.compulite.com/programming-cues-and-scenes.html\#storing-cues-to-the-master-controller}{10.5. Storing Cues to the Master Controller}
\item
  \href{https://vibemanual.compulite.com/programming-cues-and-scenes.html\#storing-qlist-cues-to-any-controller}{10.6. Storing Qlist Cues to any Controller}
\end{itemize}

\includegraphics{https://files.gitbook.com/v0/b/gitbook-x-prod.appspot.com/o/spaces\%2F3kS90tLsADGm1ocbe7q9\%2Fuploads\%2FVMwWaGG9Qr6NDEWLQKOl\%2F10.13.1\%20parameter\%20time.png?alt=media\&token=55487d5d-9035-4083-a55e-a02b0b0573a4}

Parameter Time pop-up

\hypertarget{adding-or-modifying-parameter-time-after-cue-creation}{%
\subsubsection{Adding or Modifying Parameter Time after Cue Creation}\label{adding-or-modifying-parameter-time-after-cue-creation}}

\textbf{Method 1:}

\begin{enumerate}
\def\labelenumi{\arabic{enumi}.}
\item
  Select fixtures that are active on the stage - Specific parameters may be selected before pressing {[}TIME{]}.
\item
  Press {[}TIME{]} - The Time pop-up will open.
\item
  Toggle on/of the parameters that require Parameter Time.

  \begin{itemize}
  \item
    Parameter view may be filtered by banks.
  \item
    Fixtures may be selected and detected freely in the pop-up.
  \item
    Toggling the parameter header will select reselect all fixtures.
  \end{itemize}
\item
  Set the parameter time for the selected fixtures directly using the keypad, or by using the virtual or physical time wheels.
\item
  Press {[}Enter{]} or tap \href{image.png}{} to close the pop-up.
\item
  Selected fixtures will now show a clock icon in the Live Display.
\item
  Press {[}UPDATE{]}.
\end{enumerate}

See Also: \href{https://vibemanual.compulite.com/programming-cues-and-scenes.html\#modifying-cues-and-scenes}{10.9. Modifying Cues and Scenes}

\textbf{Method 2:}

\begin{enumerate}
\def\labelenumi{\arabic{enumi}.}
\item
  Press {[}CUE{]} {[}\#{]} {[}EDIT{]} - Update will flash red and the cue will be placed in the Editor.
\item
  Select the fixtures that need parameter time editing.
\item
  Press {[}TIME{]} - The Time pop-up will open.
\item
  Toggle on/of the parameters that require Parameter Time.

  \begin{itemize}
  \item
    Parameter view may be filtered by banks.
  \item
    Fixtures may be selected and detected freely in the pop-up.
  \item
    Toggling the parameter header will select reselect all fixtures.
  \end{itemize}
\item
  Set the parameter time for the selected fixtures directly using the keypad, or by using the virtual or physical time wheels.
\item
  Press the {[}UPDATE{]} again to complete the modification. The Update light will stop flashing.
\end{enumerate}

\textbf{Remove a parameter time:}

\begin{enumerate}
\def\labelenumi{\arabic{enumi}.}
\item
  Press {[}CUE{]} {[}\#{]} {[}EDIT{]} - Update will flash red and the cue will be placed in the Editor.
\item
  Select the fixtures to release time from - Specific parameters may be selected before pressing {[}TIME{]}.
\item
  Press {[}TIME{]} - The Time pop-up opens.
\item
  If they were not already selected, toggle on/of the parameters that require Parameter Time to be released.

  \begin{itemize}
  \item
    Parameter view may be filtered by banks.
  \item
    Fixtures may be selected and detected freely in the pop-up.
  \item
    Toggling the parameter header will select reselect fixtures.
  \end{itemize}
\item
  Press either the \{Release\} key over the Time-In wheel or the \{Release\} over the Delay-In wheel.
\item
  Press the {[}UPDATE{]} again to complete the modification. The Update light will stop flashing and the parameters will be released back to cue time.
\end{enumerate}

\hypertarget{fanning-parameter-time}{%
\subsubsection{Fanning Parameter Time}\label{fanning-parameter-time}}

\textbf{Method 1:}

\begin{enumerate}
\def\labelenumi{\arabic{enumi}.}
\item
  Select fixtures that are active on the stage - Specific parameters may be selected before pressing {[}TIME{]}.
\item
  Press {[}TIME{]} - The Time pop-up will open.
\item
  Toggle on/of the parameters that require parameter or delay time.

  \begin{itemize}
  \item
    Parameter view may be filtered by banks.
  \item
    Fixtures may be selected and detected freely in the pop-up.
  \item
    Toggling the parameter header will select reselect all fixtures.
  \end{itemize}
\item
  Tap the Time-In or Delay-In virtual wheel.
\item
  Tap \{Fan\}.
\item
  Toggle the desired fan direction key.
\item
  Rotate the appropriate virtual wheel to fan the In-Time or Delay-Time.
\item
  Press {[}Enter{]} or tap \href{image.png}{} to close the Parameter Time pop-up.
\item
  Press {[}UPDATE{]} - Check that the correct cue is selected.
\item
  Press {[}Enter{]} or tap \href{image.png}{} to close the Update pop-up.
\end{enumerate}

\begin{quote}
The above sequence will also work with {[}CUE{]} {[}\#{]} {[}EDIT{]} but step 10 is redundant.
\end{quote}

\textbf{Method 2:}

\begin{enumerate}
\def\labelenumi{\arabic{enumi}.}
\item
  Select fixtures that are active on the stage - Specific parameters may be selected before pressing {[}TIME{]}.
\item
  Press {[}TIME{]} - The Time pop-up will open.
\item
  Toggle on/of the parameters that require parameter or delay time.

  \begin{itemize}
  \item
    Parameter view may be filtered by banks.
  \item
    Fixtures may be selected and detected freely in the pop-up.
  \item
    Toggling the parameter header will select reselect all fixtures.
  \end{itemize}
\item
  Tap the Time-In or Delay-In virtual wheel.
\item
  Type a time or delay range. E.g. to evenly fan delay-in time for tilt in a selection of 8 fixtures, Tap \{Delay-In\} wheel and then 0 → 8 seconds on the keypad.
\item
  Press {[}Enter{]} or tap \href{image.png}{} to close the Parameter Time pop-up.
\item
  Press {[}UPDATE{]} - Check that the correct cue is selected.
\item
  Press {[}Enter{]} or tap \href{image.png}{} to close the Update pop-up.
\end{enumerate}

\begin{quote}
The above sequence will also work with {[}CUE{]} {[}\#{]} {[}EDIT{]} but step 8 is redundant.
\end{quote}

\textbf{Remove a fanned parameter time:}

See: \href{https://vibemanual.compulite.com/programming-cues-and-scenes.html\#adding-or-modifying-parameter-time-after-cue-creation}{10.13.2. Adding or Modifying Parameter Time after Cue Creation}

\textbf{Cell Time Fan}

The following command combination enables a cell time fan without the need to touch the screen and set the time with the wheel.

To Set Fan Cell Time:

\begin{enumerate}
\def\labelenumi{\arabic{enumi}.}
\item
  Select Fixtures
\item
  Give values to a parameter
\item
  Press Time
\item
  Popup opens
\item
  Press \#→\# and \emph{ENTER} The first number is the base for the fan and the second number is the end value of the fan. In the picture below there is a fan between 2 and 6 seconds.
\end{enumerate}

\includegraphics{https://files.gitbook.com/v0/b/gitbook-x-prod.appspot.com/o/spaces\%2F3kS90tLsADGm1ocbe7q9\%2Fuploads\%2FYEcjecqiWJTijSC3aOss\%2Fimage.png?alt=media\&token=85286838-edf2-4ba4-9bb8-f6263938dca5}

\hypertarget{parameter-profile}{%
\subsection{Parameter Profile}\label{parameter-profile}}

Profile determine a parameter's behavior during a fade.
Profiles are always relative to the fade time.
Profile options are available in a popup that is triggered from a button on the toolbar named Profile. This button is valid once there is a valid parameter selection.

Profile examples:
- Linear (default) -- on go, the parameter fades in cue time to its new level
- Jump on Start (One) -- on go, the parameter jumps to its level
- Jump on End -- when the fade is complete, the parameter jumps to its new level
- Jump on 50\% (Step) -- when the fade reaches 50\% of the cue time, the parameter jumps to its new level
Any function from the system functions can be applied as a profile.

\textbf{To Set a Parameter Profile}
1. Select fixtures
2. Select parameter(s)
3. On the Editor Toolbar tap Profile
4. Popup opens

\includegraphics{https://files.gitbook.com/v0/b/gitbook-x-prod.appspot.com/o/spaces\%2F3kS90tLsADGm1ocbe7q9\%2Fuploads\%2FXCDZmHO4jg72bW1JVSVs\%2Fimage.png?alt=media\&token=ed2329aa-2f95-4960-9bd5-59f2e33e2496}

\textbf{Popup Options}

On the left side, there is a list of active parameters in the editor. Colored are the parameters that the settings will be applied to. User can switch a parameter on or off.

In the middle, there is a list of functions. User may select the desired function as the profile for the parameters.

On the right side, there are options that can be changed and explained in the table below.

User may apply the popup with nothing selected on the options or with all the options active.

\includegraphics{https://files.gitbook.com/v0/b/gitbook-x-prod.appspot.com/o/spaces\%2F3kS90tLsADGm1ocbe7q9\%2Fuploads\%2FLqWbq2vRc7qd5hTBDaHm\%2Fimage.png?alt=media\&token=2d3405fd-5406-4102-a85c-049a4a5cf82f}

\hypertarget{group-submasters}{%
\subsection{Group Submasters}\label{group-submasters}}

\textbf{Group Submasters:}

\begin{itemize}
\item
  \href{https://vibemanual.compulite.com/programming-cues-and-scenes.html\#convert-a-scene-into-a-group-submaster}{10.15.1. Convert a Scene into a Group Submaster}
\item
  \href{https://vibemanual.compulite.com/programming-cues-and-scenes.html\#assign-a-group-submater-directly-to-a-slider-controller}{10.15.2. Assign a Group Submater directly to a Slider Controller}
\item
  \href{https://vibemanual.compulite.com/programming-cues-and-scenes.html\#assign-a-group-submaster-using-existing-groups}{10.15.3. Assign a Group Submaster using existing Groups}
\end{itemize}

\hypertarget{convert-a-scene-into-a-group-submaster}{%
\subsubsection{Convert a Scene into a Group Submaster}\label{convert-a-scene-into-a-group-submaster}}

Any Scene can be easily converted into a Group \textbf{Submaster} (Known in some consoles as Inhibitive Submasters or Group Masters).

\begin{enumerate}
\def\labelenumi{\arabic{enumi}.}
\tightlist
\item
  Press {[}SETTINGS{]} {[}HERE{]} to any button of a controller that has a Scene assigned to it.
\end{enumerate}

\begin{quote}
Alternatively, tap any of the controller \{function labels\} on the controller display on the main monitor above the Scene Slider.
\end{quote}

\begin{enumerate}
\def\labelenumi{\arabic{enumi}.}
\setcounter{enumi}{1}
\tightlist
\item
  The \textbf{Slider Definitions} pop-up will appear with the \{\textbf{Controller Definitions}\} tab selected.
\end{enumerate}

\includegraphics{https://files.gitbook.com/v0/b/gitbook-x-prod.appspot.com/o/spaces\%2F3kS90tLsADGm1ocbe7q9\%2Fuploads\%2FlnX1ifXt2bPr2nggFlTN\%2F10.14.1.webp?alt=media\&token=5d515767-5197-43fd-ac8c-962105a4263a}

\hspace{0pt}

\begin{enumerate}
\def\labelenumi{\arabic{enumi}.}
\setcounter{enumi}{2}
\item
  Tap the middle \{Button label\}. By default, it should say \{Intensity Master\}. The \textbf{Choose Control} drop-down menu will appear.
\item
  Select the \{Submaster\} option with the Crown in front of it.
\item
  The middle button mimic should now say \{Submaster\}.
\item
  Close the pop-up by pressing {[}Enter{]} or tap .\\
  All fixtures in the Scene will now act as Group Submasters. Scene fixture values will be ignored.
\end{enumerate}

\includegraphics{https://files.gitbook.com/v0/b/gitbook-x-prod.appspot.com/o/spaces\%2F3kS90tLsADGm1ocbe7q9\%2Fuploads\%2FOuyVOQxYJMWbKHgwvB0k\%2F10.14.1.2.webp?alt=media\&token=30ab6c49-3e31-4c10-a2f8-72c1ed587c59}

\hypertarget{assign-a-group-submater-directly-to-a-slider-controller}{%
\subsubsection{Assign a Group Submater directly to a Slider Controller}\label{assign-a-group-submater-directly-to-a-slider-controller}}

\begin{enumerate}
\def\labelenumi{\arabic{enumi}.}
\item
  Specify a fixture selection but \textbf{do not} assign editor values.
\item
  Press {[}STORE{]} {[}SCENE{]} {[}HERE{]} to any button of the destination controller. {[}FIXTURE{]} {[}1{]} → {[}6{]} {[}STORE{]} {[}SCENE{]} press the {[}TOP Buton{]} of Fader Controller \# 1
\item
  The {[}Top Controller Button{]} will now turn {light blue} to indicate that it is now a Group Submaster.
\end{enumerate}

\begin{quote}
{When the slider handle is down, the {[}Top Button{]} will flash to indicate that levels are being inhibited. When it is at full, the button's blue backlight will go steady.}
\end{quote}

\hypertarget{assign-a-group-submaster-using-existing-groups}{%
\subsubsection{Assign a Group Submaster using existing Groups}\label{assign-a-group-submaster-using-existing-groups}}

\textbf{Assign a Group to a Slider Controller using keypad syntax:}

\begin{enumerate}
\def\labelenumi{\arabic{enumi}.}
\item
  Press {[}GROUP{]} {[}\#{]} {[}HERE{]} - To any button of the destination slider controller.
\item
  The top controller button will turn light blue to indicate that it is now a Group Submaster.
\end{enumerate}

\textbf{Assign a Group to a Slider Controller using Group softkeys:}

\begin{enumerate}
\def\labelenumi{\arabic{enumi}.}
\item
  Press a \{GROUP\} softkey.
\item
  Press {[}HERE{]} to any button of the destination slider controller.
\item
  The {[}Top Controller Button{]} will now turn light blue to indicate that it is now a Group Submaster.
\end{enumerate}

\begin{quote}
{When the slider handle is down, the {[}Top Button{]} will flash to indicate that levels are being inhibited. When it is at full, the button's blue back light will go steady.}
\end{quote}

\hypertarget{filters}{%
\section{Filters}\label{filters}}

This chapter deals with filtering Editor parameters when storing objects and setting Bank Library default filters.

\textbf{The following is covered in this chapter:}

\begin{itemize}
\tightlist
\item
  \href{https://vibemanual.compulite.com/filters.html\#on-the-fly-filters}{11.1. On the Fly Filters}
\item
  \href{https://vibemanual.compulite.com/filters.html\#general-filters-tab}{11.2. General Filters Tab}
\item
  \href{https://vibemanual.compulite.com/filters.html\#library-defaults-tab}{11.3 Library Defaults Tab}
\end{itemize}

\hypertarget{on-the-fly-filters}{%
\subsection{On the Fly Filters}\label{on-the-fly-filters}}

11.1. On the Fly Filters

On the Fly Filters are used to filter editor parameters when storing cues and scenes.

\includegraphics{https://files.gitbook.com/v0/b/gitbook-x-prod.appspot.com/o/spaces\%2F3kS90tLsADGm1ocbe7q9\%2Fuploads\%2F70gd5CGwrkMzk5SvYiQb\%2F11.1.webp?alt=media\&token=b2a747bc-afe3-4f38-898e-ea8f50b43fe8}

\begin{quote}
Green \href{image.png}{} indicates parameters that will be included in the cue or scene.
\end{quote}

\textbf{Store a cue or scene using On The Fly Filters:}

\begin{enumerate}
\def\labelenumi{\arabic{enumi}.}
\item
  Set or load values into the Editor. CUE{]} {[}\#{]} {[}EDIT{]} is also valid.
\item
  Press \protect\hyperlink{filters}{FILTERS} key.
\item
  The On The Fly Filter pop-up will open.
\item
  De-select banks and parameters that need to be filtered from the cue.
\item
  Press {[}ENTER{]} or \href{image.png}{} . to apply the filter and close the pop-up.
\item
  Store cue or scene. Deselected banks and parameters will not be included.
\end{enumerate}

Options:

\begin{itemize}
\tightlist
\item
  \{Select All\} - Selects all banks and parameters
\item
  \{Deselect All\} - Deselects all banks and parameters
\item
  \{Save as Default\} - {To be removed}
\item
  {\{Keep Filter till Reset\} - Not implemented.}
\item
  {\{Save Filter and Apply\} - Not yet implemented.}
\end{itemize}

\hypertarget{general-filters-tab}{%
\subsection{General Filters Tab}\label{general-filters-tab}}

Filters may be created ``blind'' without a fixture and parameter selection. They may be ``Include'' or ``Exclude'' depending on how the filters are created. Filters are automatically saved in the \textbf{Filters Library.}

\textbf{Create ``Include'' Filter} - Only the parameters in the \textbf{Filtered Parameters} box will be in the cue:

\begin{enumerate}
\def\labelenumi{\arabic{enumi}.}
\item
  Press {[}VIBE{]}+\protect\hyperlink{filters}{FILTERS} to open the Filters pop-up.
\item
  Make sure the \textbf{General Tab} in the lower left corner of the pop-up is selected.
\item
  Press \{New\} - A new filter will be created with the name \{1. Filter 1\}.
\item
  Accept the name or type a new name in the white data entry box at the bottom of the Filter's column.
\item
  Tap the parameters you want to be in the ``include'' filter - The parameter will move from the Available Parameters box to the Filtered Parameters box.
\item
  Press the \{Apply\} softkey to apply the filter and stay in the pop-up or press {[}ENTER{]} or \href{image.png}{} to apply the filter and close the pop-up.
\end{enumerate}

\includegraphics{https://files.gitbook.com/v0/b/gitbook-x-prod.appspot.com/o/spaces\%2F3kS90tLsADGm1ocbe7q9\%2Fuploads\%2FxJGgGofSNiTwdWFtRk9V\%2Fimage.png?alt=media\&token=a0874eb0-9da7-44de-bd08-2cf099a2ea51}

\textbf{Create ``Exclude'' Filter} - Only the parameters in the \textbf{Available Parameters} box will be used - The result is similar to \textbf{On The Fly} Filters.

\begin{enumerate}
\def\labelenumi{\arabic{enumi}.}
\item
  Press {[}VIBE{]} + \protect\hyperlink{filters}{FILTERS} to open the Blind Filters pop-up.
\item
  Make sure the \textbf{General Tab} in the lower left corner of the pop-up is selected.
\item
  Press \{New\} - A new filter will be created with the name \{1.Filter 1\}.
\item
  Accept the name or type a new name in the white data entry box at the bottom of the Filter's column.
\item
  Tap all of the parameters \textbf{except} the ones you want to be excluded from recording to move them to the \textbf{Filtered Parameters box} - This is the inverse of ``Include Filters''.
\item
  Press the \{Apply\} softkey to apply the filter and stay in the pop-up or press {[}ENTER{]} or to apply the filter and close the pop-up.
  \includegraphics{https://files.gitbook.com/v0/b/gitbook-x-prod.appspot.com/o/spaces\%2F3kS90tLsADGm1ocbe7q9\%2Fuploads\%2F5yCJCJMYORB9XLWXzCvn\%2Fimage.png?alt=media\&token=b9b9696b-e3ec-4df3-87c1-0203e434d459}
\end{enumerate}

\hypertarget{library-defaults-tab}{%
\subsection{Library Defaults Tab}\label{library-defaults-tab}}

When storing Bank Libraries, filters are used to exclude parameters from unrelated banks. The Library filter defaults are set under the \textbf{Filters} pop-up, \textbf{Library Defaults} tab.

\includegraphics{https://files.gitbook.com/v0/b/gitbook-x-prod.appspot.com/o/spaces\%2F3kS90tLsADGm1ocbe7q9\%2Fuploads\%2FBUtHaWfD8AdZQM7fSUka\%2F11.3.webp?alt=media\&token=d11601fe-f97b-4d40-b65e-a45255e4f001}

\textbf{Change Library Filter Defaults:}

\begin{enumerate}
\def\labelenumi{\arabic{enumi}.}
\item
  Press {[}VIBE{]}+\protect\hyperlink{filters}{FILTERS} to open the Filters pop-up.
\item
  Make sure the \textbf{Library Defaults} tab in the lower left corner of the pop-up is selected.
\item
  Tap the \{Library Bank\} to be edited.
\item
  Tap a parameter in the \textbf{Available Parameters} box to move it to the \textbf{Filtered Parameters box}.
\item
  Press the \{Apply\} softkey to apply the filter and stay in the pop-up or press {[}ENTER{]} or \href{image.png}{} to apply the filter and close the pop-up.
\end{enumerate}

\textbf{Restore Defaults:} - Taping \{Restore Defaults\} will reset Bank Libraries to their system defaults.

\hypertarget{libraries}{%
\section{Libraries}\label{libraries}}

This chapter deals with creating and editing Libraries.

\textbf{The following is covered in this chapter:}

\begin{itemize}
\tightlist
\item
  \href{https://vibemanual.compulite.com/libraries.html\#what-are-libraries}{12.1. What are Libraries}
\item
  \href{https://vibemanual.compulite.com/libraries.html\#store-and-modify-bank-libraries}{12.2. Store and Modify Bank Libraries}
\item
  \href{https://vibemanual.compulite.com/libraries.html\#object-libraries}{12.3. Object Libraries}
\end{itemize}

\hypertarget{what-are-libraries}{%
\subsection{What are Libraries}\label{what-are-libraries}}

\textbf{Vibe has two types of Libraries, Bank Libraries and Object Libraries:}

\begin{enumerate}
\def\labelenumi{\arabic{enumi}.}
\item
  Bank Libraries - Store commonly used parameter values for Intensity, Position, Color, Beam, Image, Shapers, and Video content. They correspond to the parameter banks. Libraries are essentially filtered cues that are referenced by recorded objects such as Scenes and Cues. There are also two special bank libraries, Custom, and Effects.

  \begin{itemize}
  \item
    Custom libraries are unfiltered and record whatever parameters are in the Editor. The system \protect\hyperlink{filters}{FILTERS} key function or user defined Filter object libraries may be used to limit which parameters are stored in Custom libraries.
  \item
    Effect libraries store effects parameters from the Editor so that they do not have to be manually rebuilt every time they are required.
  \end{itemize}
\item
  Object Libraries - Pools of system object types.

  \begin{itemize}
  \item
    Common Object Libraries are:

    \begin{itemize}
    \item
      Group
    \item
      Qlist
    \item
      Scenes-- Single non-Qliet ``Looks''(Submasters on some consoles)
    \item
      Snaps -- Snapshot controller states. Sometimes referred to as Snap Shots
    \item
      Filters -- Pool of user defined flters
    \item
      Macro
    \end{itemize}
  \end{itemize}
\end{enumerate}

\begin{quote}
Scenes are like Submasters or Group Masters on other consoles. They may be set as additive or subtractive.
\end{quote}

\hypertarget{store-and-modify-bank-libraries}{%
\subsection{Store and Modify Bank Libraries}\label{store-and-modify-bank-libraries}}

12.2. Store and Modify Bank Libraries

\begin{quote}
Compulite Libraries are similar to Presets or Palettes on other consoles.
\end{quote}

\textbf{Automatic Parameter Grouping:}

\begin{itemize}
\item
  Vibe uses tee concept of ``Auto Parameter Grouping'' parameters when storing libraries. All parameters in Position, Color, and Shaper parameter banks will always be recorded into a library to guarantee that what was viewed on the stage is the same as what is recorded in the library.
\item
  Filters may be applied to modify what is stored in a Library.
\end{itemize}

\textbf{Libraries can be assigned one of six behaviours:}

\begin{enumerate}
\def\labelenumi{\arabic{enumi}.}
\item
  \textbf{Fixture Specific} -- Values only applied only to specific stored fixtures.
\item
  \textbf{Parameter Specific} -- Values will be applied to all matching parameter types.
\item
  \textbf{Pattern} --Values will be applied to all matching parameter types and will keep the pattern of the values.
\item
  \textbf{Device + Fixture} -- the first fixture will be used as a reference for all other similar fixture types. If values are changed for specific fixtures they will be stored in the library as if they were fixture specific. If a fixture other than the first fixture is released from the library, its values will again be referenced from the first fixture. If the first fixture is released, the next fixture will become the reference.
\item
  \textbf{Device and Parameter} -- Values will be applied to fixtures of the same device type and the same parameter type.
\item
  \textbf{Device and pattern} -- Values will be applied to fixtures of the same device type and keep the pattern of values.
\end{enumerate}

Library Bank SKs have default filters and behaviours based on the way they are commonly used.

\textbf{Store directly to a Bank SK Library:}

\begin{enumerate}
\def\labelenumi{\arabic{enumi}.}
\tightlist
\item
  {[}Fixture{]} {[}\#{]} {[}+ ⟶ -{]} Select parameter \{Bank\} on Smart Screen.
\item
  Use keypad, Smart Screen pickers, and/or adjust parameter wheels, to set values.
\item
  Press {[}STORE{]} - The Library \{Store Options\} tab will appear on top right corner of SKs.
\item
  Use the Library \{Store Options\} pop-up at the bottom on the Smart Screen to make option changes.
\item
  Tap destination Library SK.
\end{enumerate}

\begin{quote}
If no option changes are required just directly tap the appropriate Library SK without tapping the Library \{Store Options\}.
\end{quote}

\textbf{Store a Library using keypad syntax:}

\begin{enumerate}
\def\labelenumi{\arabic{enumi}.}
\item
  Press {[}Fixture{]} {[}\#{]} {[}+ ⟶ -{]} - Select parameter \{Bank \} on Smart Screen.
\item
  Use Smart Screen pickers, and/or adjust parameter wheels, or type value to set values.
\item
  Press {[}LIBRARY{]} - Library \{Store Options\} pop-up will open at the bottom of the Smart Screen and \{Banks\} will appear on the Editor Toolbar.
\end{enumerate}

\includegraphics{https://files.gitbook.com/v0/b/gitbook-x-prod.appspot.com/o/spaces\%2F3kS90tLsADGm1ocbe7q9\%2Fuploads\%2FVCqjLaER2OAgqUxrCI4t\%2Fimage.png?alt=media\&token=e6e20dba-cd8a-41bd-9f43-6d8851f60b8e}

\begin{enumerate}
\def\labelenumi{\arabic{enumi}.}
\setcounter{enumi}{3}
\tightlist
\item
  Tap a parameter \{Bank\} key on the Editor Toolbar.
\end{enumerate}

\includegraphics{https://files.gitbook.com/v0/b/gitbook-x-prod.appspot.com/o/spaces\%2F3kS90tLsADGm1ocbe7q9\%2Fuploads\%2FkhULoWzjDhUgDQ6bl9Oa\%2Fimage.png?alt=media\&token=d99a3186-7685-4a5d-ad56-6893dfc3953b}

\begin{enumerate}
\def\labelenumi{\arabic{enumi}.}
\setcounter{enumi}{4}
\item
  Enter the destination library number via the keypad.
\item
  Make \{Library Store Options\} changes as required.
\item
  Press {[}STORE{]} to complete the sequence.
\end{enumerate}

\textbf{A shortcut for Bank selection is to specify a number directly after pressing {[}LIBRARY{]} key:}

1 = Intensity bank

2 = Position bank

3 = Color bank

4 = Beam bank

5 = Image

6 = Shape

7 = Custom

8 = Effect

\begin{quote}
After typing the first number, the bank name will appear on the command line. The next number typed will be the library number.
\end{quote}

\textbf{Modify a Library using {[}UPDATE{]}:}

\begin{enumerate}
\def\labelenumi{\arabic{enumi}.}
\item
  Make a change to parameter value outputting to the stage from a library referenced in an active cue or scene.
\item
  Press the {[}UPDATE{]} - The Update pop-up will open. Two columns will appear. The left column shows \textbf{Cues} available for updating and the right column shows Libraries available for updating. The source library or \textbf{libraries} should be selected.
\item
  Selected libraries may be freely deselected in the Update pop-up. If all Libraries are deselected, updates will become ``hard'' values in the source cue instead of the source libraries being updated.
\item
  Press or press {[}ENTER{]} to close the pop-up and complete the update.
\end{enumerate}

Update pop-up

\textbf{Overwrite, Update, or Release fixtures in Libraries via Softkey syntax:}

\begin{enumerate}
\def\labelenumi{\arabic{enumi}.}
\item
  Make a change to parameter value outputting to the stage from a library referenced in an active cue or scene.
\item
  Press {[}STORE{]} and tap source \{Library SK\}.
\item
  The \{Cancel, Update, Overwrite, Release\} pop-up will appear on the large screen.
\item
  Tap \{Overwrite\} to replace all fixtures with those in the Editor, \{Update\} to merge the editor into the library, or \{Release\} to remove the selected fixtures from the Library.
\end{enumerate}

\begin{quote}
Releasing fixtures is only valid if the library is \textbf{Fixture Specific} or \textbf{Device + Fixture} in the case where specific fixtures have modified recorded values.
\end{quote}

\begin{quote}
Alternately the sequence {[}LIBRARY{]} \{Bank\} from the Editor Toolbar {[}\#{]} {[}STORE{]} to an existing library will work the same as above.
\end{quote}

Object Exists pop-up

Library Settings pop-up

Library Settings may be changed at any time after they are created using the Library Settings pop-up.

\textbf{There are two ways to do this:}

\begin{itemize}
\item
  Tap the Library \{SK\} you wish to modify and press {[}SETTINGS{]}. (This is invalid with Snaps and Macros as it will trigger them) Or
\item
  Press {[}Library{]} tap \{Bank\} on the Editor Toolbar Press {[}\#{]} {[}SETTINGS{]} - The Settings pop-up will also open.

  \begin{itemize}
  \item
    If Reference is selected, modifying the library will modify the values of ALL cues that reference it.
  \item
    If Timing is selected, any parameter cell times in the Editor will be stored with the library.
  \end{itemize}
\end{itemize}

\textbf{Text labels may be assigned at any time there is NOT an active selection in the Editor. Any of the following methods are valid:}

\begin{itemize}
\item
  For Library Softkeys, Tap a Library \{SK\} - Start typing on the pull-out keyboard. The Text Entry pop-up will appear.
\item
  Tap a Library \{SK\} press {[}TEXT{]} - The Text Entry pop-up will appear.
\item
  Press {[}TEXT{]} tap a Library \{SK\} - The Text Entry pop-up will appear.
\item
  Press {[}Library{]} tap \{Bank\} on the Editor Toolbar Press {[}\#{]} {[}TEXT{]} - The Library Text pop-up will open.
\item
  Directly after storing a Library, press {[}TEXT{]} - Library text pop-up will open.
\end{itemize}

\textbf{Delete a library:}

\begin{itemize}
\item
  Method 1: Tap the Libraries' \{SK\} then press {[}DELETE{]}
\item
  Method 2: Keypad sequence for Bank Libraries: {[}LIBRARY{]} tap \{Bank\} on the Editor Toolbar {[}\#{]} {[}DELETE{]}
\item
  Method 3: Keypad sequence for deleting Object Libraries: {[}Object{]} {[}\#{]} {[}DELETE{]}
\end{itemize}

\begin{quote}
For most Delete functions, pressing {[}DELETE{]} {[}DELETE{]} will do the same as pressing {[}ENTER{]} or to complete the operation.
\end{quote}

\hypertarget{object-libraries}{%
\subsection{Object Libraries}\label{object-libraries}}

\textbf{See:}

\begin{itemize}
\item
  \href{https://vibemanual.compulite.com/programming-basics.html\#groups}{9.4. Groups}
\item
  \href{https://vibemanual.compulite.com/programming-cues-and-scenes.html\#what-are-qlists-cues-scenes-and-submasters}{10.1. What Are Qlists, Cues, Scenes, and Submasters?}
\item
  \href{https://vibemanual.compulite.com/filters.html\#general-filters-tab}{11.2. General Filters Tab}
\item
  \href{https://vibemanual.compulite.com/macros.html\#creating-macros}{16.1. Creating Macros}
\item
  \href{https://vibemanual.compulite.com/snaps-snapshots.html\#storing-snaps}{17.1. Storing Snaps}
\end{itemize}

\hypertarget{effects-1}{%
\section{Effects}\label{effects-1}}

This chapter discusses Vibe's two effect editors.

\textbf{The following is covered in this chapter:}

\begin{itemize}
\tightlist
\item
  \href{https://vibemanual.compulite.com/effects-1.html\#effect-basics}{13.1. Effect Basics}
\item
  \href{https://vibemanual.compulite.com/effects-1.html\#smart-screen-effects-editor}{13.2. Smart Screen Effects Editor}
\item
  \href{https://vibemanual.compulite.com/effects-1.html\#advanced-effects-editor}{13.3. Advanced Effects Editor}
\end{itemize}

\hypertarget{effect-basics}{%
\subsection{Effect Basics}\label{effect-basics}}

\begin{quote}
All Screen captures below are from the Advanced Effects Editor.
\end{quote}

Vibe \textbf{Effects} are made up of \textbf{Effect Events} running along horizontal \textbf{Effect Tracks}.

\textbf{Each Event Track must have:}

\begin{itemize}
\item
  At least one parameter assigned in the far left column. Additional parameters may be added but they will respond simultaneously to the Effect.
\item
  One Effect Event. Additional Effect Events may be added to the track each with its own time base or Effects may be merged and share a time base.
\end{itemize}

\textbf{Multi-track effects:}

\begin{itemize}
\item
  Effects requiring more than one parameter may be stacked on separate Effects Tracks.
\item
  Examples of Multi-track effects would be:

  \begin{itemize}
  \item
    CMY and RGB color effects
  \item
    Pan/Tilt Circles
  \item
    Blade effects
  \end{itemize}
\end{itemize}

\textbf{All Effect Events are built from the following components:}

\begin{itemize}
\item
  Functions
\item
  Primitives
\item
  Patterns
\end{itemize}

\includegraphics{https://files.gitbook.com/v0/b/gitbook-x-prod.appspot.com/o/spaces\%2F3kS90tLsADGm1ocbe7q9\%2Fuploads\%2FHDImbNucaxJ6HcluGgXp\%2F13.1.webp?alt=media\&token=ca946819-0a7e-43c6-8595-b2825a0341b3}

\textbf{Functions}:

\begin{itemize}
\tightlist
\item
  Are the basic building blocks of effects. They may be assigned to Primitives and Patterns.
\end{itemize}

\includegraphics{https://files.gitbook.com/v0/b/gitbook-x-prod.appspot.com/o/spaces\%2F3kS90tLsADGm1ocbe7q9\%2Fuploads\%2FuLl9MfvoPRQuRNhnVBzR\%2F13.1.1.webp?alt=media\&token=fa8592b2-d3da-4b0b-897d-69dfb2692758}

\textbf{Vibe uses functions as the raw building blocks for three features of the console:}

\begin{enumerate}
\def\labelenumi{\arabic{enumi}.}
\item
  Primitives - Main element of Effects
\item
  Profiles - Used for cue transitions
\item
  Curves - Used for dimmer behavior
\end{enumerate}

\begin{quote}
Vibe ships with a number of ``read only'' Functions. They may not be edited. Users may create their own functions in the Advanced Effect's \textbf{Function Editor}.
\end{quote}

\textbf{Primitives:}

\begin{itemize}
\item
  When Functions are linked to parameters on an Effect Track, the are called \textbf{Primitives}.
\item
  Primitives modulate parameter values based on their assigned \textbf{Functions}.
\item
  Primitives may be modified by a number of \textbf{Primitive settings}.
\end{itemize}

\includegraphics{https://files.gitbook.com/v0/b/gitbook-x-prod.appspot.com/o/spaces\%2F3kS90tLsADGm1ocbe7q9\%2Fuploads\%2FHOwdbSfAYgtO8DwPkda0\%2F13.1.png?alt=media\&token=7eadbadc-1bf2-49ac-850e-87d75cd619b8}

\textbf{Primitive Settings:}

\begin{itemize}
\item
  Time - The duration of time it takes each fixture to complete one cycle across the Primitive.
\item
  Size - The amplitude of the Primitive's function.
\item
  Base - The Y axis of start of the Primitive's function when each fixture crosses it.
\item
  In Base - Adjusts the direction from the parameters start point (Base) it is linked to the Swing \{Up\} {[}Center\} \{Down\} keys.
\end{itemize}

\begin{quote}
E.g. If the Base value of tilt is 50\% (0°) and the \textbf{In Base} is \textbf{Center (-50\%)}, tilt would split center with a sine wave primitive assigned to it. \textbf{Up (0\%)} would send the tilt from strait down to full out into the audience. \textbf{Down (-100\%)} would send the tilt from strait down to full up upstage.
\end{quote}

\textbf{Pattern:}

\begin{itemize}
\tightlist
\item
  Sets the behavior of fixture selection as it transitions across the Primitive.
\end{itemize}

\includegraphics{https://files.gitbook.com/v0/b/gitbook-x-prod.appspot.com/o/spaces\%2F3kS90tLsADGm1ocbe7q9\%2Fuploads\%2F1pz67CpZIa5EhHHpCO0f\%2F13.1.1.png?alt=media\&token=ccd3ba4f-10c3-48bb-98c5-f8db77a02e1a}

\textbf{Pattern Settings:}

\begin{itemize}
\item
  Spread (Offset) - Sets the percentage of offset between each fixture in a selection. 100\% would evenly divide up the selection's transition across the Primitive. A Spread of 0\% would have all fixtures transitioning across the Primitive in unison.
\item
  UP - Based on the UP group \#, Each UP grouping will complete a full cycle across the primitive before the next UP group can start.
\end{itemize}

\begin{quote}
e.g.~A selection of 8 fixtures assigned to a primitive with a sine wave function and a spread of 100\%, UP is set to 1, Fixture 1 will have to complete its full cycle across the Primitive before fixture 2 will begin. If the UP \# is set to 2 fixture 1 and 2 will have split the cycle before the next two fixtures start.
\end{quote}

\textbf{Effects:}

\begin{itemize}
\tightlist
\item
  The combination of Parameters, Primitives, and Patterns.
\end{itemize}

\includegraphics{https://files.gitbook.com/v0/b/gitbook-x-prod.appspot.com/o/spaces\%2F3kS90tLsADGm1ocbe7q9\%2Fuploads\%2Fj3rD9d6mWI7yabvQmX8O\%2F13.1.2.png?alt=media\&token=fc0ffa06-fa74-4740-b965-12353a5ff75b}

\textbf{Effects Settings:}

\begin{itemize}
\item
  Settings that affect both Primitives and Patterns (Effect Event) are called Effect settings.
\item
  Effect setting are integrated into the Smart Screen Editor and are available by tapping the empty space beside the Effect Event in the Advanced Editor.
\item
  Offset All - Moves the start time of the Effect Event along the X axis of the Effect Track.
\item
  Params Spread - Is a toggle button that enables the proportional offsetting of the start times of multiple Effect Tracks at the same time.
\end{itemize}

\begin{quote}
\{Params Spread\} is a quick way to make CMY or RGB effects where colors need to spread apart so as not to mix to black or white.
\end{quote}

\textbf{Effects Libraries:}

\begin{itemize}
\item
  Once effects have been created then may be directly stored in cues and scenes or saved in Effects Libraries for future use.
\item
  Effect Libraries are stored similar to normal Libraries \href{https://vibemanual.compulite.com/libraries.html\#store-and-modify-bank-libraries}{Store and Modify Bank Libraries}.
\item
  Effects libraries have one additional option to allow users to include the base values. This is useful when effects must start from an absolute value.
\end{itemize}

\hypertarget{smart-screen-effects-editor}{%
\subsection{Smart Screen Effects Editor}\label{smart-screen-effects-editor}}

The Smart Screen Effects Editor allows users to quickly build effects within the normal work flow of the console while still having access to the main live displays. \textbf{Effect Tracks} are created for every selected parameter but they are hidden to remove screen clutter. Some advance operations are removed to speed up programming of simple effects.

\textbf{Two pages are available in the Smart Screen Editor:}

\begin{itemize}
\item
  \textbf{Attributes} - Used for building Effect Events.
\item
  \textbf{Grouping} - Used for setting fixture \textbf{Blocks Of} (Fixture Grouping in Vector) \textbf{and Sub Blocks Of} (Blocks on some console).
\end{itemize}

\includegraphics{https://files.gitbook.com/v0/b/gitbook-x-prod.appspot.com/o/spaces\%2F3kS90tLsADGm1ocbe7q9\%2Fuploads\%2FIgg80rwQtyD8LqADi6WE\%2F13.2.webp?alt=media\&token=515e980f-453b-47db-80ca-1806b3d61f5d}

Smart Screen Attribute page

\hypertarget{building-effects-using-the-small-screen-editor}{%
\subsubsection{Building Effects Using the Small Screen Editor}\label{building-effects-using-the-small-screen-editor}}

\textbf{To build an Effect in the Smart Screen Effects Editor:}

\begin{enumerate}
\def\labelenumi{\arabic{enumi}.}
\item
  Select the the fixtures to be used in the Effect.
\item
  Select the \textbf{parameter/parameters} to be in the Effect \href{https://vibemanual.compulite.com/programming-basics.html\#what-are-parameters}{9.4.2. What are Parameters}
\item
  Press {[}EFFECT{]} - The Smart Screen Effect Editor will open in the \{Attributes\} display with a blank Effect Event.
\item
  {Tap the Pre Built tab and make a selection. }
\end{enumerate}

{Or }

\begin{enumerate}
\def\labelenumi{\arabic{enumi}.}
\setcounter{enumi}{4}
\item
  Tap the Favorite Primitive tab and select a Primitive - the Function shown on the Primitive will be loaded with a default linear Pattern.
\item
  The Pattern may be changed by by tapping any of the following:

  \begin{itemize}
  \item
    / Fixtures move in a positive direction across the selection.
  \item
    ~Fixture move in a negative direction across the selection.
  \item
    /~Fixture mirror in a positive direction.
  \item
    / Fixture mirror in a negative direction.
  \end{itemize}
\item
  Set the Offset \{Pattern Spread\} - Default is 100\%.

  \begin{itemize}
  \item
    100\% = Selection is divided 100\% transitioning across the Primitive.
  \item
    0\% = all fixtures transition in unison.
  \end{itemize}
\item
  Set Swing (In Base) as needed \href{https://vibemanual.compulite.com/effects-1.html\#effect-basics}{13.1. Effects Basics}
\item
  Adjust Size and Base to set the Low and High ranges of the values
\item
  Set the Rate - x.xx Rate is a multiplier of the time value. Default Effect Time is 4 seconds to complete the cycle
\end{enumerate}

\begin{quote}
Similar results to inverting the pattern functions may be achieved by reversing the selection direction with the \{⟵ Negative\}, \{Positive ⟶ \} keys
\end{quote}

\includegraphics{https://files.gitbook.com/v0/b/gitbook-x-prod.appspot.com/o/spaces\%2F3kS90tLsADGm1ocbe7q9\%2Fuploads\%2FMLJPn65GrCpdENgeBuI3\%2F13.2.1.webp?alt=media\&token=ac88df5b-06e7-4313-9407-bc348c14f3c3}

Smart Screen

\textbf{Grouping:}

\begin{quote}
For Grouping to be noticeable there must be an Pattern Offset Spread above 0\%.
\end{quote}

\begin{enumerate}
\def\labelenumi{\arabic{enumi}.}
\item
  Tap the \{Grouping\} key - The Grouping page will open.
\item
  If the Offset Pattern Spread is not above 0\%, tap the \{Wave On\} key. The Spread will be changes to 100\%.
\item
  Set the \{Blocks Of\} {[}\#{]} - This interleaves the Pattern into Groupings.

  \begin{itemize}
  \item
    The Default for \textbf{Blocks} and \textbf{Sub Block} is 1/0 - No Groupings.
  \item
    When \{Blocks Of\} is set to 1 or above, the Sub Blocks will move to 1 and Interleave Groups will be possible.
  \item
    e.g.~2/1 would interleave a selection into Odds/Evens.
  \item
    e.g.~3/1 would interleave a selection of 12 fixtures into {[}1+4+7+10{]} - {[}2+5+8+11{]} - {[}3+6+9+12{]}
  \end{itemize}
\item
  Set the \{Sub Blocks Of\} - Combines fixture in a selection, to act like single fixtures.

  \begin{itemize}
  \tightlist
  \item
    e.g.~12/2 would combine {[}1+2{]} - {[}3+4{]} - {[}5+6{]} - {[}7+8{]} - {[}9+10{]} - {[}11+12{]}
  \end{itemize}
\end{enumerate}

\begin{quote}
As a shortcut, Setting \{Sub Block Of\} back to 0 allows \{Blocks Of\} to combine fixtures instead of interleaving - 2/0 would combine {[}1+2{]} - {[}3+4{]} - {[}5+6{]} - {[}7+8{]} - {[}9+10{]} - {[}11+12{]}
\end{quote}

\begin{quote}
The Advanced Effects Editor may be opened at any time by pressing {[}Vibe{]} \protect\hyperlink{effects-1}{Effects}. All Effects running in the Smart Screen Effects Editor will be synchronized with the Advanced Editor
\end{quote}

\textbf{After programming an Effect, Close the editor using the} \includegraphics{https://files.gitbook.com/v0/b/gitbook-x-prod.appspot.com/o/spaces\%2F3kS90tLsADGm1ocbe7q9\%2Fuploads\%2FiqDovif3XQdSLCRmPgLg\%2Fimage.png?alt=media\&token=5fda6a67-cd96-4687-aa9e-78c3ae594e41} \textbf{in the upper right corner.}

\hypertarget{advanced-effects-editor}{%
\subsection{Advanced Effects Editor}\label{advanced-effects-editor}}

Both Effect Editors write to the same Effect Engine. Either or both may be used to create effects. Whichever is opened at the time will be synchronized to the Effects Engine, but only \textbf{one} Effect Editor may open at a time.

\textbf{Selecting multiple columns and rows:}

\begin{itemize}
\tightlist
\item
  It is possible to make simultaneous changes to Primitive, Pattern, and Effect settings by toggling them on/off.
\end{itemize}

\includegraphics{https://files.gitbook.com/v0/b/gitbook-x-prod.appspot.com/o/spaces\%2F3kS90tLsADGm1ocbe7q9\%2Fuploads\%2FoWHeezMvX1aI2sNpksZw\%2F13.3.webp?alt=media\&token=75cb2b00-1aca-49e9-b84b-e8e021ae43d7}

Advanced Effect Editor

\textbf{The Advanced Effects Editor differs from the Smart Screen Effects Editor in the following ways:}

\begin{itemize}
\item
  Parameter selection is made in the Effects Editor not before entering it.
\item
  Effect Tracks are shown.
\item
  Multiple Parameters may be assigned to the same Effect Track.
\end{itemize}

\includegraphics{https://files.gitbook.com/v0/b/gitbook-x-prod.appspot.com/o/spaces\%2F3kS90tLsADGm1ocbe7q9\%2Fuploads\%2FPyWoaSOHwBfutS6kimsm\%2F13.3.0.webp?alt=media\&token=10b42e8e-33f7-4e9f-952a-84e2b74c6071}

\begin{itemize}
\tightlist
\item
  Multiple Effects Events may be assigned to the same Effect Track.
\end{itemize}

\includegraphics{https://files.gitbook.com/v0/b/gitbook-x-prod.appspot.com/o/spaces\%2F3kS90tLsADGm1ocbe7q9\%2Fuploads\%2FuS5BTuPNoqsPu8GQbj2i\%2F13.3.0.1.webp?alt=media\&token=bcdb84a8-c9a5-4e62-9ec3-7139eb79ac48}

\begin{itemize}
\item
  Effect sharing the same Effect Track can uses separate patterns or share patterns.
\item
  Pattern Size, Rate, and Base may be spread to create randomized looking effects.
\item
  Repeat - The number of times each Effect Event repeats before moving on the the next Effects Event, when they are sharing an Effects Track.
\end{itemize}

Building Effects in the Advanced Editor differs from the Smart Screen Editor in that a only a fixture selection is needed before entering the editor. Parameters are selected using the \{Filters\} and \{Choose Parameter\} . This takes a bit longer but is more flexible.

\includegraphics{https://files.gitbook.com/v0/b/gitbook-x-prod.appspot.com/o/spaces\%2F3kS90tLsADGm1ocbe7q9\%2Fuploads\%2F5ShDQJRdKbXuqv2ZINAc\%2F13.3.0.2.webp?alt=media\&token=108c475e-6e16-44a3-a6da-e9d4808706f7}

\hypertarget{build-a-basic-effect-in-the-advanced-effects-editor}{%
\subsubsection{Build a basic Effect in the Advanced Effects Editor}\label{build-a-basic-effect-in-the-advanced-effects-editor}}

\textbf{Build a basic Effect in the Advanced Effects Editor:}

\begin{enumerate}
\def\labelenumi{\arabic{enumi}.}
\item
  Select the fixtures to be used in the Effect.
\item
  Press {[}VIBE{]}+{[}EFFECT{]} to open the Advanced Effect Editor.
\item
  Tap a parameter bank in the \textbf{FIiters} area.
\item
  Tap a parameter in the \textbf{Choose Parameter} area - the parameter background will turn purple.
\item
  Tap a blank space in the Parameter Column of an unassigned Effect Track
\end{enumerate}

\begin{itemize}
\tightlist
\item
  the parameter will be assigned and the \textbf{Attributes Column}.
\end{itemize}

\includegraphics{https://files.gitbook.com/v0/b/gitbook-x-prod.appspot.com/o/spaces\%2F3kS90tLsADGm1ocbe7q9\%2Fuploads\%2FX6HquT8ZjyknVOW4hFaV\%2F13.3.1.webp?alt=media\&token=41203d4f-ce06-4c05-9a58-f9c18ff91f44}

\begin{enumerate}
\def\labelenumi{\arabic{enumi}.}
\setcounter{enumi}{5}
\tightlist
\item
  Tap a \textbf{Function} from the Favorites tab or All Functions tab - The Function background will turn purple.
\end{enumerate}

\includegraphics{https://files.gitbook.com/v0/b/gitbook-x-prod.appspot.com/o/spaces\%2F3kS90tLsADGm1ocbe7q9\%2Fuploads\%2FCtz0h7EygEwdVYVqGu8y\%2F13.3.1.2.webp?alt=media\&token=8d0ba766-063c-4191-b479-4b9f3a720e58}

\begin{enumerate}
\def\labelenumi{\arabic{enumi}.}
\setcounter{enumi}{6}
\item
  Tap a blank space in the \textbf{Effect Event Column} - an Effect Event will be created with a Primitive and Pattern .
\item
  Adjust Pattern and Primitive settings - See \href{https://vibemanual.compulite.com/effects-1.html\#effect-basics}{13.1. Effects Basics}
\item
  Close the Advanced editor using the \includegraphics{https://files.gitbook.com/v0/b/gitbook-x-prod.appspot.com/o/spaces\%2F3kS90tLsADGm1ocbe7q9\%2Fuploads\%2FiqDovif3XQdSLCRmPgLg\%2Fimage.png?alt=media\&token=5fda6a67-cd96-4687-aa9e-78c3ae594e41} in the upper right corner.
\item
  Store final Effect directly to a cue/scene or to the Effects Library.
\end{enumerate}

\textbf{Grouping:} See \href{https://vibemanual.compulite.com/effects-1.html\#building-effects-using-the-small-screen-editor}{13.2.1. Building Effects Using the Small Screen Editor}

\textbf{Blocks by Groups:} Groups may be substituted for individual fixtures using this option.

\begin{enumerate}
\def\labelenumi{\arabic{enumi}.}
\item
  Select Groups in the order you wish the effect to use.
\item
  Build an effect as listed above.
\item
  Tap in the blank space beside the Effect Event, the blank space will highlight in blue and the \textbf{Effect Settings} will appear.
\item
  Toggle \{Blocks By Groups on\} - Make sure \{Sub Blocks Of\} is set to 0 or the groups will act as interleave instead of blocks.
\end{enumerate}

\textbf{Combining Parameters:}

\begin{enumerate}
\def\labelenumi{\arabic{enumi}.}
\item
  Tap a parameter bank in the FIiters area.
\item
  Tap a parameter in the Choose Parameter area - the parameter background will turn purple.
\item
  Tap a space in the Parameter Column already containing a parameter - the parameter will be merged into the parameter. column - All Primitives and Patterns will affect both parameters.
\end{enumerate}

\textbf{Stacking Effects Events:}

\begin{enumerate}
\def\labelenumi{\arabic{enumi}.}
\item
  Tap a Function from the Favorites tab or All Functions tab - The Function background will turn purple.
\item
  Tap the space to the right of an existing Effect Even - A new Effect Event will be stacked beside the existing one.
\end{enumerate}

\begin{itemize}
\tightlist
\item
  Each Stacked Effect Event can have its own Primitive and Pattern.
\end{itemize}

\textbf{Combining Effects Events:}

\begin{enumerate}
\def\labelenumi{\arabic{enumi}.}
\tightlist
\item
  Drag and drop the primitive of an Effect Event over an adjacent primitive
\end{enumerate}

\begin{itemize}
\tightlist
\item
  the primitives will now be combined but will \textbf{share} the pattern settings.
\end{itemize}

\textbf{Duplicate an Effects Event:}

\begin{enumerate}
\def\labelenumi{\arabic{enumi}.}
\item
  Tap either the pattern or primitive are of an Effect Event - The background of the pattern or primitive will turn red and the Effects Toolbar will open under the Effects Tracks.
\item
  Tap \{+ Duplicate\} - A duplicate of the Effect Event will be added to the right of the source Effect Event.
\end{enumerate}

\textbf{Copy an Effect Track:} Copies Effect Event settings to a new parameter (synchronize).

\begin{enumerate}
\def\labelenumi{\arabic{enumi}.}
\item
  Tap a parameter bank in the \textbf{FIiters} area.
\item
  Tap a parameter in the \textbf{Choose Parameter} area - the parameter background will turn purple.
\item
  Tap a blank space in the Parameter Column below an assigned Effect Track - the parameter will be assigned.
\item
  Press the empty area to the right of the source Effect Event - The blank area will turn light blue and the Effects Toolbar will open under the Effects Tracks.
\item
  Tap \{Copy\} - The source Effect Event's settings will be copied.
\item
  Tap in the blank area to the right of the new parameter - The area will turn blue and the Effects Toolbar will open under the Effects Tracks.
\item
  Tap \{Paste\} - The source Effect Event settings will be pasted to the new parameter.
\end{enumerate}

\textbf{Delete an Effect Event:}

\begin{enumerate}
\def\labelenumi{\arabic{enumi}.}
\item
  Tap either the pattern or primitive of the event to be deleted - the background will turn red.
\item
  Tap \{Delete\} on the Effects Toolbar below the Effect Tracks.
\end{enumerate}

\textbf{Release a parameter from an effect:} Effect Parameters may be released using Edit Cue or Update pop-up.

\begin{itemize}
\tightlist
\item
  \textbf{Using Edit Cue:}
\end{itemize}

\begin{quote}
{Cue must be active on a controller or it will be loaded to the Master Controller and bump its current Qlist out.}
\end{quote}

\begin{enumerate}
\def\labelenumi{\arabic{enumi}.}
\item
  {[}QLIST{]} {[}\#{]} {[}CUE{]} {[}\#{]} {[}EDIT{]} - Cue values will be placed in the Editor and the {[}UPDATE{]} key will flash red.
\item
  Press {[}VIBE{]}+{[}EFFECT{]} - The Advanced Effect Editor will open.
\item
  Select the fixtures requiring parameters to be released.
\item
  Tap the parameter or parameters in the parameter column to select the parameters - the parameters will turn purple.
\item
  Tap \{Release\} on the Effects Toolbar - Parameters will disappear from the Effect Editor - stage values will be returned to their recorded base values or home values if they had no programmed base values.
\item
  Press {[}UPDATE{]} to complete the edit and restore control to the source controller.
\end{enumerate}

\begin{itemize}
\tightlist
\item
  \textbf{Using Update pop-up:} Cue being edited must be active on a controller.
\end{itemize}

\begin{enumerate}
\def\labelenumi{\arabic{enumi}.}
\item
  Select the fixtures requiring parameters to be released.
\item
  Press the the parameter's push wheel, or tap the wheel display above the wheel, to select the parameter.
\item
  Press {[}VIBE{]}+{[}EFFECT{]} - opens Advanced Editor.
\item
  Tap the parameter or parameters in the parameter column to select the parameters - the parameters will turn purple.
\item
  Tap \{Release\} on the Effects Toolbar - Parameters will disappear from the Effect Editor - Stage values will be returned to their recorded base values or home values if they had no programmed base values.
\item
  Press {[}UPDATE{]} - The Update pop-up will open.
\item
  Make sure the proper cue is selected in the Cue column.
\item
  Tap or press {[}ENTER{]} - Completes the update operation.
\end{enumerate}

\hypertarget{building-effects-using-the-advanced-effects-editor}{%
\subsubsection{Building Effects Using the Advanced Effects Editor}\label{building-effects-using-the-advanced-effects-editor}}

13.3.1. Building Effects Using the Advanced Effects Editor

\textbf{Build a basic Effect in the Advanced Effects Editor:}

\begin{enumerate}
\def\labelenumi{\arabic{enumi}.}
\item
  Select the fixtures to be used in the Effect.
\item
  Press {[}VIBE{]}+{[}EFFECT{]} to open the Advanced Effect Editor.
\item
  Tap a parameter bank in the FIiters area.
\item
  Tap a parameter in the Choose Parameter area - the parameter background will turn purple.
\item
  Tap a blank space in the Parameter Column of an unassigned Effect Track
\end{enumerate}

\begin{itemize}
\tightlist
\item
  the parameter will be assigned and the Attributes Column.
\end{itemize}

\includegraphics{https://files.gitbook.com/v0/b/gitbook-x-prod.appspot.com/o/spaces\%2F3kS90tLsADGm1ocbe7q9\%2Fuploads\%2FX6HquT8ZjyknVOW4hFaV\%2F13.3.1.webp?alt=media\&token=41203d4f-ce06-4c05-9a58-f9c18ff91f44}

\begin{enumerate}
\def\labelenumi{\arabic{enumi}.}
\setcounter{enumi}{5}
\tightlist
\item
  Tap a Function from the Favorites tab or All Functions tab - The Function background will turn purple.
\end{enumerate}

\begin{enumerate}
\def\labelenumi{\arabic{enumi}.}
\setcounter{enumi}{6}
\item
  Tap a blank space in the Effect Event Column - an Effect Event will be created with a Primitive and Pattern .
\item
  Adjust Pattern and Primitive settings - See 13.1. Effects Basics
\item
  Close the Advanced editor using the \includegraphics{https://files.gitbook.com/v0/b/gitbook-x-prod.appspot.com/o/spaces\%2F3kS90tLsADGm1ocbe7q9\%2Fuploads\%2FiqDovif3XQdSLCRmPgLg\%2Fimage.png?alt=media\&token=5fda6a67-cd96-4687-aa9e-78c3ae594e41} in the upper right corner.
\end{enumerate}

10 Store final Effect directly to a cue/scene or to the Effects Library.

\textbf{Grouping:}

See \href{https://vibemanual.compulite.com/effects-1.html\#building-effects-using-the-small-screen-editor}{13.2.1. Building Effects Using the Small Screen Editor}

\textbf{Blocks by Groups:} - Groups may be substituted for individual fixtures using this option.

\begin{enumerate}
\def\labelenumi{\arabic{enumi}.}
\item
  Select Groups in the order you wish the effect to use.
\item
  Build an effect as listed above.
\item
  Tap in the blank space beside the Effect Event, the blank space will highlight in blue and the \textbf{Effect Settings} will appear.
\item
  Toggle \{Blocks By Groups on\} - Make sure \{Sub Blocks Of\} is set to 0 or the groups will act as interleave instead of blocks.
\end{enumerate}

\textbf{Combining Parameters:}

\begin{enumerate}
\def\labelenumi{\arabic{enumi}.}
\item
  Tap a parameter bank in the FIiters area.
\item
  Tap a parameter in the Choose Parameter area - the parameter background will turn purple.
\item
  Tap a space in the Parameter Column already containing a parameter - the parameter will be merged into the parameter. column - All Primitives and Patterns will affect both parameters.
\end{enumerate}

\textbf{Stacking Effects Events:}

\begin{enumerate}
\def\labelenumi{\arabic{enumi}.}
\item
  Tap a Function from the Favorites tab or All Functions tab - The Function background will turn purple.
\item
  Tap the space to the right of an existing Effect Even - A new Effect Event will be stacked beside the existing one.
\end{enumerate}

\begin{itemize}
\tightlist
\item
  Each Stacked Effect Event can have its own Primitive and Pattern.
\end{itemize}

\textbf{Combining Effects Events:}

\begin{enumerate}
\def\labelenumi{\arabic{enumi}.}
\tightlist
\item
  Drag and drop the primitive of an Effect Event over an adjacent primitive
\end{enumerate}

\begin{itemize}
\tightlist
\item
  the primitives will now be combined but will \textbf{share} the pattern settings.
\end{itemize}

\textbf{Duplicate an Effects Event:}

\begin{enumerate}
\def\labelenumi{\arabic{enumi}.}
\item
  Tap either the pattern or primitive are of an Effect Event - The background of the pattern or primitive will turn red and the Effects Toolbar will open under the Effects Tracks.
\item
  Tap \{+ Duplicate\} - A duplicate of the Effect Event will be added to the right of the source Effect Event.
\end{enumerate}

\textbf{Copy an Effect Track:} Copies Effect Event settings to a new parameter (synchronize).

\begin{enumerate}
\def\labelenumi{\arabic{enumi}.}
\item
  Tap a parameter bank in the \textbf{FIiters} area.
\item
  Tap a parameter in the \textbf{Choose Parameter} area - the parameter background will turn purple.
\item
  Tap a blank space in the Parameter Column below an assigned Effect Track - the parameter will be assigned.
\item
  Press the empty area to the right of the source Effect Event - The blank area will turn light blue and the Effects Toolbar will open under the Effects Tracks.
\item
  Tap \{Copy\} - The source Effect Event's settings will be copied.
\item
  Tap in the blank area to the right of the new parameter - The area will turn blue and the Effects Toolbar will open under the Effects Tracks.
\item
  Tap \{Paste\} - The source Effect Event settings will be pasted to the new parameter.
\end{enumerate}

\textbf{Delete an Effect Event:}

\begin{enumerate}
\def\labelenumi{\arabic{enumi}.}
\item
  Tap either the pattern or primitive of the event to be deleted - the background will turn red.
\item
  Tap \{Delete\} on the Effects Toolbar below the Effect Tracks.
\end{enumerate}

\textbf{Release a parameter from an effect:} Effect Parameters may be released using Edit Cue or Update pop-up.

\begin{itemize}
\tightlist
\item
  \textbf{Using Edit Cue:}
\end{itemize}

\begin{quote}
Cue must be active on a controller or it will be loaded to the Master Controller and bump its current Qlist out.
\end{quote}

\begin{enumerate}
\def\labelenumi{\arabic{enumi}.}
\item
  {[}QLIST{]} {[}\#{]} {[}CUE{]} {[}\#{]} {[}EDIT{]} - Cue values will be placed in the Editor and the {[}UPDATE{]} key will flash red.
\item
  Press {[}VIBE{]}+{[}EFFECT{]} - The Advanced Effect Editor will open.
\item
  Select the fixtures requiring parameters to be released.
\item
  Tap the parameter or parameters in the parameter column to select the parameters - the parameters will turn purple.
\item
  Tap \{Release\} on the Effects Toolbar - Parameters will disappear from the Effect Editor - stage values will be returned to their recorded base values or home values if they had no programmed base values.
\item
  Press {[}UPDATE{]} to complete the edit and restore control to the source controller.
\end{enumerate}

\begin{itemize}
\tightlist
\item
  \textbf{Using Update pop-up:} Cue being edited must be active on a controller.
\end{itemize}

\begin{enumerate}
\def\labelenumi{\arabic{enumi}.}
\item
  Select the fixtures requiring parameters to be released.
\item
  Press the the parameter's push wheel, or tap the wheel display above the wheel, to select the parameter.
\item
  Press {[}VIBE{]}+{[}EFFECT{]} - opens Advanced Editor.
\item
  Tap the parameter or parameters in the parameter column to select the parameters - the parameters will turn purple.
\item
  Tap \{Release\} on the Effects Toolbar - Parameters will disappear from the Effect Editor - Stage values will be returned to their recorded base values or home values if they had no programmed base values.
\item
  Press {[}UPDATE{]} - The Update pop-up will open.
\item
  Make sure the proper cue is selected in the Cue column.
\item
  Tap or press {[}ENTER{]} - Completes the update operation.
\end{enumerate}

\hypertarget{playback-controllers}{%
\section{Playback Controllers}\label{playback-controllers}}

Cues and Scenes are played back using the 15 \textbf{Motorized Slider Controllers}, non-motorized \textbf{A/B Dipless Crossfader} , 15 \textbf{Qkey Controllers}, 20 \textbf{Auxiliary Qkey Controllers}, and 5 non-motorized \textbf{Global Slider Controllers}.

\textbf{The following is covered in this chapter:}

\begin{itemize}
\tightlist
\item
  \href{https://vibemanual.compulite.com/playing-back-cues-and-scenes.html\#controller-priority-logic}{14.1. Controller Priority Logic}
\item
  \href{https://vibemanual.compulite.com/playing-back-cues-and-scenes.html\#controller-overview}{14.2. Controller Overview}
\item
  \href{https://vibemanual.compulite.com/playing-back-cues-and-scenes.html\#master-controller-1}{14.3. Master Controller}
\item
  \href{https://vibemanual.compulite.com/playing-back-cues-and-scenes.html\#goto-and-load}{14.4. GOTO and LOAD}
\item
  \href{https://vibemanual.compulite.com/playing-back-cues-and-scenes.html\#configuring-controllers}{14.5. Configuring Controllers}
\item
  \href{https://vibemanual.compulite.com/playing-back-cues-and-scenes.html\#controller-color-codes}{14.6. Controller Color Codes}
\item
  \href{https://vibemanual.compulite.com/playing-back-cues-and-scenes.html\#working-with-controllers}{14.7. Working With controllers}
\item
  \href{https://vibemanual.compulite.com/playing-back-cues-and-scenes.html\#controller-release-and-free}{14.8. Controller Release and Free}
\item
  \href{https://vibemanual.compulite.com/playing-back-cues-and-scenes.html\#controller-qlist-properties}{14.9. Controller Qlist Properties}
\item
  \href{https://vibemanual.compulite.com/playing-back-cues-and-scenes.html\#controller-modes}{14.10. Controller Modes}
\item
  \href{https://vibemanual.compulite.com/playing-back-cues-and-scenes.html\#rate-and-teach-keys}{14.11. Rate and Teach Keys}
\end{itemize}

\hypertarget{controller-priority-logic}{%
\subsection{Controller Priority Logic}\label{controller-priority-logic}}

\textbf{Controller Priority:}

\begin{itemize}
\item
  Vibe keeps an LTP priority list of which controllers own parameters and the order in which they were asserted. Every time a controller is turned on, its parameters are asserted and the controller moves to the top of the list. Parameters that are still on active controllers but are currently overridden are said to be ``robbed''. When the controller that owns the parameter is released from the stage, the last active controller to own the robbed parameter is now moved back to the top of the list and reasserted the parameter to the stage.
\item
  This feature is very useful for ``On the Fly'' programming where a main Qlist may be temporarily overridden by Qkeys or Controllers and quickly restored when they are released.
\end{itemize}

\begin{quote}
{The Editor is always the highest propriety and will override any values from controllers until they are released or The Editor is reset}.
\end{quote}

\textbf{Priority Groups:}

\begin{itemize}
\item
  Controllers assigned the same Priority Group number will be LTP amongst themselves.
\item
  Controllers assigned higher Propriety Group numbers cannot be overridden (``robbed'') by controllers with lower numbers.
\end{itemize}

\textbf{Assigning Controllers to priority Groups:}
1. Tap the top area of a controller display, the Settings pop-up will open.
2. Select the \{Controller Settings\} Tab.

OR

\begin{enumerate}
\def\labelenumi{\arabic{enumi}.}
\tightlist
\item
  Press {[}SETTINGS{]} {[}HERE{]} to any controller button, the Settings pop-up will open.
\item
  Select the \{Controller Settings\} Tab.
\item
  Rotate the virtual wheel to the desired Priority Group number or tap the center of the wheel until it turns red and type a Priority Group number from 0 - 999.
\end{enumerate}

\includegraphics{https://files.gitbook.com/v0/b/gitbook-x-prod.appspot.com/o/spaces\%2F3kS90tLsADGm1ocbe7q9\%2Fuploads\%2FsqtPJBuklsoRMifhcbTf\%2F14.1.webp?alt=media\&token=a0a4963b-265b-4915-aadc-d350a26ef118}

\hypertarget{controller-overview}{%
\subsection{Controller Overview}\label{controller-overview}}

\textbf{Controllers are used to playback or execute a number of objects:}

\begin{itemize}
\item
  Cues
\item
  Scenes
\item
  Group Submasters
\item
  Libraries
\item
  {Macros}
\item
  {Snaps }
\end{itemize}

\textbf{Vibe has five controller types:}

\begin{enumerate}
\def\labelenumi{\arabic{enumi}.}
\item
  Three button Motorized Slider Controllers
\item
  Single button Qkey Controllers
\item
  Single Auxiliary Qkey Controllers
\item
  Non-Motorized Global Slider Controllers
\item
  A/B ``theatre style'' A/B dipless crossfader pair
\end{enumerate}

\begin{quote}
Vibe software makes \textbf{no} distinction between physical and virtual controllers. This allows the system hardware to be scaled. Below is a list of total quantities of physical and virtual controllers that are supported.
\end{quote}

\textbf{Controller quantities and numbering:} Physical and Virtual

\begin{itemize}
\item
  30 pages of 100 Motorized Sliders - Numbered 1.1 ⟶ 30.100
\item
  30 pages of 100 Qkeys - Numbered 1.101 ⟶ 30.200
\item
  30 pages of 100 Auxiliary Qkeys - Numbered 1.201 ⟶ 30.300
\item
  100 Non-motorized Global Sliders - Numbered 301 ⟶ 400
\item
  1 Non-motorized A/B dipless crossfader - Numbered 401
\end{itemize}

\begin{quote}
Controllers are numbered from right to left to facilitate continuous numbering with physical and virtual controller wings.
\end{quote}

\textbf{Compound Controllers:}

\begin{itemize}
\tightlist
\item
  \textbf{Controllers may be extended in width to make larger compound controllers.}
  This is useful in creating additional crossfader pairs or adding controls for rate and effects.
  There is no limit as to how wide a controller may be, but for practical purposes, a limit of 6 wide is suggested.
\end{itemize}

\includegraphics{https://files.gitbook.com/v0/b/gitbook-x-prod.appspot.com/o/spaces\%2F3kS90tLsADGm1ocbe7q9\%2Fuploads\%2F06WOI4jVCzRUG1Op7xtO\%2F14.2.webp?alt=media\&token=40d4b958-9e44-4fb1-8fac-080edda51ca2}

\textbf{A/B Cross Faders:}

\begin{itemize}
\item
  An A/B Cross Fader Pair is provided for manual ``Theater Style'' playback of cues.
\item
  By default, the fader on the left will control dimmer parameters fading \textbf{in.}
\item
  By default, the fader on the right will control all other values including the dimmer that is fading \textbf{out.}
\item
  Parameter values that match will not dip even if the A and B are at different levels.
\end{itemize}

\hypertarget{master-controller-1}{%
\subsection{Master Controller}\label{master-controller-1}}

\textbf{Master Controller:}

\begin{itemize}
\tightlist
\item
  Any single Controller may be linked to the Master Controller by Pressing \textbf{{[}SELECT{]} {[}HERE{]}} to any of the buttons of the controller.
\end{itemize}

\includegraphics{https://files.gitbook.com/v0/b/gitbook-x-prod.appspot.com/o/spaces\%2F3kS90tLsADGm1ocbe7q9\%2Fuploads\%2FnvgTSy6VcG0Vk2ScI1UI\%2F14.3.webp?alt=media\&token=1209fc8b-b080-408c-8f8f-3623e27e799b}

\begin{itemize}
\item
  {[}GO{]} - Initiates a forward transition from one cue to another using \textbf{Cue Time}.
\item
  {[}BACK{]} - Initiates a transition to the previous cue using system \textbf{Default Back Time}.
\item
  {[}HOLD{]} - Pauses a cue in transition. Pressing {[}HOLD{]} again resumes the transition. Pressing {[}GO{]} will resume the transition as well. Pressing {[}BACK{]} will fade to the previous cue.
\end{itemize}

\begin{quote}
{The A/B is only linked to the Master Controller if the {[}SELECT{]} is assigned to A/B.}.
\end{quote}

\hypertarget{goto-and-load}{%
\subsection{GOTO and LOAD}\label{goto-and-load}}

GOTO and LOAD are used to crossfade to cues out of sequence. By default, GOTO immediately transitions to the specified cue in the system \textbf{GOTO Time}. LOAD preloads controllers. The next {[}GO{]} will transition to the preloaded cue in \textbf{Cue Time}.

\textbf{GOTO Commands: GOTO a cue on the Master Controller using system GOTO Time:}

\begin{itemize}
\tightlist
\item
  {[}CUE{]} {[}\#{]} {[}GOTO{]}
\end{itemize}

\textbf{GOTO a cue on the Master Controller using a specified Time:}

\begin{itemize}
\tightlist
\item
  {[}CUE{]} {[}\#{]} {[}TIME{]} - Enter time value {[}GOTO{]}
\end{itemize}

\textbf{GOTO a cue on the Master Controller using system Cue Time:}

\begin{itemize}
\tightlist
\item
  {[}CUE{]} {[}\#{]} {[}TIME{]} {[}GOTO{]}
\end{itemize}

\textbf{GOTO a cue on the Master Controller using Goto Cue pop-up:}
\includegraphics{https://files.gitbook.com/v0/b/gitbook-x-prod.appspot.com/o/spaces\%2F3kS90tLsADGm1ocbe7q9\%2Fuploads\%2FUx0DT82VAZ384ULKAWIf\%2F14.4.webp?alt=media\&token=fc1a8fba-8a68-4301-99f1-866ce553eeb7}

\begin{enumerate}
\def\labelenumi{\arabic{enumi}.}
\item
  Press {[}GOTO{]} - Goto Cue pop-up will open.
\item
  Select \{Qlist\} if not the current Qlist on the Master Controller.
\item
  Select the \{Cue\}.
\item
  Select behaviour.
\end{enumerate}

\begin{itemize}
\item
  \{Default GOTO time\} - Uses system default GOTO Time.
\item
  \{Cue Time\} - Uses Cue's recorded time.
\item
  \{Custom\} - Set GOTO time using the Custom Time wheel or tap centre of the wheel until it turns red and set time value from the keypad.
\item
  \{Release Master PB on Top\} - Releases the Master Controller.
\end{itemize}

\begin{enumerate}
\def\labelenumi{\arabic{enumi}.}
\tightlist
\item
  Press the \href{image.png}{} or press {[}ENTER{]} to close the pop-up and execute the GOTO.
\end{enumerate}

\begin{quote}
{If a Qlist other than the one that is currently on the Master Controller is selected, that Qlist will replace the current one.}.
\end{quote}

\textbf{GOTO a cue NOT on the Master Controller:}

\begin{itemize}
\tightlist
\item
  {[}QLIST{]} {[}\#{]} {[}CUE{]} {[}\#{]} {[}GO{]} (Controller must have a GO button assigned).
\end{itemize}

\textbf{GOTO Cue Zero -} Cue Zero is essential \{Release Master PB to Top\}.

\begin{itemize}
\item
  {[}CUE{]} {[}0{]} {[}GOTO{]} \textbf{It is also possible to go out of sequence directly to a Cue on the Master Controller (in cue time):}
\item
  {[}CUE{]} {[}\#{]} {[}GO{]}
\end{itemize}

\textbf{Load Commands:} \textbf{Load a cue on the Master Controller} - Method 1.

\begin{enumerate}
\def\labelenumi{\arabic{enumi}.}
\item
  {[}CUE{]} {[}\#{]} {[}LOAD{]} {[}HERE{]} - To Controller assigned as {[}SELECT{]}
\item
  Press {[}GO{]} - On Controller assigned as {[}SELECT{]}
\end{enumerate}

\textbf{Load a cue on the Master Controller} - Method 2.

\begin{enumerate}
\def\labelenumi{\arabic{enumi}.}
\item
  {{[}CUE{]} {[}\#{]} {[}LOAD{]} {[}LOAD{]}}
\item
  {Press {[}GO{]} - Using Master Controller GO or GO of controller assigned as {[}SELECT{]} }
\end{enumerate}

\textbf{LOAD a cue NOT on the Master Controller:}

\begin{enumerate}
\def\labelenumi{\arabic{enumi}.}
\item
  {[}QLIST{]} {[}\#{]} {[}CUE{]} {[}\#{]} {[}LOAD{]} {[}HERE{]} - Qlist and Cue will be preloaded.
\item
  Press {[}GO{]} - Controller will execute loaded cue.
\end{enumerate}

\begin{quote}
If a Qlist other than the one that is currently on the Controller is specified, that Qlist will replace the current one with the specified cue pending.
\end{quote}

\hypertarget{configuring-controllers}{%
\subsection{Configuring Controllers}\label{configuring-controllers}}

Controller buttons and sliders may be customized to allow maximum flexibility. When Qlists and Scenes are initially assigned to Controllers, the buttons are configured as per the \textbf{System Settings defaults} for Controller and Controller Action. Defaults are accessed by tapping the System Status area at the far right side of the controller display.

\hspace{0pt}

\includegraphics{https://files.gitbook.com/v0/b/gitbook-x-prod.appspot.com/o/spaces\%2F3kS90tLsADGm1ocbe7q9\%2Fuploads\%2FtcSvDf3MUkoMFKVwYlq6\%2F14.5\%20system\%20status.png?alt=media\&token=1a6f0e8b-cc31-462b-ae8f-68180f3bbbff}

\hspace{0pt}

\textbf{Assigning Buttons and Slider Behavior:}

\begin{enumerate}
\def\labelenumi{\arabic{enumi}.}
\tightlist
\item
  Press {[}SETTINGS{]} {[}HERE{]} to any button of the controller that has an object assigned to it - The \textbf{Slider Definitions} pop-up will appear with the \textbf{\{Controller Definitions\}} tab selected.
\end{enumerate}

\begin{quote}
Alternatively, tap any of the controller \{function labels\} on the controller display on the monitor above the Scene Slider.
\end{quote}

\includegraphics{https://files.gitbook.com/v0/b/gitbook-x-prod.appspot.com/o/spaces\%2F3kS90tLsADGm1ocbe7q9\%2Fuploads\%2F0Grl8Q74Ts6ZFuDbvduY\%2F14.5\%201.png?alt=media\&token=d3d44aa4-4757-4921-a53b-0d2b33a80be2}

\begin{quote}
{If a Qlist other than the one that is currently on the Master Controller is selected, that Qlist will replace the current one.}
\end{quote}

\begin{enumerate}
\def\labelenumi{\arabic{enumi}.}
\setcounter{enumi}{1}
\item
  Tap one of the three assignable \{Buttons\} or the \{Slider\} button in the centre - A context sensitive drop down menu based on the object assigned and the specific button selected will appear.
\item
  Choose a Control behaviour from the menu.
\end{enumerate}

\includegraphics{https://files.gitbook.com/v0/b/gitbook-x-prod.appspot.com/o/spaces\%2F3kS90tLsADGm1ocbe7q9\%2Fuploads\%2Fj6JDjNRFb7TWGRNNO9Y0\%2F14.5.2.webp?alt=media\&token=9639a28e-b8fe-40cc-a602-83092ac7d617}

\textbf{Assignable Button Behavior:}

\includegraphics{https://files.gitbook.com/v0/b/gitbook-x-prod.appspot.com/o/spaces\%2F3kS90tLsADGm1ocbe7q9\%2Fuploads\%2FFwTYO2QIsI5mcbflwK5f\%2F14.5.3.webp?alt=media\&token=618ef47e-5d30-429a-b872-dd6077df7b85}

\textbf{Assignable Slider Behavior:}

\includegraphics{https://files.gitbook.com/v0/b/gitbook-x-prod.appspot.com/o/spaces\%2F3kS90tLsADGm1ocbe7q9\%2Fuploads\%2Fm5ysy6BoE6l3FCWVHrYn\%2F14.5.4.webp?alt=media\&token=f6271bd6-0fe1-4576-ae64-f237d372b124}

\begin{quote}
Default settings for controllers assigned Qlists, Scenes, and Submasters may be set in the Defaults section of the System Settings Pop-up. See \href{https://vibemanual.compulite.com/system-default-settings.html\#controller-settings}{7.5.1. Controller Settings}.
\end{quote}

\hypertarget{controller-color-codes}{%
\subsection{Controller Color Codes}\label{controller-color-codes}}

\textbf{Controller Live Display color coding If {[}SELECT{]} Master Controller:}

\includegraphics{https://files.gitbook.com/v0/b/gitbook-x-prod.appspot.com/o/spaces\%2F3kS90tLsADGm1ocbe7q9\%2Fuploads\%2FYuCloo8lxuLBVkc73yQs\%2F14.6.webp?alt=media\&token=146cb343-7101-490b-a1b4-e1139a436896}

\textbf{If from a non-{[}Select{]} Controller:}

\includegraphics{https://files.gitbook.com/v0/b/gitbook-x-prod.appspot.com/o/spaces\%2F3kS90tLsADGm1ocbe7q9\%2Fuploads\%2FmWdkTDP2PrAbs3XAz1Ep\%2F14.6.1.webp?alt=media\&token=6e23d42c-cbff-4ce8-86e2-a7246be52d3f}

\textbf{Controller Display and Backlit Controller Key's color coding:}
\includegraphics{https://files.gitbook.com/v0/b/gitbook-x-prod.appspot.com/o/spaces\%2F3kS90tLsADGm1ocbe7q9\%2Fuploads\%2F2xhFbKiKqTiXNa4FPjKN\%2F14.6.2.webp?alt=media\&token=b8942fb1-76cd-44e4-a1be-83b25e1f8bda}

\hypertarget{working-with-controllers}{%
\subsection{Working With controllers}\label{working-with-controllers}}

\textbf{Controller types:} - Detailed

\includegraphics{https://files.gitbook.com/v0/b/gitbook-x-prod.appspot.com/o/spaces\%2F3kS90tLsADGm1ocbe7q9\%2Fuploads\%2FPeyRlEeHQvF5HuUF6hYJ\%2F14.7.webp?alt=media\&token=fa28a2ea-7020-4b6b-9600-666da021144b}

\includegraphics{https://files.gitbook.com/v0/b/gitbook-x-prod.appspot.com/o/spaces\%2F3kS90tLsADGm1ocbe7q9\%2Fuploads\%2FUS9LD2Ch0ruVmweAvB6p\%2F14.7.1.webp?alt=media\&token=2c8f4681-8eb8-47cc-bc60-48e1b0139624}

\includegraphics{https://files.gitbook.com/v0/b/gitbook-x-prod.appspot.com/o/spaces\%2F3kS90tLsADGm1ocbe7q9\%2Fuploads\%2Fiul8UK0l8I5GRkUgjFt0\%2F14.7.3.webp?alt=media\&token=f8b2c063-b524-43ab-9202-a9f27287647c}

\textbf{Control Keys:} Provide ``On the Fly'' access to button functions not directly assigned to controllers. Apply a Controller Key:

\begin{enumerate}
\def\labelenumi{\arabic{enumi}.}
\item
  Press a \{Control Key\}
\item
  Press any destination {[}Controller Button{]} to apply the Control Key's function
\end{enumerate}

\begin{quote}
In most cases, double press of the Control Keys locks them on until pressed again.
\end{quote}

\textbf{Control Key Functions:}

\begin{itemize}
\tightlist
\item
  \textbf{FLASH} - Momentarily turns on a Controller.
\end{itemize}

👉 Most useful when locked on with double press.

\begin{itemize}
\item
  \textbf{SOLO} - Turns off intensity parameters on all controllers excluding the destination controller.
\item
  \textbf{GO/RELEASE} - First press initiates a GO, second press initiates a RELEASE.
\end{itemize}

👉 Most useful when locked on with double press.

\begin{itemize}
\item
  \textbf{ON} - Instantly turns on a controller.
\item
  \textbf{OFF} - Instantly turns off a controller.
\item
  {\textbf{Latch} - Currently unimplemented.}
\item
  \textbf{GO} - Initiates a GO command.
\item
  \textbf{ASSERT} - Takes back control of parameters that have been ``robbed'' by other controllers.
\item
  \textbf{HOLD/BACK} - If a controller is running, pause the controller. If a controller is static, initiate a BACK command.
\item
  \textbf{\textless\textless\textless{} (Step Back)} - Steps backwards through a Qlist ignoring time 👉 {[}VIBE{]}
\item
  \textbf{+ {[}BACK{]}} will also step backwards through the list.
\item
  \textbf{\textgreater\textgreater\textgreater{} (Step Forward)} - Steps forward through a Qlist ignoring time 👉 {[}VIBE{]} + {[}GO{]} will also step forward through the list.
\item
  \textbf{\textless\textless\textless--\textgreater\textgreater\textgreater{}} - Reverses sequence direction (mainly used with loops and chasers).
\end{itemize}

\textbf{Assign a Qlist to a Controller:}

\begin{enumerate}
\def\labelenumi{\arabic{enumi}.}
\item
  {[}QLIST{]} {[}\#{]}
\item
  {[}HERE{]} to any button of the controller that will be the start point of the range.
\end{enumerate}

\begin{quote}
Blank Qlists may be assigned to any controller.
\end{quote}

\textbf{Assign a range of Qlist:}

\begin{enumerate}
\def\labelenumi{\arabic{enumi}.}
\item
  {[}QLIST{]} {[}\#{]} → {[}QLIST{]} {[}\#{]}
\item
  {[}HERE{]} to any button of the controller that will be the start point of the range.
\end{enumerate}

\textbf{Assign a Cue to a controller:}

\begin{enumerate}
\def\labelenumi{\arabic{enumi}.}
\item
  {[}QLIST{]} {[}\#{]}
\item
  {[}CUE{]} {[}\#{]}
\item
  {[}HERE{]} Press any of the controller's buttons - The Qlist and Cue will be assigned to the Controller and default controller configuration and settings will be applied. See: \href{https://vibemanual.compulite.com/system-default-settings.html\#controller-actions}{7.5.2. Controller Actions}
\end{enumerate}

\textbf{Release and unload Controllers:}

\begin{itemize}
\tightlist
\item
  See: \href{https://vibemanual.compulite.com/playing-back-cues-and-scenes.html\#controller-release-and-free}{14.8. Controller Release and Free}
\end{itemize}

\hypertarget{controller-release-and-free}{%
\subsection{Controller Release and Free}\label{controller-release-and-free}}

\begin{quote}
Vibe has two types of release functions, Editor Release and Controller Release. The following deals with Controller Release.
\end{quote}

\textbf{Off -} Instantly turns off a controller

\begin{itemize}
\tightlist
\item
  {[}OFF{]} {[}HERE{]} to any controller button.
\end{itemize}

\textbf{Release -} Turns off a controller using release time.

\begin{itemize}
\tightlist
\item
  {[}RELEASE{]} {[}HERE{]} to any controller button.
\end{itemize}

\textbf{Release All -} Turns off all controllers using release time.

\begin{itemize}
\tightlist
\item
  {[}VIBE{]} + {[}RELEASE{]}
\end{itemize}

\textbf{Release Time -}

\begin{itemize}
\item
  By default controllers use the Default Release Time set in System Settings Timing. See: \href{https://vibemanual.compulite.com/system-default-settings.html\#timing}{7.2. Timing}
\item
  Release time may also be set on a Qlist basis using the Qlist Properties Pop-up. See: \href{https://vibemanual.compulite.com/system-default-settings.html\#qlist-properties}{7.5.3. Qlist Properties}
\end{itemize}

\textbf{Free -} Unloads a controller. It does not delete the loaded object.

\begin{itemize}
\tightlist
\item
  {[}FREE{]} {[}HERE{]} to any controller button.
\end{itemize}

\textbf{Free All -} Unloads all controllers.

\begin{itemize}
\tightlist
\item
  {[}VIBE{]} + {[}FREE{]}
\end{itemize}

\hypertarget{controller-qlist-properties}{%
\subsection{Controller Qlist Properties}\label{controller-qlist-properties}}

By default, the first incident of a Qlist being assigned to a controller will take its Qlist properties from the user default Qlist Properties.
\textbf{For Qlist Properties details, see:} \href{https://vibemanual.compulite.com/system-default-settings.html\#qlist-properties}{7.5.3. Qlist Properties}

\begin{itemize}
\item
  Each new incident of assigning the same Qlist to additional controllers will initially copy the Qlist properties from the first assignment.
\item
  If changes are made to individual controller Qlist properties, those changes stay with the controller and are not copied or updated to other controller Qlist properties.
\item
  If it is desirable to update what will be copied to future assignments of the Qlist, toggle \{\textbf{Save to Qlist}\} On. Changes made to the current Qlist properties will now be passed along.
\item
  Qlist properties may be restored to their user defaults by tapping the \{\textbf{Restore Defaults}\} key.
\end{itemize}

\hypertarget{controller-modes}{%
\subsection{Controller Modes}\label{controller-modes}}

The Controller Mode keys are located at the far left of the console beside the motorized faders. The motorized faders may be assigned to any one of 4 modes. These modes allow for a quick modification to various rate and effect parameters. Controller modes are outside of the editor and are designed for ``On the Fly'' usage.

\textbf{Controller Modes allow the following to be adjusted:}

\begin{itemize}
\item
  {[}FADERS{]} - Normal behaviour for the fader as assigned in Controller Definitions
\item
  {[}RATE{]} - The Rate of the fades, loops, effects, and chasers on the controller will be proportionally adjusted. Motorized faders will move to an optimized position of 32\% allowing for proportional adjustment of a controller's rate.
\item
  {[}SIZE{]} - Proportionally adjusts the size of effects running on the controller.
\item
  {[}OFFSET{]} - Adjusts the spread of the fixtures in running effects
\item
  {{[}PROTECT{]} - When implemented, must be held down before any of the fader modes may be changed.}
\end{itemize}

\hypertarget{rate-and-teach-keys}{%
\subsection{Rate and Teach Keys}\label{rate-and-teach-keys}}

The Rate function allows users to adjust controller rates or effects parameters in ``real time''. This is very helpful in fine tuning chases and effects to synchronize with music, or to improvise performances ``on the fly''. The Rate Screen operates outside of the editor and is non destructive. Changes made in the Rate Screen may not be stored in cues but may be stored and recalled in {[}Snaps{]}.

\includegraphics{https://files.gitbook.com/v0/b/gitbook-x-prod.appspot.com/o/spaces\%2F3kS90tLsADGm1ocbe7q9\%2Fuploads\%2FASGNCmc8hdzIqKOE9fa2\%2F4.11.webp?alt=media\&token=3a476108-bcb8-4f20-8965-4339044e1870}

\textbf{Using the Rate Screen:}

\textbf{Large encoder wheels:}

\begin{enumerate}
\def\labelenumi{\arabic{enumi}.}
\item
  Press the {[}Rate Key{]} - When opened, all loaded controllers will appear in the ``Selected Controllers'' section of the display. Initially all loaded controllers will be \textbf{globally} affected by the wheels and features in the Rate Screen.
\item
  Pressing any {[}Key{]} of a controller will deselect all other controllers except the controller that was selected. Additional controllers may be toggled on or off as desired and they will appear in the selected Controllers display.
\end{enumerate}

\begin{itemize}
\item
  \{Select All\} will reselect all loaded controllers for rate control.
\item
  \{Deselect All\} will release all selected controllers from rate control
\item
  \{Show Active\} will release all controllers in the ``Selected Controllers'' section that are not currently under the control of the rate wheels.
\end{itemize}

\begin{enumerate}
\def\labelenumi{\arabic{enumi}.}
\setcounter{enumi}{2}
\item
  Wheel 1 - \{Controller Rate\} speeds up or slows down the cross fade rate of the controller. This may be used for cue transitions, chases, or effects. Range: 0\% (Stopped) ⟶ 3000\% (effectively cut time)
\item
  Wheel 2 - \{X Fade \%\} adjusts the ``fade vs jump percent of a chaser. Range: 0\% (full jump) ⟶ 100\% (full cross fade)
\item
  Wheel 3 - \{Intensity Limit\} proportionally limits the intensity output of a controller. Range 0\% (no intensity output) ⟶ 100\% \{Intensity controller full\}
\item
  { Wheel 4 - \{Library Editor Rate\} currently unimplemented (adjusts the speed that Libraries are faded into the editor) Small encoder wheels: (top to bottom) }
\end{enumerate}

\textbf{Small encoder wheels: (top to bottom)}

\begin{enumerate}
\def\labelenumi{\arabic{enumi}.}
\item
  Small Wheel 1 - \{FX Rate\} adjusts the rate of effects that are running on the selected controllers.
\item
  Small Wheel 2 - \{FX Size\} adjusts the size of effects that are running on the selected controllers.
\item
  Small Wheel 3 - \{FX Offset\} adjusts the offset (Spread) of effects that are running on the selected controllers.
\item
  Small Wheel 3 - \{FX Base\} adjusts the base values (start point) of parameters that are in running effects on the selected controllers.
\end{enumerate}

\textbf{Reset keys:}

\begin{enumerate}
\def\labelenumi{\arabic{enumi}.}
\item
  \{Reset Ctrl Rate\} - Sets all controller rates back to their default of 100\%
\item
  \{Reset FX Rate\} - Sets controlled FX rates back to their stored values.
\item
  \{Reset FX Size\} - Resets controlled size of effects back to their stored values.
\item
  \{Reset FX Offset\} - Resets controlled offsets of effects back to their stored values.
\item
  \{Reset FX Base\} - Resets controlled effects base values back to their stored values.
\end{enumerate}

\textbf{Freeze toggle keys:}

\begin{enumerate}
\def\labelenumi{\arabic{enumi}.}
\item
  \{Freeze All\} - Pauses all running chases and effects until the \{Freeze All\} key is toggled again.
\item
  \{Freeze Chase\} - Pauses all running chases until the \{Freeze Chase\} key is toggled again.
\item
  \{Freeze Effect - Pauses all running effects until the \{Freeze Effect\} key is toggled again.
\end{enumerate}

\begin{quote}
Freeze only effects chases and effects at the playback level. To permanently release an effect from a cue see: \href{https://vibemanual.compulite.com/effects-1.html\#build-a-basic-effect-in-the-advanced-effects-editor}{13.3.1. Building Effects Using the Advanced Effects Editor}
\end{quote}

\textbf{Transport Bar:}

\includegraphics{https://files.gitbook.com/v0/b/gitbook-x-prod.appspot.com/o/spaces\%2F3kS90tLsADGm1ocbe7q9\%2Fuploads\%2F3tsI1uZbjqG5pxqzYh3r\%2F14.11.2.webp?alt=media\&token=2658ea17-92cf-4afa-9028-74f319a5ce62}

When actively in the Rate Screen, physical controller keys are mapped to the rate interface and may not be used to activate Qlists, cues, or scenes. As an alternative, the transport bar may be used. To use the transport bar:

\begin{enumerate}
\def\labelenumi{\arabic{enumi}.}
\item
  Tap \{Enable Controller PB\} to activate the Transport Bar.
\item
  Select the controllers to control with the bar by toggling any of the controller's physical control keys
\item
  Use the transport Bar buttons to:
\end{enumerate}

\begin{itemize}
\item
  \{\textless\textless\} Step Back
\item
  \{\textbar\textless\} Back
\item
  \{\textbar\textbar\} Hold (Pause)
\item
  \{\textgreater\} GO
\item
  \{\textgreater\textgreater\} Step Forward
\end{itemize}

\begin{enumerate}
\def\labelenumi{\arabic{enumi}.}
\setcounter{enumi}{3}
\tightlist
\item
  \{Release\} may be tapped at any time to deactivate the selected controllers using release time.
\end{enumerate}

\textbf{Teach keys: Currently unimplemented}

\begin{enumerate}
\def\labelenumi{\arabic{enumi}.}
\item
  {\{Teach Enabled\} - Enables the {[}Teach Key{]} for sampling and adjusting tap rate of chasers and effects in ``real time''.}
\item
  {\{Tap X\} keys set the sample rate for the teach function (number of presses to be averaged). }
\item
  {When implemented, Teach time may also be set from the Rate Screen using the transport bar buttons.}
\end{enumerate}

\hypertarget{dark-parameter-positioning}{%
\section{Dark Parameter Positioning}\label{dark-parameter-positioning}}

There are three features that allow you to preset parameter values when a fixture's dimmer is dark, thus avoiding visible parameter movement:

\textbf{The following is covered in this chapter:}

\begin{itemize}
\tightlist
\item
  \href{https://vibemanual.compulite.com/dark-parameter-positioning.html\#look-ahead-move-in-dark}{15.1. Look Ahead (Move in Dark)}
\item
  \href{https://vibemanual.compulite.com/dark-parameter-positioning.html\#look-ahead-cue-mark}{15.2. Look Ahead Cue (Mark)}
\item
  \href{https://vibemanual.compulite.com/dark-parameter-positioning.html\#force-black}{15.3. Force Black}
\end{itemize}

\hypertarget{look-ahead-move-in-dark}{%
\subsection{Look Ahead (Move in Dark)}\label{look-ahead-move-in-dark}}

The Look Ahead feature is used for automatic pre-positioning of selected parameters when the fixture's dimmer is dark, thus avoiding visible p r meter movement. For Look Ahead to work, the fixture's dimmer must be at ZR, and there must be a parameter moves from one cue to the next cue. Look Ahead fades can be set for a specified default time or user time, which is different than the cue's default fade time. Before using Look Ahead, you must create a Look Ahead mask. The Look Ahead mask tells Vibe which spots, spot parameters, or scrollers are affected by the Look Ahead feature. If you have not specified parameters in the Look Ahead mask, all parameters will be affected. The Look Ahead mask can contain more than one parameter. Currently, the Look Ahead pre-positioning will happen at the \textbf{last} available opportunity before the dimmer turns on.

\begin{quote}
If in specific cues you want to prevent applying the Look Ahead mask, give dimmers a value of 1\%.
\end{quote}

\begin{quote}
{LED fixtures without physical dimmer parameters should not be included in the Look Ahead Mask.
}
\end{quote}

\textbf{Enable or Disable Look Ahead on all controllers:}

\begin{enumerate}
\def\labelenumi{\arabic{enumi}.}
\item
  Open System Settings from the Vibe Menu or tap the status area of the main monitor - The System Settings pop-up will open.
\item
  Tap the \{Timing\} tab to open the Timing Properties pop-up.
\item
  Under the \textbf{Look Ahead} header enable or disable all controllers using \{Set All Controllers ON\} or \{Set All Controllers OFF\}.
\item
  There is no confirmation of On/Off as these buttons basically toggle \{Look Ahead\} in \textbf{all} Qlist Properties pop-ups.
\item
  If Look Ahead is globally turned off it may still be enabled on a per Qlist basis in the Qlist Properties pop-up.
\end{enumerate}

\begin{quote}
Look Ahead status for individual controllers may be viewed in the controller's Qlist Properties.
\end{quote}

\textbf{Create the Look Ahead Mask:}

\begin{enumerate}
\def\labelenumi{\arabic{enumi}.}
\item
  Select the fixtures to be in the Mask.
\item
  Optionally select the parameters to be included in the Mask - If no parameters are selected, all parameters will be included.
\item
  Tap \{LOOK AHEAD MASK\} on the Editor Toolbar.
\item
  Press {[}STORE{]}.
\end{enumerate}

\textbf{Delete the Look Ahead Mask:}

\begin{enumerate}
\def\labelenumi{\arabic{enumi}.}
\item
  Select the fixtures to be deleted from the Mask.
\item
  Tap \{LOOK AHEAD MASK\} on the Editor Toolbar.
\item
  Press {[}DELETE{]}.
\end{enumerate}

A Look Ahead cue is a cue that automatically pre-positions parameters when the dimmer is off. Currently only time can be set for a Look Ahead cue.

\textbf{Set individual Look Ahead Time -} Speed at which fixtures move in dark.

\begin{enumerate}
\def\labelenumi{\arabic{enumi}.}
\item
  CUE{]} {[}\#{]} {[}Time{]} - The Time pop-up will open.
\item
  Tap \{Look Ahead\} - The indicator will turn green and Look Ahead will appear over the virtual time wheel.
\item
  Set the time value using the virtual wheel or type the value directly on the keypad.
\item
  Press {[}Enter{]} or tap to close the pop-up and store the Look Ahead time.
\end{enumerate}

\begin{quote}
Look Ahead time can also be changed using the {[}CUE {]} {[}\#{]} {[}SETTINGS{]} command. Default Look Ahead time can be set in System Settings/Defaults/Cue Settings.
\end{quote}

\textbf{Enable or Disable Look Ahead on individual controllers:}

\begin{enumerate}
\def\labelenumi{\arabic{enumi}.}
\item
  Press {[}SETTINGS{]} {[}HERE{]} to any controller button of the controller you wish to enable or disable Look Ahead on - The Settings pop-up will open.
\item
  Tap the \{Qlist properties\} tab - The Qlist Properties pop-up will open.
\item
  Toggle Look Ahead on or off.
\item
  Press {[}Enter{]} or tap to close the pop-up.
\end{enumerate}

\hypertarget{look-ahead-cue-mark}{%
\subsection{Look Ahead Cue (Mark)}\label{look-ahead-cue-mark}}

{Unimplemented}

\begin{itemize}
\item
  {Look Ahead: (sometimes referred to a Mark Cue) }

  \begin{itemize}
  \tightlist
  \item
    {Unlike having Look Ahead automatically preposition all dark parameters for all cues in a Qlist, a cue tagged in Cue Store \textbf{Options} as \{Look Ahead\} will preposition dark parameters for just the tagged cue.}
  \end{itemize}
\end{itemize}

\hypertarget{force-black}{%
\subsection{Force Black}\label{force-black}}

When it is not desirable to see parameter values move from one cue state to another, Force Black may be used to save programing time. In a cue flagged as Force Black, the dimmer parameter is first ``forced'' to Black Out, parameters them move in black, then the dimmer parameter is faded up again. The total speed of a Force Black operation is relative to the overall cue time.

\textbf{To flag a cue as a Force Black Cue:}

\begin{enumerate}
\def\labelenumi{\arabic{enumi}.}
\item
  {[}QLIST {[}\#{]} (if not on Master Controller) {[}CUE{]} {[}\#{]} {[}SETTINGS{]}.
\item
  Toggle \{Force Black\} On - The indicator will turn green and the cue will be flagged as a Force Black cue.
\item
  Press {[}Enter{]} or tap \href{image.png}{} to close the pop-up.
\end{enumerate}

\textbf{To remove a Force Black flag:}

\begin{enumerate}
\def\labelenumi{\arabic{enumi}.}
\item
  {[}QLIST {[}\#{]} (if not on Master Controller) {[}CUE{]} {[}\#{]} {[}SETTINGS{]}.
\item
  Toggle \{Force Black\} Off - The indicator will turn off and the cue will no longer act as a Force Black cue.
\item
  Press {[}Enter{]} or tap \href{image.png}{} to close the pop-up.
\end{enumerate}

\hypertarget{macros}{%
\section{Macros}\label{macros}}

This chapter deals with creating and triggering Macros.

\textbf{The following is covered in this chapter:}

\begin{itemize}
\tightlist
\item
  \href{https://vibemanual.compulite.com/macros.html\#creating-macros}{16.1. Creating Macros}
\item
  \href{https://vibemanual.compulite.com/macros.html\#triggrering-macros}{16.2. Triggrering Macros}
\item
  \href{https://vibemanual.compulite.com/macros.html\#attaching-macros}{16.3. Attaching Macros}
\end{itemize}

\hypertarget{creating-macros}{%
\subsection{Creating Macros}\label{creating-macros}}

Current Macros can only be created using the \textbf{Teach Macro} recorder. In future releases this will expand to include manually created macros. Macro time is automatically recorded with macros and the user may decide whether to enable time or ignore it.

\textbf{Record a Macro:}

\begin{enumerate}
\def\labelenumi{\arabic{enumi}.}
\item
  Press {[}TEACH{]} - Vibe will enter Teach Macro record mode and the {[}TEACH{]} key will flash red.
\item
  Press editor keys or controller buttons as required.
\item
  Press {[}TEACH{]} again to terminate the recording - the {[}TEACH{]} key will stop flashing and the key presses are now stored in the Macro clipboard.
\item
  Press {[}STORE{]} {[}HERE{]} to a Macro Softkey.
\end{enumerate}

OR

\begin{enumerate}
\def\labelenumi{\arabic{enumi}.}
\setcounter{enumi}{4}
\item
  Press {[}MACRO{]} {[}\#{]} {[}STORE{]} - The Macro is now stored.
\item
  Press {[}TEXT{]} to open the Text Pop-up and label the Macro - this is only valid directly after storing the MAcro. Alternately {[}Macro{]} {[}\#{]} {[}TEXT{]} may be used at any time to label the Macro.
\end{enumerate}

\textbf{{\textbf{Record a Macro including timing: Unimplemented}}.}

\begin{enumerate}
\def\labelenumi{\arabic{enumi}.}
\item
  {Use steps 1 ⟶ 3 to store the macro to the Macro clipboard.}
\item
  {Press {[}MACRO{]} {[}\#{]} {[}TIME{]}{[}STORE{]} - The Macro is now stored including it's timing.}
\end{enumerate}

\begin{quote}
Macros may be viewed by adding the EXAM display to any of the \textbf{Workspace Template} pages.
\end{quote}

\includegraphics{https://files.gitbook.com/v0/b/gitbook-x-prod.appspot.com/o/spaces\%2F3kS90tLsADGm1ocbe7q9\%2Fuploads\%2FswUPSRgyKZ0jqq25xzak\%2F16.1.webp?alt=media\&token=4c86738f-6797-4c6e-a628-8533c9043595}

\hypertarget{triggrering-macros}{%
\subsection{Triggrering Macros}\label{triggrering-macros}}

Macros may be directly triggered using the keypad or Macro Softkeys.

\textbf{Trigger a Macro using the keypad:}

\begin{itemize}
\tightlist
\item
  {[}MACRO{]} {[}\#{]} {[}ENTER{]}
\end{itemize}

\textbf{Trigger a Macro using Macro Softkeys:}

\begin{itemize}
\tightlist
\item
  Tap the required \{Macro\} Softkey.
\end{itemize}

{\textbf{Trigger a Macro from a Qkey: Currently Unimplemented}}
- Assign the Macro to the Qkey by pressing {[}MACRO{]} {[}\#{]} {[}HERE{]} to any Qkey button.

\begin{itemize}
\tightlist
\item
  {Press the Qkey button to activate. }
\end{itemize}

\hypertarget{attaching-macros}{%
\subsection{Attaching Macros}\label{attaching-macros}}

Macros may be attached and triggered from Qlist Cues.

\textbf{Attach a Macro to a Cue:}

\begin{itemize}
\tightlist
\item
  {[}CUE{]} {[}\#{]} {[}MACRO{]} {[}\#{]} {[}STORE{]}
\end{itemize}

\textbf{Release a specific Macro from a Cue:}

\begin{itemize}
\tightlist
\item
  {[}CUE{]} {[}\#{]} {[}MACRO{]} {[}\#{]} {[}RELEASE{]}
\end{itemize}

\textbf{Release all Macros from a Cue:}

\begin{itemize}
\tightlist
\item
  {[}CUE{]} {[}\#{]} {[}MACRO{]} {[}RLEAEASE{]}
\end{itemize}

\textbf{Macros are triggered when:}

\begin{itemize}
\item
  The Qlist is advanced to the cue with the trigger.
\item
  {[}GOTO{]} commands are used.
\item
  {[}CUE{]} {[}\#{]} {[}GO{]} is used.
\end{itemize}

\begin{quote}
Back commands will not trigger Macros or Snap.
\end{quote}

\textbf{Temporarily disable triggers:}

\begin{enumerate}
\def\labelenumi{\arabic{enumi}.}
\tightlist
\item
  Tap the controller display header for the Qlist that contains Macro or Snap triggers - The Qlist Properties pop-up will appear.
\end{enumerate}

OR

\begin{enumerate}
\def\labelenumi{\arabic{enumi}.}
\item
  {[}SETTINGS{]} {[}HERE{]} to any controller button and select the Qlist Properties tab.
\item
  Deselect the \{Trigger\} button - Triggering will be disabled for the incident of the Qlist assigned to that \textbf{specific} controller.
\end{enumerate}

\includegraphics{https://files.gitbook.com/v0/b/gitbook-x-prod.appspot.com/o/spaces\%2F3kS90tLsADGm1ocbe7q9\%2Fuploads\%2FnXGGqWyWr97MYj4YXvSe\%2F16.3.webp?alt=media\&token=961a30eb-a2db-4fdf-b0a3-9b710af7653d}

\hypertarget{snaps-snapshots}{%
\section{Snaps (Snapshots)}\label{snaps-snapshots}}

This chapter deals with creating and triggering Snapshots.

\textbf{The following is covered in this chapter:}

\begin{itemize}
\tightlist
\item
  \href{https://vibemanual.compulite.com/snaps-snapshots.html\#storing-snaps}{17.1. Storing Snaps}
\item
  \href{https://vibemanual.compulite.com/snaps-snapshots.html\#triggering-snap}{17.2. Triggering Snap}
\item
  \href{https://vibemanual.compulite.com/snaps-snapshots.html\#attaching-snaps}{17.3. Attaching Snaps}
\end{itemize}

\hypertarget{storing-snaps}{%
\subsection{Storing Snaps}\label{storing-snaps}}

\textbf{Snaps (Snapshots) - Capture and record the current state of all controllers including:}

\begin{itemize}
\item
  Loaded Qlist
\item
  Current Page - Optional
\item
  Current Master Controller - Optional
\item
  Current active cue
\item
  Slider (Fader) position
\item
  Priority list (stack) at the time of the snapshot
\item
  Controller Priority Group number
\item
  Controller slider and button assignments
\item
  Linked controllers (Combined)
\item
  Rate overrides
\item
  Effect Size overrides
\item
  Effect Offset overrides
\item
  Effect Base overrides
\end{itemize}

\textbf{Store a Snapshot:}

\begin{enumerate}
\def\labelenumi{\arabic{enumi}.}
\item
  Make sure that all controllers on \textbf{all} pages are setup, loaded, and activated exactly as you want them to be restored.
\item
  Press {[}STORE{]} {[}HERE{]} to any available \{Snap\} softkey.
\end{enumerate}

Or

\begin{enumerate}
\def\labelenumi{\arabic{enumi}.}
\setcounter{enumi}{1}
\item
  Press {[}SNAP{]} {[}\#{]} {[}STORE{]}.
\item
  Press {[}TEXT{]} to open the Text Pop-up and label the Snap - this is only valid directly after storing the Snap. Alternately {[}SNAP{]} {[}\#{]} {[}TEXT{]} may be used at any time to label the Snap.
\end{enumerate}

\hypertarget{triggering-snap}{%
\subsection{Triggering Snap}\label{triggering-snap}}

Snaps may be directly triggered using the keypad or Snap Softkeys.

\textbf{Trigger a Snap using the keypad:}

\begin{itemize}
\tightlist
\item
  {[}SNAP{]} {[}\#{]} {[}ENTER{]}
\end{itemize}

\textbf{Trigger a Snap using Snap softkeys:}

\begin{itemize}
\tightlist
\item
  Tap the required \{SNAP\} softkey.
\end{itemize}

{\textbf{Trigger a Snap from a Qkey: Currently Unimplemented.}}

\begin{itemize}
\item
  {Assign the Snap to the Qkey by pressing {[}SNAP{]} {[}\#{]} {[}HERE{]} to any Qkey button.}
\item
  {Press the Qkey button to activate. }
\end{itemize}

Snaps on Vibe are currently \textbf{absolute} - All controllers are replaced with new loaded objects and controllers without new loaded objects are cleared. {Future releases will have an option for ``Page Holdover'' where active objects will not be replaced until the controller is released.}

Snaps may be assigned a time to ``smooth'' out their transitions.

\textbf{Setting Snap Time:}

\begin{enumerate}
\def\labelenumi{\arabic{enumi}.}
\item
  Press {[}SNAP{]} {[}\#{]} {[}SETTINGS{]} - The Snap Settings pop-up will appear.
\item
  Toggle the \{Snap Time\} key On.
\item
  Adjust the Snap Time with the Virtual Time Wheel.
\item
  Close the pop-up using {[}ENTER{]} or \href{image.png}{} .
\end{enumerate}

\textbf{Additional Snap Settings:}

\begin{itemize}
\item
  Snap Label - Enter text label for Snap into the white text box.
\item
  Load Pages - Restores the controllers to the page they were on when the Snap was recorded.
\item
  Load Master PB (Controller) - Restores assignment of the Master Controller to the controller it was assigned to when the Snap was recorded.
\end{itemize}

\textbf{Controller Settings Properties that affect Snap:}

\begin{itemize}
\item
  Exclude from Snap - Controller will not be included when Snaps are recorded.
\item
  Exclude from Override - Even if recorded in a Snap, the controller will ignore Snap activation instructions.
\end{itemize}

\hypertarget{attaching-snaps}{%
\subsection{Attaching Snaps}\label{attaching-snaps}}

Snaps may be attached and triggered from Qlist Cues.

\textbf{Attach a Snap to a Cue:}

\begin{itemize}
\tightlist
\item
  {[}CUE{]} {[}\#{]} {[}SNAP{]} {[}\#{]} {[}STORE{]}
\end{itemize}

\textbf{Release a specific Snap from a Cue:}

\begin{itemize}
\tightlist
\item
  {[}CUE{]} {[}\#{]} {[}SNAP{]} {[}\#{]} {[}RELEASE{]}
\end{itemize}

\textbf{Release all Macros from a Cue:}

\begin{itemize}
\tightlist
\item
  {[}CUE{]} {[}\#{]} {[}SNAP{]} {[}RLEAEASE{]}
\end{itemize}

\textbf{Snaps are triggered when:}

\begin{itemize}
\item
  The Qlist is advanced to the cue with the Snap trigger.
\item
  {[}GOTO{]} commands are used.
\item
  {[}CUE{]} {[}\#{]} {[}GO{]} is used.
\end{itemize}

\begin{quote}
Back commands will not trigger Macros or Snap.
\end{quote}

\textbf{Temporarily disable triggers:}

\begin{enumerate}
\def\labelenumi{\arabic{enumi}.}
\tightlist
\item
  Tap the controller display header for the Qlist that contains Macro or Snap triggers - The Qlist Properties pop-up will appear.
\end{enumerate}

OR

\begin{enumerate}
\def\labelenumi{\arabic{enumi}.}
\item
  {[}SETTINGS{]} {[}HERE{]} to any controller button and select the Qlist Properties tab.
\item
  Deselect the \{Trigger\} button - Triggering will be disabled for the incident of the Qlist assigned to that \textbf{specific} controller.
\end{enumerate}

\includegraphics{https://files.gitbook.com/v0/b/gitbook-x-prod.appspot.com/o/spaces\%2F3kS90tLsADGm1ocbe7q9\%2Fuploads\%2Fd35SScD9jx2IIxZBd7cE\%2F173..webp?alt=media\&token=eaf131f2-2835-4b49-bdfd-7332d407d98f}

\hypertarget{exam-show-data}{%
\section{Exam show data}\label{exam-show-data}}

Virtually all Vibe show data can be viewed using the \textbf{Exam} display. The Exam display can be added to any Workspace Template using the Vibe Menu.

\textbf{Path}: \{Vibe\}/\{Display objects\}/\{Exam\}. It can also opened as a pop-up using the {[}EXAM{]} key. The root menu is divided into two main sub menus, \{Show\} and \{Patch\} This chapter deals with viewing show and patch data.

\textbf{The following is covered in this chapter:}

\begin{itemize}
\tightlist
\item
  \href{https://vibemanual.compulite.com/exam-viewing-show-data.html\#show-exam}{18.1. Show Exam}
\item
  \href{https://vibemanual.compulite.com/exam-viewing-show-data.html\#patch-exam}{18.2. Patch Exam}
\end{itemize}

\hypertarget{show-exam}{%
\subsection{Show Exam}\label{show-exam}}

\textbf{The Show Exam Menu has 14 items:}

\includegraphics{https://files.gitbook.com/v0/b/gitbook-x-prod.appspot.com/o/spaces\%2F3kS90tLsADGm1ocbe7q9\%2Fuploads\%2FW6Xww7DJpwyKPcb0okaD\%2F18.1.webp?alt=media\&token=c6a011cc-ad76-495d-b429-acee8616888e}

\begin{enumerate}
\def\labelenumi{\arabic{enumi}.}
\tightlist
\item
  \textbf{Qlist Exam} - List of all show Qlists.
\end{enumerate}

\begin{itemize}
\item
  Qlist \# Exam - Displays a Cue Sheet view for all cues in the Qlist.

  \begin{itemize}
  \item
    Cue Exam - Displays a track sheet for the selected cue. Fixtures may be filtered using the \textbf{Sets Tabs} at the bottom of the screen.

    \begin{itemize}
    \tightlist
    \item
      Fixture - Displays the selected fixture's information organized by \textbf{Patch, Sets, Cues}, and \textbf{Libraries} tabs at the bottom of the display.
    \end{itemize}
  \end{itemize}
\end{itemize}

\begin{enumerate}
\def\labelenumi{\arabic{enumi}.}
\setcounter{enumi}{1}
\tightlist
\item
  \textbf{Libraries Exam} -Lists all stored Libraries by type.
\end{enumerate}

\begin{itemize}
\item
  Library Type Exam - Displays all libraries for the selected library type.

  \begin{itemize}
  \item
    Library Fixtures Exam - Displays values for the fixtures stored in the library. Fixtures may be filtered using the \textbf{Sets Tabs} at the bottom of the screen.

    \begin{itemize}
    \tightlist
    \item
      Fixture - Displays the selected fixture's information organized by \textbf{Patch, Sets, Cues}, and \textbf{Libraries} tabs at the bottom of the display.
    \end{itemize}
  \end{itemize}
\end{itemize}

\begin{enumerate}
\def\labelenumi{\arabic{enumi}.}
\setcounter{enumi}{2}
\tightlist
\item
  \textbf{Group Exam} -Lists all stored groups.
\end{enumerate}

\begin{itemize}
\item
  Group \# Exam - Displays all the fixtures in the group and their recorded selection order.

  \begin{itemize}
  \tightlist
  \item
    Individual group fixtures - N/A.
  \end{itemize}
\end{itemize}

\begin{enumerate}
\def\labelenumi{\arabic{enumi}.}
\setcounter{enumi}{3}
\tightlist
\item
  \textbf{Snaps Exam} - Lists all recorded Snapshots.
\end{enumerate}

\begin{itemize}
\tightlist
\item
  Snap \# Exam - Graphically displays what controllers and their status are recorded in the the Snap.
\end{itemize}

\begin{enumerate}
\def\labelenumi{\arabic{enumi}.}
\setcounter{enumi}{4}
\tightlist
\item
  \textbf{Macro Exam} - Lists all stored Macros.
\end{enumerate}

\begin{itemize}
\tightlist
\item
  Macro \# Exam - Opens a display of all macro steps, time (if enabled), macro state, and macro command {(This will be editable in future versions).}
\end{itemize}

\begin{enumerate}
\def\labelenumi{\arabic{enumi}.}
\setcounter{enumi}{5}
\item
  {Time Lines - Not implemented yet. }
\item
  \textbf{Look Ahead Mask Exam} - Displays which fixtures will be enabled for the Look Ahead feature. (Move in Black) .
\item
  {MIDI Notes - Shows patched midi notes (currently unimplemented use patch \{I/0 Settings\} as an alternative.) }
\item
  {Filters - Not Implemented.}
\item
  \textbf{Scene Exam} - Lists all stored Scenes.
\end{enumerate}

\begin{itemize}
\item
  Scene \# Exam - Displays a track sheet for the selected Scene. Fixtures may be filtered using the \textbf{Sets Tabs} at the bottom of the screen.

  \begin{itemize}
  \tightlist
  \item
    Fixture - Displays the selected fixture's information organized by \textbf{Patch, Sets, Cues,} and \textbf{Libraries} tabs at the bottom of the display.
  \end{itemize}
\end{itemize}

\begin{enumerate}
\def\labelenumi{\arabic{enumi}.}
\setcounter{enumi}{10}
\tightlist
\item
  \textbf{Submaster Exam} - Lists all stored Scenes that have been converted to Group Submasters.
\end{enumerate}

\begin{itemize}
\tightlist
\item
  Submaster \# Exam - Displays a track sheet for the selected Group Submaster. Fixtures may be filtered using the \textbf{Sets Tabs} at the bottom of the screen.
\end{itemize}

\begin{enumerate}
\def\labelenumi{\arabic{enumi}.}
\setcounter{enumi}{11}
\item
  \textbf{Home Scene Exam} - Displays values for all fixtures stored in the user defined Home Scene.
\item
  \textbf{Highlight Scene Exam} - Displays values for all fixtures stored in the user defined Highlight Scene.
\item
  \textbf{Lowlight Scene Exam} - Displays values for all fixtures stored in the user defined Lowlight Scene.
\end{enumerate}

\hypertarget{patch-exam}{%
\subsection{Patch Exam}\label{patch-exam}}

\textbf{The Patch Exam Menu has 6 items:}

\includegraphics{https://files.gitbook.com/v0/b/gitbook-x-prod.appspot.com/o/spaces\%2F3kS90tLsADGm1ocbe7q9\%2Fuploads\%2FpdUll7y7i9CAEFIimzbC\%2F18.2.webp?alt=media\&token=102bd268-8e0b-48e8-bb18-1b4fcf78e4f4}

\begin{enumerate}
\def\labelenumi{\arabic{enumi}.}
\tightlist
\item
  \textbf{Devices Exam} - Shows a list of all imported devices and their properties.
\end{enumerate}

\begin{itemize}
\tightlist
\item
  Individual Device Exam - Graphically Displays DMX parameter allocation for the current device and mode.
\end{itemize}

\begin{enumerate}
\def\labelenumi{\arabic{enumi}.}
\setcounter{enumi}{1}
\tightlist
\item
  \textbf{Sets Exam} - Displays the ID of each set type and the number of fixtures patched to it.
\end{enumerate}

\begin{itemize}
\tightlist
\item
  Individual Set Exam - Shows a list of all fixtures patched to the set with columns for \textbf{Fixture Name, Fixture \#} (in this case unique system ID \# not actual fixture \#), \textbf{Set ID \#, \# in Set}, (User defined number in the set) , and \textbf{Device Name}.
\end{itemize}

\begin{enumerate}
\def\labelenumi{\arabic{enumi}.}
\setcounter{enumi}{2}
\tightlist
\item
  \textbf{DMX Out Exam} - Displays a list of all DMX universes that have fixtures patched to them. A table is provided with columns for \textbf{Universe \#} {(Port)}, Patched Fixture Count, and the \textbf{Number of Used DMX Address} out of 512.
\end{enumerate}

\begin{itemize}
\tightlist
\item
  Individual Universe Exam - Displays a table of patched fixtures and their DMX properties.
\end{itemize}

\begin{enumerate}
\def\labelenumi{\arabic{enumi}.}
\setcounter{enumi}{3}
\tightlist
\item
  \textbf{DMX In Exam} - Displays a table of how may DMX inputs are used on each of the two available DMX inputs.
\end{enumerate}

\begin{itemize}
\tightlist
\item
  Individual Input Exam - Displays a table of DMX input assignments.
\end{itemize}

\begin{enumerate}
\def\labelenumi{\arabic{enumi}.}
\setcounter{enumi}{4}
\tightlist
\item
  \textbf{Park Exam} - Displays a table of the number of parked fixtures and addresses.
\end{enumerate}

\begin{itemize}
\tightlist
\item
  View Park by Dimmer (Address) or Fixture - Displays a list of all parked fixtures or dimmers and their parked dimmer values.
\end{itemize}

\begin{enumerate}
\def\labelenumi{\arabic{enumi}.}
\setcounter{enumi}{5}
\tightlist
\item
  {\textbf{Curves}} \textbf{Exam} (Functions) - Displays a graphical list of all read only and user created Functions. It also displays which functions are included in the favorites list.
\end{enumerate}

\hypertarget{topo-topographical-display}{%
\section{Topo Topographical Display}\label{topo-topographical-display}}

It is sometimes useful to organize fixture selection in a graphical manner instead of numerical lists. In Vibe this is referred to as the TOPO (Topographical) map display. Other console may refer to this feature as ``Layouts'', ``Magic Sheets'', or ``Rig'' Schematics''.

\textbf{The following is covered in this chapter:}

\begin{itemize}
\tightlist
\item
  \href{https://vibemanual.compulite.com/topo-topographical-display.html\#creating-a-topo-view}{19.1. Creating a Topo View}
\item
  \href{https://vibemanual.compulite.com/topo-topographical-display.html\#topo-menu}{19.2. Topo Menu}
\item
  \href{https://vibemanual.compulite.com/topo-topographical-display.html\#creating-a-topo}{19.3. Creating a Topo}
\item
  \href{https://vibemanual.compulite.com/topo-topographical-display.html\#adding-fixtures-to-a-topo}{19.4. Adding fixtures to a Topo}
\item
  \href{https://vibemanual.compulite.com/topo-topographical-display.html\#adding-pixel-maps}{19.5. Adding Pixel Maps}
\item
  \href{https://vibemanual.compulite.com/topo-topographical-display.html\#adding-text-and-background-elements}{19.6. Adding text and background elements}
\end{itemize}

\hypertarget{creating-a-topo-view}{%
\subsection{Creating a Topo View}\label{creating-a-topo-view}}

When starting a new show any Topo views stored in Layouts will be empty until the system is patched and fixtures are assigned to a Topo. If no layout contains a Topo view, a Topo view must be added to a blank page. Virtually unlimited Topos may be created and displayed in a Topo view. Fixtures may be shared by Topos.

\includegraphics{https://files.gitbook.com/v0/b/gitbook-x-prod.appspot.com/o/spaces\%2F3kS90tLsADGm1ocbe7q9\%2Fuploads\%2FTDS2G9ZiwCKwHmLfetSe\%2F19.1\%20topo.png?alt=media\&token=4bc1efc8-ebaa-4d30-8fe0-7c5ce82c1a5e}

Topo View in \{LIVE\} mode displaying the ``Movers'' Topo

\textbf{To add a Topo view to a page:}

\begin{enumerate}
\def\labelenumi{\arabic{enumi}.}
\item
  Tap \{Vibe\} on the monitor that will display the Topo - Vibe menu will open.
\item
  Tap \{Pages\} - Page display will open.
\item
  Add a blank page from either the \{Programming\} or \{Playback\} work space templates - A blank workspace will open.
\item
  On the screen with the blank workspace, open the \{Vibe\} menu again and select\}

  ⟶ \{Display Objects\}

  ⟶ \{Topo View\} - A blank Topo view will appear on the workspace.
\item
  Size and drag the Topo display as required then tap \{unlock/lock\} icon to lock the Topo from further adjustment - Topo display is now ready to add fixtures to.
\item
  Select the desired Topo number, and give the Topo a unique name.
\item
  Enter the size of the Stage surface. The stage dimensions may be in be in Imperial or Metric. This default is set in the System Settings pop-up.
\item
  Select the stage point of view. The default is Top view.
\item
  Tap \href{image.png}{} or press {[}ENTER{]} - This closes the pop-up, enters Topo \{Edit\} mode, and the grid will be displayed.
\end{enumerate}

\hypertarget{topo-menu}{%
\subsection{Topo Menu}\label{topo-menu}}

The Topo Menu is a hierarchical menu that provides the tools for creating and editing Topos. To Edit objects, the \{EDIT\} mode button in the bottom right area of the Topo view must be enabled.

\includegraphics{https://files.gitbook.com/v0/b/gitbook-x-prod.appspot.com/o/spaces\%2F3kS90tLsADGm1ocbe7q9\%2Fuploads\%2FXFlu7uvTzBI5S6Apzy3g\%2F19.2.png?alt=media\&token=3b60e416-8286-4949-b75a-a8223de3da3a}

\hspace{0pt}

\textbf{The root menu provides the following sub-menus:}

\begin{enumerate}
\def\labelenumi{\arabic{enumi}.}
\tightlist
\item
  \textbf{Topo:}
\end{enumerate}

\includegraphics{https://files.gitbook.com/v0/b/gitbook-x-prod.appspot.com/o/spaces\%2F3kS90tLsADGm1ocbe7q9\%2Fuploads\%2F0EQ43jhWPlZ0HLJZdDHD\%2F19.2.1.webp?alt=media\&token=7277a796-bf80-4991-a343-94ad8f13f621}

\hspace{0pt}
2. \textbf{Actions:}

\begin{enumerate}
\def\labelenumi{\alph{enumi}.}
\tightlist
\item
  \{Add Elements\} - accesses the following:
\end{enumerate}

\begin{enumerate}
\def\labelenumi{\arabic{enumi}.}
\item
  \{Sets\} Opens drop down of all available fixture sets, Global fixture set, Channels, Spots, Matrix, Servers, and users created sets. \href{https://vibemanual.compulite.com/patch.html\#fixture-sets}{6.1. Fixture Sets.} , \href{https://vibemanual.compulite.com/topo-topographical-display.html\#adding-fixtures-to-a-topo}{19.4. Adding fixtures to a Topo.}
\item
  \{Pixel Mapper\} - Adds a User definable Pixel Map matrix to the Topo surface 19.5. Adding Pixel Maps
\end{enumerate}

\begin{enumerate}
\def\labelenumi{\alph{enumi}.}
\setcounter{enumi}{1}
\tightlist
\item
  \{Selection\} - May be used in Live mode or Edit mode.
\end{enumerate}

\begin{quote}
Selections must start out side of the fixture.
\end{quote}

\begin{enumerate}
\def\labelenumi{\arabic{enumi}.}
\item
  Select by creating a rectangular lasso. - Dragging from left to right or right to left to create a rectangle around or through the objects.
\item
  Select by creating a free-form path, dragging through objects.
\item
  De-select all.
\item
  Delete selected (only valid in Edit Mode).
\item
  Snap to grid - Fixtures and object will snap to the nearest grid node.
\item
  Sync to Editor - In edit mode, fixtures can be selected using the embedded keypad.
\end{enumerate}

\begin{enumerate}
\def\labelenumi{\alph{enumi}.}
\setcounter{enumi}{2}
\tightlist
\item
  \{Tools\}:
\end{enumerate}

\begin{enumerate}
\def\labelenumi{\arabic{enumi}.}
\item
  \{Align\}:
\item
  Left
\item
  Center
\item
  Right
\item
  Top
\item
  Middle
\item
  Bottom
\item
  Circular - When selected, a set of modifier horizontal virtual wheels will be available for the following:
\end{enumerate}

\begin{itemize}
\item
  fixture location on the circumference
\item
  Radius
\item
  End point
\end{itemize}

\begin{enumerate}
\def\labelenumi{\arabic{enumi}.}
\setcounter{enumi}{7}
\item
  Point to Point
\item
  \{Align to Point\} - A base point must be specified on the Topo surface as a starting reference. The options are the same as Align except there is no point to point.
\item
  \{Spread\} (Fan) - When Spread is selected, a horizontal virtual modifier wheel will appear. The following spread options are available:
\item
  Spread horizontally from left
\item
  Spread horizontally from center
\item
  Spread horizontally from right
\item
  spread vertically from top
\item
  Spread vertically from middle
\item
  Spread vertically from bottom
\item
  \{Transform\} - Used to modify size and rotation of Topo objects. Horizontal virtual wheels will appear for the following:
\item
  Scale
\item
  Rotate
\item
  Presets are supplied for:

  \begin{itemize}
  \item
    Rotate 45° counterclockwise
  \item
    Rotate 45° clockwise
  \item
    Reset
  \end{itemize}
\item
  \{Arrange Calibration\} - When tapped, selected Topo fixtures get their positions from the XYZ calibration. \href{https://vibemanual.compulite.com/xyz-fixture-calibration.html}{20. XYZ Fixture Calibration.}
\end{enumerate}

\hypertarget{creating-a-topo}{%
\subsection{Creating a Topo}\label{creating-a-topo}}

\textbf{To create a new Topo:}

Navigate to a Topo view

\begin{enumerate}
\def\labelenumi{\arabic{enumi}.}
\tightlist
\item
  Tap the \{\href{image.png}{}\} key in the Topo Menu at the right hand side of the Topo view display - The ``Create New Topo'' pop-up will appear
\end{enumerate}

\includegraphics{https://files.gitbook.com/v0/b/gitbook-x-prod.appspot.com/o/spaces\%2F3kS90tLsADGm1ocbe7q9\%2Fuploads\%2FdbnMzNTAiuRmv8Gch2g2\%2F19.3.webp?alt=media\&token=ca062ed5-c7e8-4c08-ac9b-ff74f9a497ba}

\begin{enumerate}
\def\labelenumi{\arabic{enumi}.}
\setcounter{enumi}{1}
\item
  Select the desired Topo number, and give the Topo a unique name.
\item
  Enter the size of the Stage surface. The stage dimensions may be in be in Imperial or Metric. This default is set in the System Settings pop-up.
\item
  Select the stage point of view. The default is Top view.
\item
  Tap \href{image.png}{} aor press {[}ENTER{]} - This closes the pop-up, enters Topo \{Edit\} mode, and the grid will be displayed.
\end{enumerate}

\hypertarget{adding-fixtures-to-a-topo}{%
\subsection{Adding fixtures to a Topo}\label{adding-fixtures-to-a-topo}}

There are two ways to add fixtures to a Topo

\textbf{Using keypad selection}:

\begin{enumerate}
\def\labelenumi{\arabic{enumi}.}
\item
  Press zoom option \{FIT STAGE TO SCREEN\} on the toolbar at the bottom of the Topo View to see the whole stage area.
\item
  Select the fixtures using the keypad or a group.
\item
  Tap \{FROM EDITOR\} on the toolbar at the bottom of the Topo View.
\item
  Selected fixtures will now appear in the top right section of the Topo.
\item
  initially, the fixtures will be selected (outlined in red), and can be dragged freely into place.
\item
  Double tap in a clear space on the TOPO to deselect.
\end{enumerate}

\begin{quote}
{If the \{From Editor\} key is selected with a long press (held for a few seconds) the fixtures may be directly dragged from the \{From Editor\} area. Only a maximum of 5 fixture icons will be shown but the whole selection will actually be there. Drag the fixtures to a location on the Topo view and release. The actual fixtures will now appear on the Topo surface. \textbf{The reference point for dropping is the fixtures is the top left corner of the dragged fixture icons.}}
\end{quote}

\textbf{Using \{ADD ELELMENTS\} from the Topo menu:}

\begin{enumerate}
\def\labelenumi{\arabic{enumi}.}
\item
  Create the Topo
\item
  From the Topo menu tap \{ADD ELEMENTS\}
\item
  From the drop down, tap \{SETS\}
\item
  From the next drop down, select a fixture set - A list of the fixture in the set will drop down.
\item
  From the fixture list, select the fixtures to drag onto the Topo.
\item
  Press, hold and drag the fixtures onto the Topo surface.
\item
  Double tap in a blank area of the Topo to deselect the fixtures. Topo tools are provided to edit topo fixture placement and appearance.\href{https://vibemanual.compulite.com/topo-topographical-display.html\#topo-menu}{19.2. Topo Menu.}
\end{enumerate}

\hypertarget{adding-pixel-maps}{%
\subsection{Adding Pixel Maps}\label{adding-pixel-maps}}

The pixel mapper creates a matrix of fixtures. The aspect ratio of the pixel map is controlled by the number of columns and rows that are created. Matrix fixtures may be selected by blocks using the rectangle lasso selection tool, or free-form path tool.

\textbf{To create a new pixel mapper:}

\begin{enumerate}
\def\labelenumi{\arabic{enumi}.}
\item
  From the Topo Actions menu, tap \{Add Elements\}
\item
  From the \{Add Elements\} drop down, tap \{Pixel Mapper\} - The Create New Pixel Mapper pop-up will appear.
\end{enumerate}

\includegraphics{https://files.gitbook.com/v0/b/gitbook-x-prod.appspot.com/o/spaces\%2F3kS90tLsADGm1ocbe7q9\%2Fuploads\%2Ff9zc1asl0cfTClSVXoI1\%2F19.5.webp?alt=media\&token=6349e1a5-6671-4d96-9be8-09a19688a2c1}

\begin{enumerate}
\def\labelenumi{\arabic{enumi}.}
\setcounter{enumi}{2}
\item
  Fill in the number of columns and rows and set the start position and direction.
\item
  Press {[}Enter{]} or tap \href{image.png}{} to close the pop-up and create the blank pixel map.
\item
  👉 {Currently, the pixel map will appear in the top right corner and in some cases, a zoom out will be needed to find the pixel map and drag it back to the desired location.}.
\item
  Using the same methods for adding fixtures to the Topo, drag the fixtures into the pixel map matrix. Usually, this will be to the top left corner. \href{https://vibemanual.compulite.com/topo-topographical-display.html\#adding-fixtures-to-a-topo}{19.4. Adding fixtures to a Topo}
\item
  Release the fixtures and they will populate the pixel map.
\end{enumerate}

👉 {The number of fixtures must not exceed the number of pixel mapper cells.}.

\begin{enumerate}
\def\labelenumi{\arabic{enumi}.}
\setcounter{enumi}{4}
\tightlist
\item
  The populated matrix in \{Edit\} Mode will display the fixture numbers in each cell. In \{Live\} Mode, the fixture's color mix will be shown. CMY, RGB, and HSV color spaces are supported.
\end{enumerate}

\includegraphics{https://files.gitbook.com/v0/b/gitbook-x-prod.appspot.com/o/spaces\%2F3kS90tLsADGm1ocbe7q9\%2Fuploads\%2FebGZvb6DGOYtyqaGKSdZ\%2F19.5.1.webp?alt=media\&token=5f033d65-2db7-4ba0-918e-7980aade9cc9}

\includegraphics{https://files.gitbook.com/v0/b/gitbook-x-prod.appspot.com/o/spaces\%2F3kS90tLsADGm1ocbe7q9\%2Fuploads\%2FM8tXegFlpqyJ9QGo4FY0\%2F91.5.2.webp?alt=media\&token=dd690b9b-aa5a-4f0a-a8e2-a8f137b1f78a}

\begin{enumerate}
\def\labelenumi{\arabic{enumi}.}
\item
  Multiple pixel mappers may be added to the same Topo along with other objects.
\item
  \begin{quote}
  👉 The current Pixel Mapper is for the selection of fixtures and displaying of live output but future versions will add the ability to map content to individual pixel maps.
  \end{quote}
\end{enumerate}

\hypertarget{adding-text-and-background-elements}{%
\subsection{Adding text and background elements}\label{adding-text-and-background-elements}}

Additional elements may be added to a Topo. Currently available are \{Text Label\} and \{Add Background\} To add

\textbf{Text labels:}

\begin{enumerate}
\def\labelenumi{\arabic{enumi}.}
\item
  From the Topo Actions menu, tap \{Add Elements\}
\item
  From the drop down, tap \{Text Label\} - The Create Label pop-up will appear.
\end{enumerate}

\includegraphics{https://files.gitbook.com/v0/b/gitbook-x-prod.appspot.com/o/spaces\%2F3kS90tLsADGm1ocbe7q9\%2Fuploads\%2FHsLe6IRkyI2sKrNgJoaU\%2F19.6.png?alt=media\&token=9c3717a4-6eee-4d68-8cd5-49c6273f7953}

\begin{enumerate}
\def\labelenumi{\arabic{enumi}.}
\setcounter{enumi}{2}
\item
  Close the pop-up and the test label will appear on the Topo surface.
\item
  Drag the label to the desired location.
\item
  Transpose tools may be used to size and rotate the label.
\end{enumerate}

Backgrounds such as studio layouts and rig elements may be added to the Topo.

\textbf{To add a Background to a Topo:}

\begin{enumerate}
\def\labelenumi{\arabic{enumi}.}
\item
  Prepare a USB stick with a JPEG, BMP, or PNG piece of content.
\item
  Insert the USB stick into one of the Vibe's USB ports.
\item
  From the Topo Actions menu, tap \{Add Elements\}.
\item
  Tap \{Add Background\} - A browser pop-up will appear.
\item
  Select the USB stick and click on the desired content - Once imported, the content may be found in the default local directory on the hard-drive under D:\Vibe\Compulite\WorkDir\show\Images\Topo\\
\item
  Tap \{Load\}, {[}Enter{]} or tap to close the pop-up and place the content as a background to the Topo.
\item
  To remove the background tap \{Remove Background\}
\end{enumerate}

\hypertarget{xyz-fixture-calibration}{%
\section{XYZ Fixture Calibration}\label{xyz-fixture-calibration}}

Fixtures can be calibrated easily using XYZ values of the Visualizer. When fixtures are calibrated, you can specify a point of the stage in Vibe, and all fixtures will point to it. There are few things that need to be done in order to calibrate the fixtures, and it depends on which visualizer you use.

\textbf{The following is covered in this chapter:}

\begin{itemize}
\tightlist
\item
  \href{https://vibemanual.compulite.com/xyz-fixture-calibration.html\#calibration-in-wysiwyg}{20.1. Calibration in WYSIWYG}
\item
  \href{https://vibemanual.compulite.com/xyz-fixture-calibration.html\#calibration-in-capture}{20.2. Calibration in Capture}
\end{itemize}

\hypertarget{calibration-in-wysiwyg}{%
\subsection{Calibration in WYSIWYG}\label{calibration-in-wysiwyg}}

\textbf{Calibrate using WYSIWYG}

\begin{enumerate}
\def\labelenumi{\arabic{enumi}.}
\item
  On Vibe, go to \protect\hyperlink{patch}{PATCH} page and tap on \{FIXTURE CALIBRATION\}
\item
  Tap on \{Stage\} button and set the stage dimensions.
\item
  Specify the drawing origin point at the center of the stage. For this, it will be necessary to work in the CAD tab using the TOP view. Press on the top left corner (that connects the vertical ruler and the horizontal ruler -- (see {Blue Frame}), and select the option of 'Mover Ruler (Set View Origin)ʼ.
\item
  Select the center of your stage (see {Red Arrow} in the Picture)
\end{enumerate}

\includegraphics{https://files.gitbook.com/v0/b/gitbook-x-prod.appspot.com/o/spaces\%2F3kS90tLsADGm1ocbe7q9\%2Fuploads\%2FXCTVpvo6MK4hT6KirGey\%2F20.1.webp?alt=media\&token=539a482f-4d3b-40ea-a25e-3199c557973b}

\begin{enumerate}
\def\labelenumi{\arabic{enumi}.}
\setcounter{enumi}{4}
\item
  Go to `Data tab'.
\item
  When in the Data tab, press on Option (in the toolbar) ⟶ Document Options ⟶ Draw Defaults - and change the units to metric.
\item
\end{enumerate}

\includegraphics{https://files.gitbook.com/v0/b/gitbook-x-prod.appspot.com/o/spaces\%2F3kS90tLsADGm1ocbe7q9\%2Fuploads\%2FevwfnkgAvMLdIU4zgZBr\%2F20.1.1.webp?alt=media\&token=2ef469e1-b321-48d2-89f2-da906b386daf}

\hypertarget{calibration-in-capture}{%
\subsection{Calibration in Capture}\label{calibration-in-capture}}

\textbf{Calibrate using Capture}

\begin{enumerate}
\def\labelenumi{\arabic{enumi}.}
\item
  On Vibe, go to patch and tap on `Fixture Calibration'.
\item
  Tap on `Stage' button and set the stage dimensions.
\item
  Place your center of the stage in the intersect point of the 2 Bold Axis (when the CAD displays a view from the Top).
\end{enumerate}

\includegraphics{https://files.gitbook.com/v0/b/gitbook-x-prod.appspot.com/o/spaces\%2F3kS90tLsADGm1ocbe7q9\%2Fuploads\%2Fn5Wiw7EBsgJNFC9bYVRx\%2F20.2.webp?alt=media\&token=3224f209-dfa7-440e-8e84-4b5e88866f58}

\begin{enumerate}
\def\labelenumi{\arabic{enumi}.}
\setcounter{enumi}{3}
\item
  Go to Tools ⟶ Options and change the measurements to metric.
\item
  Select a fixture in the Capture, and you will see the XYZ coordinates when the table is showing 'Selected Itemsʼ.
  \includegraphics{https://files.gitbook.com/v0/b/gitbook-x-prod.appspot.com/o/spaces\%2F3kS90tLsADGm1ocbe7q9\%2Fuploads\%2F35qZUCLtAwnVv9zjMkVV\%2F20.2.1.webp?alt=media\&token=d469868d-cd70-496f-9573-6623cd50040f}
\item
  Copy these values (manually) to the Fixture calibration table in Vibe.
\item
  Copy the column X to the column X the column Y to the column Z the column Z to the column Y
\item
  After all values are in the Vibe fixture calibration table, select the X column (by tapping the header) and press {[}-{]}, and select the Y column and press {[}-{]}. Fixtures should now be calibrated.
\end{enumerate}

\hypertarget{time-code}{%
\section{Time Code}\label{time-code}}

Time code is used to synchronize Vibe events with the internal timeline or external SMPTE linear timecode (LTC) or MIDI timecode.

\begin{quote}
{This feature is under development and this section is preliminary.}
\end{quote}

\textbf{The following is covered in this section:}

\begin{itemize}
\item
  \href{https://vibemanual.compulite.com/working-with-time-code.html\#timecode-and-timeline-basics}{21.1. Timecode and Timeline basics}
\item
  \href{https://vibemanual.compulite.com/working-with-time-code.html\#create-a-timeline-view}{21.2. Create a Timeline View}
\item
  \href{https://vibemanual.compulite.com/working-with-time-code.html\#configuring-timecode}{21.3. Configuring Timecode}
\item
  \href{https://vibemanual.compulite.com/working-with-time-code.html\#recording-timeline-tracks}{21.4. Recording Timeline Tracks}
\item
  \href{https://vibemanual.compulite.com/working-with-time-code.html\#executing-and-editing-events}{21.5. Executing and Editing Events}
\item
  \href{https://vibemanual.compulite.com/working-with-time-code.html\#markers}{21.6. Markers}
\item
  \href{https://vibemanual.compulite.com/time-code.html\#time-format}{21.7. Time Format}
\end{itemize}

\hypertarget{timecode-and-timeline-basics}{%
\subsection{Timecode and Timeline basics}\label{timecode-and-timeline-basics}}

Vibe uses a philosophy of Timeline Tracks. There is no practice limit to the number of Timeline Tracks that may be recorded in a show but graphic performance may suffer with more than 16 Timeline Tracks. Each timeline may have its own timecode source.

Vibe can read the following sources:

\begin{itemize}
\item
  Virtual Local - Internally generated timecode.
\item
  Physical Local - SMPTE or MIDI timecode received at the SMPTE audio input or via the DIN 5 pin MIDI connector on the back of the console.
\item
  USB MIDI class compliant timecode.
\item
  MIDI over Ethernet via Compulite VC protocol.
\item
  Input sources from remote multi-User consoles and devices
\end{itemize}

Unlike some consoles, Vibe can record data from multiple controllers simultaneously. Additional timeline subtracks will be created on a per controller basis. There is no need to create event lists. Vibe also supports fader automation on timeline tracks.

\includegraphics{https://files.gitbook.com/v0/b/gitbook-x-prod.appspot.com/o/spaces\%2F3kS90tLsADGm1ocbe7q9\%2Fuploads\%2Fxr8dXGMjsyLAZDoyef5I\%2F21.1.webp?alt=media\&token=b585f40e-5bdb-4475-8098-b00bb3ebce08}

Vibe automatically matches frame rates to the incoming source.

Supported frame rates are:

\begin{itemize}
\item
  24 FPS
\item
  25 FPS
\item
  29.97 FPS drop-frame
\item
  30 FPS
\end{itemize}

\hypertarget{create-a-timeline-view}{%
\subsection{Create a Timeline View}\label{create-a-timeline-view}}

When starting a new show any Timeline views stored in Layouts will be empty and no timecode devices will be connected to the Timeline View. If no layout contains a Timeline view, a view must be added to a blank page. Virtually unlimited timecode views may be created but they will all be synchronized and show the same tracks and options.

\includegraphics{https://files.gitbook.com/v0/b/gitbook-x-prod.appspot.com/o/spaces\%2F3kS90tLsADGm1ocbe7q9\%2Fuploads\%2FnGLpgXe9CBaHIDAeyvZq\%2Fimage.png?alt=media\&token=86dd2f10-6d2a-4905-bfeb-baec212c85db}

Timeline view with two timeline tracks, the cursor, and a marker

\textbf{To add a Timeline view to a page:}

\begin{enumerate}
\def\labelenumi{\arabic{enumi}.}
\item
  Tap \{Vibe\} on the monitor that will display the Timeline - Vibe menu will open.
\item
  Tap \{Pages\} - Page display will open.
\item
  Add a blank page from either the \{Programming\} or \{Playback\} work space templates - A blank workspace will open.
\item
  Tap the Lock/Unlock icon to enable adding a new view.
\item
  On the screen with the blank workspace, open the \{Vibe\} menu again and select\}

  ⟶ \{Display Objects\}
  ⟶ \{Timeline View\} - A blank timeline view will appear on the workspace.
\item
  Size and drag the Timeline view as required then tap \{unlock/lock\} icon to lock the Topo from further adjustment - The Timeline view is now ready to add tracks to.
\end{enumerate}

\hypertarget{configuring-timecode}{%
\subsection{Configuring Timecode}\label{configuring-timecode}}

\textbf{Vibe has Timecode connectors for the following:}

\begin{itemize}
\item
  Audio input Combi Jack with:

  \begin{itemize}
  \item
    Female unbalanced line level input on 1/4'' Phone jack
  \item
    Female 3 pin XLR balanced line level input.
  \end{itemize}
\item
  5 Pin DIN connector for MIDI and MIDI timecode input.
\end{itemize}

\textbf{Additionally, the following general purpose ports may be used for Timecode:}

\begin{itemize}
\item
  USB 2 and USB 3 ports can be used for USB class compliant MIDI over USB.
\item
  Compulite devices that support MIDI over Ethernet can also be used.
\end{itemize}

\textbf{Recommended quality and level for SMPTE:}

\begin{itemize}
\item
  SMPTE is very sensitive to distortion and also requires adequate gain. A clean audio signal with a gain ranging between 0db to +5db is recommended.
\item
  As the console SMPTE input is high impedance, If long runs are to be used, it is recommended to send a 150 - 600 ohm balancing line level feed to a transformer close to the console and convert to high impedance.
\end{itemize}

\includegraphics{https://files.gitbook.com/v0/b/gitbook-x-prod.appspot.com/o/spaces\%2F3kS90tLsADGm1ocbe7q9\%2Fuploads\%2FvlG4GstkXKltPKieBniE\%2F21.3.webp?alt=media\&token=4d637f49-7313-454e-8b8e-9214d8d1b707}

1/O Settings Pop-up in Timecode Tab

\textbf{Enable Timecode:}

\begin{enumerate}
\def\labelenumi{\arabic{enumi}.}
\item
  From the patch toolbar tap \{I/O Settings\}
\item
  For Tap \{General\} tab and select:

  \begin{itemize}
  \item
    \{Local MIDI\} to enable the DIN connector MIDI.
  \item
    \{External MIDI\} to enable USB MIDI or MIDI over Ethernet.
  \end{itemize}
\item
  Tap the \{Timecode\} tab, the Timecode pop-up will appear.
\item
  Enable \{MTC In\} to enable any type of MIDI timecode input.
\item
  Enable \{SMPTE In\} to receive LTC audio timecode. (Frames are automatically matched to the incoming frame rate\}
\item
  {Currently, Vibe is only capable of being a timecode Slave but will be able to be the MIDI timecode Master in future releases.}
\end{enumerate}

\hypertarget{recording-timeline-tracks}{%
\subsection{Recording Timeline Tracks}\label{recording-timeline-tracks}}

After a Timeline View has been created, timeline tracks may be recorded. Currently, timeline events are recorded in ``real time'' and then edited.

\textbf{To create a timeline track:}

\includegraphics{https://files.gitbook.com/v0/b/gitbook-x-prod.appspot.com/o/spaces\%2F3kS90tLsADGm1ocbe7q9\%2Fuploads\%2FAiiso4f15WODVxfvLcZf\%2F21.4.webp?alt=media\&token=6d042dc3-2c94-47b0-910d-e5481466fa85}

\begin{enumerate}
\def\labelenumi{\arabic{enumi}.}
\item
  Open a Time Line View.
\item
  Tap \{\href{image.png}{}\} - The ``Create New Timeline'' pop-up will open.
\item
  Select a timeline number and name.
\item
  Choose Start from \{Blank\} Timeline as an option.
\item
  Tap the \{Select Clock\} drop down in the top left corner and select the timecode source.

  \begin{itemize}
  \item
    Virtual Local - Use internal timecode clock.
  \item
    Physical Local - Use the SMPTE or MTC timecode inputs on the console.
  \item
    Additional options will populate the list depending on the interfaced timecode device.
  \end{itemize}
\item
  Enable the red \{Record\} button or the \{+ Record\} button.

  \includegraphics{https://files.gitbook.com/v0/b/gitbook-x-prod.appspot.com/o/spaces\%2F3kS90tLsADGm1ocbe7q9\%2Fuploads\%2Fl4fNaYR34NAoNMJxt3HL\%2F21.4.1.webp?alt=media\&token=fb24e6fe-6798-432f-a7d7-b454cf666b76}

  \begin{itemize}
  \item
    Record overwrites existing events.
  \item
    (+) Record adds events to existing events.
  \end{itemize}
\item
  If using internal TC, press Play.
\item
  If using external TC start the timecode source.
\item
  The timeline will start to move and all controller presses and motorized fader moves will now be recorded on sub-tracks, one for each controller's events.
\item
  When all recording is done, either stop the external timecode source or press the Stop button if the clock source is internal.
\item
  👉 {Disable the Record button to prevent accidental overwrite of events.}
\end{enumerate}

\textbf{Timeline Features}

To trigger Timeline commands via the panel and Editor Toolbar there is an ``Active Timeline View'' indicator:

\textbf{Grid View Filters}

In the grid view of the Timeline, it is now possible to see all events from all timelines (as before) and in addition to filter events by Timeline and see only its own events.

\includegraphics{https://files.gitbook.com/v0/b/gitbook-x-prod.appspot.com/o/spaces\%2F3kS90tLsADGm1ocbe7q9\%2Fuploads\%2F1GcF6cL317OyaygZLFZV\%2Fimage.png?alt=media\&token=7b352b8c-dfaf-402d-a9fc-5e986b395212}

\hypertarget{executing-and-editing-events}{%
\subsection{Executing and Editing Events}\label{executing-and-editing-events}}

\textbf{To Execute timeline events:}

\begin{itemize}
\item
  If the source is external MIDI or SMPTE timecode, press {[}Play{]} on the timecode source. The events will synchronize to the source timecode.
\item
  If the source is internal, tap \{Play\} on the timecode view's control bar. The internal timecode will start and events will execute.
\item
  The control bar may be used to stop and rewind events when the source is internal Editing timecode events
\item
  Record may be enabled and disabled ``on the fly'' to overwrite sections of a timeline track.
\item
  Events may also be moved forwards and backward using the Set Delta pop-up in the Actions drop down.
\end{itemize}

\includegraphics{https://files.gitbook.com/v0/b/gitbook-x-prod.appspot.com/o/spaces\%2F3kS90tLsADGm1ocbe7q9\%2Fuploads\%2FI3VD6eGtDp8buWi0NKAg\%2F21.5.webp?alt=media\&token=985e10f0-a809-4482-8112-763a2766724b}

\textbf{Moving Events}

It is possible to change selected event times via the physical wheels.

There are 2 new buttons to assign the delta time for the selected events to the physical wheel and then move it.

This option is available under the \emph{Move Selection} sub-menu of the Timeline view (\emph{Wheel On, Wheel Off}).

\includegraphics{https://files.gitbook.com/v0/b/gitbook-x-prod.appspot.com/o/spaces\%2F3kS90tLsADGm1ocbe7q9\%2Fuploads\%2F0PRTpcVpuXr0euSzyomY\%2Fimage.png?alt=media\&token=5e0244e3-c571-48b0-b1c1-2d72b0778999}

Then there will be 2 physical wheels that are available for use:

\includegraphics{https://files.gitbook.com/v0/b/gitbook-x-prod.appspot.com/o/spaces\%2F3kS90tLsADGm1ocbe7q9\%2Fuploads\%2FZYiJxaCTcXmcfomXlZuM\%2Fimage.png?alt=media\&token=f1407bc6-fecc-4f8c-9615-f8a7541966dd}

The left one is for moving the selected events.

The right one is to change the time resolution for the current time movement.

\textbf{Events Selection}

To Select timeline events:

\begin{itemize}
\item
  Tap on a single event -- toggles the event selection
\item
  Tap on a playback box -- toggles the playback selection
\item
  Tap on a marker -- toggles the marker selection
\item
  Rectangle draw using touch and drag -- toggles all the events inside the rectangle
\item
  Double-tap an empty area -- clear the selection
\item
  Double tap on the time bar -- sets the current time to the selected time
\item
  Long click an event and drag -- moves the position of all the selected events
\item
  Double finger vertical scroll on the view -- will scroll the Timelines
\item
  Double finger horizontal scroll on the time bar -- will scroll the time bar
\end{itemize}

{\textbf{Important!}}

Selecting one object type (i.e.~Playback, Event, Marker\ldots) will clear all other objects selection.

Selection can have only one type of objects at a time.

\textbf{State Machine Support}

To trigger Timeline commands via the panel and Editor Toolbar there is an ``Active Timeline View'' indicator:

\includegraphics{https://files.gitbook.com/v0/b/gitbook-x-prod.appspot.com/o/spaces\%2F3kS90tLsADGm1ocbe7q9\%2Fuploads\%2FlKeiPK6tnS5KxOAvwPwp\%2Fimage.png?alt=media\&token=87396a0f-bdd3-4e7c-aee1-65341332d0e3}

If there are more than 1 Timeline Views on the layout, only 1 of them can be the ``Active'' one.

Tapping the white header of the view will select the current view as the ``Active'' and will remove the Active sign from the other views.

{\textbf{Important!}}

Each Timeline View has an active clock. All Timeline Views that have the same clock will be automatically synced.

The available commands from the panel or the Editor Toolbar are split into 2 levels:

\textbf{Commands for the View}

These commands will affect the Active Timeline View.

\begin{enumerate}
\def\labelenumi{\arabic{enumi}.}
\item
  Go to idle state
\item
  Press \emph{Timeline} from the panel
\item
  You see on the Editor Toolbar the available commands:
\end{enumerate}

\includegraphics{https://files.gitbook.com/v0/b/gitbook-x-prod.appspot.com/o/spaces\%2F3kS90tLsADGm1ocbe7q9\%2Fuploads\%2FWK0w3FLBG2M0FsJd9FEY\%2Fimage.png?alt=media\&token=cdbedd39-d8ff-44a3-8f84-68a0ed7a2204}

\textbf{Commands for Timelines}

These commands will affect the selected timelines in the sequence. Some sequences will affect the selected timelines regardless of their current view and some will affect them on the ``Active Timeline View''.

\begin{enumerate}
\def\labelenumi{\arabic{enumi}.}
\item
  Go to idle state
\item
  Press Timeline \# (+\#, ⟶\#, etc.)
\item
  You see on the Editor Toolbar the available commands:
\end{enumerate}

\includegraphics{https://files.gitbook.com/v0/b/gitbook-x-prod.appspot.com/o/spaces\%2F3kS90tLsADGm1ocbe7q9\%2Fuploads\%2FzgT27J7izwv0ermWLplh\%2Fimage.png?alt=media\&token=8ebdbcb1-4cfc-4f5d-a4db-a219ef3f88d3}

\begin{enumerate}
\def\labelenumi{\arabic{enumi}.}
\setcounter{enumi}{3}
\item
  \textbf{Sequence may also end with:}

  \textbf{a.} Store -- will create the timelines if they are not already created

  \textbf{b.} Delete -- will delete the timelines if they exist

  \textbf{c.} Text -- will open text box to give a name to the selected timelines

  \textbf{d.} Copy -- will add the timelines to the clipboard

  \textbf{e.} Paste -- will paste the copied timelines from the clipboard
\end{enumerate}

\hypertarget{markers}{%
\subsection{Markers}\label{markers}}

To aid in timeline navigation, Markers may be inserted.

\includegraphics{https://files.gitbook.com/v0/b/gitbook-x-prod.appspot.com/o/spaces\%2F3kS90tLsADGm1ocbe7q9\%2Fuploads\%2FSou2Iz4861ze8cyJBkg4\%2F21.6.webp?alt=media\&token=346cd8e4-e42b-492b-acf2-8dbb33e9ab00}

To insert a marker:

\begin{itemize}
\item
  Press the + at the desired location when the timeline is running. A marker will be inserted.
\item
  Press ⟵ to move backwards through the markers.
\item
  Press ⟶ to move forwards through the markers.
\item
  Press 🗑️ to delete a marker
\end{itemize}

\hypertarget{time-format}{%
\subsection{Time Format}\label{time-format}}

It is possible to view the timeline and events data in different time
formats.

The default time format is milliseconds, but it is also possible to use FPS format. The time format selection is possible via the view's menu.

\includegraphics{https://files.gitbook.com/v0/b/gitbook-x-prod.appspot.com/o/spaces\%2F3kS90tLsADGm1ocbe7q9\%2Fuploads\%2Fbo29Yc2ouM4IM1uP84Nl\%2Fimage.png?alt=media\&token=103f9e9d-1202-4ccc-bf97-7694f1762c3a}

\hypertarget{midi}{%
\section{MIDI}\label{midi}}

This chapter deals with configuring and assigning MIDI

\textbf{The following is covered in this chapter:}

\begin{itemize}
\tightlist
\item
  \href{https://vibemanual.compulite.com/working-with-midi.html\#what-is-midi}{22.1. What is MIDI}
\item
  \href{https://vibemanual.compulite.com/working-with-midi.html\#general-midi-configuration}{22.2. General MIDI Configuration}
\item
  \href{https://vibemanual.compulite.com/working-with-midi.html\#configuring-midi-notes-and-commands}{22.3 Configuring MIDI Notes and Commands}
\item
  \href{https://vibemanual.compulite.com/working-with-midi.html\#midi-show-control}{22.4. MIDI Show Control}
\item
  \href{https://vibemanual.compulite.com/working-with-midi.html\#midi-time-code}{22.5 MIDI Time Code}
\end{itemize}

\hypertarget{what-is-midi}{%
\subsection{What is MIDI}\label{what-is-midi}}

MIDI (Musical Instrument Digital Interface) is a protocol that was originally developed by the music industry to allow multiple MIDI sound devices to be triggered from just one source, usually a master keyboard or synthesizer. It is a control protocol and does not transmit audio. Over the years the MIDI protocol has evolved to also be a popular way to synchronize and ``Play'' lighting equipment. MIDI is based around a 7-bit architecture and therefore has a theoretical 127 possible notes for a keyboard. (Most pianos use 88 notes and synthesizers use even less)

\textbf{In the basic MIDI implementation it is possible to have:}

\begin{itemize}
\item
  16 independent MIDI channels - The channels systems listen on.
\item
  127 Note ONs - Pressing down on a keyboard.
\item
  127 Note OFFs - Lifting off a keyboard.
\item
  Note On Velocity of 1 ⟶ 127 - How hard the key is struck.
\item
  After Touch pressure of 0 ⟶ 127 - Ho muce pressure is applied after the key reaches the bottom.
\item
  127 controllers each with a range of 0 ⟶ 127 - Encoder wheels or faders.
\item
  Patch changes from 0 ⟶ 127 - change from one instrument to another.
\item
  Pitch bend wheel that has a 14-bit resolution.
\item
  A section for MIDI time code, and other manufacturer's ``System Exclusive'' information. In lighting, we often assign matching notes and controllers to common keys and wheels to synchronize playback.
\end{itemize}

\begin{quote}
E.g. A GO Button on console A is configured to send Note \#46, and a GO button on console B is configured to receive note \#46. Pressing GO on console A console will press the GO on console B. In EDM shows, DJ's frequently control lighting via MIDI touchpad controllers.
\end{quote}

\textbf{MIDI Show Control or MSC -} {Not yet implemented}

\begin{itemize}
\item
  is a significant Real Time System Exclusive extension of the international Musical Instrument Digital Interface (MIDI) standard. MSC enables all types of entertainment equipment to easily communicate with each other through the process of show control.
\item
  The MIDI Show Control protocol is an industry standard ratified by the MIDI Manufacturers Association in 1991 which allows all types of entertainment control devices to talk with each other and with computers to perform show control functions in live and canned entertainment applications. Just like musical MIDI, MSC does not transmit the actual show media - it simply transmits digital information about a multimedia performance.. * Courtesy Wikipedia
\end{itemize}

\textbf{MIDI Time Code} - {Not yet implemented}

\begin{itemize}
\item
  MIDI time code is similar to the SMPTE time code and is used to synchronize audio, video, and lighting equipment. Unlike SMPTE it does not use an audio signal.
\item
  MIDI time code is available in the following frame rates:

  \begin{itemize}
  \item
    24 frame/s (standard rate for film work)
  \item
    25 frame/s (standard rate for PAL video)
  \item
    29.97 frame/s (drop-frame timecode for NTSC video)
  \item
    30 frame/s (non-drop timecode for NTSC video)
  \end{itemize}
\end{itemize}

\begin{quote}
Vibe can receive MIDI time code but does not generate MIDI time code.
\end{quote}

\textbf{Local MIDI:}

\begin{itemize}
\tightlist
\item
  MIDI is generated and received from the 5 pin DIN connectors on the back of the console.
\end{itemize}

\textbf{MIDI over USB:}

\begin{itemize}
\tightlist
\item
  Transmits and receives MIDI over standard USB connectors. The device must be MIDI over USB class compliant to work with Windows without a driver. If the device has a Windows driver available it is recommended to install the driver for optimal results.
\end{itemize}

\textbf{MIDI over Ethernet:}

\begin{itemize}
\tightlist
\item
  Windows class compliant MIDI over Ethernet is supported. Compulite VC MIDI over Ethernet is also supported.
\end{itemize}

\hypertarget{general-midi-configuration}{%
\subsection{General MIDI Configuration}\label{general-midi-configuration}}

\textbf{I/O General:}

\includegraphics{https://files.gitbook.com/v0/b/gitbook-x-prod.appspot.com/o/spaces\%2F3kS90tLsADGm1ocbe7q9\%2Fuploads\%2FLfFsHYyFscSxarwfepfu\%2F22.2.webp?alt=media\&token=7b0b4729-7618-47fb-819f-c2cc7c95c2ec}

\textbf{General configuring Vibe MIDI:}

\begin{enumerate}
\def\labelenumi{\arabic{enumi}.}
\item
  Connect all appropriate MIDI devices and power them on.
\item
  Toggle a MIDI Connection source
\end{enumerate}

\begin{itemize}
\item
  Local MIDI.
\item
  MIDI over USB/MIDI over Ethernet.
\item
  Refresh looks for new connections if devices are added later.
\end{itemize}

\begin{enumerate}
\def\labelenumi{\arabic{enumi}.}
\setcounter{enumi}{2}
\tightlist
\item
  Press Apply to stay in I/O settings or Press the \href{image.png}{} or press {[}ENTER{]} to close the pop-up.
\end{enumerate}

\hypertarget{configuring-midi-notes-and-commands}{%
\subsection{Configuring MIDI Notes and Commands}\label{configuring-midi-notes-and-commands}}

\textbf{MIDI Notes and Commands:}

\begin{itemize}
\item
  The numbers 0 ⟶ 127 on the MIDI grid represent 10 3/4 octaves of notes from C-0 ⟶ G-10.
\item
  Tabs at the bottom filter the type of MIDI message to be received.
\item
  Channel \# may also be selected to filter just messages being received on the selected channel.
\end{itemize}

\begin{quote}
The MIDI grid is used to select any of the 127 values available for the MIDI message types selected with the tabs.
\end{quote}

\includegraphics{https://files.gitbook.com/v0/b/gitbook-x-prod.appspot.com/o/spaces\%2F3kS90tLsADGm1ocbe7q9\%2Fuploads\%2F4VPqNUiOZuUAJ3DO9bYY\%2F22.3.webp?alt=media\&token=46e4c67b-244b-42f9-b9cc-d364a6c10b64}

\textbf{Assign MIDI Notes to Controllers:}

\begin{enumerate}
\def\labelenumi{\arabic{enumi}.}
\item
  Enable Notes in or Notes out.
\item
  By default, all 16 available channels will be selected. Deselect as needed.
\item
  Select the tab for the type of MIDI message to be received. In this case \textbf{Note} (Note ON = press down on controller key, Note off = release of controller key).
\item
  Tap \{Teach \} - The Teach key will turn purple.
\item
  Tap the desired note box in the MIDI grid - The box will turn purple.
\item
  Press any one of the controller keys - The location and button function of the controller will appear in the box.
\item
  Tap the \{Teach\} key again to toggle it off.
\item
  Press the \href{image.png}{} or press {[}ENTER{]} to close the pop-up.
\end{enumerate}

\textbf{Assign MIDI Continuous Controller to Vibe Controller:}

\begin{enumerate}
\def\labelenumi{\arabic{enumi}.}
\item
  Select the channel that the MIDI message will be received on.
\item
  Select the tab for the type of MIDI message to be received. In this case \textbf{Continuous Controller}.
\item
  Tap \{Teach \} -The Teach key will turn purple.
\item
  Tap in the desired continuous controller \# in the MIDI grid - The box will turn purple.
\item
  Move any of the Vibe slider faders - The location and controller type will appear in the box.
\item
  Tap the \{Teach\} key again to toggle it off.
\item
  Press the or press {[}ENTER{]} to close the pop-up.
\end{enumerate}

\textbf{Assign Macros to a MIDI note:}

\begin{enumerate}
\def\labelenumi{\arabic{enumi}.}
\item
  Select the channel that the MIDI message will be received on.
\item
  Select the tab for the type of MIDI message to be received. In this case \textbf{Note}.
\item
  Tap \{Teach \} - The Teach key will turn purple.
\item
  Tap the desired note box in the MIDI grid - The box will turn purple.
\item
  Press {[}MACRO{]} - The Macro selector pop-up will appear.
\item
  Select or type a previously recorded \{Macro \#\} - the Macro \# will appear in the the box.
\item
  Tap the \{Teach\} key again to toggle it off.
\item
  Press the or press {[}ENTER{]} to close the pop-up.
\end{enumerate}

\textbf{Additional functions:}

\begin{itemize}
\item
  Tap \{Clear\} followed by a MIDI grid box to clear its assignment.
\item
  Tap \{Clear All\} to clear all MIDI assignments.
\item
  Tap \{Copy\} followed by the source MIDI assignment and then the destination MIDI assignment to copy from one note to another.
\item
  Tap \{Move\} followed by the source MIDI assignment and then the destination MIDI assignment to move the assignment from one note to another.
\end{itemize}

\hypertarget{midi-show-control}{%
\subsection{MIDI Show Control}\label{midi-show-control}}

Midi Show Control was developed as a simple standard method of triggering basic commands on dissimilar devices. Basic MSC commands are:

\begin{itemize}
\item
  Go
\item
  Stop
\item
  Fire - Generally used for macros
\end{itemize}

Also supported on \textbf{receive} are:

\begin{itemize}
\item
  All fixtures
\item
  General lights
\item
  Moving lights
\end{itemize}

\includegraphics{https://files.gitbook.com/v0/b/gitbook-x-prod.appspot.com/o/spaces\%2F3kS90tLsADGm1ocbe7q9\%2Fuploads\%2F1PsdpcqmxQCd0gcxgb4B\%2F22.4.webp?alt=media\&token=60eaf457-aa95-4167-bc11-47d3aa3f3f1a}

\textbf{To receive MSC:}

\begin{enumerate}
\def\labelenumi{\arabic{enumi}.}
\item
  Tap \{Enable In\}
\item
  Match the \{Device \#\} to number of the peer that will be sending the control commands.
\item
  If the Peer \# is above 111, select a Group \# that matches and all devices set to that Group \# will receive MSC packets.
\item
  If the Peer \# is 127 (Broadcast Mode) packets will be received on all devices regardless of their Device or Group numbers.
\end{enumerate}

{To Transmit MSC: This function is currently under development. }

Recommended reading: \url{http://jac.michaeldrolet.net/SCS_10_Help/scs_options_mid_msc.htm}

\hypertarget{midi-time-code}{%
\subsection{MIDI Time Code}\label{midi-time-code}}

See:
\href{https://vibemanual.compulite.com/time-code.html\#configuring-timecode}{21.3. Configuring Timecode}

\hypertarget{media-servers}{%
\section{Media Servers}\label{media-servers}}

\textbf{The following is covered in this chapter:}

\begin{itemize}
\tightlist
\item
  \href{https://vibemanual.compulite.com/media-servers.html\#media-bank}{23.1 - Media Bank}
\end{itemize}

\hypertarget{media-bank}{%
\subsection{Media Bank}\label{media-bank}}

There is a bank on the Small Screen Called Media. This bank was created to be able to handle media servers in a smart and easy way. In order to see content in this bank, users will need to create Media Server devices such as Green Hippo's Hippotizer. All virtual controls are also mapped to the physical wheels for easy use.

\includegraphics{https://files.gitbook.com/v0/b/gitbook-x-prod.appspot.com/o/spaces\%2F3kS90tLsADGm1ocbe7q9\%2Fuploads\%2FlVvUqSqPrRFKgcECEGjN\%2Fimage.png?alt=media\&token=374479f8-7747-4c66-84fd-a559a7981644}

This bank has 5 types of sub-banks:

\textbf{Media -- Media Bank}

This sub-bank includes a quick way to trigger media files from the media server. It shows all the folders and files inside the server and lets the user decide which one to play. By default, all the folders and subfolders are visible, but once synchronizing the data with the media server through CITP (see: \href{https://vibemanual.compulite.com/connecting-to-external-devices.html\#citp}{CITP Popup}) then thumbnails will be shown.

\includegraphics{https://files.gitbook.com/v0/b/gitbook-x-prod.appspot.com/o/spaces\%2F3kS90tLsADGm1ocbe7q9\%2Fuploads\%2FZuTcpOKXTIeetfKrHMc9\%2Fimage.png?alt=media\&token=cb71bf19-0489-4924-8a80-038c875e9548}

\textbf{Media -- FX (Effect) Bank}

This sub-bank includes the effects available on the media server. There are media servers with more than 1 effect engine (layer) and for each effect\textbf{,} there will be a numbered FX Bank button created. The FX Bank includes a list of effects that are possible to trigger on the server and a list of parameters to control the effect. The parameters will be dynamically changed according to the selected effect.

\includegraphics{https://files.gitbook.com/v0/b/gitbook-x-prod.appspot.com/o/spaces\%2F3kS90tLsADGm1ocbe7q9\%2Fuploads\%2FN9YimXUoD6NgyqxQFclL\%2Fimage.png?alt=media\&token=b58925d2-9e85-4671-9698-942d0984502a}

\textbf{Media -- Keystone Bank}

This sub-bank includes a graphic interface to control the keystoning of the server. There are 4 points available for movement in order to control the position of each corner of the keystone control. The selected point that is assigned to the physical wheels is marked in red.

\includegraphics{https://files.gitbook.com/v0/b/gitbook-x-prod.appspot.com/o/spaces\%2F3kS90tLsADGm1ocbe7q9\%2Fuploads\%2FNnHxiuEnYcaxvhnSctcK\%2Fimage.png?alt=media\&token=429bd8dc-45c2-40b8-b7cf-19b1f3a4ba17}

\textbf{Media -- Generator Bank}

This sub- bank is specifically made for Hippotizer. On this bank, users can control the picture that is sent from the server's layer on the Generator engine. The bank acts the same as the Media FX bank by selecting the Generator from the Generators list and then moving the Generator's parameters accordingly.

\includegraphics{https://files.gitbook.com/v0/b/gitbook-x-prod.appspot.com/o/spaces\%2F3kS90tLsADGm1ocbe7q9\%2Fuploads\%2FTP8i9jK7Qn2LqYocUTXt\%2Fimage.png?alt=media\&token=9b3bc2d2-daa5-4f5d-a121-234e9e6c8798}

\textbf{Media -- Geometry Bank}

This sub-bank includes an interface to control most of the geometric functions of the media server. There are options to control:

\begin{itemize}
\item
  Aspect Mode of the layer.
\item
  Zoom
\item
  Rotation, Indexed or Rotating with speed control
\item
  Aspect Ratio
\item
  Positioning, both X and Y
\end{itemize}

In addition there is a graphic preview to show the approximate result of the geometric parameters.

\includegraphics{https://files.gitbook.com/v0/b/gitbook-x-prod.appspot.com/o/spaces\%2F3kS90tLsADGm1ocbe7q9\%2Fuploads\%2FA7nVUjNgsguL4OhIFk9N\%2Fimage.png?alt=media\&token=15b6d68d-dff6-4d1f-b4d6-2515a479acdb}

\hypertarget{network-1}{%
\section{Network}\label{network-1}}

This chapter deals with configuring, transmitting and receiving DMX over Ethernet.

\textbf{The following is covered in this chapter:}

\begin{itemize}
\tightlist
\item
  \href{https://vibemanual.compulite.com/connecting-to-external-devices.html\#network-basics}{24.1. Network Basics}
\item
  \href{https://vibemanual.compulite.com/connecting-to-external-devices.html\#connecting-to-compulite-eports}{24.2. Connecting to Compulite ePorts}
\item
  \href{https://vibemanual.compulite.com/connecting-to-external-devices.html\#connecting-to-art-net-and-sacn-nodes}{24.3. Connecting to Art-Net and sACN Nodes}
\item
  \href{https://vibemanual.compulite.com/connecting-to-external-devices.html\#rdm}{24.4. RDM}
\item
  \href{https://vibemanual.compulite.com/connecting-to-external-devices.html\#citp}{24.5. CITP}
\end{itemize}

\hypertarget{network-basics}{%
\subsection{Network Basics}\label{network-basics}}

24.1. Network Basics

\textbf{Vibe has two user accessible Ethernet networks:}

\begin{itemize}
\item
  Data Network - Accessible via the two etherCON connectors on the rear panel.

  ◾ Used for Art-Net, sACN, Compulite VC protocol nodes and Master/Slave backup.
\item
  General Networks - Accessible via the two RJ45 connectors on the rear panel.

  ◾ Used for Network storage of show files and remote diagnostics.
\end{itemize}

Each network device must have its own unique static IP address in the form of four triplets xxx.xxx.xxx.xxx, to identify it on the network. In Vibe, this can be set in the System Settings pop-up. It is not necessary to have three digits in each segment of the IP but there must be at least one digit in each segment. Vibe ships with a default Subnet Mask of 255.0.0.0 which means that it is only important that the first triplets of the IP, (the start address), must match on all devices in the network. Devices on the same network must \textbf{not} have identical IP addresses.

\textbf{Common Start addresses are:}

\begin{itemize}
\item
  10.x.x.x - Default for most ETC networks and alternate for Art-Net network IP.
\item
  2.x.x.x - Default for Art-Net networks.
\item
  192.x.x.x -Default for most MA consoles.
\end{itemize}

\textbf{Subnet Mask} - Identifies the network itself and in conjunction with the IP address can be used to properly route network devices.

\begin{itemize}
\item
  255.0.0.0. - Compulite Default. Only the first triplet must match.
\item
  255.255.0.0 - Also common.
\end{itemize}

\begin{quote}
Compulite uses the third triplet as a console identifier in some of the products.
\end{quote}

\begin{quote}
The Vibe can be used in networks using a DHCP server but cannot act as a DHCP server.
\end{quote}

\begin{quote}
{In all cases, the first triplets must match for devices to communicate with one another.}.
\end{quote}

\textbf{Switches}:

\begin{itemize}
\item
  Vibe has two internal switches so single direct device to device connections are possible without a cross cable.
\item
  When more than one device must be connected to the Vibe a good quality gigabit switch should be used.
\end{itemize}

\begin{quote}
{Many a system has gone down due to bad Ethernet cabling and cheap switches. It is recommended that etherCON connectors be used where possible in conjunction with professional grade switches designed for lighting and audio applications.}.
\end{quote}

\hypertarget{connecting-to-compulite-eports}{%
\subsection{Connecting to Compulite ePorts}\label{connecting-to-compulite-eports}}

Compulite manufactures the \textbf{ePort} brand of DMX over Ethernet nodes. \textbf{Supported Protocols:}

\begin{itemize}
\item
  ePort 41 - Prior to Sept 2016 (check serial number with Compulite to confirm)

  ◾ Compulite VC protocol

  ◾ Art-Net v2
\item
  ePort-41 - Current

  ◾ sACN pre-release and release versions

  ◾ Art-Net 3

  ◾ Compulite VC protocol
\item
  ePort-8 and ePort-2

  ◾ sACN pre-release and release versions

  ◾ Art-Net 3

  ◾ Compulite VC protocol For more product information, See: ePort Documents
\end{itemize}

Before an ePort can be connected it should be configured using the CNET Manager software, available at the above link.

\includegraphics{https://files.gitbook.com/v0/b/gitbook-x-prod.appspot.com/o/spaces\%2F3kS90tLsADGm1ocbe7q9\%2Fuploads\%2FazG0oFLuOppI8XrwMiK8\%2F24.2.webp?alt=media\&token=439bc413-9d0a-487f-8fff-bc31e829c581}

\textbf{Configuring ePorts:}

\begin{enumerate}
\def\labelenumi{\arabic{enumi}.}
\item
  Make sure that the Vibe IP and the ePort IP are in the same range.
\item
  Assign universes to each physical port by typing the universe number in the box below the icon of the physical connector.
\item
  Select the DMX- ON-Ethernet protocol in the protocol list below the graphical display of the ePort.
\end{enumerate}

\begin{quote}
If the ePort is a newer model, sACN will also appear in the list. If Art-Net is selected make sure the Art-Net Subnet is also set. In most cases this will be set to 0 which allows the first 16 universes. It must be incremented by one with each new range of 16 universes
\end{quote}

\begin{enumerate}
\def\labelenumi{\arabic{enumi}.}
\setcounter{enumi}{3}
\tightlist
\item
  Tap \{Update\} to apply all changes.
\end{enumerate}

\hypertarget{connecting-to-art-net-and-sacn-nodes}{%
\subsection{Connecting to Art-Net and sACN Nodes}\label{connecting-to-art-net-and-sacn-nodes}}

Vibe currently supports transmission of 256 universes of DMX over Ethernet using the industry standards of Art-Net versions 1 ⟶ 3, ad eACN (ANSI E1.31-2016 standard).

\textbf{Enable Art-Net or sACN output:}

\begin{itemize}
\tightlist
\item
  Tap \{DMX SETTINGS\} in the patch toolbar found at the bottom of the Patch Workspace Template. The DMX Properties Pop-up will open.
\end{itemize}

\includegraphics{https://files.gitbook.com/v0/b/gitbook-x-prod.appspot.com/o/spaces\%2F3kS90tLsADGm1ocbe7q9\%2Fuploads\%2FzMGK8HmzzbLpRxRNYeh8\%2F24.3.webp?alt=media\&token=4f56ee15-cc74-4574-b077-257de6f78a21}

\includegraphics{https://files.gitbook.com/v0/b/gitbook-x-prod.appspot.com/o/spaces\%2F3kS90tLsADGm1ocbe7q9\%2Fuploads\%2FRZbgxqeSNTYs9CfcabMm\%2F24.3.1.webp?alt=media\&token=2fd64c6c-b903-4168-87d8-f19188f7342d}

\begin{enumerate}
\def\labelenumi{\arabic{enumi}.}
\item
  Under the Protocol heading, Tap \{Art-Net\} or \{sACN\} - The selection will turn red.
\item
  Under Broadcast Options (VC, Art-Net, sACN) tap \{Out\} - The status indicator light will turn green. The console will now be outputting the selected protocol on the etherCON Data ports.
\end{enumerate}

The Vibe is capable of transmitting more than one protocol at a time. To save bandwidth it is also possible to enable only the universes for each protocol that are required. Vibe has separate universe patches for VCs, Art-Net, and sACN. By default, universes are patched 1 to 1 for each protocol. It is possible to reroute universe output destinations. Normally fixtures patched to console source universe 1 would be transmitted on Ethernet destination universe 1 but sometimes when multiple consoles are on line it is desirable to reroute universes to avoid conflicts. Source universe 1 for example could be rerouted to destination universe 101.

\begin{quote}
The Vibe is licensed for 64, 96, 128, or 255 console source universes but these universes may be freely rerouted to any destination universe from 0 ⟶ 256 providing the total amount of universes does not exceed the purchased license.
\end{quote}

\textbf{Reroute Console Universes:}

\begin{enumerate}
\def\labelenumi{\arabic{enumi}.}
\item
  Under the \textbf{Source Universe} header, scroll the list to the source console universe number you wish to route \textbf{from}. Once tapped, the selection will turn red. A universe number may also be typed from the keypad.
\item
  Under the \textbf{Destination Universe} header, tap one of the cells in the destination universe grid. The source universe number will now be routed through the destination universe number.
\end{enumerate}

\textbf{Additional functions:}

\begin{itemize}
\item
  Clear a destination universe - Tap \{Universe \#\} in the destination grid. Tap \{Clear\}.
\item
  Clear all destination universes - Select the protocol to clear. Tap \{Clear All\}.
\item
  Reset default route for a single destination universe - Tap \{Universe \#\} Tap \{Config\}.
\item
  Reset all defaults for the selected protocol - Tap \{Config All\}.
\end{itemize}

\textbf{Art-Net Subnet View:}

\begin{itemize}
\tightlist
\item
  Art-Net is designed in blocks of 16 universes. The blocks starts at subnet 0 and increments by 1 every 16 universes. The Subnet number and fixture number in the subnet may be viewed by toggling \{Subnet View\}.
\end{itemize}

\textbf{Unicast Option for Art-Net and VC}

It is possible to set specific IPs to send the output to. If there is a port configured to a specific IP it will not be sent broadcast. Ports that are not defined as unicast will be still sent as broadcast. Each port can be configured to multiple IPs. Few ports can be sent to the same IP.

\textbf{To configure Unicast send}

\begin{enumerate}
\def\labelenumi{\arabic{enumi}.}
\item
  Tap Vibe menu button
\item
  Tap Settings button
\item
  Tap \emph{DMX Settings} button
\item
  Go to \emph{VC's Out} or \emph{Art-Net}
\item
  Change the toggle to \emph{Unicast Settings}
\item
  Tap Add to add a new configuration line
\end{enumerate}

\includegraphics{https://files.gitbook.com/v0/b/gitbook-x-prod.appspot.com/o/spaces\%2F3kS90tLsADGm1ocbe7q9\%2Fuploads\%2FUygtX7FrLFBuX7Jpr2t1\%2Fimage.png?alt=media\&token=d54d8b49-b039-488e-a894-99812a32711e}

On the first line -- ports 1 ⟶ 5 will be sent to IP 172.16.10.10

On the second line -- port 5 will be sent also to IP 172.16.10.11

On the third line -- ports 6 ⟶ 10 will be sent to IP 172.16.10.12

\hypertarget{rdm}{%
\subsection{RDM}\label{rdm}}

Currently, RDM is fully supported on the 8 physical DMX output ports and Compulite VC protocol. \textbf{View RDM information:}

\begin{enumerate}
\def\labelenumi{\arabic{enumi}.}
\item
  Open Patch Workspace.
\item
  Tap the RDM tab in the lower right corner of the workspace.
\item
  Tap \{Get Devices\} - The system will scan and display RDM compliant devices.
\item
  If new devices are added, press \{Re Scan\}. Full RDM management will be available in future releases.
\end{enumerate}

\hypertarget{citp}{%
\subsection{CITP}\label{citp}}

\textbf{CITP Popup}

This popup gathers all the CITP capabilities of the Vibe together.

\textbf{To open the CITP Popup}

\begin{enumerate}
\def\labelenumi{\arabic{enumi}.}
\item
  Tap Vibe menu button
\item
  In Settings, Tap CITP Settings
\end{enumerate}

In the popup there are 4 tabs:

\textbf{General CITP Settings}

This tab shows the general information about the console and gives the option to enable/disable CITP communication from/to this console.

\includegraphics{https://files.gitbook.com/v0/b/gitbook-x-prod.appspot.com/o/spaces\%2F3kS90tLsADGm1ocbe7q9\%2Fuploads\%2FwXqFHABn8Y02rKS0aClC\%2Fimage.png?alt=media\&token=54609c14-83fb-4f2a-a0cc-c94f4407e39e}

\textbf{Media Servers}

This tab shows a list of the detected Media Servers on the network. In order to connect to a media server and start a session, select a media server from the list and press \emph{Connect}.

In order to link a media server to a specific fixture, select a connected media server from the list and tap \emph{Configure}. A popup will appear where you can choose which fixtures you want to link.

In order to get all the available data from a media server, select a connected and linked media server from the list, and tap \emph{Get Data}. All available data such as thumbnails, effects etc. will be synced from the media server to the fixtures on your show.

\includegraphics{https://files.gitbook.com/v0/b/gitbook-x-prod.appspot.com/o/spaces\%2F3kS90tLsADGm1ocbe7q9\%2Fuploads\%2Fh3TOxEoJXcctMpwn2sJZ\%2Fimage.png?alt=media\&token=9eb3bec8-bf79-43bf-97e1-dd79baf7da90}

\textbf{Visualizers}
This tab shows a list of the detected Visualizers on the network. In order to connect to a visualizer and start a session, select it from the list and tap Connect. In order to get the patch from a connected visualizer, select it from the list and tap Get Show. This action will clear the current show, get the devices, fixtures, numbers, patch, XYZ information from the visualizer and automatically create it on your show.

\includegraphics{https://files.gitbook.com/v0/b/gitbook-x-prod.appspot.com/o/spaces\%2F3kS90tLsADGm1ocbe7q9\%2Fuploads\%2FUENsFjNTKcMOaF7bbOwM\%2Fimage.png?alt=media\&token=39bb4fba-348e-4130-9df8-344239a287c5}

\textbf{Consoles}

This tab shows a list of the detected consoles on the network that are using CITP and their state.
\includegraphics{https://files.gitbook.com/v0/b/gitbook-x-prod.appspot.com/o/spaces\%2F3kS90tLsADGm1ocbe7q9\%2Fuploads\%2FFcpkKcW84Yay2mK4SL4y\%2Fimage.png?alt=media\&token=70780c8c-d341-42d3-846f-36fe9e6be416}

\textbf{Auto Patch from Capture}

It is possible to sync the show from Capture. In order to do that you need to set the Capture CITP settings to use Standard Multicast and select the relevant IP address.

\includegraphics{https://files.gitbook.com/v0/b/gitbook-x-prod.appspot.com/o/spaces\%2F3kS90tLsADGm1ocbe7q9\%2Fuploads\%2FCt3bPG8w7qkGAHfPUj7l\%2Fimage.png?alt=media\&token=ff89b3eb-5b62-4981-80f9-1a0c7cf599b0}

In addition, make sure Vibe and Capture are on the same subnet. After finishing preparing the show on Capture, go to Vibe, open the CITP popup, enable the CITP and connect to the Capture from the Visualizers tab. Then \emph{Get Data} from Capture and you should be ready to go.

\end{document}
